\chapter*{Some solutions}

%\section*{Lecture 1}
%\section*{Lecture 2}
\section*{Lecture 3}

\begin{sol}{xca:simple=>primitive}
	Since $R$ is unitary, there exists a maximal left ideal $I$ and, moreover, $R$ is regular.
	By Proposition~\ref{proposition:R/I}, $R/I$ is a simple $R$-module. 
	Since $\Ann_R(R/I)$ is an ideal of $R$ and $R$ is simple, either $\Ann_R(R/I)\in\{0\}$ or 
	$\Ann_R(R/I)=R$. Moreover, since 
	$1\not\in\Ann(R/I)$, it follows that 
	$\Ann_R(R/I)=\{0\}$. 
\end{sol}

\begin{sol}{xca:M_n(R)primitive}
    Assume that $M_n(R)$ is primitive. Let $W$ be a faithful 
    simple $M_n(R)$-module. Let 
    $V=\{w\in W:L_1v=0\}$, where $L_1$ is a matrix
    with the first column equal to zero. Then $V$ is a subgroup
    of $W$.
    
    We claim that $V\ne\{0\}$. In fact, if 
    $x$ is a matrix where all the rows are zero except the first one, 
    then $L_1x=0$. Hence $0\ne x\in V$....
\end{sol}

\begin{sol}{xca:prim+conm=cuerpo}
	If $R$ is a field, then $R$ is primitive because it is a unitary simple ring, see  
	Exercise~\ref{xca:simple=>prim}. If $R$ is a primitive commutative ring, Proposition~\ref{proposition:R/I} implies that there exists a maximal regular ideal $I$
	such that  
	$R/I$ is a faithful simple $R$-module. 
	Since $I\subseteq \Ann_R(R/I)=\{0\}$ and $I$ is regular, there exists $e\in R$ such that 
	$r=re=er$. Therefore $R$ is a unitary commutative ring. Since $I=\{0\}$ is a maximal ideal, 
	$R$ is a field. 
\end{sol}

\section*{Lecture 4}

\begin{sol}{xca:maximal=>prim}
	Let $R$ be a ring with identity and $M$ be a maximal ideal of $R$. Then 
	$R/M$ is a simple unitary ring by 
	Proposition~\ref{proposition:R/I}. Then $R/M$ is primitive by
	Exercise~\ref{xca:simple=>prim}. By Lemma~\ref{lemma:primitivo}, 
	$M$ is primitive. 
\end{sol}

%\section*{Lecture 5}
%\section*{Lecture 6}
%\section*{Lecture 7}
%\section*{Lecture 8}
%\section*{Lecture 9}
\section*{Lecture 10}

\begin{sol}{xca:invertible_algebraic}
	Since $a$ is algebraic, 
	\[
		a^n(1+\lambda_1a+\cdots+\lambda_ma^m)=0
	\]
	for some minimal $n\geq0$ and scalars $\lambda_1,\dots,\lambda_m$. If  
	$n>0$, then 
	\[
	b=(1+\lambda_1a+\cdots+\lambda_ma^m)a^{n-1}\ne 0
	\]
	is such that $ab=ba=0$. If $n=0$, then  
	\[
		c=-\lambda_1-\lambda_2a-\cdots-\lambda_ma^{m-1}\ne 0
	\]
	is such that $ac=ca=1$. 
\end{sol}

%\section*{Lecture 9}
%\section*{Lecture 10}
%\section*{Lecture 11}
%\section*{Lecture 12}
%\section*{Lecture 13}
