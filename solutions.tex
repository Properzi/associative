\chapter{Some solutions}

\section*{Lecture 1}
\section*{Lecture 2}
\section*{Lecture 3}

\begin{sol}{xca:simple=>prim}
	Since $R$ is unitary, there exists a maximal left ideal $I$ and, moreover, $R$ is regular.
	By Proposition~\ref{proposition:R/I}, $R/I$ is a simple $R$-module. 
	Since $\Ann_R(R/I)$ is an ideal of $R$ and $R$ is simple, either $\Ann_R(R/I)\in\{0\}$ or 
	$\Ann_R(R/I)=R$. Moreover, since 
	$1\not\in\Ann(R/I)$, it follows that 
	$\Ann_R(R/I)=\{0\}$. 
\end{sol}

\begin{sol}{xca:prim+conm=cuerpo}
	If $R$ is a field, then $R$ is primitive because it is a unitary simple ring, see  
	Exercise~\ref{xca:simple=>prim}. If $R$ is a primitive commutative ring, Proposition~\ref{proposition:R/I} implies that there exists a maximal regular ideal $I$
	such that  
	$R/I$ is a faithful simple $R$-module. 
	Since $I\subseteq \Ann_R(R/I)=\{0\}$ and $I$ is regular, there exists $e\in R$ such that 
	$r=re=er$. Therefore $R$ is a unitary commutative ring. Since $I=\{0\}$ is a maximal ideal, 
	$R$ is a field. 
\end{sol}

\section*{Lecture 4}
\section*{Lecture 5}
\section*{Lecture 6}
\section*{Lecture 7}
\section*{Lecture 8}
\section*{Lecture 9}
\section*{Lecture 10}
\section*{Lecture 9}
\section*{Lecture 10}
\section*{Lecture 11}
\section*{Lecture 12}
\section*{Lecture 13}
