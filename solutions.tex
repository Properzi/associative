\chapter*{Some solutions}

%\section*{Lecture 1}
%\section*{Lecture 2}
%\section*{Lecture 3}

% ?????

% 1.10
\begin{sol}{xca:G_zero_divisors}
    Let $g\in G$ be an element of order $n>1$.  
    Then $G$ has zero divisors, as 
    \[
    0=g^n-1=(g-1)(g^{n-1}+\cdots+g+1).
    \]
\end{sol}

% 1.11
\begin{sol}{xca:units_UP}
    Let 
    \[
    \varphi\colon K[G]\to A,\quad  
    \varphi\left(\sum_{g\in G}\lambda_gg\right)=\sum_{g\in G}\lambda_gf(g).
    \]
    Routine calculations show that $\varphi$ is a well-defined
    algebra homomorphism. By definition, $\varphi(g)=f(g)$ for all $g\in G$, that is $\varphi|_G=f$. 
\end{sol}

\begin{sol}{xca:K_cyclic}
    Let $\varphi\colon K[X]\to K[G]$, $\sum\lambda_iX^i\mapsto \sum\lambda_ig^i$. Then 
    $\varphi$ is a subjective 
    algebra homomorphism with kernel $(X^n-1)$. To prove that 
    $\ker\varphi=(X^n-1)$ we proceed as follows. The inclusion 
    $(X^n-1)\subseteq\ker\varphi$ is trivial. Conversely, 
    if $p(X)\in\ker\varphi$, divide $p(X)$ by 
    $X^n-1$ to obtain polynomials $q(X)$ and $r(X)$ such that
    \[
    p(X)=(X^n-1)q(X)+r(X),
    \]
    where $r(X)=0$ or $\deg r(X)<n$. Write $r(X)=\sum_{i=0}^{n-1}\lambda_iX^i$. 
    Then
    \[
    0=\varphi(p(X))=\varphi(X^n-1)\varphi(q(X))+\varphi(r(X))
    =\sum_{i=1}^{n-1}\lambda_ig^i.
    \]
    Since $\{1,g,\dots,g^{n-1}\}$ is linearly independent, 
    $\lambda_0=\cdots=\lambda_{n-1}=0$ and hence $r(X)=0$. 
    Thus $p(X)\in (X^n-1)$. The isomorphism theorem 
    now implies that 
    \[
    K[X]/(X^n-1)\simeq K[G].
    \]
\end{sol}

\begin{sol}{xca:invertible_subgroups}
Note that $K[H]$ is included in $K[G]$. This proves the easy implications. For the non-trivial implication, we use the map 
\[ 
\pi_H\colon K[G]\to K[H],\quad \pi_H\left(\sum_{g\in G}\lambda_gg\right)=\sum_{g\in H}\lambda_gg.
\] 
Assume that $\alpha\in K[G]$ is invertible. Then there exists $\beta\in K[G]$ such that \[ 
\alpha\beta=\beta\alpha=1.
\]
Then $\alpha\pi_H(\beta)=\pi_H(\alpha\beta)=\pi_H(1)=1$ and similarly $\pi_H(\beta)\alpha=\pi_H(\beta\alpha)=1$. 

Now if $\alpha\beta=0$ for some non-zero $\beta\in K[G]$, then take $g\in G$ with $1\in\supp \beta g$. Since $\alpha(\beta g)=0$, it follows that 
\[ 
\alpha\pi_H(\beta g)=\pi_H(\alpha\beta g)=\pi_H(0)=0. 
\]
But $\pi_H(\beta g)\ne 0$, as $1\in\supp\beta g$. 
\end{sol}

\begin{sol}{xca:isos_dihedral}
Let $G$ be the group $\langle r,s:r^3=s^2=1,srs=r^{-1}\rangle$ 
and $\omega$ be a primitive cubic root of one. 
\begin{enumerate}
    \item Consider the map 
\[
\varphi\colon \C[G]\to \C\times \C\times M_2(\C),\quad 
r\mapsto (1,1,R),\quad 
s\mapsto (1,-1,S),
\]
where 
\[
R=\begin{pmatrix}
    \omega & 0\\
    0 & \omega^2
\end{pmatrix},\quad 
S=\begin{pmatrix}
    0 & 1 \\
    1 & 0
    \end{pmatrix}. 
\] 
Since $R^3=S^2=\begin{pmatrix}1&0\\0&1\end{pmatrix}$ and $SRS=R^{-1}$, the map 
is a well-defined bijective algebra homomorphism. 
\item Consider the map 
\begin{align*} 
\Q[G]\to\Q\times\Q\times M_2(\Q),&&
r\mapsto \left(1,1,\begin{pmatrix}0&1\\-1&-1\end{pmatrix}\right),&&
s\mapsto \left(1,-1,\begin{pmatrix}1&1\\0&-1\end{pmatrix}\right). 
\end{align*}
This map is a bijective algebra homomorphism. 
\end{enumerate}
\end{sol}


\begin{sol}{xca:simple=>prim} 
	Since $R$ is unitary, there exists a maximal left ideal $I$ and $R$ is regular.
	By Proposition~\ref{proposition:R/I}, $R/I$ is a simple $R$-module. 
	Since $\Ann_R(R/I)$ is an ideal of $R$ and $R$ is simple, either $\Ann_R(R/I)\in\{0\}$ or 
	$\Ann_R(R/I)=R$. Moreover, since 
	$1\not\in\Ann(R/I)$, it follows that 
	$\Ann_R(R/I)=\{0\}$. 
\end{sol}

\begin{sol}{xca:prim+conm=cuerpo}
	If $R$ is a field, then $R$ is primitive because it is a unitary simple ring, see  
	Exercise~\ref{xca:simple=>prim}. If $R$ is a primitive commutative ring, Proposition~\ref{proposition:R/I} implies that there exists a maximal regular ideal $I$
	such that  
	$R/I$ is a faithful simple $R$-module. 
	Since $I\subseteq \Ann_R(R/I)=\{0\}$ and $I$ is regular, there exists $e\in R$ such that 
	$r=re=er$. Therefore $R$ is a unitary commutative ring. Since $I=\{0\}$ is a maximal ideal, 
	$R$ is a field. 
\end{sol}

\begin{sol}{xca:maximal=>primitive}
	Let $R$ be a ring with identity and $M$ be a maximal ideal of $R$. Then 
	$R/M$ is a simple unitary ring by 
	Proposition~\ref{proposition:R/I}. Then $R/M$ is primitive by
	Exercise~\ref{xca:simple=>prim}. By Lemma~\ref{lemma:primitivo}, 
	$M$ is primitive. 
\end{sol}




% \begin{sol}{xca:M_n(R)primitive}
%     Assume that $M_n(R)$ is primitive. Let $W$ be a faithful 
%     simple $M_n(R)$-module. Let 
%     $V=\{w\in W:L_1v=0\}$, where $L_1$ is a matrix
%     with the first column equal to zero. Then $V$ is a subgroup
%     of $W$.
    
%     We claim that $V\ne\{0\}$. If 
%     $x$ is a matrix where all the rows are zero except the first one, 
%     then $L_1x=0$. Hence $0\ne x\in V$....
% \end{sol}

\begin{sol}{xca:M_n(R)primitive}
    Let $W$ be a faithful simple $M_n(R)$-module. Let 
    $L_1$ be the subset of $M_n(R)$ of matrices 
    with the first column equal to zero and 
    $V=\{w\in W:L_1v=0\}$. Then $V$ is a subgroup of $W$. 

    We claim that $V\ne\{0\}$. Let $x$ be 
    a non-zero matrix where only the first row is non-zero. Then 
    $L_1x=0$ and hence $x\in V$. 

    We claim that $xW\ne\{0\}$. 
    Let $R\times V\to W$, $(r,v)\mapsto E_{11}(r)v$, where $E_{11}(r)$ is the matrix 
    with $r$ in position $(1,1)$ and zero elsewhere. Since 
    $L_1E_{11}(r)=0$, $rv\in V$. Routine calculations show
    that $V$ is an $R$-module. 
    
    We claim that $V$ is simple. 
    If $v\in V\setminus\{0\}$, then 
    $M_n(R)V=W$. In particular, if $u\in V$, 
    there exists $a\in M_n(R)$ 
    such that $av=u$. Write $a=E_{11}(r)+l_1+c$ for
    some $r\in R$, $l_1\in L_1$ and a matrix 
    $c=(c_{ij})$ with $c_{i1}=0$ for all $i\geq2$. Then
    \[
    u=av=E_{11}(r)v+l_1v+cv.
    \]
    Since $l_1v=0$, $cv=u-E_{11}(r)v\in V$. Let $b\in M_n(R)$ 
    be such that $b=l_1'+d$ for $l_1'\in L_1$ and a matrix $d$
    with only the first column different from zero. Since $cv\in V$, 
    $l_1'cv=0$ and $dvc=0$. It follows that
    $bcv=(m+d)cv=0$. Hence $M_n(R)cv=0$ and 
    therefore $cv=0$, so $u=E_{11}(r)v=rv$. This implies that 
    $Rv=V$ and $RV=V$. Thus $\{0\}$ and $V$ are the only 
    submodules of $V$. 

    Now we prove that $V$ is faithful. If $rV=\{0\}$, then
    $E_{11}(r)V=\{0\}$. Let $v\in V\setminus\{0\}$. Then
    $M_n(R)v=W$. If $w\in W$, then there exists 
    a matrix $f$ with only the first column different from zero
    such that $fv=w$. Then
    \[
    E_{11}(r)w=E_{11}(r)fv=E_{11}(r)E_{11}(f_{11})v=0.
    \]
    It follows that $E_{11}rW=\{0\}$ and
    hence $E_{11}(r)=0$, which means $r=0$.
\end{sol}
%

% 12.6
\begin{sol}{xca:Z_semiprimitive}
    We show that $\Z$ is a non-trivial subdirect product of the family $\{\Z/p:p\text{ prime}\}$. 
    Let $f\colon\Z\to \prod_{p}\Z/p$, $f(m)=(m\bmod p)_{p}$. Since $\cap_p\Z/p=\{0\}$, routine calculations show that 
    $f$ is an injective homomorphism. For each prime number $q$, let $\pi_q\colon\prod_{p}\Z/p\to\Z/q$ be the canonical map. A straightforward computation 
    shows that 
    $\pi_pf$ is a surjective ring homomorphism. 
\end{sol}

% 12.13
\begin{sol}{xca:D_semiprime_semiprimitive}
\begin{enumerate}
    \item Let $R=D[X]$. We first prove that $R$ is semiprime. If $fRf=0$, in particular, 
    $f^2=0$. This implies that $f=0$. 
    We now prove that $R$ is semiprimitive. Let $f\in J(R)$. Then $f$ is not invertible. By Exercise \ref{xca:Jcon1}, $1-f$ is invertible. Thus $\deg(1-f)=0$ and therefore
    $f=0$. 

    \item Let $R=D[\![X]\!]$. Since $R$ is a domain, 
    it is semiprime. Let us prove that, however, $R$ is not
    semiprimitive. Let $I$ be ideal of $R$ 
    generated by $X$. 
    Since $f=\sum_{n\geq0}a_nX^n\in R$ is invertible if and only 
    if $a_0\ne 0$, one obtains that $f\in I$ for every non-invertible $f$. 
    This implies
    that every proper left ideal is contained in $I$. Hence 
    $I$ is the unique maximal left ideal of $R$. Therefore 
    $J(R)=I$. 
    \end{enumerate}
\end{sol}


\begin{sol}{xca:idempotents_modpm}
    Let $x\ne 0$ be an idempotent of $\Z/p^m$. Then $x^2\equiv x\bmod p^m$. 
    Write $x=p^ry$ for some $r$ and $y$ such that $\gcd(p,y)=1$. Then 
    \[
    p^{2r}y^2=x^2\equiv x=p^ry\bmod p^m,
    \]
    which means that $p^{m-r}$ divides $y(p^ry-1)$. Since 
    $\gcd(p,y)=1$, it follows that 
    $p^ry\equiv 1\bmod p^{m-r}$. This implies that $r=0$ and 
    therefore $x=y=1\bmod p^m$.
\end{sol}

\begin{sol}{xca:idempotents_modn}
    Let $n=p_1^{\alpha_1}\cdots p_k^{\alpha_k}$ be the decompositions 
    of $n$ into primes. For each $i$, let $R_i=\Z/p_i^{\alpha_i}$. By the 
    Chinese remainder theorem, $R\simeq R_1\times\cdots\times R_k$. Exercise~\ref{xca:idempotents_modpm} states that each $R_i$ has exactly  
    two idempotents. Thus $R$ has $2^k$ idempotents. 
\end{sol}

\begin{sol}{xca:lifting_idempotents}
    Let $x+I\in R/I$ be an idempotent. Then $x^2-x\in I$. Since $I$ is nil, 
    there exists $n$ such that $(x^2-x)^n=0$. Thus 
    \begin{align*}    
    0=(x^2-x)^n&=\sum_{i=0}^n (-1)^i\binom{n}{i}x^{n-i}
    =\sum_{i=0}^n(-1)^i\binom{n}{i}x^{n+i}\\
    &=x^n+\sum_{i=1}^n(-1)^i\binom{n}{i}x^{n+i}
    =x^n-x^{n+1}\sum_{i=1}^n\binom{n}{i}(-1)^{i-1}x^{i-1}.
    \end{align*}
    Let $y=\sum_{i=1}^n (-1)^{i-i}\binom{n}{i}r^{n-i}$. Then 
    $xy=yx$ and $x^n=x^{n+1}y$. Let $e=(xy)^n$. 

    We claim that $e^2=e$. Since $x$ and $y$ commute, 
    \[
    e^2=(xy)^{2n}=x^{2n}y^{2n}=x^{n+1}yx^{n-1}y^{2n-1}=x^{2n-1}y^{2n-1}=\cdots=x^ny^n=e.
    \]
    
    Since $x+I=x^2+I$, 
    by induction one proves that $x+I=x^k+I$ for all $k\geq1$. Since 
    \[
    x+I=x^n+I=x^{n+1}y+I=(x^{n+1}+I)(y+I)=xy+I, 
    \]
    one obtains that 
    \[
    x+I=x^n+I=(x+I)^n=(xy+I)^n=(xy)^n+I=e+I.
    \]
\end{sol}

%\section*{Lecture 5}
%\section*{Lecture 6}
%\section*{Lecture 7}
%\section*{Lecture 8}
%\section*{Lecture 9}
%\section*{Lecture 10}

\begin{sol}{xca:invertible_algebraic}
	Since $a$ is algebraic, 
	\[
		a^n(1+\lambda_1a+\cdots+\lambda_ma^m)=0
	\]
	for some minimal $n\geq0$ and scalars $\lambda_1,\dots,\lambda_m$. If  
	$n>0$, then 
	\[
	b=(1+\lambda_1a+\cdots+\lambda_ma^m)a^{n-1}\ne 0
	\]
	is such that $ab=ba=0$. If $n=0$, then  
	\[
		c=-\lambda_1-\lambda_2a-\cdots-\lambda_ma^{m-1}\ne 0
	\]
	is such that $ac=ca=1$. 
\end{sol}

\begin{sol}{xca:C_p:local}
    Note that $K[G]\simeq K[X]/(X^p-1)$ by 
    Exercise \ref{xca:K_cyclic}. Since $K$ has characteristic $p>0$, 
    $(X^p-1)=(X-1)^p$. Thus 
    \[
    K[G]\simeq K[X]/(X^p-1)=K[X]/( (X-1)^p)
    \]
    is a commutative local ring with maximal ideal $(X-1)/(X-1)^p$. 
\end{sol}

\begin{sol}{xca:K[G]_domain_easy}
    If $G$ is trivial, then $K[G]\simeq K$ is a domain. Conversely, 
    assume that $G$ is not trivial. Let $g\in G$ be an element of
    order $n>1$. Then $g^n=1$ and 
    $1,g,\dots,g^{n-1}$ are all distinct. Then $K[G]$ is not a domain, as 
  \[
  (1-g)(1+g+\cdots+g^{n-1})=1-g^n=0
  \]   
  but $1-g\ne 0$ and $1+g+\cdots+g^{n-1}\ne 0$. 
\end{sol}


\begin{sol}{xca:reduced}

\end{sol}

\begin{sol}{xca:reduced_RX}

\end{sol}

\begin{sol}{xca:reduced_central}
\end{sol}

\begin{sol}{xca:x^3=x}
\end{sol}

%\section*{Lecture 9}
%\section*{Lecture 10}
%\section*{Lecture 11}
%\section*{Lecture 12}
%\section*{Lecture 13}
