\chapter*{Some solutions}

%\section*{Lecture 1}
%\section*{Lecture 2}
%\section*{Lecture 3}

\begin{sol}{xca:simple=>primitive}
	Since $R$ is unitary, there exists a maximal left ideal $I$ and $R$ is regular.
	By Proposition~\ref{proposition:R/I}, $R/I$ is a simple $R$-module. 
	Since $\Ann_R(R/I)$ is an ideal of $R$ and $R$ is simple, either $\Ann_R(R/I)\in\{0\}$ or 
	$\Ann_R(R/I)=R$. Moreover, since 
	$1\not\in\Ann(R/I)$, it follows that 
	$\Ann_R(R/I)=\{0\}$. 
\end{sol}

\begin{sol}{xca:M_n(R)primitive}
    Assume that $M_n(R)$ is primitive. Let $W$ be a faithful 
    simple $M_n(R)$-module. Let 
    $V=\{w\in W:L_1v=0\}$, where $L_1$ is a matrix
    with the first column equal to zero. Then $V$ is a subgroup
    of $W$.
    
    We claim that $V\ne\{0\}$. If 
    $x$ is a matrix where all the rows are zero except the first one, 
    then $L_1x=0$. Hence $0\ne x\in V$....
\end{sol}

\begin{sol}{xca:prim+conm=cuerpo}
	If $R$ is a field, then $R$ is primitive because it is a unitary simple ring, see  
	Exercise~\ref{xca:simple=>prim}. If $R$ is a primitive commutative ring, Proposition~\ref{proposition:R/I} implies that there exists a maximal regular ideal $I$
	such that  
	$R/I$ is a faithful simple $R$-module. 
	Since $I\subseteq \Ann_R(R/I)=\{0\}$ and $I$ is regular, there exists $e\in R$ such that 
	$r=re=er$. Therefore $R$ is a unitary commutative ring. Since $I=\{0\}$ is a maximal ideal, 
	$R$ is a field. 
\end{sol}

\section*{Lecture 4}

\begin{sol}{xca:maximal=>prim}
	Let $R$ be a ring with identity and $M$ be a maximal ideal of $R$. Then 
	$R/M$ is a simple unitary ring by 
	Proposition~\ref{proposition:R/I}. Then $R/M$ is primitive by
	Exercise~\ref{xca:simple=>prim}. By Lemma~\ref{lemma:primitivo}, 
	$M$ is primitive. 
\end{sol}

\begin{sol}{xca:M_n(R)primitive}
    Let $W$ be a faithful simple $M_n(R)$-module. Let 
    $L_1$ be the subset of $M_n(R)$ of matrices 
    with the first column equal to zero and 
    $V=\{w\in W:L_1v=0\}$. Then $V$ is a subgroup of $W$. 

    We claim that $V\ne\{0\}$. Let $x$ be 
    a non-zero matrix where only the first row is non-zero. Then 
    $L_1x=0$ and hence $x\in V$. 

    We claim that $xW\ne\{0\}$. 
    Let $R\times V\to W$, $(r,v)\mapsto E_{11}(r)v$, where $E_{11}(r)$ is the matrix 
    with $r$ in position $(1,1)$ and zero elsewhere. Since 
    $L_1E_{11}(r)=0$, $rv\in V$. Routine calculations show
    that $V$ is an $R$-module. 
    
    We claim that $V$ is simple. 
    If $v\in V\setminus\{0\}$, then 
    $M_n(R)V=W$. In particular, if $u\in V$, 
    there exists $a\in M_n(R)$ 
    such that $av=u$. Write $a=E_{11}(r)+l_1+c$ for
    some $r\in R$, $l_1\in L_1$ and a matrix 
    $c=(c_{ij})$ with $c_{i1}=0$ for all $i\geq2$. Then
    \[
    u=av=E_{11}(r)v+l_1v+cv.
    \]
    Since $l_1v=0$, $cv=u-E_{11}(r)v\in V$. Let $b\in M_n(R)$ 
    be such that $b=l_1'+d$ for $l_1'\in L_1$ and a matrix $d$
    with only the first column different from zero. Since $cv\in V$, 
    $l_1'cv=0$ and $dvc=0$. It follows that
    $bcv=(m+d)cv=0$. Hence $M_n(R)cv=0$ and 
    therefore $cv=0$, so $u=E_{11}(r)v=rv$. This implies that 
    $Rv=V$ and $RV=V$. Thus $\{0\}$ and $V$ are the only 
    submodules of $V$. 

    Now we prove that $V$ is faithful. If $rV=\{0\}$, then
    $E_{11}(r)V=\{0\}$. Let $v\in V\setminus\{0\}$. Then
    $M_n(R)v=W$. If $w\in W$, then there exists 
    a matrix $f$ with only the first column different from zero
    such that $fv=w$. Then
    \[
    E_{11}(r)w=E_{11}(r)fv=E_{11}(r)E_{11}(f_{11})v=0.
    \]
    It follows that $E_{11}rW=\{0\}$ and
    hence $E_{11}(r)=0$, which means $r=0$.
\end{sol}
%\section*{Lecture 5}
%\section*{Lecture 6}
%\section*{Lecture 7}
%\section*{Lecture 8}
%\section*{Lecture 9}
%\section*{Lecture 10}

\begin{sol}{xca:invertible_algebraic}
	Since $a$ is algebraic, 
	\[
		a^n(1+\lambda_1a+\cdots+\lambda_ma^m)=0
	\]
	for some minimal $n\geq0$ and scalars $\lambda_1,\dots,\lambda_m$. If  
	$n>0$, then 
	\[
	b=(1+\lambda_1a+\cdots+\lambda_ma^m)a^{n-1}\ne 0
	\]
	is such that $ab=ba=0$. If $n=0$, then  
	\[
		c=-\lambda_1-\lambda_2a-\cdots-\lambda_ma^{m-1}\ne 0
	\]
	is such that $ac=ca=1$. 
\end{sol}

\begin{sol}{xca:reduced}

\end{sol}

\begin{sol}{xca:reduced_RX}

\end{sol}

\begin{sol}{xca:reduced_central}
\end{sol}

\begin{sol}{xca:x^3=x}
\end{sol}

%\section*{Lecture 9}
%\section*{Lecture 10}
%\section*{Lecture 11}
%\section*{Lecture 12}
%\section*{Lecture 13}
