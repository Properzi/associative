
\section{Locally indicable groups}
\label{section:LI}

\begin{definition}
\index{Group!indicable}
    A group $G$ is \emph{indicable} if
    there exists a non-trivial group homomorphism $G\to\Z$.
\end{definition}

We know that braid groups are indicable.
The free group $F_n$ in $n$ letters is indicable. 

\begin{definition}
\index{Group!locally indicable}
    A group $G$ is \emph{locally indicable} if every 
    non-trivial finitely generated subgroup is indicable.
\end{definition}

% \index{Burns--Hale's theorem}
% Burns--Hale's theorem (see \cite[Theorem 1.50]{MR3560661}) states that a group $G$ is left-ordered if and only if
% for every non-trivial finitely generated subgroup $H$ of $G$ there exists 
% a left-ordered group $L$ and a non-trivial group homomorphism $H\to L$. 

\begin{theorem}[Burns--Hale]
\label{thm:BurnsHale}
\index{Burns--Hale theorem}
Let $G$ be a group. Then $G$ is left ordered if and only if for each finitely generated non-trivial subgroup $H$ of $G$ there exists a left ordered group $L$ 
and a non-trivial group homomorphism $H\to L$.  
\end{theorem}

\begin{proof}
	If $G$ is a left ordered group, take $L=G$. 
	
	Conversely, suppose that for each finitely generated non-trivial subgroup $H$ of $G$ there exists a left ordered group $L$ 
and a non-trivial group homomorphism $H\to L$. We claim that for all $\{x_1,\dots,x_n\}\subseteq G\setminus\{1\}$ 
	there exist $\epsilon_1,\dots,\epsilon_n\in\{-1,1\}$ such that 
	\[
	1\not\in S(x_1^{\epsilon_1},\dots,x_n^{\epsilon_n}),
	\]
	where $S(x_1^{\epsilon_1},\dots,x_n^{\epsilon_n})$ denotes the semigroup generated by 
	the set $\{x_1^{\epsilon_1},\dots,x_n^{\epsilon_n}\}$. 
	We proceed by induction on $n$. If $n=1$, then $x_1\in G\setminus\{1\}$. Let $\epsilon_1=1$. If
	$1\in S(x_1)$, then $x_1$ is an element of finite order and hence $\langle x_1\rangle\to L$ is the trivial homomorphism for every left ordered group $L$. Hence $1\notin S(x_1)$.
    Now assume that the claim holds for some $n\geq 1$. Let $\{x_1,\dots,x_{n+1}\}\subseteq G\setminus\{1\}$. 
	By assumption, there exists a non-trivial group homomorphism 
	$h\colon\langle x_1,\dots,x_{n+1}\rangle\to L$ for some left ordered group $L$. In particular, $h(x_i)\ne 1$ for some $i\in\{1,\dots,n+1\}$. Without loss
	of generality, we may assume that there exists an integer $1\leq k\leq n+1$ such that $h(x_j)\ne 1$ for all $j\in\{1,\dots,k\}$ and 
	$h(x_j)=1$ for all $j>k$. Suppose that $L$ is left ordered with respect to a total order $\leq$. Since $h(x_j)\ne 1$ for all $j\leq k$, there
	are elements $\epsilon_j\in\{-1,1\}$ such that $1\leq h(x_j^{\epsilon_j})$ for all $j\leq k$. By the inductive hypothesis, 
	there are elements $\epsilon_{k+1},\dots,\epsilon_{n+1}\in\{-1,1\}$ such that 
	$1\not\in S(x_{k+1}^{\epsilon_{k+1}},\dots,x_{n+1}^{\epsilon_{n+1}})$. Note that for every  $x\in S(x_1^{\epsilon_1},\dots,x_{n+1}^{\epsilon_{n+1}})\setminus S(x_{k+1}^{\epsilon_{k+1}},\dots,x_{n+1}^{\epsilon_{n+1}})$, $1\leq h(x)\neq 1$. Hence $1\notin S(x_1^{\epsilon_1},\dots ,x_{n+1}^{\epsilon_{n+1}})$, and the claim follows by induction.
	
	Consider the set $\mathcal{F}$ of pairs $(F,f)$, where $F$ is a finite subset  of $G\setminus\{ 1\}$ and $f\colon F\to\{ -1, 1\}$ is a map such that for every finite subset $B$ of $G\setminus\{ 1\}$ containing $F$, there exists a map  $g\colon B\to \{ -1,1\}$ such that  $1\notin S(a^{g(a)} : a\in B)$ and $g(x)=f(x)$ for all $x\in F$. 
	
	Let $\mathcal{C}=\{ (A,f) : A\subseteq G\setminus\{ 1\}$ and $f\colon A\to \{ -1,1\}$ such that  $(F,f|_F)\in\mathcal{F}$ for all finite subset $F\text{ of }A \}$. We define and order on $\mathcal{C}$ by $(A,f)\leq (B,g)$ if and only if $A\subseteq B$ and $g(a)=f(a)$ for all $a\in A$, i. e. $f=g|_A$. Note that there is a unique map $f_{\emptyset}\colon \emptyset\to \{-1,1\}$. We have shown that $(\emptyset,f_{\emptyset})\in\mathcal{C}$. Hence $\mathcal{C}\neq \emptyset$. Furthermore, every chain of elements in $\mathcal{C}$ has an upper bound in $\mathcal{C}$. Thus, by Zorn's lemma, there exists a maximal element $(A,f)\in\mathcal{C}$. Suppose that $A\neq G\setminus\{ 1\}$. Let $x\in G\setminus (A\cup\{ 1\})$. Let $g_1\colon A\cup\{ x\}\to\{ -1,1\}$ and $g_{-1}\colon A\cup\{ x\}\to\{ -1,1\}$ be the maps defined $g_i(a)=f(a)$ for all $a\in A$ and $g_i(x)=i$ for $i\in\{ -1,1\}$. By the maximality of $(A,f)$, we have that $(A\cup\{ x\}, g_i)\not\in\mathcal{C}$. Hence there exist finite subsets $F_1$ and $F_{-1}$ of $A\cup \{ x\}$ and finite subsets $B_1$ and $B_{-1}$ of $G\setminus\{ 1\}$ such that $F_1\subseteq B_1$, $F_{-1}\subseteq B_{-1}$, $1\in S(a^{h_1(a)}: a\in B_1)$ and $1\in S(a^{h_{-1}(a)}:a\in B_2)$ for all $h_1\colon B_1\to\{ -1,1\}$ and all $h_{-1}\colon B_{-1}\to \{ -1,1\}$ such that $g_i(a)=h_i(a)$ for all $a\in F_i$. Let $C=\bigcup_{i\in\{ -1,1\}}(A\cap F_i)$. Note that $C\cup\{ x\}=F_1\cup F_{-1}\subseteq B_1\cup B_{-1}$. Since $(C,f|_{C})\in \mathcal{F}$, there exists $h\colon B_1\cup B_{-1}\to\{ -1,1\}$ such that $1\notin S(a^{h(a)}: a\in B_1\cup B_{-1})$ and $h(a)=f(a)$ for all $a\in C$. Let $i=h(x)\in\{ -1,1\}$. We have that $h(x)=g_i(x)$, and thus $h(a)=g_i(a)$ for all $a\in F_i$, a contradiction because $S(a^{h(a)}:a\in B_i)\subseteq S(a^{h(a)}: a\in B_1\cup B_{-1})$.
	Therefore $A=G\setminus\{ 1\}$. 
	
	Let $P=\{a\in G\setminus \{1\} : f(a)=1\}$. Note that if $b\in G\setminus\{ 1\}$ then 
	\[1\notin S(b^{f(b)},(b^{-1})^{f(b^{-1})}),\] 
	and thus $f(b)f(b^{-1})=-1$. Hence $G$ is the disjoint union of $P$, $P^{-1}=\{ a^{-1} : a\in P\}$ and $\{ 1\}$. Note that for all $a,b\in P$, $1\notin S(a,b,(ab)^{f(ab)})$. Hence $f(ab)=1$ and thus $ab\in P$. This proves that $P$ is a subsemigroup of $G$. We define a binary relation $\leq$ on $G$ by, for all $a,b\in G$,
	\[ a\leq b\text{ if and only if }a^{-1}b\in P\cup\{ 1\}.\]
	It is straightforward to check that $\leq$ is a total order on $G$ and that $G$ is a left ordered group with respect to $\leq$.
\end{proof}

% An immediate corollary:

% \begin{corollary}\label{cor:LIimpliesLO}
% 	Locally indicable groups are left ordered groups. 
% \end{corollary}

As a consequence, 
locally indicable groups are left-ordered. 

\begin{example}
    Since subgroups of free groups are free, 
    it follows that $F_n$ is locally indicable. 
\end{example}

There are groups that are left-ordered and not locally indicable, see
for example \cite{MR1084707}. The braid group
$\B_n$ for $n\geq5$ is another example of a left-ordered group that is not locally indicable.

\begin{proposition}
\label{pro:LI_exact}
    Let
    \[\begin{tikzcd}
	1 & K & G & Q & 0
	\arrow[from=1-1, to=1-2]
	\arrow["\alpha", from=1-2, to=1-3]
	\arrow["\beta", from=1-3, to=1-4]
	\arrow[from=1-4, to=1-5]
\end{tikzcd}
\]
    be an exact sequence of groups and group homomorphisms. 
    If $K$ and $Q$ are
    locally indicable, then $G$ is locally indicable.
\end{proposition}

\begin{proof}
    Let $g_1,\dots,g_n\in G$ and $L=\langle g_1,\dots,g_n\rangle$. 
    Assume first that $\beta(L)\ne\{1\}$. Since $Q$ is locally indicable, 
    there exists a non-trivial group homomorphism $\beta(L)\to\Z$. Then the 
    composition $L\to\beta(Q)\to\Z$ is then a non-trivial group homomorphism. Assume now
    that $\beta(L)=\{1\}$. Then there exist $k_1,\dots,k_n\in K$ 
    such that $\alpha(k_i)=g_i$ for all $i\in\{1,\dots,n\}$. Note that
    $\alpha\colon \langle k_1,\dots,k_n\rangle\to L$ is a group isomorphism. Since
    $K$ is locally indicable, there exits a non-trivial group 
    homomorphism $\langle k_1,\dots,k_n\rangle\to\Z$. 
    Thus the composition \[
    L\to\langle k_1,\dots,k_n\rangle\to\Z
    \]
    is a non-trivial
    group homomorphism 
    and hence $G$ is locally indicable. 
\end{proof}

As a consequence of the previous proposition, 
if $G$ and $H$ are locally indicable groups and 
$\sigma\colon G\to\Aut(H)$ is a group homomorphism, then 
$G\rtimes_\sigma H$ is locally indicable. In particular, the 
direct product of locally indicable groups is locally indicable.

\begin{example}
    The group $G=\langle x,y:x^{-1}yx=y^{-1}\rangle$ is
    locally indicable. We know that $G$ is torsion-free. Let 
    $K=\langle y\rangle\simeq\Z$. Then $G/K\simeq\Z$ and 
    then, since there is an exact sequence
    $1\to\Z\to G\to\Z\to1$ 
    it follows from Proposition \ref{pro:LI_exact} 
    that $G$ is locally indicable.
\end{example}

% $(x,y)\mapsto(x+1,y)$, $(x,y)\mapsto(-x,y+1)$
% A=2312, B=200-2
% Let $K=\langle y\rangle$. Then $G/K\simeq\Z$ and 
% the second result follows from the previous proposition. 

\begin{exercise}
\label{xca:B3_LI}
    Prove that $\B_3$ is locally indicable. 
\end{exercise}

The previous exercise uses the fact that $[\B_3,\B_3]$ is isomorphic to the free group in two letters, see
Exercise \ref{xca:derivedB3}.
An alternative solution to the previous fact goes as follows: $\B_3$ is the fundamental group
of the trefoil knot and fundamental groups of knots are locally indicable. 

\begin{exercise}
    Prove that $\B_4$ is locally indicable.
\end{exercise}

The previous exercise might be harder than Exercise \ref{xca:B3_LI}. One possible solution
is based on using the Reidemeister--Schreier method to prove that 
$[\B_4,\B_4]$ is a certain semidirect product 
between free groups in two generators. Another solution: Let 
$f\colon\B_4\to\B_3$ be the group homomorphism given by $f(\sigma_1)=f(\sigma_3)=\sigma_1$ 
and $f(\sigma_2)=\sigma_2$. Then 
\[
\ker f=\langle \sigma_1\sigma_3^{-1},\sigma_2\sigma_1\sigma_3^{-1}\sigma_2^{-1}\rangle
\]
is isomorphic to the free group in two letters. Now use the exact sequence
$1\to \ker f\to\B_4\to\B_3\to1$. 

\begin{exercise}
\label{xca:relations}
    Let $n\geq5$. Consider the elements of $\B_n$ given by 
    \begin{align*}
        &\beta_1=\sigma_1^{-1}\sigma_2,
        &&\beta_2=\sigma_2\sigma_1^{-1}, 
        &&\beta_3=\sigma_1\sigma_2\sigma_1^{-2},
        &&\beta_4=\sigma_3\sigma_1^{-1}, 
        &&\beta_5=\sigma_4\sigma_1^{-1}.
    \end{align*}
    Prove the following relations:
    \begin{enumerate}
        \item $\beta_1\beta_5=\beta_5\beta_2$.
        \item $\beta_2\beta_5=\beta_5\beta_3$.
        \item $\beta_1\beta_3=\beta_2$.
        \item $\beta_1\beta_4\beta_3=\beta_4\beta_2\beta_4$.
        \item $\beta_4\beta_5\beta_4=\beta_5\beta_4\beta_5$.
    \end{enumerate}
\end{exercise}

\begin{exercise}
    Let $n\geq 5$. 
    Prove that $\B_n$ is not locally indicable.
\end{exercise}

For the previous exercise one needs to show that
every group homomorphism 
\[
f\colon \langle\beta_1,\dots,\beta_5\rangle\to\Z
\]
is trivial. Hint: consider
the abelianization of $\langle\beta_1,\dots,\beta_5\rangle$.

