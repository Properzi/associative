\chapter{}

\begin{theorem}
Let $A$ be a finite-dimensional algebra and let 
$S_1,\dots,S_k$ be the simple modules over $A$. 
If 
\[
M\simeq n_1S_1\oplus\cdots\oplus n_kS_k,
\]
then each $n_j$ is uniquely determined.  
\end{theorem}

\begin{proof}
	Since each $S_j$ is simple and $S_i\not\simeq S_j$ if $i\ne j$, Schur's lemma implies that 
	$\Hom_A(S_i,S_j)=\{0\}$ whenever $i\ne j$. 
	For each $j\in\{1,\dots,k\}$, 
	\begin{align*}
		\Hom_A(M,S_j) &\simeq \Hom_A\left(\bigoplus_{i=1}^k n_i S_i,S_j\right)
		\simeq n_j\Hom_A(S_j,S_j). 
	\end{align*} 
	Since $M$ and $S_j$ are finite-dimensional vector spaces, $\Hom_A(M,S_j)$ and $\Hom_A(S_j,S_j)$ 
	are finite-dimensional vector spaces.  
	Moreover, since 
	$\id\in\Hom_A(S_j,S_j)$, it follows that 
	$\dim\Hom_A(S_j,S_j)\geq 1$. 
	Thus each $n_j$ is uniquely determined, as  
	\[ 
	n_j=\frac{\dim\Hom_A(M,S_j)}{\dim\Hom_A(S_j,S_j)}.\qedhere
	\]
\end{proof}

If $A$ is an algebra, the \textbf{opposite algebra} $A^{\op}$ is the vector space 
$A$ with multiplication $(a,b)\mapsto ba=a\cdot_{\op}b$. 

\begin{lemma}
	\label{lem:A^op}
    If $A$ is an algebra, then $A^{\op}\simeq\End_A(A)$ as algebras.  
\end{lemma}

\begin{proof}
	Note that $\End_A(A)=\{\rho_a:a\in A\}$, where $\rho_a\colon
	A\to A$, $x\mapsto xa$. Indeed, if $f\in\End_A(A)$, then 
	$f(1)=a\in A$. Moreover, $f(b)=f(b1)=bf(1)=ba$ and hence 
	$f=\rho_a$. The bijection $\End_A(A)\to A^{\op}$ is an algebra homomorphism, as 
    \[
		\rho_a\rho_b(x)=\rho_a(\rho_b(x))=\rho_a(xb)=x(ba)=\rho_{ba}(x).\qedhere
    \]
\end{proof}

\begin{lemma}
	\label{lem:Mn_op}
	If $A$ is an algebra and $n\in\N$, then $M_n(A)^{\op}\simeq
	M_n(A^{\op})$ as algebras.   
\end{lemma}

\begin{proof}
	Let $\psi\colon M_n(A)^{\op}\to M_n(A^{\op})$, $X\mapsto X^T$,
	where $X^T$ is the transpose matrix of $X$. Since $\psi$ is a bijective linear map, it is enough
	to see that $\psi$ is a homomorphism. If $i,j\in\{1,\dots,n\}$, $a=(a_{ij})$ and $b=(b_{ij})$, then 
	\begin{align*}
		(\psi(a)\psi(b))_{ij}&=\sum_{k=1}^n \psi(a)_{ik}\psi(b)_{kj}=\sum_{k=1}^n a_{ki}\cdot_{\op}b_{jk}\\
		&=\sum_{k=1}^n b_{jk}a_{ki}=(ba)_{ji}=((ba)^T)_{ij}=\psi(a\cdot_{\op}b)_{ij}.\qedhere
	\end{align*}
\end{proof}

\begin{lemma}
	\label{lem:simple}
	If $S$ is a simple module and $n\in\N$, then  
	\[
		\End_A(nS)\simeq M_n(\End_A(S))
	\]
	as algebras.
\end{lemma}

\begin{proof}
	Let $(\varphi_{ij})$ be a matrix with enties in $\End_A(S)$. We define a map
	$nS\to nS$ as follows:
	\[
	\begin{pmatrix}
	x_1\\
	\vdots\\
	x_n	
	\end{pmatrix}
	\mapsto 
		\begin{pmatrix}
			\varphi_{11} & \cdots & \varphi_{1n}\\
			\cdots & \ddots & \vdots\\
			\varphi{n1} & \cdots & \varphi_{nn}
		\end{pmatrix}
		\begin{pmatrix}
		x_1\\
		\vdots\\
		x_n	
		\end{pmatrix}
		=\begin{pmatrix}
			\varphi_{11}(x_1)+\cdots+\varphi_{1n}(x_n)\\
			\vdots\\
			\varphi_{n1}(x_1)+\cdots+\varphi_{nn}(x_n)
		\end{pmatrix}.
	\]
	The reader should prove that the map is an injective algebra homomorphism 
	\[
		M_n(\End_A(S))\to\End_A(nS).
	\]
	It is surjective: if $\psi\in\End(nS)$ and 
	$i,j\in\{1,\dots,n\}$ one defines $\psi_{ij}$ by 
	\[
		\psi\begin{pmatrix}
		x\\
		0\\
		\vdots\\
		0	
		\end{pmatrix}
		=\begin{pmatrix}
		\psi_{11}(x)\\
		\psi_{21}(x)\\
		\vdots\\
		\psi_{n1}(x)
		\end{pmatrix},\dots,
		\psi\begin{pmatrix}
		0\\
		0\\
		\vdots\\
		x	
		\end{pmatrix}
		=\begin{pmatrix}
		\psi_{1n}(x)\\
		\psi_{2n}(x)\\
		\vdots\\
		\psi_{nn}(x)
		\end{pmatrix}.\qedhere
	\]
\end{proof}


\begin{theorem}[Artin--Wedderburn]
\index{Teorema!de Artin--Wedderburn}
Sea $A$ un álgebra semisimple y de dimensión finita, digamos con 
$k$ clases de isomorfismos de $A$-módulos simples. Entonces 
\[
A\simeq M_{n_1}(D_1)\times\cdots\times M_{n_k}(D_k)
\]
para ciertos $n_1,\dots,n_k\in\N$ y ciertas álgebras de división $D_1,\dots,D_k$.
\end{theorem}

\begin{proof}
	Al agrupar los finitos
	submódulos simples de la representación regular de $A$ podemos escribir 
	\[
	A=\bigoplus_{i=1}^k n_iS_i,
	\]
	donde los $S_i$ son submódulos simples tales que $S_i\not\simeq S_j$ si
	$i\ne j$. Dejamos como ejercicio verificar que, gracias al lema de Schur, tenemos 
	\begin{align*}
		\End_A(A)\simeq\End_A\left(\bigoplus_{i=1}^kn_iS_i\right)
		\simeq \prod_{i=1}^k\End_A(n_iS_i)
		\simeq\prod_{i=1}^kM_{n_i}(\End_A(S_i)), 
	\end{align*}
	donde cada $D_i=\End_A(S_i)$ es un álgebra de división. 
	%En particular, 
	%$A$ tiene $k$ submódulos simples. 
	Tenemos entonces que %álgebras de división $D_1,\dots,D_k$ tales que 
	\[
		\End_A(A)\simeq\prod_{i=1}^kM_{n_i}(D_i).
	\]
	Como $\End_A(A)\simeq
	A^{\op}$, entonces 
	\begin{align*}
		A=(A^{\op})^{\op}\simeq \prod_{i=1}^kM_{n_i}(D_i)^{\op}\simeq \prod_{i=1}^kM_{n_i}(D_i^{\op}).
	\end{align*}
	Como además 
	cada $D_i$ es un álgebra de división, cada $D_i^{\op}$ también lo es.
\end{proof}

Utilizaremos el teorema de Wedderburn en el caso de los números complejos. 
%\begin{lemma}
%Si $A$ es un álgebra compleja de dimensión finita y $S$ es un $A$-módulo simple, entonces
%$\End_A(S)\simeq\C$. 	
%\end{lemma}
%
%\begin{proof}
%Si $\varphi\in\End_A(S)$, entonces $\varphi$ tiene un autovalor $\lambda\in\C$. Como entonces 
%$\varphi-\lambda\id$ no es un isomorfismo, el lema de Schur implica que $\varphi-\lambda\id=0$, 
%es decir $\varphi=\lambda\id$. 	
%\end{proof}

\begin{corollary}[Mollien]
	Si $A$ es un álgebra compleja de dimensión finita semisimple, entonces
	\[
	A\simeq\prod_{i=1}^k M_{n_i}(\C)
	\]  
	para ciertos $n_1,\dots,n_k\in\N$. 
\end{corollary}

\begin{proof}
	Vimos en la demostración del teorema de Wedderburn que 
	\[
	A\simeq \prod_{i=1}^k M_{n_i}(\End_A(S_i)),
	\]
	donde $S_1,\dots,S_k$ son representantes de las clases de 
	isomorfismos de los $A$-módulos simples y cada $\End_A(S_i)$ es un álgebra de división. 
	Veamos que 
	\[
	\End_A(S_i)=\{\lambda\id:\lambda\in\C\}\simeq\C
	\]
	para todo $i\in\{1,\dots,k\}$. En efecto, si 
	$f\in\End_A(S_i)$, entonces $f$ tiene un autovalor $\lambda\in\C$. Como entonces 
	$f-\lambda\id$ no es un isomorfismo, el lema de Schur implica que $f-\lambda\id=0$, 
	es decir $f=\lambda\id$. Luego $\End_A(S_i)\to\C$, $\varphi\mapsto\lambda$, 
	es un isomorfismo de álgebras. En particular, 
	\[
	A\simeq \prod_{i=1}^k M_{n_i}(\C).\qedhere
	\]
\end{proof}

\begin{exercise}
Sean $A$ y $B$ álgebras. Demuestre que los ideales de $A\times B$ son 
de la forma $I\times J$, donde $I$ es un ideal de $A$ y $J$ es un ideal de $B$. 
\end{exercise}

%\index{Módulo!fiel}
%\index{Anulador!de un módulo}
%Recordemos que un $A$-módulo $M$ se dice \textbf{fiel} si el \textbf{anulador} 
%\[
%\Ann(M)=\{a\in A:a\cdot M=0\}
%\]
%de $M$ es nulo. Observemos que $\Ann(M)$ es un ideal de $A$.
\begin{definition}
\index{Álgebra!simple}  
Un álgebra $A$ se dice \textbf{simple} si sus únicos ideales son $\{0\}$ y $A$. 
\end{definition}

%Sabemos que toda álgebra $A$ posee al menos un ideal maximal.  
%
%\begin{example}
%Sea $A$ un álgebra. Sabemos que exist un ideal maximal $I$. El cociente $A/I$ es un $A$-módulo simple. 	
%\end{example}

\begin{proposition}
	Sea $A$ un álgebra simple de dimensión finita. Entonces existe un ideal a
	izquierda no nulo $I$ de dimensión minimal.  Este ideal es un $A$-módulo
	simple y todo $A$-módulo simple es isomorfo a $I$. 	
\end{proposition}

\begin{proof}
	Como $A$ es de dimensión finita y $A$ es un ideal a izquierda de $A$, existe un ideal a izquierda no nulo $I$ de dimensión minimal. 
	La minimalidad de $\dim I$ implica que $I$ es simple como $A$-módulo. 
	
	Sea $M$ un $A$-módulo simple. En particular, $M\ne\{0\}$. 
	Como 
	\[
	\Ann(M)=\{a\in A:a\cdot M=\{0\}\}
	\]
	es un ideal de $A$ y además $1\in A\setminus\Ann(M)$, la simplicidad de $A$ implica que
	$\Ann(M)=\{0\}$ y luego $I\cdot M\ne \{0\}$ (pues $I\cdot m\ne 0$ para todo $m\in M$ implica que 
	$I\subseteq\Ann(M)$ e $I$ es no nulo, una contradicción).  
	Sea $m\in M$ tal que $I\cdot m\ne\{0\}$. La función
	\[
	\varphi\colon I\to M,\quad
	x\mapsto x\cdot m,
	\]
	es un morfismo de módulos. Como $I\cdot m\ne\{0\}$, el morfismo $\varphi$ es no nulo. 
	Como $I$ y $M$ son $A$-módulos simples, el lema de Schur implica que $\varphi$ es un isomorfismo. 
\end{proof}

Si $D$ es un álgebra de división, 
el álgebra de matrices $M_n(D)$ es un álgebra simple. La proposición anterior nos dice
en particular que $M_n(D)$ tiene una única clase de isomorfismos de $M_n(D)$-módulos simples. Como sabemos, 
estos módulos son isomorfos a $D^n$. 

\begin{proposition}
Sea $A$ un álgebra de dimensión finita. Si $A$ es simple, entonces $A$ es semisimple.	
\end{proposition}

\begin{proof}
	Sea $S$ la suma de los submódulos simples de la representación regular de $A$. 
	Afirmamos que $S$ es un ideal de $A$. Sabemos
	que $S$ es un ideal a izquierda, pues los submódulos de la representación regular de $A$ son exactamente los ideales a izquierda de $A$. 
	Para ver que $Sa\subseteq S$ para todo $a\in A$, debemos
	demostrar que $Ta\subseteq S$ para todo submódulo simple $T$ de $A$. Si $T\subseteq A$ es un submódulo simple y $a\in A$, 
	sea $f\colon T\to Ta$, $t\mapsto ta$. Como $f$ es un morfismo de $A$-módulos y $T$ es simple, $\ker f=\{0\}$ o bien $\ker T=T$. Si $\ker T=T$, entonces
	$f(T)=Ta=\{0\}\subseteq S$. Si $\ker f=\{0\}$, entonces $T\simeq f(T)=Ta$ y luego $Ta$ es simple y entonces $Ta\subseteq S$. 
	
	Como $S$ es un ideal de $A$ y $A$
	es un álgebra simple, entonces $S=\{0\}$ o bien $S=A$.  Como $S\ne\{0\}$, pues 
	existe un ideal a izquierda no nulo $I$ de $A$ tal que $I\ne\{0\}$ de dimensión minimal,  
	se concluye que $S=A$, es decir la representación regular de $A$ 
	es semisimple (por ser suma de submódulos simples) y luego el álgebra 
	$A$ es semisimple. 
\end{proof}

%\begin{lemma}
%	Sea $A=B\times C$ 
%	un producto directo de álgebras. Si $K$ es un ideal de $A$ si y sólo si $K=I\times J$ 
%	para algún ideal $I$ de $A$ y un ideal $J$ de $B$. 
%\end{lemma}
%
%\begin{proof}
%	Consideramos los morfismos de anillos
%	\begin{align*}
%		& p_B\colon B\times C\to B, &&p_B(b,c)=b,\\
%		& p_C\colon B\times C\to C, && p_C(b,c)=c.  	
%	\end{align*}
%	Si $K$ es un ideal de $A$, definimos
%	$I=p_B(K)$ y $J=p_C(K)$. Veamos que $I$ es un ideal de $B$. Como $K$ es un subgrupo aditivo de $A$ y $p_B$ es un morfismo, entonces
%	$I$ es un subgrupo aditivo de $B$. Si $x\in I=p_B(K)$, entonces
%	existe $y\in C$ tal que $(x,y)\in K$. Como $(bx,0)=(b,0)(x,y)\in K$, entonces 
%	\[
%	bx=p_B(b,0)p_B(x,y)=p_B((b,0)(x,y))\in p_B(K)=I.
%	\]
%	Similarmente se demuestra que $J$ es un ideal de $C$. Afirmamos ahora que $K=I\times J$. 
%	Si $(x,y)\in K$, entonces trivialente $x\in I$ e $y\in J$. Por otro lado, 
%	si $(x,y)\in I\times J$, entonces $x\in I=p_B(K)$ y luego existe $c\in C$ tal que
%	$(x,c)\in K$. Como $K$ es un ideal, $(1,0)(x,c)=(x,0)\in K$. Similarmente vemos que 
%	$(0,y)\in K$. 
%	En consecuencia, $(x,y)=(x,0)+(0,y)\in K$.   
%%\end{proof}
%
%El resultado anterior puede extenderse por inducción. Si $A=A_1\times\cdots\times A_k$ es producto directo de álgebras, 
%todo ideal de $A$ es de la forma $I_1\times\cdots I_k$, donde $I_j$ es un ideal de $A_j$ para todo $j\in\{1,\dots,k\}$. 
\begin{theorem}[Wedderburn]
\index{Teorema!de Wedderburn}
	Sea $A$ un álgebra de dimensión finita. 
	Si $A$ es simple, entonces $A\simeq M_n(D)$ para algún $n\in\N$ y alguna álgebra de división $D$.
\end{theorem}


\begin{proof}
	Como $A$ es simple, entonces $A$ es semisimple. El teorema de Artin--Wedderburn implica que $A\simeq\prod_{i=1}^k M_{n_i}(D_i)$ 
	para ciertos $n_1,\dots,n_k$ y ciertas álgebras de división $D_1,\dots,D_k$. Además $A$ tiene
	$k$ clases de isomorfismos de módulos simples. Como $A$ es simple,
	$A$ tiene solamente una clase de isomorfismos de módulos simples. Luego $k=1$ y entonce
	$A\simeq M_n(D)$ para algún $n\in\N$ y alguna álgebra de división $D$. 
\end{proof}








Let $A$ be an algebra over $K$. If $I$ is a left ideal of the ring $A$, then 
$I$ is a subspace (over $K$), as $\lambda a=\lambda(1_Aa)=(\lambda 1_A)a$ 
for all $\lambda\in K$ and $a\in A$.  



An important example of a module is given by the left representation. The 
algebra $A$ is a module over $A$ with the left multiplication. 

