\lecture{}

\begin{definition}
    An algebra $A$ is \textbf{semisimple} if every finitely-generated $A$-module is semisimple. 
\end{definition}

\begin{proposition}
Let $A$ be a finite-dimensional algebra. Then $A$ is semisimple if and only if 
the regular representation of $A$ is semisimple. 
\end{proposition}

\begin{proof}
Let us prove the non-trivial implication. Let $M$ be a finitely-generated module, say $M=(m_1,\dots,m_k)$. 
The map
\[
\bigoplus_{i=1}^k A\to M,\quad
(a_1,\dots,a_k)\mapsto \sum_{i=1}^k a_i\cdot m_i,
\]
is a surjective homomorphism of modules, where $A$ is considered as a module with the regular
representation. Since 
$A$ is semisimple, it follows that $\oplus_{i=1}^kA$ is semisimple. 
Thus $M$ is semisimple, as it is isomorphic to the quotient of a semisimple module.
\end{proof}

\begin{theorem}
Let $A$ be a finite-dimensional semisimple algebra. 
Assume that the regular representation can be decomposed as $\prescript{}{A}A=\oplus_{i=1}^k S_i$ where each $S_i$ is a simple submodule.  
If $S$ is a simple module, then $S\simeq S_i$ for some $i\in\{1,\dots,k\}$. 
\end{theorem}

\begin{proof}
Let $s\in S\setminus\{0\}$. The map $\varphi\colon A\to S$, $a\mapsto a\cdot s$, is a surjective module homomorphism. Since 
$\varphi\ne 0$, there exists $i\in\{1,\dots,k\}$ such that some restriction 
$\varphi|_{S_i}\colon S_i\to S$ is non-zero. By Schur's lemma, it follows that  
$\varphi|_{S_i}$ is an isomorphism.   	
\end{proof}

As a corollary, a finite-dimensional semisimple algebra admits 
only finitely many isomorphism classes of simple modules. When we say that 
the $S_1,\dots,S_k$ are the simple modules of an algebra, this means that the $S_i$ are the representatives
of isomorphism classes of all simple modules of the algebra, that is that each simple module is isomorphic to
some $S_i$ and, moreover,  
$S_i\not\simeq S_j$ whenever $i\ne j$. 

\begin{exercise}
If $A$ and $B$ are algebras, $M$ is a module over $A$ and $N$ is a module over $B$, then 
    $M\oplus N$ is a module over $A\times B$ with 
    \[
    (a,b)\cdot (m,n)=(a\cdot m,b\cdot n).
    \]
\end{exercise}


A \textbf{division algebra} $D$ is an algebra such that every non-zero element 
is invertible, that is for all $x\in D\setminus\{0\}$ there exists $y\in D$ such that $xy=yx=1$.  
Modules over division algebras are very much like vector spaces.  For example, 
every finitely-generated module $M$ over a division algebra has a basis. 
Moreover, every linearly independent subset of
$M$ can be extended into a basis of $M$. 

\begin{proposition}
	Let $D$ be a division algebra and $V$ be a finitely-generated module over $D$. Then 
	$V$ is a simple module over $\End_D(V)$ and there exits $n\in\Z_{>0}$ such that  
	$\End_D(V)\simeq nV$ is semisimple.
\end{proposition}

\begin{proof}[Sketch of the proof]
	Let $\{v_1,\dots,v_n\}$ be a basis of $V$. A direct calculation shows that the map 
	\[
		\End_D(V)\to\bigoplus_{i=1}^nV=nV,\quad
		f\mapsto (f(v_1),\dots,f(v_n)),
	\]
	is an injective homomorphism of $\End_D(V)$-modules.
	Since
	\[
	\dim\End_D(V)=n^2=\dim(nV),
	\]
	it follows that the map is an isomorphism. 
	Thus 
	\[
		\End_D(V)\simeq \bigoplus_{i=1}^nV.
	\]
	
	It remains to show that $V$ is simple. It is enough to prove that $V=(v)$ 
	for all $v\in V\setminus\{0\}$. Let $v\in V\setminus\{0\}$. If $w\in V$, then 
	there exists $f\in\End_D(V)$ such that $f\cdot v=f(v)=w$. 
	Thus $w\in (v)$ and therefore $V=(v)$.  
\end{proof}

The proposition states that if $D$ is a division algebra, then  
$D^{n}$ is a simple $M_n(D)$-module and that $M_n(D)\simeq n D^n$ as $M_n(D)$-modules. 

\begin{exercise}
    Let $M$, $N$ and $X$ be modules. Prove that 
    \begin{align}
        \Hom_A(M\oplus N,X)=\Hom_A(M,X)\times\Hom_A(N,X).
%        \Hom_A(X,M\oplus N)=\Hom_A(X,M)\times\Hom_A(X,N).
    \end{align}
\end{exercise}

\begin{theorem}
Let $A$ be a finite-dimensional algebra and let 
$S_1,\dots,S_k$ be the simple modules over $A$. 
If 
\[
M\simeq n_1S_1\oplus\cdots\oplus n_kS_k,
\]
then each $n_j$ is uniquely determined.  
\end{theorem}

\begin{proof}
	Since each $S_j$ is simple and $S_i\not\simeq S_j$ if $i\ne j$, Schur's lemma implies that 
	$\Hom_A(S_i,S_j)=\{0\}$ whenever $i\ne j$. 
	For each $j\in\{1,\dots,k\}$, routine calculations show that 
	\begin{align*}
		\Hom_A(M,S_j) &\simeq \Hom_A\left(\bigoplus_{i=1}^k n_i S_i,S_j\right)
		\simeq n_j\Hom_A(S_j,S_j). 
	\end{align*} 
	Since $M$ and $S_j$ are finite-dimensional vector spaces, $\Hom_A(M,S_j)$ and $\Hom_A(S_j,S_j)$ 
	are finite-dimensional vector spaces.  
	Moreover, since 
	$\id\in\Hom_A(S_j,S_j)$, it follows that 
	$\dim\Hom_A(S_j,S_j)\geq 1$. 
	Thus each $n_j$ is uniquely determined, as  
	\[ 
	n_j=\frac{\dim\Hom_A(M,S_j)}{\dim\Hom_A(S_j,S_j)}.\qedhere
	\]
\end{proof}

If $A$ is an algebra, the \textbf{opposite algebra} $A^{\op}$ is the vector space 
$A$ with multiplication $A\times A\to A$, $(a,b)\mapsto ba=a\cdot_{\op}b$. Clearly,
$A$ is commutative if and only if $A=A^{\op}$. 

\begin{lemma}
	\label{lem:A^op}
    If $A$ is an algebra, then $A^{\op}\simeq\End_A(A)$ as algebras.  
\end{lemma}

\begin{proof}
	Note that $\End_A(A)=\{\rho_a:a\in A\}$, where $\rho_a\colon
	A\to A$, $x\mapsto xa$. Indeed, if $f\in\End_A(A)$, then 
	$f(1)=a\in A$. Moreover, $f(b)=f(b1)=bf(1)=ba$ and hence 
	$f=\rho_a$. The map $A^{\op}\to \End_A(A)$, $a\mapsto\rho_a$, 
	is bijective and it is an algebra homomorphism, as 
    \[
		\rho_a\rho_b(x)=\rho_a(\rho_b(x))=\rho_a(xb)=x(ba)=\rho_{ba}(x).\qedhere
    \]
\end{proof}

\begin{lemma}
	\label{lem:Mn_op}
	If $A$ is an algebra and $n\in\Z_{>0}$, then $M_n(A)^{\op}\simeq
	M_n(A^{\op})$ as algebras.   
\end{lemma}

\begin{proof}
	Let $\psi\colon M_n(A)^{\op}\to M_n(A^{\op})$, $X\mapsto X^T$,
	where $X^T$ is the transpose matrix of $X$. Since $\psi$ is a bijective linear map, it is enough
	to see that $\psi$ is a homomorphism. If $i,j\in\{1,\dots,n\}$, $a=(a_{ij})$ and $b=(b_{ij})$, then 
	\begin{align*}
		(\psi(a)\psi(b))_{ij}&=\sum_{k=1}^n \psi(a)_{ik}\psi(b)_{kj}=\sum_{k=1}^n a_{ki}\cdot_{\op}b_{jk}\\
		&=\sum_{k=1}^n b_{jk}a_{ki}=(ba)_{ji}=((ba)^T)_{ij}=\psi(a\cdot_{\op}b)_{ij}.\qedhere
	\end{align*}
\end{proof}

\begin{lemma}
	\label{lem:simple}
	If $S$ is a simple module and $n\in\Z_{>0}$, then  
	\[
		\End_A(nS)\simeq M_n(\End_A(S))
	\]
	as algebras.
\end{lemma}

\begin{proof}
	Let $(\varphi_{ij})$ be a matrix with entries in $\End_A(S)$. We define a map
	$nS\to nS$ as follows:
	\[
	\begin{pmatrix}
	x_1\\
	\vdots\\
	x_n	
	\end{pmatrix}
	\mapsto 
		\begin{pmatrix}
			\varphi_{11} & \cdots & \varphi_{1n}\\
			\cdots & \ddots & \vdots\\
			\varphi{n1} & \cdots & \varphi_{nn}
		\end{pmatrix}
		\begin{pmatrix}
		x_1\\
		\vdots\\
		x_n	
		\end{pmatrix}
		=\begin{pmatrix}
			\varphi_{11}(x_1)+\cdots+\varphi_{1n}(x_n)\\
			\vdots\\
			\varphi_{n1}(x_1)+\cdots+\varphi_{nn}(x_n)
		\end{pmatrix}.
	\]
	The reader should prove that the map  
	\[
		M_n(\End_A(S))\to\End_A(nS)
	\]
	is an injective algebra homomorphism. 
	It is surjective. Indeed, if $\psi\in\End(nS)$ and 
	$i,j\in\{1,\dots,n\}$ one defines $\psi_{ij}$ by 
	\[
		\psi\begin{pmatrix}
		x\\
		0\\
		\vdots\\
		0	
		\end{pmatrix}
		=\begin{pmatrix}
		\psi_{11}(x)\\
		\psi_{21}(x)\\
		\vdots\\
		\psi_{n1}(x)
		\end{pmatrix},\dots,
		\psi\begin{pmatrix}
		0\\
		0\\
		\vdots\\
		x	
		\end{pmatrix}
		=\begin{pmatrix}
		\psi_{1n}(x)\\
		\psi_{2n}(x)\\
		\vdots\\
		\psi_{nn}(x)
		\end{pmatrix}.\qedhere
	\]
\end{proof}

\begin{exercise}
    Let $M$, $N$ and $X$ be modules. Prove that 
    \begin{align}
%        \Hom_A(M\oplus N,X)=\Hom_A(M,X)\times\Hom_A(N,X).
        \Hom_A(X,M\oplus N)=\Hom_A(X,M)\times\Hom_A(X,N).
    \end{align}
\end{exercise}

\begin{theorem}[Artin--Wedderburn]
Let $A$ be a finite-dimensional semisimple algebra, say with  
$k$ isomorphism classes of simple modules. Then 
\[
A\simeq M_{n_1}(D_1)\times\cdots\times M_{n_k}(D_k)
\]
for some $n_1,\dots,n_k\in\Z_{>0}$ and some division algebras $D_1,\dots,D_k$.
\end{theorem}

\begin{proof}
    Decompose the regular representation as a sum of simple modules and
    gather the simples by isomorphism classes to get 	
    \[
	A=\bigoplus_{i=1}^k n_iS_i,
	\]
	where each $S_i$ is simple and $S_i\not\simeq S_j$ whenever 
	$i\ne j$. Schur's lemma implies that  
	\begin{align*}
		\End_A(A)\simeq\End_A\left(\bigoplus_{i=1}^kn_iS_i\right)
		\simeq \prod_{i=1}^k\End_A(n_iS_i)
		\simeq\prod_{i=1}^kM_{n_i}(\End_A(S_i)), 
	\end{align*}
	where each $D_i=\End_A(S_i)$ is a division algebra. 
    Thus
    \[
		\End_A(A)\simeq\prod_{i=1}^kM_{n_i}(D_i).
	\]
	Since $\End_A(A)\simeq
	A^{\op}$, it follows that  
	\begin{align*}
		A=(A^{\op})^{\op}\simeq \prod_{i=1}^kM_{n_i}(D_i)^{\op}\simeq \prod_{i=1}^kM_{n_i}(D_i^{\op}).
	\end{align*}
	Since each $D_i$ is a division algebra, each $D_i^{\op}$ is also a division algebra.
\end{proof}

\begin{corollary}[Mollien]
	If $A$ is a finite-dimensional complex semisimple algebra, then 
	\[
	A\simeq\prod_{i=1}^k M_{n_i}(\C)
	\]  
	for some $n_1,\dots,n_k\in\Z_{>0}$. 
\end{corollary}

\begin{proof}
	By Wedderburn's theorem,  
	\[
	A\simeq \prod_{i=1}^k M_{n_i}(\End_A(S_i)^{\op}),
	\]
	where $S_1,\dots,S_k$ are representatives of the isomorphism classes of simple modules
	and each $\End_A(S_i)$ is a division algebra. We claim that 
	\[
	\End_A(S_i)=\{\lambda\id:\lambda\in\C\}\simeq\C
	\]
	for all $i\in\{1,\dots,k\}$. If  
	$f\in\End_A(S_i)$, then $f$ has an eigenvector $\lambda\in\C$. Since  
	$f-\lambda\id$ is not an isomorphism, Schur's lemma implies that $f-\lambda\id=0$, 
	that is $f=\lambda\id$. Thus $\End_A(S_i)\to\C$, $\varphi\mapsto\lambda$, 
	is an algebra isomorphism. In particular,  
	\[
	A\simeq \prod_{i=1}^k M_{n_i}(\C).\qedhere
	\]
\end{proof}

% \begin{exercise}
%     Let $A$ and $B$ be algebras. Prove that the ideals of $A\times B$ are of the form 
%     $I\times J$, where $I$ is an ideal of $A$ and $J$ is an ideal of $B$.
% \end{exercise}



