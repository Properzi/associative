\chapter{}


\begin{exercise}
If $A$ and $B$ are algebras, $M$ is a module over $A$ and $N$ is a module over $B$, then 
    $M\oplus N$ is a module over $A\times B$ with 
    \[
    (a,b)\cdot (m,n)=(a\cdot m,b\cdot n).
    \]
\end{exercise}


\index{Division algebra}
A \textbf{division algebra} $D$ is an algebra such that every non-zero element 
is invertible, that is for all $x\in D\setminus\{0\}$ there exists $y\in D$ such that $xy=yx=1$.  
Modules over division algebras are very much like vector spaces.  For example, 
every finitely generated module $M$ over a division algebra has a basis. 
Moreover, every linearly independent subset of
$M$ can be extended into a basis of $M$. 

\begin{proposition}
	Let $D$ be a division algebra, and $V$ be a finitely generated module over $D$. Then 
	$V$ is a simple module over $\End_D(V)$ and there exits $n\in\Z_{>0}$ such that  
	$\End_D(V)\simeq nV$ is semisimple.
\end{proposition}

\begin{proof}[Sketch of the proof]
	Let $\{v_1,\dots,v_n\}$ be a basis of $V$. A direct calculation shows that the map 
	\[
		\End_D(V)\to\bigoplus_{i=1}^nV=nV,\quad
		f\mapsto (f(v_1),\dots,f(v_n)),
	\]
	is an injective homomorphism of $\End_D(V)$-modules.
	Since
	\[
	\dim_D\End_D(V)=n^2=\dim_D(nV),
	\]
	it follows that the map is an isomorphism. 
	Thus 
	\[
		\End_D(V)\simeq \bigoplus_{i=1}^nV.
	\]
	
	It remains to show that $V$ is simple. It is enough to prove that $V=\End_D(V)\cdot v=(v)$ 
	for all $v\in V\setminus\{0\}$. Let $v\in V\setminus\{0\}$. If $w\in V$, then 
	there exists $f\in\End_D(V)$ such that $f\cdot v=f(v)=w$. 
	Thus $w\in (v)$ and therefore $V=(v)$.  
\end{proof}

The proposition states that if $D$ is a division algebra, then  
$D^{n}$ is a simple $M_n(D)$-module and that $M_n(D)\simeq n D^n$ as $M_n(D)$-modules. 

\begin{exercise}
    Let $M$, $N$, and $X$ be modules. Prove that 
    \begin{align}
        \Hom_A(M\oplus N,X)\simeq\Hom_A(M,X)\times\Hom_A(N,X).
    \end{align}
\end{exercise}

\begin{theorem}
Let $A$ be a finite-dimensional algebra and let 
$S_1,\dots,S_k$ be the simple modules over $A$. 
If 
\[
M\simeq n_1S_1\oplus\cdots\oplus n_kS_k,
\]
then each $n_j$ is uniquely determined.  
\end{theorem}

\begin{proof}
	Since each $S_j$ is simple and $S_i\not\simeq S_j$ if $i\ne j$, 
    Schur's lemma implies that 
	$\Hom_A(S_i,S_j)=\{0\}$ whenever $i\ne j$. 
	For each $j\in\{1,\dots,k\}$, routine calculations show that 
	\begin{align*}
		\Hom_A(M,S_j) &\simeq \Hom_A\left(\bigoplus_{i=1}^k n_i S_i,S_j\right)
		\simeq n_j\Hom_A(S_j,S_j). 
	\end{align*} 
	Since $M$ and $S_j$ are finite-dimensional vector spaces, it follows that
	$\Hom_A(M,S_j)$ and $\Hom_A(S_j,S_j)$ 
	are both finite-dimensional vector spaces.  
	Moreover, the identity $\id\colon S_j\to S_j$ 
	is a module homomorphism and hence  
%	$\id\in\Hom_A(S_j,S_j)$ and hence 
	$\dim\Hom_A(S_j,S_j)\geq 1$. 
	Thus each $n_j$ is uniquely determined, as  
	\[ 
	n_j=\frac{\dim\Hom_A(M,S_j)}{\dim\Hom_A(S_j,S_j)}.\qedhere
	\]
\end{proof}

If $A$ is an algebra, the \textbf{opposite algebra} $A^{\op}$ is the vector space 
$A$ with multiplication $A\times A\to A$, $(a,b)\mapsto ba=a\cdot_{\op}b$. Clearly,
$A$ is commutative if and only if $A=A^{\op}$. 

\begin{lemma}
	\label{lem:A^op}
    If $A$ is an algebra, then $A^{\op}\simeq\End_A(A)$ as algebras.  
\end{lemma}

\begin{proof}
	Note that $\End_A(A)=\{\rho_a:a\in A\}$, where $\rho_a\colon
	A\to A$, $x\mapsto xa$. Indeed, if $f\in\End_A(A)$, then 
	$f(1)=a\in A$. Moreover, $f(b)=f(b1)=bf(1)=ba$ and hence 
	$f=\rho_a$. The map $A^{\op}\to \End_A(A)$, $a\mapsto\rho_a$, 
	is bijective and it is an algebra homomorphism, as 
    \[
		\rho_a\rho_b(x)=\rho_a(\rho_b(x))=\rho_a(xb)=x(ba)=\rho_{ba}(x).\qedhere
    \]
\end{proof}

\begin{lemma}
	\label{lem:Mn_op}
	If $A$ is an algebra and $n\in\Z_{>0}$, then $M_n(A)^{\op}\simeq
	M_n(A^{\op})$ as algebras.   
\end{lemma}

\begin{proof}
	Let $\psi\colon M_n(A)^{\op}\to M_n(A^{\op})$, $X\mapsto X^T$,
	where $X^T$ is the transpose matrix of $X$. Since $\psi$ is a bijective linear map, it is enough
	to see that $\psi$ is a homomorphism. If $i,j\in\{1,\dots,n\}$, $a=(a_{ij})$ and $b=(b_{ij})$, then 
	\begin{align*}
		(\psi(a)\psi(b))_{ij}&=\sum_{k=1}^n \psi(a)_{ik}\psi(b)_{kj}=\sum_{k=1}^n a_{ki}\cdot_{\op}b_{jk}\\
		&=\sum_{k=1}^n b_{jk}a_{ki}=(ba)_{ji}=((ba)^T)_{ij}=\psi(a\cdot_{\op}b)_{ij}.\qedhere
	\end{align*}
\end{proof}

\begin{lemma}
	\label{lem:simple}
	If $S$ is a simple module and $n\in\Z_{>0}$, then  
	\[
		\End_A(nS)\simeq M_n(\End_A(S))
	\]
	as algebras.
\end{lemma}

\begin{proof}[Sketch of the proof]
	Let $(\varphi_{ij})$ be a matrix with entries in $\End_A(S)$. We define a map
	$nS\to nS$ as follows:
	\[
	\begin{pmatrix}
	x_1\\
	\vdots\\
	x_n	
	\end{pmatrix}
	\mapsto 
		\begin{pmatrix}
			\varphi_{11} & \cdots & \varphi_{1n}\\
			\cdots & \ddots & \vdots\\
			\varphi{n1} & \cdots & \varphi_{nn}
		\end{pmatrix}
		\begin{pmatrix}
		x_1\\
		\vdots\\
		x_n	
		\end{pmatrix}
		=\begin{pmatrix}
			\varphi_{11}(x_1)+\cdots+\varphi_{1n}(x_n)\\
			\vdots\\
			\varphi_{n1}(x_1)+\cdots+\varphi_{nn}(x_n)
		\end{pmatrix}.
	\]
	The reader should prove that the map  
	\[
		M_n(\End_A(S))\to\End_A(nS)
	\]
	is an injective algebra homomorphism. 
	It is surjective. Indeed, if $\psi\in\End_A(nS)$ and 
	$i,j\in\{1,\dots,n\}$ one defines $\psi_{ij}$ by 
	\[
		\psi\begin{pmatrix}
		x\\
		0\\
		\vdots\\
		0	
		\end{pmatrix}
		=\begin{pmatrix}
		\psi_{11}(x)\\
		\psi_{21}(x)\\
		\vdots\\
		\psi_{n1}(x)
		\end{pmatrix},\dots,
		\psi\begin{pmatrix}
		0\\
		0\\
		\vdots\\
		x	
		\end{pmatrix}
		=\begin{pmatrix}
		\psi_{1n}(x)\\
		\psi_{2n}(x)\\
		\vdots\\
		\psi_{nn}(x)
		\end{pmatrix}.\qedhere
	\]
\end{proof}

\begin{exercise}
    Prove Lemma \ref{lem:simple}.
\end{exercise}

\begin{exercise}
    Let $M$, $N$, and $X$ be modules. Prove that 
    \begin{align}
        \Hom_A(X,M\oplus N)\simeq\Hom_A(X,M)\times\Hom_A(X,N).
    \end{align}
\end{exercise}

\begin{theorem}[Artin--Wedderburn]
\index{Artin--Wedderburn theorem}
Let $A$ be a finite-dimensional semisimple algebra with  
$k$ isomorphism classes of simple modules. Then 
\[
A\simeq M_{n_1}(D_1)\times\cdots\times M_{n_k}(D_k)
\]
for some $n_1,\dots,n_k\in\Z_{>0}$ and some division algebras $D_1,\dots,D_k$.
\end{theorem}

\begin{proof}
    Decompose the regular representation as a sum of simple modules and
    gather the simples by isomorphism classes to get 	
    \[
	A=\bigoplus_{i=1}^k n_iS_i,
	\]
	where each $S_i$ is simple and $S_i\not\simeq S_j$ whenever 
	$i\ne j$. Schur's lemma implies that  
	\begin{align*}
		\End_A(A)\simeq\End_A\left(\bigoplus_{i=1}^kn_iS_i\right)
		\simeq \prod_{i=1}^k\End_A(n_iS_i)
		\simeq\prod_{i=1}^kM_{n_i}(\End_A(S_i)), 
	\end{align*}
	where each $D_i=\End_A(S_i)$ is a division algebra by Schur's lemma. 
    Thus
    \[
		\End_A(A)\simeq\prod_{i=1}^kM_{n_i}(D_i).
	\]
	Since $\End_A(A)\simeq
	A^{\op}$, it follows that  
	\begin{align*}
		A=(A^{\op})^{\op}\simeq \prod_{i=1}^kM_{n_i}(D_i)^{\op}\simeq \prod_{i=1}^kM_{n_i}(D_i^{\op}).
	\end{align*}
	Since each $D_i$ is a division algebra, each $D_i^{\op}$ is also a division algebra.
\end{proof}

\begin{corollary}[Mollien]
\index{Mollien's theorem}
	If $A$ is a finite-dimensional complex semisimple algebra
	with $k$ isomorphism classes of simple modules, 
	then 
	\[
	A\simeq\prod_{i=1}^k M_{n_i}(\C)
	\]  
	for some $n_1,\dots,n_k\in\Z_{>0}$. 
\end{corollary}

\begin{proof}
	By Wedderburn's theorem,  
	\[
	A\simeq \prod_{i=1}^k M_{n_i}(\End_A(S_i)^{\op}),
	\]
	where $S_1,\dots,S_k$ are representatives of the isomorphism classes of simple modules
	and each $\End_A(S_i)$ is a division algebra. We claim that 
	\[
	\End_A(S_i)=\{\lambda\id:\lambda\in\C\}\simeq\C
	\]
	for all $i\in\{1,\dots,k\}$. If  
	$f\in\End_A(S_i)$, then $f$ has an eigenvalue $\lambda\in\C$. Since  
	$f-\lambda\id$ is not an isomorphism, Schur's lemma implies that $f-\lambda\id=0$, 
	that is $f=\lambda\id$. Thus $\End_A(S_i)\to\C$, $f\mapsto\lambda$, 
	is an algebra isomorphism. In particular,  
	\[
	A\simeq \prod_{i=1}^k M_{n_i}(\C).\qedhere
	\]
\end{proof}

% \begin{exercise}
%     Let $A$ and $B$ be algebras. Prove that the ideals of $A\times B$ are of the form 
%     $I\times J$, where $I$ is an ideal of $A$ and $J$ is an ideal of $B$.
% \end{exercise}

\topic{Group algebras}

Let $K$ be a field, and $G$ be a group. The \textbf{group algebra} $K[G]$ 
is the vector space (over $K$) with basis $\{g:g\in G\}$ 
and the algebra structure is given by the multiplication
\[
	\left(\sum_{g\in G}\lambda_gg\right)\left(\sum_{h\in G}\mu_hh\right)
	=\sum_{g,h\in G}\lambda_g\mu_h(gh).
\]
Every element of $K[G]$ is a finite sum of the form $\sum_{g\in G}\lambda_gg$.

\begin{exercise}
\label{xc:K[G]notsimple}
    If $G$ is non-trivial, then $K[G]$ is not simple. 
\end{exercise}

\begin{exercise}
	Let $G=C_n$ be the (multiplicative) cyclic group of order $n$. Prove that 
	$K[G]\simeq K[X]/(X^n-1)$. 
\end{exercise}

\begin{exercise}
	Let $G$ be a finitely-generated torsion-free abelian group. Prove that 
	$K[G]$ is a domain. 
\end{exercise}

\begin{exercise}
	Let $G$ be a group and $H$ be a subgroup of $G$. Let $\alpha\in K[H]$. Prove that 
    $\alpha$ is invertible (resp. a left zero divisor) in $K[H]$ if and only if 
	$\alpha$ is invertible (resp. a left zero divisor) in
	$K[G]$.
\end{exercise}

\begin{exercise}
	Let $G$ be a group and $\alpha=\sum_{g\in G}\lambda_gg\in K[G]$.  
	The \textbf{support} of $\alpha$ is the set 
	\[
		\supp\alpha=\{g\in G:\lambda_g\ne 0\}.
	\]
	Prove that if $g\in G$, then 
	$\supp(g\alpha)=g(\supp\alpha)$ and $\supp(\alpha g)=(\supp\alpha)g$.
\end{exercise}

% El objetivo de esta sección es calcular el radical de Jacobson del álgebra de
% grupo de un grupo finito. Comenzamos con un ejemplo:

\begin{exercise}
	Let $G=C_2=\langle g\rangle\simeq\Z/2$ the (multiplicative) 
	group with two elements. Note that every element of $K[G]$ is of the form
	$a+bg$ for some $a,b\in K$. Prove the following statements:
	\begin{enumerate}
	    \item If the characteristic of $K$ is different from two, then 
	    \[
		K[G]\to K\times K,
		\quad
		a1+bg\mapsto (a+b,a-b),
	\]
	is an algebra isomorphism. 
	\item If the characteristic of $K$ is two, then 
	\[
	K[G]\to \begin{pmatrix}
			K & K\\
			0 & K
		\end{pmatrix},
		\quad
		a1+bg\mapsto\begin{pmatrix}
			a+b & b\\
			0 & a+b
		\end{pmatrix},
	\]
	is an algebra isomorphism. 
	\end{enumerate}
\end{exercise}

If $A$ is an algebra over $K$ and $\rho\colon G\to \mathcal{U}(A)$
is a group homomorphism, where $\mathcal{U}(A)$ is the group of units of $A$, then 
the map \[
	K[G]\to A,\quad 
\sum_{g\in G}\lambda_gg\mapsto\sum_{g\in G}\lambda_g\rho(g),
\]
is an algebra homomorphism. 

\begin{exercise}
	Let $G=C_3$ be the (multiplicative) group of three elements. Prove that
	$\R[G]\simeq\R\times\C$.
% 	Escribamos $G=\langle g:g^3=1\rangle$ y sea 
% 	\[
% 		\varphi\colon\R[G]\to\R\times\C,
% 		\quad
% 		g\mapsto (1,\omega),
% 	\]
% 	donde $\omega$ es una raíz cúbica primitiva de la unidad. Entonces
% 	$\varphi$ es inyectivo pues
% 	$0=\varphi(a1+bg+cg^2)=(a+b+c,a+b\omega+c\omega^2)$ implica que $a=b=c=0$.
% 	Luego $\varphi$ es un isomorfismo pues
% 	$\dim_\R\R[G]=\dim_\R(\R\times\C)=3$. 
\end{exercise}

\begin{exercise}
	Let $G=\langle r,s:r^3=s^2=1,\,srs=r^{-1}\rangle$ be the dihedral group of six elements. 
	Prove the following statements:
	\begin{enumerate}
	    \item $\C[G]\simeq\C\times\C\times M_2(\C)$.
	    \item $\Q[G]\simeq\Q\times\Q\times M_2(\Q)$.
	\end{enumerate}  
% 	Sea $\omega$ una raíz cúbica de la unidad y sean  
% 	\[
% 		R=\begin{pmatrix}
% 			\omega & 0\\
% 			0 & \omega^2
% 		\end{pmatrix},
% 		\quad
% 		S=\begin{pmatrix}
% 			0 & 1\\
% 			1 & 0
% 		\end{pmatrix}.
% 	\]
% 	Un cálculo sencillo muestra que $R^2=S^2=I$ y que $SRS=R^{-1}$. Sea
% 	\[
% 		\varphi\colon\C[G]\to\C\times\C\times M_2(\C),\quad
% 		r\mapsto (1,1,R),\quad
% 		s\mapsto (1,-1,S).
% 	\]
% 	Es fácil ver que $\varphi$ es un morfismo de álgebras. Veamos que es
% 	biyectivo. Como $\dim_{\C}\C[G]=\dim_{\C}(\C\times\C\times M_2(\C))=6$,
% 	basta ver que $\varphi$ es inyectivo. Si 
% 	\[
% 		\alpha=a_0+a_1r+a_2r^2+(b_0+b_1r+b_2r^2)s\in\ker\varphi,
% 	\]
% 	entonces 
% 	\[
% 		0=\varphi(\alpha)=\left(u,v,\begin{pmatrix} \alpha_{11} & \alpha_{12}\\\alpha_{21}&\alpha_{22}\end{pmatrix}\right), 
% 	\]
% 	donde
% 	\begin{align*}
% 		&u = a_0+a_1+a_2+b_0+b_1+b_2, && v = a_0+a_1+a_2-b_0-b_1-b_2,\\
% 		&\alpha_{11}=a_0+a_1\omega+a_2\omega^2, && \alpha_{12}=b_0+b_1\omega+b_2\omega^2,\\
% 		&\alpha_{21}=b_0+b_2\omega+b_1\omega^2, && \alpha_{22}=a_0+a_2\omega+a_1\omega^2.
% 	\end{align*}
% 	Un cálculo sencillo muestra que estas ecuaciones implican que
% 	$\alpha=0$ y luego $\varphi$ es inyectiva.  
\end{exercise}



Maschke's theorem states that, if $G$ is a finite group, 
then the group algebra $\C[G]$ is semisimple. By Mollien's theorem, 
\[
\C[G]\simeq \prod_{i=1}^k M_{n_i}(\C),
\]
where $k$ is the number of (isomorphism classes of) 
simple $\C[G]$-modules. Moreover, 
\[
|G|=\dim\C[G]=\sum_{i=1}^k n_i^2.
\]

\begin{theorem}
    Let $G$ be a finite group. The number of simple 
    modules of $\C[G]$ coincides with the number of conjugacy classes of $G$. 
\end{theorem}

\begin{proof}
    By Mollien's theorem, $\C[G]\simeq\prod_{i=1}^kM_{n_i}(\C)$. Thus 
    \[
		Z(\C[G])\simeq\prod_{i=1}^kZ(M_{n_i}(\C))\simeq\C^k.
	\]
	In particular, $\dim Z(\C[G])=k$. If $\alpha=\sum_{g\in
	G}\lambda_gg\in Z(\C[G])$, then $h^{-1}\alpha h=\alpha$ for all $h\in
	G$. Thus 
	\[
		\sum_{g\in G}\lambda_{hgh^{-1}}g=
		\sum_{g\in g}\lambda_g h^{-1}gh=\sum_{g\in G}\lambda_gg
	\]
	and hence $\lambda_{g}=\lambda_{hgh^{-1}}$ for all $g,h\in G$. A basis for 
	$Z(\C[G])$ is given by elements of the form 
	\[
		\sum_{g\in K}g,
	\]
	where $K$ is a conjugacy class of $G$. Therefore $\dim Z(\C[G])$ is equal to 
	the number of conjugacy classes of $G$.
\end{proof}

\begin{example}
\label{exa:C4}
    Let $G=C_4$ be the cyclic group of order four. Then
    $G$ has four simple modules and 
    $\C[G]\simeq\C^4$. 
\end{example}

\begin{example}
\label{exa:S3}
    Let $G=\Sym_3$. Then $G$ has three simple modules and
    \[
    \C[G]\simeq\C\times\C\times M_2(\C).
    \]
\end{example}

\begin{problem}[Brauer]
\index{Brauer's problem}
    Which algebras are group algebras? 
\end{problem}

This question might be impossible to answer, but it is extremely interesting. 
Examples \ref{exa:C4} and \ref{exa:S3} show
that $\C^4$ and $\C^2\times M_2(\C)$ are complex group algebras. 

\begin{exercise}
    Is $\C^2\times M_2(\C)\times M_3(\C)$ a complex group algebra?  
\end{exercise}

% No. Let $G$ be a group of order 15. Since groups of order 15 are abelian,
% $G$ has 15 conjugacy classes. 
