\section{Project: An analytic proof of Rickart's theorem}
\label{section:Rickart} 

We will present a different proof 
of Rickart's theorem \ref{thm:Rickart}. 

\begin{definition}
	\index{Ring!with an involution}
	\index{Involution}
	Let $R$ be a ring. An \emph{involution} 
    in $R$ is an additive map  
    $R\to R$, $x\mapsto x^*$, such that $x^{**}=x$ and  $(xy)^*=y^*x^*$ for all 
	$x,y\in R$.
\end{definition}

It follows that if $R$ is a unitary ring, then 
$1^*=1$.

\begin{example}
	The conjugation $z\mapsto\overline{z}$ is an 
    involution of $\C$.
\end{example}

\begin{example}
	The transpose map $X\mapsto X^T$ is an involution of $M_n(K)$.
\end{example}

\begin{example}
	Let $G$ be a group. 
    Then \[
    \left(\sum_{g\in G}\alpha_gg\right)^*=\sum_{g\in G}\overline{\alpha_g}g^{-1}
    \]
	is an involution of the group ring $\C[G]$.
\end{example}

For a group $G$, one defines the \emph{trace} 
of an element 
\[
\alpha=\sum_{g\in G}\alpha_gg\in K[G]
\]
as $\trace(\alpha)=\alpha_1$. One proves that 
the map 
\[
\trace\colon K[G]\to K,\quad 
\alpha\mapsto\trace(\alpha)
\]
is $K$-linear such that   
$\trace(\alpha\beta)=\trace(\beta\alpha)$ for 
all $\alpha,\beta\in K[G]$. 

\begin{exercise}
	Let $G$ be a finite group and $K$ 
 be a field of characteristic not dividing the 
 order of $G$.
	Prove the following statements: 
	\begin{enumerate}
		\item If $\alpha\in K[G]$ is nilpotent, then $\trace(\alpha)=0$.
		\item If $\alpha\in K[G]$ is idempotent, then $\trace(\alpha)=\dim
			(K[G]\alpha)/|G|$.
	\end{enumerate}
\end{exercise}

\begin{exercise}
	Prove that  
	$\langle\alpha,\beta\rangle=\trace(\alpha\beta^*)$, $\alpha,\beta\in\C[G]$, 
	defines an inner product in $\C[G]$.
\end{exercise}

\begin{lemma}
	\label{lem:algebraico}
	Let $G$ be a group. Prove that if $J(\C[G])\ne 0$, then there exists $\alpha\in J(\C[G])$ such that  
	$\trace(\alpha^{2^m})\in\R_{\geq1}$ 
	for all $m\geq1$.
\end{lemma}

\begin{proof}
	Let $\alpha=\sum_{g\in G}\alpha_gg\in\C[G]$. Then 
	\[
		\trace(\alpha^*\alpha)
		=\sum_{g\in G}\overline{\alpha_g}\alpha_g
		=\sum_{g\in G}|\alpha_g|^2\geq|\alpha_1|^2
		=|\trace(\alpha)|^2.
	\]
	By induction, one proves that, if 
    $\alpha$ is such that $\alpha^*=\alpha$, then  
    \[
    \trace(\alpha^{2^m})\geq|\trace(\alpha)|^{2^m}
    \]
	for all $m\geq1$. 

	Let $\beta=\sum_{g\in G}\beta_gg\in J(\C[G])$ be such that $\beta\ne0$. Since 
	$\trace(\beta^*\beta)=\sum_{g\in G}|\beta_g|^2\ne0$ and $J(\C[G])$ is an ideal, 
	\[
		\alpha=\frac{\beta^*\beta}{\trace(\beta^*\beta)}\in J(\C[G]).
	\]
	Then $\alpha$ is such that $\alpha^*=\alpha$ and $\trace(\alpha)=1$.
	Hence $\trace(\alpha^{2^m})\geq 1$  for all $m\geq1$.
\end{proof}

Exercise~\ref{exa:norma} implies that $\C[G]$ with 
$\operatorname{dist}(\alpha,\beta)=|\alpha-\beta|$ is a metric space. In this metric space, the map 
$\C[G]\to\C$, $\alpha\mapsto \trace(\alpha)$, is a continuous map. 

\begin{lemma}
	\label{lem:phi_diferenciable}
	Let $\alpha\in J(\C[G])$. The map 
	\[
		\varphi\colon\C\to\C[G],\quad
		\varphi(z)=(1-z\alpha)^{-1},
	\]
	is continuous, differentiable and $\varphi(z)=\sum_{n\geq0}\alpha^nz^n\in\C[G]$ if  $|z|$
	is sufficiently small. 
\end{lemma}

\begin{proof}	
	Let $y,z\in\C$. Since $\varphi(y)$ and $\varphi(z)$ commute, 
	\begin{equation}
		\label{eq:Rickart}
		\begin{aligned}
			\varphi(y)-\varphi(z)&=\left( (1-z\alpha)-(1-y\alpha)\right)(1-y\alpha)^{-1}(1-z\alpha)^{-1}\\
			&=(y-z)\alpha\varphi(y)\varphi(z).
		\end{aligned}
	\end{equation}
	Hence $|\varphi(y)|\leq|\varphi(z)|+|y-z||\alpha\varphi(y)||\varphi(z)|$ and therefore 
	\[
		|\varphi(y)|\left( 1-|y-z||\alpha\varphi(z)|\right)\leq|\varphi(z)|.
	\]
	Fix $z$ and choose $y$ sufficiently close to $z$ in such a way that $1-|y-z||\alpha\varphi(z)|\geq1/2$. Then
	$|\varphi(y)|\leq2|\varphi(z)|$. Using~\eqref{eq:Rickart} one obtains that 
	$|\varphi(y)-\varphi(z)|\leq2|y-z||\alpha||\varphi(z)|^2$. 
	Hence $\varphi$ is continuous. By~\eqref{eq:Rickart}, 
	\[
	\varphi'(z)
	=\lim_{y\to z}\frac{\varphi(y)-\varphi(z)}{y-z}
	=\lim_{y\to z}\alpha\varphi(y)\varphi(z)
	=\alpha\varphi(z)^2
	\]
	for all $z\in\C$.

	If $z$ is such that $|z||\alpha|=|z\alpha|<1$, then  
	\[
		\varphi(z)-\sum_{n=0}^Nz^n\alpha^n
		=\varphi(z)\left(1-(1-z\alpha)\sum_{n=0}^Nz^n\alpha^n\right)
		=\varphi(z)(z\alpha)^{N+1}.
	\]
	Then 
	\[
		\left|\varphi(z)-\sum_{n=0}^Nz^n\alpha^n\right|\leq|\varphi(z)||z\alpha|^{N+1}.
	\]
	Since $\varphi(z)$ is bounded close to $z=0$, we conclude that 
	\[
 \left|\varphi(z)-\sum_{n=0}^Nz^n\alpha^n\right|\to0
 \]
 if $N\to\infty$.
\end{proof}

We now provide an alternative proof of Rickart's theorem: 

\begin{theorem}[Rickart]
	\index{Rickart's theorem}
	If $G$ is a group, then $J(\C[G])=\{0\}$.
\end{theorem}

\begin{proof}
	Let $\alpha\in J(\C[G])$ and $\varphi(z)=(1-\alpha z)^{-1}$. Let 
	\[
    f\colon\C\to \C,\quad  
	f(z)=\trace\varphi(z)=\trace\left((1-z\alpha)^{-1}\right).
    \]
    By Lemma~\ref{lem:phi_diferenciable},
	$f(z)$ is an entire function such that  $f'(z)=\trace(\alpha\varphi(z)^2)$ and 
	\begin{equation}
		\label{eq:Taylor}
		f(z)=\sum_{n=0}^\infty z^n\trace(\alpha^n)
	\end{equation}
	if $|z|$ is sufficiently small. In particular,~\eqref{eq:Taylor} is the Taylor series of $f(z)$ around the origin. This implies that the series converges to $f(z)$ for all $z\in\C$. In particular,
	\begin{equation}
		\label{eq:limite}
		\lim_{n\to\infty}\trace(\alpha^n)=0.
	\end{equation}
	On the other hand, if $\alpha\ne0$, Lemma~\ref{lem:algebraico} implies that 
	$\trace(\alpha^{2^m})\geq1$ for all $m\geq0$. This contradicts the limit computed in~\eqref{eq:limite}. Hence $\alpha=0$.
\end{proof}

For a corollary, we need a consequence of 
Nakayama's lemma \ref{lem:Nakayama}.


\begin{lemma}
	\label{lem:Rickart}
    Let $R$ and $S$ be unitary rings and 
	$\iota\colon R\to S$ be a ring homomorphism. If  
	\[
	S=\iota(R)x_1+\cdots+\iota(R)x_n,
	\]
	where each $x_j$ is such that
    $x_jy=yx_j$ for all $y\in\iota(R)$, then
	$\iota(J(R))\subseteq J(S)$.
\end{lemma}

\begin{proof}
	We claim that $J=\iota(J(R))$ acts trivially on each simple $S$-module $M$.
	Let $M$ be a simple $S$-module. Write 
    $M=Sm$ for some $m\ne0$. Then $M$ is 
    an $R$-module with $r\cdot m=\iota(r)m$. Since 
    \[
		M=Sm=(\iota(R)x_1+\cdots+\iota(R)x_n)m=\iota(R)(x_1m)+\cdots+\iota(R)(x_nm),
	\]
	$M$ is finitely generated as an $\iota(R)$-module. Moreover, $J(R)\cdot
	M=JM=\iota(J)M$ is an $S$-submodule of $M$, as 
	\[
		x_j(JM)=(x_jJ)M=(Jx_j)M=J(x_jM)\subseteq JM.
	\]
	Since $M\ne\{0\}$, Nakayama's lemma implies that $J(R)\cdot M\subsetneq M$. Since 
	$M$ is a simple $S$-module, we conclude that 
    $J(R)M=\{0\}$.
\end{proof}

\begin{corollary}
	If $G$ is a group, then $J(\R[G])=\{0\}$. 
\end{corollary}

\begin{proof}
	Let $\iota\colon \R[G]\to\C[G]$ be the canonical inclusion. Since 
 	\[
	\C[G]=\R[G]+i\R[G],
	\]
	Now Lemma \ref{lem:Rickart} and Rickart's theorem imply that 
	$\iota(J(\R[G]))\subseteq J(\C[G])=\{0\}$. We conclude that $J(\R[G])=0$, as the map $\iota$ is injective. 
\end{proof}
