\section{Project: The Skolem--Noether theorem}

We now present an elementary proof of 
the Skolem--Noether theorem. We refer to 
\cite{MR3308118} for more information. 

\begin{definition}
\index{Algebra!central}
    Let $K$ be a field. 
	An algebra $A$ (over $K$) is \emph{central} if $Z(A)=K$. 
\end{definition}

If $K$ is a field, then $M_n(K)$ is a central algebra.

\begin{proposition}
	Let $A$ be a unitary algebra and $n\geq1$. Then
	$A$ is central if and only if $M_n(A)$ is central.
\end{proposition}

\begin{proof}
	If $M_n(A)$ is central and $z\in Z(A)$, then
	$zI\in Z(M_n(A))=KI$. Thus
	$z\in K$. Conversely, if $X\in Z(M_n(A))$, then, since 
	$XE_{kl}=E_{kl}X$ for all $k\ne l$, $X=aI$ for some $a\in A$. 
	Moreover, 
	$XaE_{11}=aE_{11}X$. Hence $a\in Z(A)=K1$. 
\end{proof}

\begin{example}
	$\mathbb{H}$ is a real central algebra.
\end{example}

\begin{example}
	$\C$ is a complex central algebra but it is not a real central 
	algebra. 
\end{example}

A celebrated theorem by Frobenius states that every 
finite-dimensional  
real central division algebra is isomorphic to $\R$ or $\mathbb{H}$ (see Theorem \ref{thm:Frobenius}). 

\begin{proposition}
	Every simple unitary ring is an algebra over its center. 
\end{proposition}

\begin{proof}
	Let $R$ be a simple unitary ring. It is enough to show that
	$Z(R)$ is a field. If $z\in
	Z(R)\setminus\{0\}$ then $zR$ is a non-zero ideal of $R$. Since $R$
	is simple, $zR=R$. Thus $z$ is invertible. 
\end{proof}

For an algebra $A$, let $L\colon A\to\End_K(A)$,
$a\mapsto L_a$, and $R\colon A\to\End_K(A)$, $a\mapsto R_a$, be given by 
$L_a(x)=ax$ and $R_a(x)=xa$. Then both $L$ and $R$ are linear maps such that 
\begin{align*}
	L_{ab}=L_aL_b, && R_{ab}=R_bR_a, &&	L_aR_b=R_bL_a
\end{align*}
for all $a,b\in A$.

\begin{definition}
	\index{Algebra!of multipliers}
	Let $A$ be an algebra. The \emph{algebra of multipliers} of $A$ 
	is 
	\[
		M(A)=\left\{\sum_{j=1}^n L_{a_i}R_{b_i}:n\in\Z_{\geq0},\,a_1,\dots,a_n,b_1,\dots,b_n\in A\right\}.
	\]
\end{definition}

It is an exercise to show that $M(A)$ is a subalgebra of $\End_K(A)$. Moreover,
if $A$ is unitary, then $M(A)$ is generated by the $L_a$ and the $R_b$ for 
$a,b\in A$.

\begin{lemma}
	\label{lem:SkolemNoether}
	Let $A$ be an algebra and  
	$f\in M(A)$. Then there exists $n\geq0$ and 
	$a_1,\dots,a_n\in A$ and $b_1,\dots,b_n\in A$ such that 
	\[
		f=\sum_{i=1}^n L_{a_i}R_{b_i}
	\]
	and $\{b_1,\dots,b_n\}$ is linearly independent. 
\end{lemma}

\begin{proof}
    Write $f=\sum_{i=1}^n L_{a_i}R_{b_i}$ with $n$ be minimal. If 
    $b_n=\sum_{j=1}^{n-1}\lambda_jb_j$, then
    \[
    f=\sum_{i=1}^{n-1}L_{a_i+\lambda_ia_n}R_{b_i},
    \]
    a contradiction.
\end{proof}

\begin{lemma}
    \label{lem:SkolemNoether1}
    Let $A$ be a central simple algebra. 
    If $\sum_{i=1}^n L_{a_i}R_{b_i}=0$ and $\{b_1,\dots,b_n\}$ 
    (resp. $\{a_1,\dots,a_n\}$) is linearly independent, 
    then $a_i=0$ (resp. $b_i=0$) for all $i$.
\end{lemma}

\begin{proof}
The result holds for $n=1$. We want to prove that 
if  $a_1xb_1=0$ for all $x\in A$ and $b_1\ne0$, then
	$a_1=0$. Assume that $a_1\ne 0$. The ideal of $A$ generated by 
	$a_1$ is non-zero, and hence it is equal to $A$. Thus there exist 
	$u_1,\dots,u_m,v_1,\dots,v_m\in A$ such that $1=\sum_{j=1}^m u_ja_1v_j$.
	Write  
	\[
		0=\sum_{j=1}^m L_{u_j}(L_{a_1}R_{b_1})L_{v_j}
        =\sum_{j=1}^m L_{u_ja_1v_j}R_{b_1}=R_{b_1}.
	\]
	Hence $b_1=0$. 

	Assume that the lemma is not true and let $n>1$ be the smallest positive
 integer where the lemma is false. 
    Assume that $a_n\ne 0$. Since $A$ is simple, the ideal generated by 
	 $a_n$ is $A$. Then there exist 
  	$u_1,\dots,u_m,v_1,\dots,v_m\in A$ such that $1=\sum_{j=1}^m u_ja_1v_j$ and 
	\[
		0=\sum_{j=1}^m L_{u_j}\left(\sum_{i=1}^n L_{a_i}R_{b_i}\right)L_{v_j}=\sum_{i=1}^n\sum_{j=1}^m L_{u_ja_iv_j}R_{b_i}=\sum_{i=1}^n L_{c_i}R_{b_i},
	\]
	where $c_i=\sum_{j=1}^m u_ja_iv_j$ and $c_n=1$. Since 
	\[
		0=L_x\left(\sum_{i=1}^n L_{c_i}R_{b_i}\right)-\left(\sum_{i=1}^n L_{c_i}R_{b_i}\right)L_x=\sum_{i=1}^{n-1}L_{xc_i-c_ix}R_{b_i}
	\]
	for all $x\in A$, it follows that $xc_i-c_ix=0$ for all 
	 $x\in A$. Since $A$ is central, $c_i\in k$ for all
	$i\in\{1,\dots,n-1\}$. Evaluate $0=\sum_{i=1}^n L_{c_i}R_{b_i}$ in $1_A$
	we obtain that $0=c_1b_1+\cdots+c_nb_n$, a contradiction since 
	$\{b_1,\dots,b_n\}$ is linearly independent. 
\end{proof}

\begin{lemma}
	\label{lem:SkolemNoether2}
	If $A$ is a finite-dimensional central simple algebra, then 
        \[
        M(A)=\End_K(A).
        \]
\end{lemma}

\begin{proof}
	Let $\{a_1,\dots,a_n\}$ be a basis of $A$. We claim that  
	$\{L_{a_i}R_{a_j}:1\leq i,j\leq n\}$ is linearly independent. If 
	\[
        \sum_{i,j=1}^n\lambda_{ij}L_{a_i}R_{a_j}=0,
        \]
        then 
	$\sum_{i=1}^nL_{a_i}R_{c_i}=0$, where
	$c_i=\sum_{j=1}^n\lambda_{ij}R_{a_j}$. Since the $a_i$'s are linearly 
 independent, Lemma~\ref{lem:SkolemNoether1} implies that $c_i=0$ for all 
	 $i\in\{1,\dots,n\}$, a contradiction since the $a_j$'s are linearly independent.   
	Hence $\dim_kM(A)\geq n^2=\dim\End_K(A)$.
\end{proof}

\begin{definition}
	\index{Automorfism!inner}
	Let $R$ be a unitary ring. An automorphism $f\in\Aut(R)$ is 
	\emph{inner} is there exists an invertible $r\in R$ such that 
	$f(x)=rxr^{-1}$ for all $x\in R$.
\end{definition}

For example, $\C\to\C$, $z\mapsto\overline{z}$, is not inner.

\begin{example}
	Let $\lambda\in k\setminus\{0\}$ and $R=k[X]$. Then 
 \[
 k[X]\to
	k[X],\quad f(X)\mapsto f(X+\lambda),
 \]
 is not inner. 
\end{example}

\begin{example}
	Let $R$ be a ring. Then $R\times R\to R\times R$, $(x,y)\mapsto
	(y,x)$, is not inner. 
\end{example}

\begin{theorem}[Skolem--Noether]
	\index{Skolem--Noether theorem}
	\label{thm:SkolemNoether}
	If $A$ is a finite-dimensional central simple algebra, 
        every automorphism of $A$ is inner. 
\end{theorem}

\begin{proof}	
	Let $f\in\Aut(A)$. By Lemma~\ref{lem:SkolemNoether2}, 
	$f=\sum_{i=1}^n	L_{a_i}R_{b_i}$. 
	Without loss of generality, we may assume that  $a_1\ne 0$ and 
	that $\{b_1,\dots,b_n\}$ is linearly independent. 
	Since $f$ is a homomorphism, 
 	$L_{f(x)}f=fL_x$ for all $x\in A$. Then 
	\[
		0=\sum_{i=1}^n L_{f(x)a_i-a_ix}R_{b_i}. 
	\]
	By Lemma~\ref{lem:SkolemNoether1}, $f(x)a_1-a_1x=0$ for all 
	$x\in A$. We claim that $a_1$ is invertible. 
	Since $a_1\ne 0$ and $A$ is simple, the ideal of $A$ generated by $a_1$ is $A$.
        Write $1=\sum_{i=1}^m u_ja_1v_j$. Thus $a_1$ is invertible, 
        as  
	\[
		\left(\sum_{j=1}^m u_jf(v_j)\right)a_1=a_1\left(\sum_{j=1}^m f^{-1}(u_j)v_j\right)=1.\qedhere
	\]
\end{proof}


%\begin{corollary}
%	\label{cor:SkolemNoether1}
%	Sea $A$ es un álgebra de dimensión finita unitaria. Si $a\in A$, entonces
%	$a$ es inversible o es un divisor de cero. 
%\end{corollary}
%
%\begin{proof}
%	Como $A$ es de dimensión finita, $A$ es algebraica. Existe entonces un
%	polinomio $f=\sum_{j=1}^n \lambda_jX^j\in k[X]$ (que podemos suponer de grado
%	mínimo) tal que $f(a)=0$. Al escribir
%	\[
%		0=f(a)=a(\lambda_na^{n-1}+\cdots+\lambda_2a+\lambda_1)+a_0
%	\]
%	vemos que existe un polinomio $g=\lambda_nX^{n-1}+\cdots+\lambda_1\in k[X]$
%	tal que $g(a)\ne 0$ (por la minimalidad de $n$) y $ag(a)=-\lambda_0$. Si
%	$a$ no es un divisor de cero, entonces $\lambda_0\ne 0$ y luego
%	$a^{-1}=-\lambda_0^{-1}g(a)$. 
%\end{proof}
%
%\begin{corollary}
%	Sea $A$ un álgebra de dimensión finita unitaria y sean $a,b\in A$. Si
%	$ab=1$, entonces $ba=1$.
%\end{corollary}
%
%\begin{proof}
%	Es consecuencia inmediata del corolario~\ref{cor:SkolemNoether1}.
%\end{proof}
%
%\begin{corollary}
%	Sea $D$ un álgebra de división de dimensión finita y sea $A$ una subálgebra
%	de $D$. Entonces $A$ es un álgebra de división.
%\end{corollary}
%
%\begin{proof}
%	Sea $a\in A\setminus\{0\}$. Como existe $d\in D$ tal que $ad=1$. Como $a$ es algebraico, 
%	existe $f\in k[X]$ de grado mínimo tal que $f(a)=0$. Luego $a$ es inversible con 
%	$a^{-1}=-\lambda_0^{-1}g(a)$ para algún $g\in k[X]$ tal que $g(a)\ne 0$. En
%	particular, $a^{-1}\in A$ y además $A$ es unitaria. 
%\end{proof}


