\section{}

\begin{definition}
\index{Ring!radical}
A ring $R$ is said to be \textbf{radical} if $J(R)=R$. 
\end{definition}

\begin{example}
	If $R$ is a ring, then $J(R)$ is a radical ring, by Proposition~\ref{proposition:J(I)}.
\end{example}

\begin{example}
	The Jacobson radical of $\Z/8$ is $\{0,2,4,6\}$. 
\end{example}

There are several characterizations of radical rings. 

\begin{theorem}
	\label{theorem:anillo_radical}
	Let $R$ be a ring. The following statements are equivalent: 
	\begin{enumerate}
		\item $R$ is radical.
		\item $R$ admits no simple $R$-modules. 
		\item $R$ does not have regular maximal left ideals.
		\item $R$ does not have primitive left ideals.
		\item Every element of $R$ is quasi-regular. 
		\item $(R,\circ)$ is a group. 
	\end{enumerate}
\end{theorem}

\begin{exercise}
    Prove Theorem \ref{theorem:anillo_radical}. 
\end{exercise}
%\begin{proof}
%	The equivalence $(1)\Longleftrightarrow(5)$ follows from 
%	Theorem~\ref{thm:casireg_eq}. 
%    
%    The equivalence $(5)\Longleftrightarrow(6)$ is left as an exercise. 
%
%	Let us prove that $(1)\implies(2)$. Assume that there exists a simple $R$-module $N$. Since 
%	$R=J(R)\subseteq\Ann_R(N)$, $R=\Ann_R(N)$. 
%	Hence $R\cdot N=\{0\}$, a contradiction to the simplicity of $N$.
%	
%	To prove $(2)\implies(3)$ we note that for each regular and maximal left ideal 
%	$I$, the quotient $R/I$ is a simple $R$-module by
%	Proposition~\ref{proposition:R/I}. 
%	
%	To prove $(1)\implies(4)$ assume that there is a primitive left ideal 
%	$I=\Ann_R(M)$, where $M$ is some simple $R$-module. Since $R=J(R)\subseteq I$, it follows that  
%    $I=R$, a contradiction to the simplicity of $M$.
%
%	Finally we prove $(4)\implies(2)$. If $M$ is a simple $R$-module, then 
%	$\Ann_R(M)$ is a primitive left ideal.
%\end{proof}

\begin{example}
	Let 
	\[
	A=\left\{\frac{2x}{2y+1}:x,y\in\Z\right\}.
	\]
	Then $A$ is a radical ring, as the inverse of the element $\frac{2x}{2y+1}$
	with respect to the circle operation 
	$\circ$ is 
	\[
	\left(\frac{2x}{2y+1}\right)'=\frac{-2x}{2(x+y)+1}.
	\]
\end{example}

\subsection{Commutative rings with no maximal ideals}

There are rings with no maximal ideals. 

\begin{exercise}
\label{xca:Q_no_maximals}
    Prove that the additive group of rational numbers is an abelian group
with no maximal subgroups.
\end{exercise}

One can turn the additive group $\Q$ of rational into a non-unitary 
ring by considering the zero multiplication $xy=0$ for all $x,y\in\Q$. This 
ring has no maximal ideals. 

% Let $R$ be a commutative ring with no non-zero proper ideals. 
% Prove that if $R$ is not a field, then there exists a prime number $p$ 
% such that $R=\Z/p$ and $xy=0$ for all $x,y\in R$. 

\begin{exercise}
\label{xca:R/I_field_or_zero}
    Let $R$ be a commutative ring 
    and $I$ be an ideal of $R$. Prove that
    $I$ is maximal if and only if $R/I$ is a field
    or a ring isomorphic to 
    $\Z/p$ with zero multiplication for some prime number $p$. 
\end{exercise}

\begin{exercise}
    \label{xca:J(R)_fields}
    Let $R$ be a commutative ring. Prove that
    $J(R)$ equals the intersection of maximal ideals 
    such that $R/M$ is a field. 
\end{exercise}

\index{Characteristic of a ring}
Recall that the \textbf{characteristic of a ring} is defined as the least positive
integer $n$ such that $nx=0$ for all $x$. If no such $n$ exists, 
then we say that the ring is of characteristic zero. 

\begin{exercise}
\label{xca:characteristic}
    % Let $R$ be a ring and $n=\lcm\{ |x|:x\in R\}$, where 
    % $|x|$ denotes the additive order of $x$. If $n<\infty$, then 
    % $R$ has characteristic $n$. If $n=\infty$, then $R$ has characteristic zero. 
    % is the least common multiple of 
    % the additive orders of the elements of $R$, then  
    Let $R$ be a ring such and $p$ be a prime number. 
    If $px=0$ for all $x\in R$, then $R$ has characteristic $p$. 
\end{exercise}

% \begin{sol}{xca:characteristic}
%     Let $n$ be the characteristic of $R$. 
%     For each $x\in R$, $x\ne 0$, $px=0$. Since the additive order $|x|$ of $x$
%     divides $p$, it follows that $|x|=p$. Since $nx=0$ for all $x$, 
%     $p$ divides $n$. The minimality of the characteristic implies that $p=n$. 
% Assume that $n<\infty$. 
% Let $X=\{n:nx=0\text{ for all $x\in R$}\}$. By definition, 
% \[ 
% m\in X
% \Longleftrightarrow |x|\text{ divides }m
% \Longleftrightarrow n\leq m
% \Longleftrightarrow n\in X.
% \]
%\end{sol}

% Let $n>0$ be the characteristic of $R$. 
% If $px=0$ for all $x$, then $n=p$. Since $px=0$ for all $x$, $p\geq n$. 
% Assume that $p>n$, then $p+t=n$ for some $n$. 
% Then $0=px=(n-t)x=nx-tx=-tx$ and hence $tx=0$ for all $x$, a contradiction. 
% If $p\nmid n$, then 
% $n=pm$. Then $px=$

We now characterize commutative rings with no maximal ideals.
The result appeared in~\cite{MR0424776}. 

\begin{theorem}[Henriksen]
\label{thm:Henriksen}
\index{Henriksen's theorem}
Let $R$ be a commutative ring. Then $R$ has
no maximal ideals if and only if 
$J(R)=R$ and $R^2+pR=R$ for all prime number $p$. 
\end{theorem}

\begin{proof}
    Assume first that $R$ has no maximal ideals. Then $J(R)=R$ by 
    Exercise~\ref{xca:J(R)_fields}. Let $p$ be a prime number
    such that $I=R^2+pR\ne R$. Then $I$ is a proper ideal of $R$. 
    Let $\pi\colon R\to R/I$ be the canonical map. Since $R^2\subseteq I$, 
    $0=\pi(xy)=\pi(x)\pi(y)$ for all $x,y\in R$. Thus $R/I$ has zero multiplication. 
    Moreover, by Exercise~\ref{xca:characteristic}, 
    $R/I$ has characteristic $p$, as $pR\subseteq I$.  
    %as $0=\pi(px)=p\pi(x)$ for all $x\in R$, 
    % since $pR\subseteq I$. 
    Thus $R/I$ is a vector space 
    over the field $\Z/p$.
    Let $\{x_\alpha:\alpha\in\Lambda\}$ be a basis
    of $R/I$. Every element $x\in R/I$ can be written uniquely
    as a finite sum of the form 
    $x=\sum \lambda_{\alpha}x_{\alpha}$ for scalars $\lambda_\alpha$. 
    Let $A$ be the ring with underlying additive group $\Z/p$ and zero multiplication. 
    For a fixed
    $\beta\in\Lambda$, the map 
    \[
    \gamma\colon R/I\to A,\quad 
    x=\sum \lambda_{\alpha}x_{\alpha}\mapsto \lambda_{\beta}
    \]
    is a ring homomorphism. 
    The composition $f=\gamma\pi\colon R\to R/I\to A$ is a ring homomorphism. By Exercise~\ref{xca:R/I_field_or_zero}, $\ker f$ is a maximal ideal, a contradiction. 

    Conversely, let $M$ be a maximal ideal of $R$. If $R/M$ is a field, 
    then $J(R)\subseteq M\ne R$, a contradiction. 
    By Exercise~\ref{xca:R/I_field_or_zero},  there exists a prime
    number $p$ such that $R/M\simeq\Z/p$ as abelian groups and 
    zero multiplication (i.e. $xy\in M$ for all $x,y\in R$). 
    Let us write $A$ to denote this ring and  
    $\pi\colon R\to R/M$ be the canonical map. Note that 
    $R^2\subseteq M$. Moreover, $pR\subseteq M$, as 
    $\pi(px)=p\pi(x)=0$ for all $x\in R$. Thus $R^2+pR\subseteq M\ne R$, a contradiction. 
\end{proof}

We now present a non-trivial concrete example of a ring 
with no maximal ideals. For that purpose, we will use
the field of fractions $\R(X)$ of the real polynomial ring $\R[X]$.  

\begin{exercise}
    \label{xca:example}
    Let $R$ be the set of rational real functions of the form 
    $f(X)/g(X)$, where $f(X),g(X)\in\R(X)$ and $g(0)\ne 0$.  Prove the following statements:
    \begin{enumerate}
        \item $R$ is an integral domain with a unique maximal ideal $M=XR$.
        \item $M$ has no maximal ideals. 
    \end{enumerate}  
\end{exercise}

%The construction of the previous example does not work 
%if the field of real numbers is replaced by a field of positive characteristic. 

\begin{definition}
\index{Ring!nil}
A ring $R$ is said to be \textbf{nil} if for every $x\in R$ there
exists $n=n(x)$ such that $x^n=0$. 
\end{definition}

\begin{exercise}
    Prove that a nil ring is a radical ring. 
\end{exercise}

\begin{exercise}
    Let $\R[\![X]\!]$ be the ring of power series with real coefficients. Prove that the ideal 
    $X\R[\![X]\!]$ consisting of power series with zero constant term is a radical ring
    that is not nil. 
\end{exercise}

\begin{theorem}
	\label{thm:J(R/J)=0}
	If $R$ is a ring, then $J(R/J(R))=\{0\}$.
\end{theorem}

\begin{proof}
	If $R$ is radical, the result is trivial. Suppose then that 
	$J(R)\ne R$. Let $M$ be a simple $R$-module. Then $M$ is 
	a simple module over $R/J(R)$ with 
	\[
		(x+J(R))\cdot m=x\cdot m,\quad
		x\in R,\,m\in M.
	\]
	If $x+J(R)\in J(R/J(R))$, then  $x\cdot M=(x+J(R))\cdot M=\{0\}$. Then $x\in J(R)$, as 
	$x$ annihilates any simple module over $R$.
\end{proof}

\begin{theorem}
	Let $R$ be a ring and $n\in\Z_{>0}$. Then $J(M_n(R))=M_n(J(R))$. 
\end{theorem}

\begin{proof}
	We first prove that $J(M_n(R))\subseteq M_n(J(R))$. 
	If $J(R)=R$, the theorem is clear. Let us assume that $J(R)\ne R$ and let  
	$J=J(R)$. 
	If $M$ is a simple $R$-module, then $M^n$ is a simple $M_n(R)$-module with the usual multiplication. 
	Let $x=(x_{ij})\in J(M_n(R))$ and $m_1,\dots,m_n\in M$. Then
	\[
		x\colvec{3}{m_1}{\vdots}{m_n}=0.
	\]
	In particular, $x_{ij}\in\Ann_R(M)$ for all $i,j\in\{1,\dots,n\}$. Hence 
	$x\in M_n(J)$. 

	We now prove that $M_n(J)\subseteq J(M_n(R))$. Let 
	\[
		J_1=\begin{pmatrix}
			J & 0 & \cdots & 0\\
			J & 0 & \cdots & 0\\
			\vdots & \vdots & \ddots & \vdots\\
			J & 0 & \cdots & 0
		\end{pmatrix}
		\quad\text{and}\quad
		x=\begin{pmatrix}
			x_1 & 0 & \cdots & 0\\
			x_2 & 0 & \cdots & 0\\
			\vdots & \vdots & \ddots & \vdots\\
			x_n & 0 & \cdots & 0
		\end{pmatrix}\in J_1.
	\]
	Since $x_1$ is quasi-regular, there exists $y_1\in R$ such that $x_1+y_1+x_1y_1=0$.
	If
	\[
		y=\begin{pmatrix}
			y_1 & 0 & \cdots & 0\\
			0 & 0 & \cdots & 0\\
			\vdots & \vdots & \ddots & \vdots\\
			0 & 0 & \cdots & 0
		\end{pmatrix}, 
	\]
	then $u=x+y+xy$ is lower triangular, as  
	\[
		u=\begin{pmatrix}
			0 & 0 & \cdots & 0\\
			x_2y_1 & 0 & \cdots & 0\\
			x_3y_1 & 0 & \cdots & 0\\
			\vdots & \vdots & \ddots & \vdots\\
			x_ny_1 & 0 & \cdots & 0
		\end{pmatrix}.
	\]
	Since  
	$u^n=0$, the element
	\[
	v=-u+u^2-u^3+\cdots+(-1)^{n-1} u^{n-1}
	\]
	is such that 
	$u+v+uv=0$. Thus $x$ is right quasi-regular, as  
	\begin{align*}
		x+(y+v+yv)+x(y+v+yv)&=0,
	\end{align*}
	and therefore $J_1$ is right quasi-regular. Similarly one proves that 
	each $J_i$ is right quasi-regular and hence $J_i\subseteq J(M_n(R))$ for all 
	$i\in\{1,\dots,n\}$. In conclusion, 
	\[
	J_1+\cdots+J_n\subseteq J(M_n(R))
	\]
	and therefore $M_n(J)\subseteq J(M_n(R))$.
\end{proof}

\begin{exercise}
	Let $R$ be a unitary ring. Then  
	\[
	J(R)=\bigcap\{M:\text{$M$ is a left maximal ideal}\}.
	\]
\end{exercise}

\begin{exercise}
\label{xca:Jcon1}
	Let $R$ be a unitary ring. The
	following statements are equivalent: 
	\begin{enumerate}
		\item $x\in J(R)$.
		\item $x\cdot M=\{0\}$ for all simple $R$-module $M$.
		\item $x\in P$ for all primitive left ideal $P$.
		\item $1+rx$ is invertible for all $r\in R$.
		\item $1+\sum_{i=1}^n r_ixs_i$ is invertible 
		    for all $n$ and all $r_i,s_i\in R$.
		\item $x$ belongs to every maximal ideal maximal. 
	\end{enumerate}
\end{exercise}

The following exercise is entirely optional. 
It somewhat shows a recent application of radical rings
to solutions of the celebrated Yang--Baxter equation. 

\begin{exercise}
A pair $(X,r)$ is a \textbf{solution} to the 
Yang--Baxter equation if $X$ is a set and
$r\colon X\times X\to X\times X$ is a bijective map such that  
\[
	(r\times\id)\circ (\id\times r)\circ (r\times\id)
	=(\id\times r)\circ (r\times\id)\circ (\id\times r).
\]
The solution $(X,r)$ is said to be \textbf{involutive} 
if $r^2=\id$. By convention, we write 
\[
	r(x,y)=(\sigma_x(y),\tau_y(x)).
\]
The solution $(X,r)$ is said to be \textbf{non-degenerate}  
$\sigma_x\colon X\to X$ and 
$\tau_x\colon X\to X$ are bijective for all $x\in X$.

\begin{enumerate}
    \item Let $X$ be a set and $\sigma\colon X\to X$ be a bijective map. Prove that  
          the pair $(X,r)$, where 
          $r(x,y)=(\sigma(y),\sigma^{-1}(x))$, is an involutive non-degenerate solution. 
\end{enumerate}
Let $R$ be a radical ring. For $x,y\in R$ let 
\begin{align*}
	&\lambda_x(y)=-x+x\circ y=xy+y,\\
	&\mu_y(x)=\lambda_x(y)'\circ x\circ y=(xy+y)'x+x
\end{align*}
Prove the following statements:
\begin{enumerate}
    \setcounter{enumi}{1}
		\item $\lambda\colon (R,\circ)\to\Aut(R,+)$, $x\mapsto
			\lambda_x$, is a group homomorphism.
		\item $\mu\colon (R,\circ)\to\Aut(R,+)$, $y\mapsto\mu_y$,
    		is a group antihomomorphism.
	    \item The map 
    	\[
	        r\colon R\times R\to R\times R,\quad
	        r(x,y)=(\lambda_x(y),\mu_y(x)),
	    \]
	is an involutive non-degenerate solution to the Yang--Baxter equation. 
\end{enumerate}
\end{exercise}

%\begin{exercise}
%	Sea $A$ un anillo radical. Para $a,b\in A$ se define 
%	\[
%		\mu_b(a)=\lambda_a(b)'\circ a\circ b=(ab+b)'a+a.
%	\]
%	Demuestre que la función $\mu\colon (A,\circ)\to\Aut(A,+)$,
%	$b\mapsto\mu_b$, está bien definida y es un antimorfismo de grupos.
%\end{exercise}

\begin{exercise}
    If $D$ is a division ring and $R=D[X_1,\dots,X_n]$, then
    $J(R)=\{0\}$. 
%     Como las unidades de $R$ son los elementos no nulos de $D$,
% 	$J(R)$ es un ideal de $D$. Como $D$ es simple, $J(R)\in\{0,D\}$. Si
% 	$J(R)=D$, entonces existe  $f\in R$ tal que $-1+f+(-1)f=0$ y luego $-1=0$,
% 	una contradicción. Luego $J(R)=0$.
\end{exercise}


\begin{example}
\index{Ring!local}
    A commutative and unitary ring $R$ is \textbf{local} if it contains
    only one maximal ideal. 
	If $R$ is a local ring and $M$ is its maximal ideal, then $J(R)=M$. Some particular cases: 
	\begin{enumerate}
		\item If $K$ is a field and $R=K[\![X]\!]$, then $J(R)=(X)$. 
		\item If $p$ is a prime number and $R=\Z/p^n$, then $J(R)=(p)$. 
	\end{enumerate}
\end{example}

We finish the discussion on the Jacobson radical with 
some results in the case of unitary algebras. We first need an application of Zorn's lemma. 

\begin{exercise}
\label{xca:maximal_regular}
    Let $I$ be a proper left ideal that is left regular. Prove that $I$ is contained in a maximal left ideal 
    which is regular. 
\end{exercise}

% explain ideals of algebras without one. 

\begin{proposition}
	Let $A$ be a $K$-algebra and $I$ be a subset of $A$. Then $I$ is 
	a regular maximal left ideal of the algebra $A$ if and only if $I$ is 
	a regular maximal left ideal of the ring $A$.
\end{proposition}

\begin{proof}
	Let $I$ be a left regular maximal ideal of the ring $A$. We claim that
	$\lambda I\subseteq I$ for all $\lambda\in K$. Assume that 
	$\lambda I\not\subseteq I$ for some $\lambda$. Then $I+\lambda I$
	is an ideal of the ring $A$ that contains $I$, as 
	\[
	a(I+\lambda I)=aI+a(\lambda I)\subseteq I+\lambda (aI)\subseteq I+\lambda I.
	\]
	Since $I$ is maximal, it follows that $I+\lambda I=A$. 
	The left regularity of $I$ implies that there exists $e\in A$
	such that 
	$a-ae\in I$ for all $a\in A$. Write $e=x+\lambda y$ for $x,y\in
	I$. Then 
	\[
		e^2=e(x+\lambda y)=ex+e(\lambda y)=ex+(\lambda e)y\in I.
	\]
	Since $e-e^2\in I$ and $e^2\in I$, it follows that $e\in I$. Thus $A=I$, as
	$a-ae\in I$ for all $a\in A$, a contradiction.

	Conversely, if $I$ is a left regular maximal ideal of the algebra $A$, then 
	$I$ is a left regular ideal of the ring $A$. We claim that $I$ is a maximal left ideal of the ring of $A$. 
	There exists a regular maximal left ideal $M$ 
	of the ring $A$ that contains $I$. Since 
	$M$ is regular, it follows that $M$ is a regular maximal ideal of the algebra $A$. Thus 
	$M=I$ because $I$ is a maximal left ideal of the algebra $A$. 
% 	ejercicio~\ref{xca:Zorn:regular} sabemos que existe un ideal a izquierda
% 	maximal $L$ del anillo $A$ que contiene a $I$. Como $L$ es regular, la
% 	implicación demostrada nos dice que $L$ es un ideal a izquierda maximal y
% 	regular del anillo $A$. Luego $L=I$ por la maximalidad de $I$.
\end{proof}

For algebras, the Jacobson radical of an 
algebra can be defined as 
the intersection of the left ideals (of the algebra) 
that are maximal and regular. The previous 
proposition then implies that the Jacobson radical of an algebra coincides
with the Jacobson radical of the underlying ring. 

% \begin{exercise}
%     Let $A$ be an algebra. Prove that the Jacobson
%     radical of the ring $A$ coincides with the Jacobson radical of the algebra $A$. 
% \end{exercise}

% \begin{proof}
% 	Es consecuencia del teorema anterior y de que el radical de Jacobson es la
% 	intersección de los ideales a izquierda maximales y regulares.
% \end{proof}

\subsection{Amitsur's theorem}

We now prove an important result of Amitsur that
has several interesting applications. We first need a lemma. 

\begin{lemma}
	\label{lemma:algebraico=nil}
	Let $A$ be an algebra with one and let $x\in J(A)$. 
	Then $x$ is algebraic if and only if $x$ is nilpotent. 
\end{lemma}

\begin{proof}
    Since $x$ is algebraic, there exist $a_0,\dots,a_n\in K$ 
    not all zero such that 
    \[
		a_0+a_1x+\cdots+a_nx^n=0.
	\]
	Let $r$ be the smallest integer such that $a_r\ne 0$. Then 
	\[
		x^r(1+b_1x+\cdots+b_mx^m)=0,
	\]
	for some $b_1,\dots,b_m\in K$. Since $1+b_1x+\cdots+b_mx^m$ is a unit by 
	Exercise~\ref{xca:Jcon1}, it follows that $x^r=0$.
\end{proof}

An application:

\begin{proposition}
	\label{pro:algebraica=>Jnil}
	If $A$ is an algebraic algebra with one, then $J(A)$ is the largest nil ideal of $A$.
\end{proposition}

\begin{proof}
	The previous lemma implies that $J(A)$ is a nil ideal. 
	Proposition~\ref{pro:nilJ} now implies that $J(A)$ is the largest nil ideal of $A$. 
\end{proof}

\begin{theorem}[Amitsur]
	\label{thm:Amitsur}
	\index{Amitsur's theorem}
	Let $A$ be a $K$-algebra with one such that $\dim_KA<|K|$ (as cardinals). Then 
	$J(A)$ is the largest nil ideal of $A$. 
\end{theorem}

\begin{proof}
	If $K$ is finite, then $A$ is a finite-dimensional algebra. In particular, $A$ is algebraic and
	hence $J(A)$ is a nil ideal by Proposition~\ref{pro:algebraica=>Jnil}.

	Assume that $K$ is infinite and let $a\in J(A)$. Exercise~\ref{xca:Jcon1} implies that 
	every element of the form 
	$1-\lambda^{-1}a$, $\lambda\in K\setminus\{0\}$, is invertible. Thus  
	\[
		a-\lambda=-\lambda(1-\lambda^{-1}a)
	\]
	is invertible for all $\lambda\in K\setminus\{0\}$. Let
	$S=\{(a-\lambda)^{-1}:\lambda\in K\setminus\{0\}\}$. Since 
	\[
	(a-\lambda)^{-1}=(a-\mu)^{-1}\Longleftrightarrow\lambda=\mu,
	\]
	it follows that $|S|=|K\setminus\{0\}|=|K|>\dim_KA$. Then $S$ a 
	is linearly dependent set, so there are $\beta_1,\dots,\beta_n\in K$
	not all zero and distinct elements $\lambda_1,\dots,\lambda_n\in K$ such that 
	\begin{equation}
		\label{eq:Amitsur}
		\sum_{i=1}^n \beta_i(a-\lambda_i)^{-1}=0.
	\end{equation}
	Multiplying~\eqref{eq:Amitsur} by $\prod_{i=1}^n(a-\lambda_i)$ we get 
	\[
		\sum_{i=1}^n\beta_i\prod_{j\ne i}(a-\lambda_j)=0.
	\]
	We claim that $a$ is algebraic over $K$. Indeed,  
	\[
		f(X)=\sum_{i=1}^n\beta_i\prod_{j\ne i}(X-\lambda_j)
	\]
	is non-zero, as, for example, if $\beta_1\ne1$, then  
	$f(\lambda_1)=\beta_1(\lambda_1-\lambda_2)\cdots(\lambda_1-\lambda_n)\ne0$
	and $f(a)=0$. Since $a\in J(A)$ is algebraic, it follows
	$a$ is nilpotent by Lemma~\ref{lemma:algebraico=nil}.
\end{proof}

Amitsur's theorem implies the following result. 

\begin{corollary}
Let $K$ be a non-countable field. If $A$ is an algebra
over $K$ with a countable basis, then 
$J(A)$ is the largest nil ideal of $A$.
\end{corollary}

% \begin{proof}
% 	Es consecuencia del teorema de Amitsur pues $\dim_KA<|K|$. 
% \end{proof}

%We now finish the lecture with some big open problems. 


\subsection{Jacobson's conjecture}

We now conclude the lecture
with two big open problems related to the Jacobson radical. The first
one is Jacobson's conjecture. 

\begin{problem}[Jacobson]
\label{prob:Jacobson}
\index{Jacobson conjecture}
\index{Jacobson--Herstein conjecture}
Let $R$ be a noetherian ring. Is then 
\[
\bigcap_{n\geq1}J(R)^n=\{0\}?
\]
\end{problem}

Open problem \ref{prob:Jacobson} was originally formulated by Jacobson in 1956 \cite{MR0222106} 
for one-sided noetherian rings. In 1965 Herstein \cite{MR188253} found a counterexample
in the case of one-sided noetherian rings 
and reformulated the conjecture as it appears here. 

\begin{exercise}[Herstein]
Let $D$ be the ring of rationals with odd denominators. Let
$R=\begin{pmatrix}
    D & \Q\\
    0 & \Q
\end{pmatrix}$. Prove that $R$ is right noetherian and 
$J(R)=\begin{pmatrix}
J(D) & \Q\\
0 & 0
\end{pmatrix}$. Prove that 
$J(R)^n\supseteq\begin{pmatrix}0&\Q\\0&0\end{pmatrix}$ and hence $\bigcap_nJ(R)^n$ is non-zero. 
\end{exercise}

\subsection{K\"othe's conjecture}

The following problem is maybe the most important open 
problem in non-commutative ring theory. 

\begin{problem}[K\"othe]
\label{prob:Koethe}
\index{K\"othe conjecture}
Let $R$ be a ring. Is the sum 
of two arbitrary nil left ideals of $R$ is nil?
\end{problem}

Open problem~\ref{prob:Koethe} is the well-known K\"othe's conjecture. 
The conjecture was first formulated in 1930, see \cite{MR1545158}. It is known to be true
in several cases. In full generality, the problem is still open. In~\cite{MR306251} 
Krempa proved that
the following statements are equivalent:
\begin{enumerate}
    \item K\"othe's conjecture is true.  
    \item If $R$ is a nil ring, then $R[X]$ is a radical ring. 
    \item If $R$ is a nil ring, then $M_2(R)$ is a nil ring. 
    \item Let $n\geq2$. If $R$ is a nil ring, then $M_n(R)$ is a nil ring. 
\end{enumerate}

In 1956 Amitsur formulated the following conjecture, see for example
\cite{MR0347873}: If $R$ is a nil ring, then $R[X]$ is a nil ring. In~\cite{MR1793911} 
Smoktunowicz found a counterexample to Amitsur's conjecture. 
This counterexample suggests that K\"othe's conjecture might be false. 
A simplification of Smoktunowicz's example
appears in~\cite{MR3169522}. See \cite{MR1879880,MR2275597} for more
information on K\"othe's conjecture and related topics. 

