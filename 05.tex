
chapter{}

\begin{theorem}
	Let $R$ be a ring and $n\in\Z_{>0}$. Then $J(M_n(R))=M_n(J(R))$. 
\end{theorem}

\begin{proof}
	We first prove that $J(M_n(R))\subseteq M_n(J(R))$. 
	If $J(R)=R$, the theorem is clear. Let us assume that $J(R)\ne R$ and let  
	$J=J(R)$. 
	If $M$ is a simple $R$-module, then $M^n$ is a simple $M_n(R)$-module with the usual multiplication. 
	Let $x=(x_{ij})\in J(M_n(R))$ and $m_1,\dots,m_n\in M$. Then
	\[
		x\colvec{3}{m_1}{\vdots}{m_n}=0.
	\]
	In particular, $x_{ij}\in\Ann_R(M)$ for all $i,j\in\{1,\dots,n\}$. Hence 
	$x\in M_n(J)$. 

	We now prove that $M_n(J)\subseteq J(M_n(R))$. Let 
	\[
		J_1=\begin{pmatrix}
			J & 0 & \cdots & 0\\
			J & 0 & \cdots & 0\\
			\vdots & \vdots & \ddots & \vdots\\
			J & 0 & \cdots & 0
		\end{pmatrix}
		\quad\text{and}\quad
		x=\begin{pmatrix}
			x_1 & 0 & \cdots & 0\\
			x_2 & 0 & \cdots & 0\\
			\vdots & \vdots & \ddots & \vdots\\
			x_n & 0 & \cdots & 0
		\end{pmatrix}\in J_1.
	\]
	Since $x_1$ es quasi-regular, there exists $y_1\in R$ such that $x_1+y_1+x_1y_1=0$.
	If
	\[
		y=\begin{pmatrix}
			y_1 & 0 & \cdots & 0\\
			0 & 0 & \cdots & 0\\
			\vdots & \vdots & \ddots & \vdots\\
			0 & 0 & \cdots & 0
		\end{pmatrix}, 
	\]
	then $u=x+y+xy$ is lower triangular, as  
	\[
		u=\begin{pmatrix}
			0 & 0 & \cdots & 0\\
			x_2y_1 & 0 & \cdots & 0\\
			x_3y_1 & 0 & \cdots & 0\\
			\vdots & \vdots & \ddots & \vdots\\
			x_ny_1 & 0 & \cdots & 0
		\end{pmatrix}.
	\]
	Since  
	$u^n=0$, the element
	\[
	v=-u+u^2-u^3+\cdots+(-1)^{n-1} u^{n-1}
	\]
	is such that 
	$u+v+uv=0$. Thus $x$ is right quasi-regular, as  
	\begin{align*}
		x+(y+v+yv)+x(y+v+yv)&=0,
	\end{align*}
	and therefore $J_1$ is right quasi-regular. Similarly one proves that 
	each $J_i$ is right quasi-regular and hence $J_i\subseteq J(M_n(R))$ for all 
	$i\in\{1,\dots,n\}$. In conclusion, 
	\[
	J_1+\cdots+J_n\subseteq J(M_n(R))
	\]
	and therefore $M_n(J)\subseteq J(M_n(R))$.
\end{proof}

\begin{exercise}
	Let $R$ be a unitary ring. Then  
	\[
	J(R)=\bigcap\{M:\text{$M$ is a left maximal ideal}\}.
	\]
\end{exercise}

\begin{exercise}
\label{xca:Jcon1}
	Let $R$ be a unitary ring. The
	following statements are equivalent: 
	\begin{enumerate}
		\item $x\in J(R)$.
		\item $xM=0$ for all simple $R$-module $M$.
		\item $x\in P$ for all primitive left ideal $P$.
		\item $1+rx$ is invertible for all $r\in R$.
		\item $1+\sum_{i=1}^n r_ixs_i$ is invertible for all $n$ and all $r_i,s_i\in R$.
		\item $x$ belongs to every left maximal ideal maximal. 
	\end{enumerate}
\end{exercise}

The following exercise is entirely optional. It somewhat shows a recent application of radical rings
to solutions of the celebrated Yang--Baxter equation. 

\begin{exercise}
A pair $(X,r)$ is a \textbf{solution} to the 
Yang--Baxter equation if $X$ is a set and
$r\colon X\times X\to X\times X$ is a bijective map such that  
\[
	(r\times\id)\circ (\id\times r)\circ (r\times\id)
	=(\id\times r)\circ (r\times\id)\circ (\id\times r)
\]
The solution $(X,r)$ is said to be \textbf{involutive} 
if $r^2=\id$. By convention we write 
\[
	r(x,y)=(\sigma_x(y),\tau_y(x)).
\]
The solution $(X,r)$ is said to be \textbf{non-degenerate}  
$\sigma_x\colon X\to X$ and 
$\tau_x\colon X\to X$ are bijective for all $x\in X$.

\begin{enumerate}
    \item Let $X$ be a set and $\sigma\colon X\to X$ be a bijective map. Prove that  
          the pair $(X,r)$, where 
          $r(x,y)=(\sigma(y),\sigma^{-1}(x))$, is an involutive non-degenerate solution. 
\end{enumerate}
Let $R$ be a radical ring. For $x,y\in R$ let 
\begin{align*}
	&\lambda_x(y)=-x+x\circ y=xy+y,\\
	&\mu_y(x)=\lambda_x(y)'\circ x\circ y=(xy+y)'x+x
\end{align*}
Prove the following statements:
\begin{enumerate}
    \setcounter{enumi}{1}
		\item $\lambda\colon (R,\circ)\to\Aut(R,+)$, $x\mapsto
			\lambda_x$, is a group homomorphism.
		\item $\mu\colon (R,\circ)\to\Aut(R,+)$, $y\mapsto\mu_y$,
    		is a group antihomomorphism.
	    \item The map 
    	\[
	        r\colon R\times R\to R\times R,\quad
	        r(x,y)=(\lambda_x(y),\mu_y(x)),
	    \]
	is an involutive non-degenerate solution. 
\end{enumerate}
\end{exercise}

%\begin{exercise}
%	Sea $A$ un anillo radical. Para $a,b\in A$ se define 
%	\[
%		\mu_b(a)=\lambda_a(b)'\circ a\circ b=(ab+b)'a+a.
%	\]
%	Demuestre que la función $\mu\colon (A,\circ)\to\Aut(A,+)$,
%	$b\mapsto\mu_b$, está bien definida y es un antimorfismo de grupos.
%\end{exercise}

\begin{exercise}
    If $D$ is a division ring and $R=D[X_1,\dots,X_n]$, then
    $J(R)=\{0\}$. 
%     Como las unidades de $R$ son los elementos no nulos de $D$,
% 	$J(R)$ es un ideal de $D$. Como $D$ es simple, $J(R)\in\{0,D\}$. Si
% 	$J(R)=D$, entonces existe  $f\in R$ tal que $-1+f+(-1)f=0$ y luego $-1=0$,
% 	una contradicción. Luego $J(R)=0$.
\end{exercise}


\begin{example}
\index{Ring!local}
    A commutative and unitary ring $R$ is \textbf{local} if it contains
    only one maximal ideal. 
	If $R$ is a local ring and $M$ be its maximal ideal, then $J(R)=M$. Some particular cases: 
	\begin{enumerate}
		\item If $K$ is a field and $R=K\left[ [X] \right]$, then $J(R)=(X)$. 
		\item If $p$ is a prime number and $R=\Z/p^n$, then $J(R)=(p)$. 
	\end{enumerate}
\end{example}

We finish the discussion on the Jacobson radical with 
some results in the case of unitary algebras. We first need an application of Zorn's lemma. 

\begin{exercise}
\label{xca:maximal_regular}
    Let $I$ be a proper left ideal that is left regular. Prove that $I$ is contained in a maximal left ideal 
    which is regular. 
\end{exercise}

\begin{theorem}
	Let $A$ be a $K$-algebra and $I$ be a subset of $A$. Then $I$ is 
	a left regular maximal ideal of the algebra $A$ if and only if $I$ is 
	a left regular maximal ideal of the ring $A$.
\end{theorem}

\begin{proof}
	Let $I$ be a left regular maximal ideal of the ring $A$. We claim that
	$\lambda I\subseteq I$ for all $\lambda\in K$. Assume that 
	$\lambda I\not\subseteq I$ for some $\lambda$. Then $I+\lambda I$
	is an ideal of the ring $A$ that contains $I$, as 
	\[
	a(I+\lambda I)=aI+a(\lambda I)\subseteq I+\lambda (aI)\subseteq I+\lambda I.
	\]
	Since $I$ is maximal, it follows that $I+\lambda I=A$. 
	The left regularity of $I$ implies that there exists $e\in R$
	such that 
	$a-ae\in I$ for all $a\in A$. Write $e=x+\lambda y$ for $x,y\in
	I$. Then 
	\[
		e^2=e(x+\lambda y)=ex+e(\lambda y)=ex+(\lambda e)y\in I.
	\]
	Since $e-e^2\in I$ and $e^2\in I$, it follows that $e\in I$. Thus $A=I$, as
	$a-ae\in I$ for all $a\in A$, a contradiction.

	Conversely, if $I$ is a left regular maximal ideal of the algebra $A$, then 
	$I$ is a left regular ideal of the ring $A$. We claim that $I$ is maximal. 
	There exists a left regular maximal ideal $M$ of the ring $A$ that contains $I$. Since 
	$M$ is left regular, it follows that $M$ is a left regular maximal ideal of the ring $A$. Thus 
	$M=I$ because $I$ is maximal. 
% 	ejercicio~\ref{xca:Zorn:regular} sabemos que existe un ideal a izquierda
% 	maximal $L$ del anillo $A$ que contiene a $I$. Como $L$ es regular, la
% 	implicación demostrada nos dice que $L$ es un ideal a izquierda maximal y
% 	regular del anillo $A$. Luego $L=I$ por la maximalidad de $I$.
\end{proof}

\begin{exercise}
    Let $A$ be an algebra. Prove that the Jacobson
    radical of the ring $A$ coincides with the Jacobson radical of the algebra $A$. 
\end{exercise}

% \begin{proof}
% 	Es consecuencia del teorema anterior y de que el radical de Jacobson es la
% 	intersección de los ideales a izquierda maximales y regulares.
% \end{proof}

\topic{Amitsur's theorem}

We now prove an important result of Amitsur that
has several interesting applications. We first need a lemma. 

\begin{lemma}
	\label{lemma:algebraico=nil}
	Let $A$ be an algebra with one and let $x\in J(A)$. 
	Then $x$ is algebraic if and only if $x$ is nil. 
\end{lemma}

\begin{proof}
    Since $x$ is algebraic, there exist $a_0,\dots,a_n\in K$ 
    not all zero such that 
    \[
		a_0+a_1x+\cdots+a_nx^n=0.
	\]
	Let $r$ be the smallest integer such that $a_r\ne 0$. Then 
	\[
		x^r(1+b_1x+\cdots+b_mx^m)=0,
	\]
	for some $b_1,\dots,b_m\in K$. Since $1+b_1x+\cdots+b_mx^m$ is a unit by 
	Exercise~\ref{xca:Jcon1}, it follows that $x^r=0$.
\end{proof}

An application:

\begin{proposition}
	\label{pro:algebraica=>Jnil}
	If $A$ is an algebraic algebra with one, then $J(A)$ is the largest nil ideal of $A$.
\end{proposition}

\begin{proof}
	The previous lemma implies that $J(A)$ is a nil ideal. 
	Proposition~\ref{pro:nilJ} now implies that $J(A)$ is the largest nil ideal of $A$. 
\end{proof}

\begin{theorem}[Amitsur]
	\label{thm:Amitsur}
	Let $A$ be a $K$-algebra with one such that $\dim_KA<|K|$ (as cardinals). Then 
	$J(A)$ is the largest nil ideal of $A$. 
\end{theorem}

\begin{proof}
	If $K$ is finite, then $A$ is a finite-dimensional algebra. In particular, $A$ is algebraic and
	hence $J(A)$ is a nil ideal by Proposition~\ref{pro:algebraica=>Jnil}.

	Assume that $K$ is infinite and let $a\in J(A)$. Exercise~\ref{xca:Jcon1} implies that 
	every element of the form 
	$1-\lambda^{-1}a$, $\lambda\in K\setminus\{0\}$, is invertible. Thus  
	\[
		a-\lambda=-\lambda(1-\lambda^{-1}a)
	\]
	is invertible for all $\lambda\in K\setminus\{0\}$. Let
	$S=\{(a-\lambda)^{-1}:\lambda\in K\setminus\{0\}\}$. Since 
	\[
	(a-\lambda)^{-1}=(a-\mu)^{-1}\Longleftrightarrow\lambda=\mu,
	\]
	it follows that $|S|=|K\setminus\{0\}|=|K|>\dim_KA$. Then $S$ 
	is linearly dependent, so there are $\beta_1,\dots,\beta_n\in K$
	not all zero and distinct elements $\lambda_1,\dots,\lambda_n\in K$ such that 
	\begin{equation}
		\label{eq:Amitsur}
		\sum_{i=1}^n \beta_i(a-\lambda_i)^{-1}=0.
	\end{equation}
	Multiplying~\eqref{eq:Amitsur} by $\prod_{i=1}^n(a-\lambda_i)$ we get 
	\[
		\sum_{i=1}^n\beta_i\prod_{j\ne i}(a-\lambda_j)=0.
	\]
	We claim that $a$ is algebraic over $K$. Indeed,  
	\[
		f(X)=\sum_{i=1}^n\beta_i\prod_{j\ne i}(X-\lambda_j)
	\]
	is non-zero, as, for example, if $\beta_1\ne1$, then  
	$f(\lambda_1)=\beta_1(\lambda_1-\lambda_2)\cdots(\lambda_1-\lambda_n)\ne0$
	and $f(a)=0$. Since $a\in J(A)$ is algebraic, it follows
	$a$ is nil by Lemma~\ref{lemma:algebraico=nil}.
\end{proof}

Amitsur's theorem implies the following result. 

\begin{corollary}
Let $K$ be a non-countable field. If $A$ is an algebra
over $K$ with a countable basis, then 
$J(A)$ is the largest nil ideal of $A$.
\end{corollary}

% \begin{proof}
% 	Es consecuencia del teorema de Amitsur pues $\dim_KA<|K|$. 
% \end{proof}

%We now finish the lecture with some big open problems. 


\topic{Two open problems}

We now conclude the lecture
with two big open problems related with the Jacobson radical.

\begin{openproblem}[Jacobson--Herstein]
\label{prob:Jacobson}
\index{Jacobson conjecture}
\index{Jacobson--Herstein conjecture}
Let $R$ be a noetherian ring. Is then 
\[
\bigcap_{n\geq1}J(R)^n=\{0\}?
\]
\end{openproblem}

Open problem \ref{prob:Jacobson} was originally formulated by Jacobson in 1956 \cite{MR0222106} 
for one-sided noetherian rings. In 1965 Herstein \cite{MR188253} found a counterexample
in the case of one-sided noetherian rings 
and reformulated the conjecture as it appears here. 

\begin{exercise}[Herstein]
Let $D$ be the ring of rationals with odd denominators. Let
$R=\begin{pmatrix}
    D & \Q\\
    0 & \Q
\end{pmatrix}$. Prove that $R$ is right noetherian and 
$J(R)=\begin{pmatrix}
J(D) & \Q\\
0 & 0
\end{pmatrix}$. Prove that 
$J(R)^n\supseteq\begin{pmatrix}0&\Q\\0&0\end{pmatrix}$ and hence $\bigcap_nJ(R)^n$ is non-zero. 
\end{exercise}

The following problem is maybe the most important open 
problem in non-commutative ring theory. 

\begin{openproblem}[K\"othe]
\label{prob:Koethe}
\index{K\"othe conjecture}
Let $R$ be a ring. Is the sum 
of two arbitrary nil left ideals of $R$ is nil?
\end{openproblem}

Open problem~\ref{prob:Koethe} is the well-known K\"othe's conjecture. 
The conjecture was first formulated in 1930, see \cite{MR1545158}. It is known to be true
in several cases. In full generality, the problem is still open. In~\cite{MR306251} 
Krempa proved that
the following statements are equivalent:
\begin{enumerate}
    \item K\"othe's conjecture is true.  
    \item If $R$ is a nil ring, then $R[X]$ is a radical ring. 
    \item If $R$ is a nil ring, then $M_2(R)$ is a nil ring. 
    \item Let $n\geq2$. If $R$ is a nil ring, then $M_n(R)$ is a nil ring. 
\end{enumerate}

In 1956 Amitsur formulated the following conjecture, see for example
\cite{MR0347873}: If $R$ is a nil ring, then $R[X]$ is a nil ring. In~\cite{MR1793911} 
Smoktunowicz found a counterexample to Amitsur's conjecture. 
This counterexample suggests that K\"othe's conjecture might be false. 
A simplification of Smoktunowicz's example
appears in~\cite{MR3169522}. See \cite{MR1879880,MR2275597} for more
information on K\"othe's conjecture and related topics. 

