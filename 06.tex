\chapter{}

\section*{\S5. Artinian modules}

\begin{definition}
	Let $R$ be a ring. A module $N$ is \textbf{artinian} if every decreasing sequence 
	$N_1\supseteq N_2\supseteq\cdots$ of submodules of $N$ stabilizes, that is
	there exists $n\in\Z_{>0}$ such that 
	$N_n=N_{n+k}$ for all $k\in\Z_{>0}$.
\end{definition}

Let $X$ be a set and $\mathcal{S}$ be a set of subsets of $X$. 
We say that $A\in\mathcal{S}$ is a \textbf{minimal element} of $\mathcal{S}$
if there is no $Y\in\mathcal{S}$ such that $Y\subsetneq A$. 

\begin{proposition}
\label{pro:artinian_minimal}
	A module $N$ is artinian if and only if every non-empty subset of submodules of $N$ 
	contains a minimal element. 
\end{proposition}

\begin{proof}
	Assume that $N$ is artinian. Let $\mathcal{S}$ be the non-empty set of submodules of $N$. 
	Suppose that $\mathcal{S}$ has no minimal element and let $N_1\in\mathcal{S}$. 
	Since $N_1$ is not minimal, there exists 
	$N_2\in\mathcal{S}$ such that $N_1\supsetneq N_2$. Now assume the 
	submodules 
	\[
	N_1\supsetneq N_2\supseteq\cdots\supsetneq N_k
	\]
	we chosen. 
	Since $N_k$ is not minimal, there exists $N_{k+1}$ such that $N_k\supsetneq N_{k+1}$.
	This procedure produces a sequence $N_1\supsetneq
	N_2\supsetneq\cdots$ that cannot stabilize, a contradiction. 
	
	If $N_1\supseteq N_2\supseteq\cdots$ is a sequence of submodules, then 
	$\mathcal{S}=\{N_j:j\geq1\}$ has a minimal element, say $N_n$. Then
	$N_n=N_{n+k}$ for all $k$. 
\end{proof}

A modulo $N$ is \textbf{noetherian} if for every sequence 
$N_1\subseteq N_2\subseteq\cdots$ of submodules of $N$ there exists $n\in\Z_{>0}$ such that 
$N_n=N_{n+k}$ for all $k\in\Z_{>0}$. 

% Let $X$ be a set and $\mathcal{S}$ be a set of subsets of $X$. We say that 
% $B\in\mathcal{S}$ is a \textbf{maximal element} of $\mathcal{S}$ if
% there is no $Z\in\mathcal{S}$ such that $B\subsetneq Z$.

\begin{exercise}
    Let $M$ be a module. The following statements are equivalent:
    \begin{enumerate}
        \item $M$ is noetherian.
        \item Every submodule of $M$ is finitely generated. 
        \item Every non-empty subset $\mathcal{S}$ of submodules of $M$ contains a maximal element, that is
            an element $X\in\mathcal{S}$ such that there is no $Z\in\mathcal{S}$ such that $X\subseteq Z$.  
    \end{enumerate}
\end{exercise}

\begin{exercise}
\label{xca:AN_exact}
	Let 
	\[
	\begin{tikzcd}
		0 \arrow{r}
		& A \arrow{r}{f}
		& B \arrow{r}{g}
		& C \arrow{r}
		& 0
	\end{tikzcd}
	\]
	be an exact sequence of modules. Prove that $B$ is noetherian (resp.
	artinian) if and only if $A$ and $C$ are noetherian (resp. artinian).
\end{exercise}

% \begin{definition}
% 	Un anillo $R$ se dice \textbf{noetheriano a izquierda} si el módulo 
% 	$\prescript{}{R}R$ es noetheriano.
% \end{definition}
%Similarly one defines right noetherian rings.

\begin{definition}
	A ring $R$ is \textbf{left artinian} if the module 
	$\prescript{}{R}R$ is artinian.
\end{definition}

Similarly one defines right artinian rings. 

\begin{example}
	The ring $\Z$ is noetherian. It is not artinian, as the sequence
	\[
	2\Z\supseteq
	4\Z\supseteq 8\Z\supseteq\cdots
	\]
	does not stabilize. 
\end{example}

\begin{definition}
	\label{def:serie_de_composicion}
	A \textbf{composition series} of the module $M$ is a sequence 
	\[
		\{0\}=M_0\subsetneq M_1\subsetneq M_2\subsetneq\cdots\subsetneq M_n=M
	\]
	of submodules of $M$ such that each $M_i/M_{i-1}$ is non-zero and has no proper submodules. 
	In this case 
	$n$ is the length of $M$ and $M$ is said to have \textbf{finite length}.
\end{definition}

The previous definition makes sense also for non-unitary rings. That is why
it is required that each quotient $M_i/M_{i-1}$ has no proper submodules.

\begin{theorem}
	\label{thm:serie_de_composicion}
	A non-zero module admits a composition series if and only if it is artinian and noetherian.
\end{theorem}

\begin{proof}
	Let $M$ be a non-zero module and let $\{0\}=M_0\subsetneq
	M_1\subsetneq\cdots\subsetneq M_n=M$ be a composition series for $M$.
	We claim that each $M_i$ is artinian and noetherian. We proceed by induction on $i$. The case
	$i=0$ is trivial. Let us assume that $M_i$ is artinian and noetherian. Since 
	$M_i/M_{i+1}$ has no proper submodules and the sequence 
	\[
	\begin{tikzcd}
		0 \arrow{r}
		& M_i \arrow{r}
		& M_{i+1} \arrow{r}
		& M_{i+1}/M_i \arrow{r}
		& 0
	\end{tikzcd}
	\]
	is exact, it follows that 
	$M_{i+1}$ is artinian and noetherian, see Exercise \ref{xca:AN_exact}. 

    Conversely, let $M$ be an artinian and noetherian module. Let $M_0=\{0\}$ and 
    $M_1$ be minimal among the submodules of $M$ (it exists by Proposition \ref{pro:artinian_minimal}.
    If $M_1\ne M$, let 
	$M_2$ be minimal among those submodules of $M$ such that $M_1\subsetneq M_2$. This procedure
	produces a sequence 
	\[
		\{0\}=M_0\subsetneq M_1\subsetneq M_2\subsetneq\cdots
	\]
	of submodules of $M$, where each $M_{i+1}/M_i$ is non-zero and admits no
	proper submodules. Since $M$ is noetherian, the sequence stabilizes and
	hence it follows that $M_n=M$ for some $n$. 
\end{proof}

\begin{definition}
    Let $M$ be a module. 
	We say that the composition series
	\[
	M=V_0\supseteq V_1\supseteq\cdots\supseteq V_k=\{0\},
	\quad
	M=W_0\supseteq W_1\supseteq\cdots\supseteq W_l=\{0\},
	\]
	are \textbf{equivalent} if $k=l$ and there exists 
	$\sigma\in\Sym_n$ such that 
	$V_{i}/V_{i-1}\simeq W_{\sigma(i)}/W_{\sigma(i)-1}$
	for all $i\in\{1,\dots,k\}$.
\end{definition}

\begin{theorem}[Jordan--H\"older]
	\label{thm:JordanHolder}
	Any two composition series for a module are equivalent. 
\end{theorem}

\begin{proof}
    Let $M$ be a module and
    \[
		M=V_0\supseteq V_1\supseteq\cdots\supseteq V_k=\{0\},
		\quad
		M=W_0\supseteq W_1\supseteq\cdots\supseteq W_l=\{0\},
	\]
	be composition series of $M$. 
	We claim that these composition series are equivalent. 
	We proceed by induction on $k$. The case $k=1$ is trivial, as 
	in this case $M$ has no proper submodules and $M\supseteq\{0\}$ 
	is the only possible composition series for $M$. So
	assume the result holds for modules with composition series of length $<k$. If $V_1=W_1$, then 
	$V_1$ has composition series of lengths $k-1$ and $l-1$. The inductive hypothesis implies that 
	$k=l$ and we are done. So assume that $V_1\ne W_1$. Since $V_1$ and $W_1$ are submodules of $M$, the
	sum $V_1+W_1$ is also a submodule of $M$. Moreover, $V/V_1$ has no non-zero proper submodules
	and hence 
	$V_1+W_1=V$. Then 
	\[
		V/V_1=\frac{V_1+W_1}{V_1}\simeq\frac{V_1}{V_1\cap W_1}.
	\]
	Since $V_1$ has a composition series, $V_1$ is artinian and
	noetherian by Theorem~\ref{thm:serie_de_composicion}. The submodule $U=V_1\cap W_1$ is also 
	artinian and noetherian and hence, by Theorem \ref{thm:serie_de_composicion}, 
	it admits a composition series 
	\[
		U=U_0\supseteq U_1\supseteq\cdots\supseteq U_r=\{0\}.
	\]
    Thus
    $V_1\supseteq\cdots\supseteq V_k=\{0\}$ and  
	$V_1\supseteq U\supseteq U_1\supseteq\cdots\supseteq U_r=\{0\}$ are both composition 
	series for $V_1$. The inductive hypothesis implies that 
	$k-1=r+1$ and that these composition series are equivalent. Similarly, 
	\[
		W_1\supseteq W_1\supseteq\cdots\supseteq W_l=\{0\},
		\quad
		W_1\supseteq U\supseteq U_1\supseteq\cdots\supseteq U_{r}=\{0\},
	\]
    are both composition series for $W_1$ and hence $l-1=r+1$ and these composition 
    series are equivalent. Therefore $l=k$ and the proof is completed. 
\end{proof}

Jordan--H\"older's theorem allows us to define the 
length of modules that admit a composition series. 

\begin{definition}
    Let $M$ be a module with a composition series. 
    The \textbf{length} $\ell(M)$ of $M$ is defined as the length of any composition series of $M$. 
\end{definition}

A module is said to be of finite length if it admits a composition series. 

\begin{exercise}
	If $N$ and $Q$ are modules with composition series and  
	\[
	\begin{tikzcd}
		0 \arrow[r]
		& N \arrow{r}{f}
		& M \arrow{r}{g}
		& Q \arrow[r]
		& 0
	\end{tikzcd}
	\]
	is an exact sequence of modules, then $\ell(M)=\ell(N)+\ell(Q)$.
\end{exercise}

%\begin{proof}
%	Sean $Q=Q_0\supsetneq Q_1\supsetneq\cdots\supsetneq Q_m=0$ y
%	$N=N_0\supsetneq N_1\supseteq\cdots\supsetneq N_n=0$ series de composición
%	para $Q$ y $N$ respectivamente. Entonces
%	\[
%		M=g^{-1}(Q_0)\supsetneq g^{-1}(Q_1)\supsetneq\cdots\supsetneq g^{-1}(Q_m)=f(N_0)\supsetneq f(N_1)\supsetneq\cdots\supsetneq f(N_n)=0
%	\]
%	es una serie de composición para $M$ y luego $c(M)=c(N)+c(Q)$.
%\end{proof}

\begin{exercise}
	If $A$ and $B$ are finite-length submodules of $M$, then  
	\[
	\ell(A+B)+\ell(A\cap B)=\ell(A)+\ell(B).
	\]
\end{exercise}

\section*{\S6. Semisimple modules}

In the first lectures we studied semisimple modules over finite-dimensional 
algebras. Let us now review the theory of semisimple modules over rings. 
A (finitely generated) module $M$ (over a ring $R$) is \textbf{semisimple} 
if it isomorphic to a (finite) direct sum of simple modules. 

\begin{definition}
    Let $R$ be a ring. A left ideal $L$ is said to be \textbf{minimal}
    if $L\ne\{0\}$ and there is no left ideal $J$
    such that $\{0\}\subsetneq J\subsetneq I$.
\end{definition}

The ring $\Z$ contains no minimal left ideals. If $I$ is a non-zero 
left ideal of $\Z$, then
$I=(n)$ for some $n>0$ and $I=(n)\supsetneq (2n)$. 

\begin{proposition}
    Let $R$ be a left artinian ring. 
    Then every non-zero left ideal contains a minimal left ideal. 
\end{proposition}

\begin{proof}
    Let $X$ be the family of non-zero left ideals contained in $I$. Then $X$ is non-empty, as 
    $I\in X$. Then $X$ contains a minimal element by Proposition \ref{pro:artinian_minimal}. 
\end{proof}

% \begin{proposition}
% 	Let $R$ be a unitary ring and $M$ be a unitary semisimple module. 
% 	The following statements are equivalent:
% 	\begin{enumerate}
% 		\item $M$ is noetherian.
% 		\item $M$ is artinian.
% 		\item $M$ is a direct sum of finitetely many simple modules. 
% 	\end{enumerate}
% \end{proposition}

% \begin{proof}
% 	We first prove that $3)\Longleftrightarrow1)$ and that 
% 	$3)\Longleftrightarrow2)$. Como cada submódulo simple es artiniano y
% 	noetheriano, $M$ resulta artiniano y noetheriano. Recíprocamente, si $M$ es
% 	artiniano, $I$ debe ser finito pues de lo contrario podríamos elegir
% 	elementos $i_1,i_2,i_3,\dots$ de $I$ tales que la sucesión
% 	\[
% 		\bigoplus_{i\in I}M_i\supsetneq \bigoplus_{i\in I\setminus\{i_1\}}M_i\supsetneq\bigoplus_{i\in I\setminus\{i_1,i_2\}}M_i\supsetneq\cdots
% 	\]
% 	nunca se estabiliza. Análogamente, si $M$ es noetheriano, podríamos elegir
% 	elementos $i_1,i_2,i_3,\dots\in I$ tales que la sucesión
% 	\[
% 		M_{i_1}\subsetneq M_{i_1}\oplus M_{i_2}\subsetneq\cdots
% 	\]
% 	nunca se estabiliza.
% \end{proof}

A ring $R$ with identity is \textbf{semisimple} if it is a direct sum of finitely many minimal left ideals. Note
that $\prescript{}{R}{R}$ is finitely generated by $\{1\}$. Minimal left ideals of $R$ 
are exactly the simple submodules of $\prescript{}{R}{R}$. 
This means that 
the ring $R$ is semisimple if and only if the module
$\prescript{}{R}{R}$ is semisimple.  

\begin{proposition}
    Let $R$ be a semisimple ring. Then $R$ is noetherian and artinian.
\end{proposition}

\begin{proof}
    Write $R$ as a direct sum $R=L_1\oplus\cdots\oplus L_n$ of minimal left ideals. Since 
    each $L_j$ is a simple submodule of $\prescript{}{R}{R}$, it follows that 
    \[
    L_1\oplus\cdots\oplus L_n\supsetneq L_2\oplus\cdots\oplus L_n\supsetneq\cdots\supsetneq L_n\supsetneq\{0\}
    \]
    is a composition series for $\prescript{}{R}{R}$ with composition factors
    $L_1,\dots,L_n$. Since $\prescript{}{R}{R}$ admits a composition
    series, it is artinian and noetherian by Theorem \ref{thm:serie_de_composicion}.
\end{proof}

Now it is possible to prove Artin--Wedderburn's theorem for rings. 
If $R$ is a semisimple ring, then
\[
R\simeq \prod_{i=1}^k M_{n_i}(D_i)
\]
for some $n_1,\dots,n_k\geq1$ and some
division rings $D_1,\dots,D_k$. 
The proof is somewhat
the same we did for finite-dimensional algebras.

\begin{theorem}
	\label{thm:SSartin=J}
	Si $R$ es un anillo unitario, entonces $R$ es semisimple si y sólo si $R$
	es artiniano a izquierda y $J(R)=0$.
\end{theorem}

We shall need a lemma.

\begin{lemma}
	\label{lem:Jartiniano}
	Let $R$ be a unitary left artinian ring. There exists finitely many maximal ideals 
	$I_1,\dots,I_n$ of $R$ such that 
	$J(R)=I_1\cap\cdots\cap I_n$.
\end{lemma}

\begin{proof}
	Since $R$ is
	unitary, $J(R)$ is the intersection of all maximal ideals of $R$. Since $R$ is left artinian,
	Proposition~\ref{pro:artinian_minimal} implies that 
	the set of ideals of the form
	$I_1\cap\cdots\cap I_n$ for finitely many maximal ideals $I_1,\dots,I_n$ of $R$ 
	contains a minimal element, say 
	$J=\bigcap_{i=1}^k I_i$. We claim that $J=J(R)$. If not, let $x\in
	J(R)\setminus J$. Then there exists a maximal ideal $M$ such that $x\not\in
	M$. This implies that $J\cap M\subsetneq J$, a contradiction to the minimality of 
    $J$. 
\end{proof}

We now prove the theorem. 

\begin{proof}[Proof of Theorem \ref{thm:SSartin=J}]
	AssumSupongamos primero que $R$ es semisimple. Por el teorema de Wedderburn,
	existen enteros positivos $n_1,\dots,n_k$ y anillos de división
	$D_1,\dots,D_k$ tales que \[
		R\simeq\prod_{i=1}^kM_{n_i}(D_i).
	\]
	En particular, $R$ es
	artiniano a izquierda y $J(R)=\prod_{i=1}^kJ(M_{n_i}(D_i))=0$ pues
	cada $M_{n_i}(D_i)$ es simple. 

	Recíprocamente, por el lema anterior sabemos que $0=J(R)=I_1\cap\cdots\cap I_k$ para
	finitos ideales maximales $I_1,\dots,I_k$.  Como cada cociente $R/I_i$ es
	simple, $\prod_{i=1}^k R/M_i$ es semisimple. Como $I_1\cap\cdots\cap I_k=0$,
	el morfismo $R\to \prod_{i=1}^k R/M_i$ es inyectivo y luego $R$ es también
	semisimple.
\end{proof}

Como consecuencia tenemos el siguiente resultado:

\begin{proposition}
	Sea $G$ un grupo. Entonces $\C[G]$ es artiniana a izquierda si y sólo si
	$G$ es finito. 
\end{proposition}

\begin{proof}
	Si $G$ es finito sabemos que $\C[G]$ es artiniano a izquierda por ser de
	dimensión finita.  Recíprocamente, si $G$ es infinito, sabemos que
	$J(\C[G])=0$ (por el teorema de Rickart) y que $\C[G]$ no es semisimple
	(por la proposición~\ref{pro:nunca_SS}). Luego $\C[G]$ no es artiniana a izquierda por el
	teorema~\ref{thm:SSartin=J}.
\end{proof}

Concluimos la sección con el siguiente teorema:

\begin{theorem}[Hopkins--Levitszki]
	\label{thm:Hopkins-Levitski}
	Si $R$ es un anillo unitario artiniano a izquierda, entonces $R$ es
	noetheriano a izquierda.
\end{theorem}

\begin{proof}
	Sea $J=J(R)$. Como $R$ es artiniano a izquierda, $J$ es un ideal nilpotente
	por el teorema~\ref{thm:Jnilpotente}, digamos $J^n=0$. Consideremos la
	sucesión
	\[
		R\supsetneq J\supsetneq J^2\supsetneq\cdots\supsetneq J^{n-1}\supsetneq J^n=0.
	\]
	Cada $J^{i}/J^{i+1}$ es un $R$-módulo anulado por $J$. Luego cada 
	$J^i/J^{i+1}$ es un $(R/J)$-módulo. Como $R/J$ es artiniano (pues $R$ lo
	es) y $J(R)=0$, $R/J$ es semisimple. Luego cada $J^{i}/J^{i+1}$ es
	semisimple y entonces es noetheriano a izquierda.  Inductivamente se
	demuestra entonces que cada $J^i$ es noetheriano a izquierda y luego $R$
	también lo es.
\end{proof}


\section*{\S5. Rickart's theorem}

Let $K$ be a field and $G$ be a group. The \textbf{group algebra} $K[G]$ 
is the vector space (over $K$) with basis $\{g:g\in G\}$ 
and the algebra structure given by the multiplication
\[
	\left(\sum_{g\in G}\lambda_gg\right)\left(\sum_{h\in G}\mu_hh\right)
	=\sum_{g,h\in G}\lambda_g\mu_h(gh).
\]
Note that every element of $K[G]$ is a finite sum of the form $\sum_{g\in G}\lambda_gg$.

\begin{exercise}
\label{xc:K[G]notsimple}
    If $G$ is non-trivial, then $K[G]$ is not simple. 
\end{exercise}

\begin{exercise}
	Let $G=C_n$ be the (multiplicative) cyclic group of order $n$. Prove that 
	$K[G]\simeq K[X]/(X^n-1)$. 
\end{exercise}

\begin{exercise}
	Let $G$ be a finitely-generated torsion-free abelian group. Prove that 
	$K[G]$ is a domain. 
\end{exercise}

\begin{exercise}
	Let $G$ be a group and $H$ be a subgroup of $G$. Let $\alpha\in K[H]$. Prove that 
    $\alpha$ is invertible (resp. left zero divisor) in $K[H]$ if and only if 
	$\alpha$ is invertible (resp. left zero divisor) in
	$K[G]$.
\end{exercise}

\begin{exercise}
	Let $G$ be a group and $\alpha=\sum_{g\in G}\lambda_gg\in K[G]$.  
	The \textbf{support} of $\alpha$ is the set 
	\[
		\supp\alpha=\{g\in G:\lambda_g\ne 0\}.
	\]
	Prove that if $g\in G$, then 
	$\supp(g\alpha)=g(\supp\alpha)$ and $\supp(\alpha g)=(\supp\alpha)g$.
\end{exercise}

% El objetivo de esta sección es calcular el radical de Jacobson del álgebra de
% grupo de un grupo finito. Comenzamos con un ejemplo:

\begin{exercise}
	Let $G=C_2=\langle g\rangle\simeq\Z/2$ the (multiplicative) 
	group with two elements. Note that every element of $K[G]$ is of the form
	$a1+bg$ for some $a,b\in K$. Prove the following statements:
	\begin{enumerate}
	    \item If the characteristic of $K$ is different from two, then 
	    \[
		K[G]\to K\times K,
		\quad
		a1+bg\mapsto (a+b,a-b),
	\]
	is an algebra isomorhism. 
	\item If the characteristic of $K$ is two, then 
	\[
	K[G]\to \begin{pmatrix}
			K & K\\
			0 & K
		\end{pmatrix},
		\quad
		a1+bg\mapsto\begin{pmatrix}
			a+b & b\\
			0 & a+b
		\end{pmatrix},
	\]
	is an algebra isomorphism. 
	\end{enumerate}
\end{exercise}

Veamos otros ejemplo un poco más difíciles. La idea a utilizar es la siguiente:
Si $A$ es una $K$-álgebra y $\rho\colon G\to U(A)$ es un morfismo de grupos,
donde $U(A)$ es el grupo de unidades de $A$, entonces la función $K[G]\to A$,
$\sum_{g\in G}\lambda_gg\mapsto\sum_{g\in G}\lambda_g\rho(g)$, es un morfismo
de álgebras.

\begin{exercise}
	Let $G=C_3$ be the (multiplicative) group of three elements. Prove that
	$\R[G]\simeq\R\times\C$.
% 	Escribamos $G=\langle g:g^3=1\rangle$ y sea 
% 	\[
% 		\varphi\colon\R[G]\to\R\times\C,
% 		\quad
% 		g\mapsto (1,\omega),
% 	\]
% 	donde $\omega$ es una raíz cúbica primitiva de la unidad. Entonces
% 	$\varphi$ es inyectivo pues
% 	$0=\varphi(a1+bg+cg^2)=(a+b+c,a+b\omega+c\omega^2)$ implica que $a=b=c=0$.
% 	Luego $\varphi$ es un isomorfismo pues
% 	$\dim_\R\R[G]=\dim_\R(\R\times\C)=3$. 
\end{exercise}

\begin{exercise}
	Let $G=\langle r,s:r^3=s^2=1,\,srs=r^{-1}\rangle$ be the dihedral group of six elements. 
	Prove the following statements:
	\begin{enumerate}
	    \item $\C[G]\simeq\C\times\C\times M_2(\C)$.
	    \item $\Q[G]\simeq\Q\times\Q\times M_2(\Q)$.
	\end{enumerate}  
% 	Sea $\omega$ una raíz cúbica de la unidad y sean  
% 	\[
% 		R=\begin{pmatrix}
% 			\omega & 0\\
% 			0 & \omega^2
% 		\end{pmatrix},
% 		\quad
% 		S=\begin{pmatrix}
% 			0 & 1\\
% 			1 & 0
% 		\end{pmatrix}.
% 	\]
% 	Un cálculo sencillo muestra que $R^2=S^2=I$ y que $SRS=R^{-1}$. Sea
% 	\[
% 		\varphi\colon\C[G]\to\C\times\C\times M_2(\C),\quad
% 		r\mapsto (1,1,R),\quad
% 		s\mapsto (1,-1,S).
% 	\]
% 	Es fácil ver que $\varphi$ es un morfismo de álgebras. Veamos que es
% 	biyectivo. Como $\dim_{\C}\C[G]=\dim_{\C}(\C\times\C\times M_2(\C))=6$,
% 	basta ver que $\varphi$ es inyectivo. Si 
% 	\[
% 		\alpha=a_0+a_1r+a_2r^2+(b_0+b_1r+b_2r^2)s\in\ker\varphi,
% 	\]
% 	entonces 
% 	\[
% 		0=\varphi(\alpha)=\left(u,v,\begin{pmatrix} \alpha_{11} & \alpha_{12}\\\alpha_{21}&\alpha_{22}\end{pmatrix}\right), 
% 	\]
% 	donde
% 	\begin{align*}
% 		&u = a_0+a_1+a_2+b_0+b_1+b_2, && v = a_0+a_1+a_2-b_0-b_1-b_2,\\
% 		&\alpha_{11}=a_0+a_1\omega+a_2\omega^2, && \alpha_{12}=b_0+b_1\omega+b_2\omega^2,\\
% 		&\alpha_{21}=b_0+b_2\omega+b_1\omega^2, && \alpha_{22}=a_0+a_2\omega+a_1\omega^2.
% 	\end{align*}
% 	Un cálculo sencillo muestra que estas ecuaciones implican que
% 	$\alpha=0$ y luego $\varphi$ es inyectiva.  
\end{exercise}

We now consider the following problem. 

\begin{openproblem}
Let $G$ be a group and $K$ be a field. When $J(K[G])=\{0\}$?
\end{openproblem}

As an application of Amitsur's theorem we prove that 
complex group algebras have null Jacobson radical.
This is known as 
Rickart's theorem. The original proof found by Rickart 
uses complex analysis. Here, however, 
we present an algebraic proof. 


\begin{theorem}[Rickart]
\label{thm:J(C[G])=0}
    Let $G$ be a group. Then $J(\C[G])=\{0\}$.
\end{theorem}

To prove the theorem we need a lemma.

\begin{lemma}
Let $G$ be a group. Then $J(\C[G])$ is nil.        
\end{lemma}

\begin{proof}
    We need to show that every element of $J(\C[G])$ is nilpotent. 
    If $G$ is countable, then the result follows from Amitsur's theorem. So assume that 
    $G$ is not countable. Let $\alpha\in J(\C[G])$, say
    \[
    \alpha=\sum_{i=1}^n\lambda_ig_i,
    \]
    where $\lambda_1,\dots,\lambda_n\in\C$ and $g_1,\dots,g_n\in G$. Let $H=\langle g_1,\dots,g_n\rangle$.
    Then $g\in \C[H]$ and $H$ is countable. We claim that $g\in J(\C[H])$. Decompose
    $G$ as a disjoint union 
    \[
    G=\bigcup_\lambda x_\lambda H
    \]
    of cosets of $H$ in $G$. Then $\C[G]=\bigoplus_\lambda x_\lambda\C[H]$ and
    hence $\C[G]=\C[H]\oplus K$ for some right module $K$ over $\C[H]$. Since $\alpha\in J(\C[G])$, for each 
    $\beta\in\C[H]$ there exists $\gamma\in\C[G]$ such that 
    $\gamma(1-\beta\alpha)=1$. Write $\gamma=\gamma_1+\kappa$ for $\gamma_1\in\C[H]$ and $\kappa\in K$. Then
    \[
    1=\gamma(1-\beta\alpha)=\gamma_1(1-\beta\alpha)+\kappa(1-\beta\alpha)
    \]
    and hence $\kappa(1-\beta\alpha)\in K\cap \C[H]=\{0\}$. Since $1=\gamma_1(1-\beta\alpha)$, it follows that
    $\alpha\in J(\C[H])$ and the lemma follows from Amitsur's theorem.  
\end{proof}

We now prove the theorem. 

\begin{proof}[Proof of Theorem \ref{thm:J(C[G])=0}]
    For $\alpha=\sum_{i=1}^n\lambda_ig_i\in\C[G]$ let 
    \[
    \alpha^*=\sum_{i=1}^n\overline{\lambda_i}g_i^{-1}.
    \]
    Then $\alpha\alpha^*=0$ if and only if $\alpha=0$ and, moreover, 
    $(\alpha\beta)^*=\beta^*\alpha^*$ for all $\beta\in\C[G]$. 
    Assume that $J(\C[G])\ne\{0\}$ and let $\alpha\in J(\C[G])\setminus\{0\}$. Then
    $\beta=\alpha\alpha^*\in J(\C[G])$, as $J(\C[G])$ is an ideal of $\C[G]$. Moreover, $\beta\ne 0$, as 
    \[
    (\beta^m)^*=(\beta^*)^m=\beta^m
    \]
    for all $m\geq1$. If there exists $k\geq2$ such that $\beta^k=0$ and $\beta^{k-1}\ne 0$, then
    \[
    \beta^{k-1}\left(\beta^{k-1}\right)^*=\beta^{2k-2}=0
    \]
    and hence $\beta^{k-1}=0$, a contradiction. Thus $\beta=0$ and therefore $\alpha=0$. 
\end{proof}

To obtain a consequence of Rickart's theorem we need two lemmas. 

\begin{lemma}[Nakayama]
	\label{lem:Nakayama}
	Let $R$ be a unitary ring and $M$ be a finitely generated module. If 
	$J(R)\cdot M=M$, then $M=\{0\}$.
\end{lemma}

\begin{proof}
    Since $M$ is finitely generated, we may assume that 
	$M=(x_1,\dots,x_n)$. Since $x_n\in M=J(R)\cdot M$, 
	there exist $r_1,\dots,r_n\in J(R)$ such that $x_n=r_1\cdot x_1+\cdots+r_n\cdot x_n$, that is 
	$(1-r_n)\cdot x_n=\sum_{j=1}^{n-1}r_j\cdot x_j$. 
	Since $1-r_n$ is invertible, there exists $s\in R$ such that $s(1-r_n)=1$. Thus 
	$x_n=\sum_{j=1}^{n-1}(sr_j)\cdot x_j$ 
	and hence $M=(x_1,\dots,x_{n-1})$. Repeating this procedure several times 
	one obtains $M=\{0\}$.
\end{proof}

\begin{lemma}
	\label{lem:Rickart}
	Let $\iota\colon R\to S$ be a homomorphism of unitary rings. If	
	\[
	S=\iota(R)x_1+\cdots+\iota(R)x_n,
	\]
	where each $x_j$ is such that $x_jy=yx_j$ for all $y\in\iota(R)$, then 
	$\iota(J(R))\subseteq J(S)$.
\end{lemma}

\begin{proof}
	We claim that $J=\iota(J(R))$ acts trivially on each simple $S$-module $M$.
	If is $M$ is a simple module over $S$, then, in particular, $M=S\cdot m$ for some $m\ne0$. 
	Now $M$ is a module over $R$ with $r\cdot m=\iota(r)\cdot m$. Since 
	\[
		M=S\cdot m=(\iota(R)x_1+\cdots+\iota(R)x_n)\cdot m=\iota(R)\cdot (x_1\cdot m)+\cdots+\iota(R)\cdot (x_n\cdot m),
	\]
	it follows that 
	$M$ is finitely generated as a module over $\iota(R)$. Moreover, 
	\[
	J(R)\cdot
	M=J\cdot M=\iota(J)\cdot M
	\]
	is an $S$-submodule of $M$, as 
	\[
		x_j\cdot (J\cdot M)=(x_j J)\cdot M=(J x_j)\cdot M=J\cdot (x_j\cdot M)\subseteq J\cdot M.
	\]
	Since $M\ne\{0\}$, Nakayama's lemma implies that $J(R)\cdot M\subsetneq M$. The simplicity of 
	the $S$-module $M$ implies that $J(R)\cdot M=\{0\}$.
\end{proof}

We now obtain the following consequence of Rickart's theorem. 

\begin{theorem}
	If $G$ is a group, then $J(\R[G])=0$. 
\end{theorem}

\begin{proof}
	Let $\iota\colon \R[G]\to\C[G]$ be the canonical inclusion. Since 
	\[
	\C[G]=\R[G]+i\R[G],
	\]
	Lemma~\ref{lem:Rickart} and Rickart's theorem imply that 
	$\iota(J(\R[G]))\subseteq J(\C[G])=0$. Thus $J(\R[G])=0$, as $\iota$ is injective. 
\end{proof}

\section*{\S6. Maschke's theorem}

We now present another instance of the Jacobson semisimplicity problem.
In this case, our result is for finite groups. 

\begin{theorem}[Maschke]
	Let $G$ be a finite group. Then $J(K[G])=0$ if and only 
	if the characteristic of $K$ is zero 
	or does not divide the order of $G$. 
\end{theorem}

\begin{proof}
	Supongamos que $G=\{g_1,\dots,g_n\}$ con $g_1=1$. Sea $\rho\colon K[G]\to
	K$ dada por $\alpha\mapsto\trace(L_{\alpha})$, donde
	$L_{\alpha}(\beta)=\alpha\beta$. Tenemos $\rho(g_1)=n$ y $\rho(g_i)=0$ para
	todo $i\in\{2,\dots,n\}$ pues,  como $L_{g_i}(g_j)=g_{i}g_j\ne g_j$, la
	matriz de $L_{g_i}$ en la base $\{g_1,\dots,g_n\}$ tiene ceros en la
	diagonal.

	Supongamos que $J=J(K[G])$ es no nulo y sea
	$\alpha=\sum_{i=1}^n\lambda_ig_i\in J\setminus\{0\}$. Sin pérdida de
	generalidad podemos suponer que $\lambda_1\ne 0$ (pues si $\lambda_1=0$ hay
	algún $\lambda_i\ne 0$ y alcanza con tomar $g_i^{-1}\alpha\in J$). Entonces
	\[
		\rho(\alpha)=\sum_{i=1}^n \lambda_i\rho(g_i)=n\lambda_1.
	\]
	Como $G$ es un grupo finito, $K[G]$ es un álgebra de dimensión finita y
	luego $K[G]$ es artiniana a izquierda. Como el radical de Jacobson $J$ es
	un ideal nilpotente, en particular $\alpha$ es un elemento nil. Luego
	$L_{\alpha}$ es nilpotente y entonces $0=\rho(\alpha)=n\lambda_1$. Esto
	implica que la característica del cuerpo $K$ divide a $n$. 

	Recíprocamente, supongamos que la característica de $K$ es un número primo
	que divide a $n$ y sea $\alpha=\sum_{i=1}^ng_i$. Como $\alpha
	g_j=g_j\alpha=\alpha$ para todo $j\in\{1,\dots,n\}$, el conjunto
	$I=K[G]\alpha$ es un ideal de $K[G]$. Como además 
	\[
		\alpha^2=\sum_{i=1}^n g_i\alpha=n\alpha=0,
	\]
	se concluye que $I$ es un ideal no nulo y nilpotente. Luego $J(K[G])\ne 0$
	pues por la proposición~\ref{pro:nilJ} sabemos que $I\subseteq J(K[G])$.
\end{proof}

\begin{corollary}
	\label{cor:GfinitoNOnil}
	Sea $G$ un grupo finito. Entonces $K[G]$ no contiene ideales a izquierda
	nil no nulos.
\end{corollary}

\begin{proof}
	Es consecuencia inmediata del teorema de Maschke ya que $J(K[G])$ contiene a
	todo ideal a izquierda nil.	
\end{proof}

%\index{Anillo!semisimple}
%Recordemos que un anillo unitario $R$ se dice \textbf{semisimple} si para cada
%ideal $I$ de $R$ existe un ideal $J$ de $R$ tal que $R=I\oplus J$.
%
%%\begin{corollary}
%%	Sea $G$ un grupo finito y $K$ un cuerpo de característica coprima con el
%%	orden de $G$. Entonces $K[G]$ es semisimple.
%%\end{corollary}
%%
%%\begin{proof}
%%	
%%\end{proof}
%
%\begin{theorem}
%	Si $G$ es un grupo infinito, entonces $K[G]$ nunca es semisimple.
%\end{theorem}
%
%\begin{proof}
%	Sea $R=K[G]$ y supongamos que $R$ es semisimple.  Si $I$ es el ideal de
%	aumentación de $R$, existe un ideal no nulo $J$ de $R$ tal que $R=I\oplus
%	J$. Como $R$ es unitario, existen $e\in I$, $f\in J$ tales que $1=e+f$. Si
%	$x\in I$, entonces $x=xe+xf$ y luego $xf=x-xe\in I\cap J=\{0\}$. Como
%	entonces $x=xe$ para todo $x\in I$, en particular $e_1=e_1^2$. Análogamente
%	vemos que $e_2^2=e_2$. Además $ef=0$ pues $ef\in I\cap J=\{0\}$.  Como $I$
%	es el ideal de aumentación y $If=(Re)f=R(ef)=0$, se concluye que $(g-1)f=0$
%	para todo $g\in G$ pues $g-1\in I$. Si suponemos que $f=\sum_{h\in
%	G}\lambda_hh$, entonces 
%	\[
%	f=gf=\sum_{h\in G}\lambda_h(gh)=\sum_{h\in
%	G}\lambda_{g^{-1}h}h.
%	\]
%	Luego $\lambda_h=\lambda_{g^{-1}h}$ para todo $g,h\in G$, una contradicción
%	pues como $f\ne 0$ la suma que define a $f$ es infinita. 
%\end{proof}


\section*{\S6. Herstein's theorem}

El objetivo de esta sección responderemos la siguiente pregunta: ¿Cuándo un
álgebra de grupo es un álgebra algebraica? Una respuesta parcial está dada por
el teorema de Herstein. 

\begin{definition}
	\index{Grupo!localmente finito}
	Un grupo $G$ se dice \textbf{localmente finito} si todo subgrupo de $G$
	finitamente generado es finito.
\end{definition}

Si $G$ es un grupo localmente finito, entonces todo $g\in G$ tiene orden finito
(pues el subgrupo $\langle g\rangle$ es finito por ser finitamente generado). 

\begin{example}
	Todo grupo finito es obviamente localmente finito.
\end{example}

\begin{example}
	El grupo $\Z$ no es localmente finito pues es libre de torsión.
\end{example}

\begin{example}
	\index{Grupo!de Pr\"ufer}
	Sea $p$ un primo.  El \textbf{grupo de Pr\"ufer}
	\[
		\Z(p^{\infty})=\{z\in\Z:z^{p^n}=1\text{ para algún $n\in\N$}\}
	\]
	de todas
	las raíces $p$-ésimas de uno es localmente finito.
\end{example}

\begin{example}
	Sean $X$ un conjunto infinito y $\Sym_X$ el conjunto de biyecciones $X\to
	X$ que mueven únicamente una cantidad finita de elementos de $X$. Entonces
	$\Sym_X$ es localmente finito.
\end{example}

Antes de demostrar el teorema de Herstein vamos a dar una familia de ejemplos
de grupos localmente finitos. Para eso necesitamos un lema:

\begin{lemma}
	\label{lem:solvable_torsion=>lf}
	Sea $G$ un grupo y sea $N$ un subgrupo normal de $G$. Si $N$ y $G/N$ son
	localmente finitos, entonces $G$ es localmente finito.
\end{lemma}

\begin{proof}
	Sea $\pi\colon G\to G/N$ el morfismo canónico. Sea $\{g_1,\dots,g_n\}$ un
	subconjunto finito de $G$. Como $G/N$ es localmente finito, el subgrupo $Q$
	de $G/N$ generado por $\pi(g_1),\dots,\pi(g_n)$ es finito, digamos
	\[
		Q=\{\pi(g_1),\dots,\pi(g_n),\pi(g_{n+1}),\dots,\pi(g_m)\}.
	\]
	Para cada $i,j\in\{1,\dots,n\}$ sabemos que existen $u_{ij}\in N$ y
	$k\in\{1,\dots,m\}$ tales que $g_ig_j=u_{ij}g_k$. Sea $U$ el subgrupo de
	$G$ generado por los $u_{ij}$. Como $N$ es localmente finito, $U$ es un
	subgrupo finito. Como además cada elemento $g_ig_jg_l$ puede escribirse como
	\[
		g_ig_jg_l=u_{ij}g_kg_l=u_{ij}u_{kl}g_t=ug_t
	\]
	para algún $u\in U$ y algún $t\in\{1,\dots,m\}$, se concluye que el
	subgrupo $H$ de $G$ generado por $\{g_1,\dots,g_n\}$ es finito pues
	$|H|\leq m|U|$. 
\end{proof}

\index{Grupo!resoluble}
Veamos una aplicación a los grupos resolubles. Recordemos que un grupo $G$ se
dice \textbf{resoluble} si existe una sucesión de subgrupos 
\begin{equation}
	\label{eq:resoluble}
	1=G_0\subsetneq G_1\subsetneq \cdots\subsetneq G_n=G
\end{equation}
donde cada $G_i$ es normal en $G_{i+1}$ y cada cociente $G_i/G_{i-1}$ es
abeliano.

\begin{proposition}
	Si $G$ es un grupo resoluble y de torsión, entonces $G$ es localmente
	finito.
\end{proposition}

\begin{proof}
	Procederemos por inducción en la longitud $n$ de la sucesión de
	resolubilidad~\eqref{eq:resoluble}. Si $n=1$ entonces $G$ es finito por ser
	abeliano y de torsión. Supongamos que el resultado vale para grupos
	resolubles de longitud $n-1$ y sea $G$ un grupo resoluble tal
	que~\eqref{eq:resoluble}. Por hipótesis inductiva, el subgrupo normal
	$G_{n-1}$ de $G$ es localmente finito. Entonces, como $G/G_{n-1}$ es
	localmente finito por ser abeliano y de torsión, el resultado se obtiene
	del lema~\ref{lem:solvable_torsion=>lf}.
\end{proof}

\begin{theorem}[Herstein]
	\index{Teorema!de Herstein}
	Si $G$ es un grupo localmente finito, entonces $K[G]$ es algebraica.
	Recíprocamente, si $K[G]$ es algebraica y $K$ es de característica cero,
	entonces $G$ es localmente finito.
\end{theorem}

\begin{proof}
	Supongamos que $G$ es localmente finito y sea $\alpha\in K[G]$. El subgrupo
	$H=\langle\supp\alpha\rangle$ es finitamente generado y luego finito. Como
	$\alpha\in K[H]$ y $\dim_KK[H]<\infty$, el conjunto
	$\{1,\alpha,\alpha^2,\dots\}$ es linealmente dependiente. Luego $\alpha$ es
	algebraico sobre $K$.

	Sea $\{x_1,\dots,x_m\}$ un subconjunto finito de $G$. Si agregamos los
	inversos, podemos suponer que $\{x_1,\dots,x_m\}$ genera al subgrupo
	$H=\langle x_1,\dots,x_m\rangle$ como semigrupo. Si
	$\alpha=x_1+\dots+x_m\in K[G]$, entonces, como $\alpha$ es algebraico sobre
	$K$, 
	\[
		\alpha^{n+1}=a_0+a_1\alpha+\cdots+a_n\alpha^n
	\]
	para algún $n\geq0$ y escalares $a_0,\dots,x_n\in K$. Sea $w=x_{i_1}\cdots
	x_{i_{n+1}}\in H$ una palabra de longitud $n+1$. Observemos que existen enteros
	positivos $c_{i_1\cdots i_m}$ tales que 
	\[
		\alpha^{n+1}=(x_1+\cdots+x_m)^{n+1}
		=\sum_{\substack{{i_1+\cdots+i_m=n+1}\\{\text{$i_j$ enteros positivos}}}} c_{i_1\cdots i_m}x_1^{i_1}\cdots x_{m}^{i_m}.
	\]
	Como $K$
	es de característica cero, se concluye que $w\in\supp(\alpha^{n+1})$.  Pero
	como además $\alpha^{n+1}=\sum_{j=0}^na_j\alpha^j$, entonces
	$w\in\supp(\alpha^j)$ para algún $j\in\{0,\dots,n\}$. Demostramos entonces
	que toda palabra en las $x_j$ de longitud $n+1$ puede escribirse como una
	palabra en las $x_j$ de longitud a lo sumo $n$.  Luego $H$ es finito y
	entonces $G$ es localmente finito.
\end{proof}

\section*{\S6.Formanek's theorem}

Veremos un resultado de Formanek que puede entenderse como una generalización
del teorema de Herstein. 

\begin{exercise}
	Sea $A$ un álgebra algebraica y sea $a\in A$. Demuestre las siguientes
	afirmaciones:
	\begin{enumerate}
		\item $a$ es un divisor de cero a izquierda si y sólo si $a$ es un
			divisor de cero a derecha.
		\item $a$ es inversible a izquierda si y sólo si $a$ es inversible a
			derecha.
		\item $a$ es inversible si y sólo si $a$ no es un divisor de cero.
	\end{enumerate}
\end{exercise}

%\begin{proof}
%	Como $a$ es algebraico, podemos escribir 
%	\[
%		a^n(1+\lambda_1a+\cdots+\lambda_ma^m)=0
%	\]
%	para algún $n\geq0$ minimal y escalares $\lambda_1,\dots,\lambda_m$. Si 
%	$n>0$, entonces 
%	\[
%	b=(1+\lambda_1a+\cdots+\lambda_ma^m)a^{n-1}\ne 0
%	\]
%	cumple que $ab=ba=0$. Si $n=0$, entonces 
%	\[
%		c=-\lambda_1-\lambda_2a-\cdots-\lambda_ma^{m-1}\ne 0
%	\]
%	cumple que $ac=ca=1$. 
%\end{proof}

\begin{exercise}
	\label{exa:norma}
	Si $\alpha=\sum_{g\in G}\alpha_gg\in\C[G]$ se define $|\alpha|=\sum_{g\in
	G}|\alpha_g|\in\R$. Demuestre que valen las siguientes propiedades:
	\begin{enumerate}
		%\item $|\trace(\alpha)|\leq |\alpha|$, 
		\item $|\alpha+\beta|\leq|\alpha|+|\beta|$, y 
		\item $|\alpha\beta|\leq|\alpha||\beta|$ 
	\end{enumerate}
	para todo $\alpha,\beta\in\C[G]$.
\end{exercise}

\begin{theorem}[Formanek, primera versión]
	\label{thm:FormanekQ}
	\index{Teorema!de Formanek}
	Sea $G$ un grupo y supongamos que todo elemento de $\Q[G]$ es inversible o
	un divisor de cero. Entonces $G$ es localmente finito.
\end{theorem}

\begin{proof}
	Sea $\{x_1,\dots,x_n\}$ un subconjunto finito de $G$. Si agregamos los
	inversos, podemos suponer que $\{x_1,\dots,x_n\}$ genera al subgrupo
	$H=\langle x_1,\dots,x_n\rangle$ como semigrupo. Sea
	\[
		\alpha=\frac{1}{2n}(x_1+\cdots+x_n)\in\Q[G]
	\]

	Veamos que $1-\alpha\in\Q[G]$ es inversible. Si no, entonces es un divisor de cero. 
	Si existe $\delta\in\Q[G]$ tal que $\delta(1-\alpha)=0$, entonces
	$\delta=\delta\alpha$ y luego, como 
	\[
		|\delta|=|\delta\alpha|\leq|\delta||\alpha|=|\delta|/2,
	\]
	se concluye que $\delta=0$. Similarmente se demuestra que $(1-\alpha)\delta=0$ implica que
	$\delta=0$. 
	
	Sea $\beta=(1-\alpha)^{-1}\in\Q[G]$.  Para cada $k$ definimos 
	\[
		\gamma_k=(1+\alpha+\cdots+\alpha^k)-\beta.
	\]
	Entonces 
	\begin{align*}
		\gamma_k(1-\alpha)&=(1+\alpha+\cdots+\alpha^k-\beta)(1-\alpha)\\
		&=(1+\alpha+\cdots+\alpha^k)(1-\alpha)-\beta(1-\alpha)=-\alpha^{k+1}
	\end{align*}
	y luego 
	$\gamma_k=-\alpha^{k+1}\beta$. Como 
	\[
		|\gamma_k|=|-\alpha^{k+1}\beta|\leq|\beta||\alpha^{k+1}|=\frac{|\beta|}{2^{k+1}},
	\]
	se concluye que $\lim_{k\to\infty}|\gamma_k|=0$. 

	Para terminar veamos que $H\subseteq\supp\beta$. Si
	$H\not\subseteq\supp\beta$, sea $h\in H\setminus\supp\beta$.  Supongamos
	que $h=x_{i_1}\cdots x_{i_m}$ es una palabra de longitud $m$ en los $x_j$.
	Sea $c_j$ el coeficiente de $h$ en $\alpha^j$. Entonces $c_0+\cdots+c_k$ es
	el coeficiente de $h$ en $\gamma_k$, pero
	\[
		|\gamma_k|\geq c_0+c_1+\cdots+c_k\geq c_m>0
	\]
	para todo $k\geq m$ pues cada $c_j$ es no negativo, una contradicción pues
	demostramos que $|\gamma_k|\to 0$ si $k\to\infty$.
\end{proof}

A continuación explicaremos por qué el teorema de Formanek se considera una
generalización del teorema de Herstein. En el teorema~\ref{thm:FormanekQ} nos
concentramos en álgebras de grupo sobre los números racionales. ¿Cómo podemos
extender este resultado a álgebras de grupo sobre cuerpos de característica
cero? Para extender el cuerpo de base sobre el que se trabaja necesitamos
definir el producto tensorial de espacios vectoriales y el producto tensorial
de álgebras.

\begin{definition}
	\index{Producto tensorial!de espacios vectoriales}
	El \textbf{producto tensorial} de los $K$-espacios vectoriales $U$ y $V$ es
	el espacio vectorial cociente $K[U\times V]/T$, donde $K[U\times V]$ es el
	espacio vectorial con base $\{(u,v):u\in U,v\in V\}$ y $T$ es el subespacio
	generado por los elementos de la forma
	\[
		(\lambda u+\mu u',v)-\lambda(u,v)-\mu(u',v),\quad
		(u,\lambda v+\mu v')-\lambda(u,v)-\mu(u,v')
	\]
	para $\lambda,\mu\in K$, $u,u'\in U$ y $v,v'\in V$.
\end{definition}

El producto tensorial de $U$ y $V$ será denotado por $U\otimes_KV$ o por
$U\otimes V$ si la referencia al cuerpo $K$ puede omitirse. Dados $u\in U$
y $v\in V$ escribiremos $u\otimes v$ para denotar a la coclase $(u,v)+T$.

\begin{theorem}
\index{Producto tensorial!propiedad universal}
	Sean $U$ y $V$ espacios vectoriales.  Existe entonces una función bilineal
	$U\times V\to U\otimes V$, $(u,v)\mapsto u\otimes v$, tal que todo
	elemento de $U\otimes V$ es una suma finita de la forma
	\[
		\sum_{i=1}^N u_i\otimes v_i
	\]
	para $u_1,\dots,u_N\in U$ y $v_1,\dots,v_N\in V$. 
	Más aún, dado un espacio vectorial $W$ y una función
	bilineal $\beta\colon U\times V\to W$, existe una función lineal
	$\overline{\beta}\colon U\otimes V\to W$ tal que $\overline{\beta}(u\otimes
	v)=\beta(u,v)$ para todo $u\in U$ y $v\in V$.
\end{theorem}

\begin{proof}
	Por la definición del producto tensorial, la función 
	\[
	U\times V\to U\otimes V,\quad
	(u,v)\mapsto u\otimes v,
	\]
	es bilineal. También de la definición se deduce inmediatamente que todo
	elemento de $U\otimes V$ es una combinación lineal finita de elementos de
	la forma $u\otimes v$, donde $u\in U$ y $v\in V$. Como $\lambda(u\otimes
	v)=(\lambda u)\otimes v$ para todo $\lambda\in K$, la primera afirmación
	queda demostrada.

	Como $U\times V$ es base de $K[U\times V]$, existe una transformación lineal 
	\[
		\gamma\colon K[U\times V]\to W,\quad
	\gamma(u,v)=\beta(u,v). 
	\]
	Como $\beta$ es bilineal por hipótesis, $T\subseteq\ker\gamma$. Existe
	entonces una transformación lineal $\overline{\beta}\colon U\otimes V\to
	W$ tal que 
	\[
	\begin{tikzcd}
		K[U\times V] \arrow[r]\arrow[d] & W \\
		U\otimes V\arrow[ur, dashrightarrow]
	\end{tikzcd}
	\]
	conmuta. En particular, $\overline{\beta}(u\otimes v)=\beta(u,v)$. 
\end{proof}

\begin{exercise}
	\label{xca:tensorial_unicidad}
	Demuestre que las propiedades mencionadas en el teorema anterior
	caracterizan el producto tensorial salvo isomorfismo.
\end{exercise}

Veamos algunas propiedades del producto tensorial de espacios vectoriales. 
%Observemos
%que todo elemento de $U\otimes V$ es una suma finita
%de la forma 
%\[
%	\sum_{i=1}^N u_i\otimes v_i
%\]
%para $N\in\N$, $u_i\in U$ y $v_i\in V$. Esta expresión no es única. Vale además
%que $u\otimes 0=0=0\otimes v$ para todo $u\in U$ y $v\in V$.

\begin{lemma}
	\index{Producto tensorial!de transformaciones lineales}
	Sean $\varphi\colon U\to U'$ y $\psi\colon V\to V'$ transformaciones
	lineales. Existe entonces una única transformación lineal
	$\varphi\otimes\psi\colon U\otimes V\to U'\otimes V'$ tal que
	\[
		(\varphi\otimes\psi)(u\otimes v)=\varphi(u)\otimes\psi(v)
	\]
	para todo $u\in U$ y $v\in V$.
\end{lemma}

\begin{proof}
	Como la función $U\times V\to U\otimes V$,
	$(u,v)\mapsto\varphi(u)\otimes\psi(v)$, es bilineal, existe una
	transformación lineal $U\otimes V\to U\otimes V$, $u\otimes
	v\to\varphi(u)\otimes\psi(v)$. Luego la función
	\[
		\sum u_i\otimes v_i\mapsto\sum\varphi(u_i)\otimes\psi(v_i)
	\]
	está bien definida. 
\end{proof}

\begin{exercise}
	Demuestre las siguientes afirmaciones:
	\begin{enumerate}
		\item $(\varphi\otimes\psi)(\varphi'\otimes\psi')=(\varphi\varphi')\otimes(\psi\psi')$.
		\item Si $\varphi$ y $\psi$ son isomorfismos, entonces
			$\varphi\otimes\psi$ es un isomorfismo. 
		\item $(\lambda\varphi+\lambda'\varphi')\otimes\psi=\lambda\varphi\otimes\psi+\lambda'\varphi'\otimes\psi$.
		\item $\varphi\otimes(\lambda\psi+\lambda'\psi')=\lambda\varphi\otimes\psi+\lambda'\varphi\otimes\psi'$.
		\item Si $U\simeq U'$ y $V\simeq V'$, entonces $U\otimes V\simeq U'\otimes V'$.
	\end{enumerate}
\end{exercise}

\begin{lemma}
	Si $U$ y $V$ son espacios vectoriales, entonces 
	$U\otimes V\simeq V\otimes U$.
\end{lemma}

\begin{proof}
	Como la función $U\times V\to V\otimes U$, $(u,v)\mapsto v\otimes u$,
	existe una transformación lineal $U\otimes V\to V\otimes U$, $u\otimes
	v\mapsto v\otimes u$. Similarmente se demuestra que existe una
	transformación lineal $V\otimes U\to U\otimes V$, $v\otimes u\mapsto
	u\otimes v$. Luego $U\otimes V\simeq V\otimes U$.
\end{proof}

\begin{exercise}
	\label{xca:UxVxW}
	Demuestre que $(U\otimes V)\otimes W\simeq U\otimes(V\otimes W)$.
\end{exercise}

\begin{exercise}
	\label{xca:UxK}
	Demuestre que $U\otimes K\simeq K\simeq K\otimes U$.
\end{exercise}

\begin{lemma}
	\label{lem:U_LI}
	Sea $\{u_1,\dots,u_n\}\subseteq U$ un conjunto linealmente independiente y
	sean $v_1,\dots,v_n\in V$ tales que $\sum_{i=1}^n u_i\otimes v_i=0$.
	Entonces $v_i=0$ para todo $i\in\{1,\dots,n\}$.
\end{lemma}

\begin{proof}
	Sea $i\in\{1,\dots,n\}$ y sea $f_i\colon U\to K$, $f_i(u_j)=\delta_{ij}$.
	Como la función $U\times V\to V$, $(u,v)\mapsto f_i(u)v$, es bilineal, existe una función
	$\alpha_i\colon U\otimes V\to V$ lineal tal que $\alpha_i(u\otimes
	v)=f_i(u)v$. Luego
	\[
		v_i=\sum_{j=1}^n\alpha_i(u_j\otimes v_j)=\alpha_i\left(\sum_{j=1}^nu_j\otimes v_j\right)=0.
	\]
\end{proof}

\begin{exercise}
	\label{xca:uxv=0}
	Demuestre que si $u\otimes v=0$ y $v\ne 0$, entonces $u=0$.
\end{exercise}

\begin{theorem}
	Si $\{u_i:i\in I\}$ es una base de $U$ y $\{v_j:j\in J\}$ es una base de
	$V$, entonces $\{u_i\otimes v_j:i\in I,j\in J\}$ es una base de $U\otimes
	V$.
\end{theorem}

\begin{proof}
	Los $u_i\otimes v_j$ forman un conjunto de generadores pues  
	si $u=\sum_i\lambda_iu_i$ y $v=\sum_j\mu_jv_j$, entonces
	$u\otimes v=\sum_{i,j}\lambda_i\mu_ju_i\otimes v_j$. 
	Veamos ahora que los $u_i\otimes v_j$ son linealmente independientes. Para
	eso, queremos ver que cualquier subconjunto finito de los $u_i\otimes v_j$
	es linealmente independiente. Si $\sum_k\sum_l\lambda_{kl}u_{i_k}\otimes
	v_{j_l}=0$, entonces
	$0=\sum_{k}u_{i_k}\otimes\left(\sum_{l}\lambda_{kl}v_{j_l}\right)$ y luego,
	como los $u_{i_k}$ son linealmente indepentientes, el lema~\ref{lem:U_LI}
	implica que $\sum_{l}\lambda_{kl}v_{j_l}=0$. Luego $\lambda_{kl}=0$ para
	todo $k,l$ pues los $v_{j_l}$ son linealmente independientes.
\end{proof}

El teorema anterior implica inmediatamente que si $U$ y $V$ son espacios
vectoriales de dimensión finita entonces
\[
	\dim(U\otimes V)=(\dim U)(\dim V).
\]

\begin{corollary}
	Si $\{u_i:i\in I\}$ es base de $U$, entonces todo elemento de $U\otimes V$
	se escribe unívocamente como una suma finita $\sum_{i}u_i\otimes v_i$.
\end{corollary}

\begin{proof}
	Sabemos que todo elemento de $U\otimes V$ es una suma finita
	$\sum_i x_i\otimes y_i$, donde $x_i\in U$ y $y_i\in V$. Si escribimos 
	$x_i=\sum_j\lambda_{ij}u_j$, entonces
	\[
		\sum_i x_i\otimes y_i=\sum_i\left(\sum_j\lambda_{ij}u_j\right)\otimes y_i		
		=\sum_j u_j\otimes\left(\sum_i\lambda_{ij}y_i\right).
	\]
\end{proof}

%\begin{corollary}
%	Todo elemento no nulo de $U\otimes V$ puede escribirse como una suma finita
%	$\sum_{i=1}^N u_i\otimes v_i$ para un conjuntos $\{u_i:1\leq i\leq
%	N\}\subseteq U$ y $\{v_i:1\leq i\leq N\}\subseteq V$ linealmente
%	independientes.
%\end{corollary}
%
%\begin{proof}
%	tomar $N$ minimal	
%\end{proof}

\index{Producto tensorial!de álgebras}
El siguiente lema nos permite definir el \textbf{producto tensorial de
álgebras}.

\begin{lemma}
	Si $A$ y $B$ son álgebras, entonces $A\otimes B$ es un álgebra con el
	producto
	\[
		(a\otimes b)(x\otimes y)=ax\otimes by.
	\]
\end{lemma}

\begin{proof}
	Para $x\in A$, $y\in B$ consideramos $R_x\otimes R_y\in\End_K(A\otimes B)$.
	Como la función $A\times B\to\End_K(A\otimes B)$, $(x,y)\mapsto R_x\otimes
	R_y$, es bilineal, existe una función lineal $\varphi\colon A\otimes
	B\to\End_K(A\otimes B)$, $\varphi(x\otimes y)=R_x\otimes R_y$. Para $u,v\in A\otimes B$ definimos
	\[
		uv=\varphi(v)(u).
	\]
	Esta operación es bilineal pues por ejemplo
	\[
		u(v+w)=\varphi(v+w)(u)=(\varphi(v)+\varphi(w))(u)=\varphi(v)(u)+\varphi(w)(u)=uv+uw.
	\]
	Además
	$(a\otimes b)(x\otimes y)=\varphi(x\otimes y)(a\otimes b)=(R_x\otimes R_y)(a\otimes b)=ax\otimes by$.
	Un cálculo sencillo muestra que este producto es asociativo.
\end{proof}

\begin{exercise}
	Demuestre que para álgebras valen las siguientes afirmaciones:
	\begin{enumerate}
		\item $A\otimes B\simeq B\otimes A$.
		\item $(A\otimes B)\otimes C\simeq A\otimes(B\otimes C)$.
		\item $A\otimes K\simeq A\simeq K\otimes A$.
		\item Si $A\otimes A'$ y $B\otimes B'$ entonces $A\otimes B\simeq A'\otimes B'$.
	\end{enumerate}
\end{exercise}

Veamos algunos ejemplos:

\begin{proposition}
	Si $G$ y $H$ son grupos, entonces $K[G]\otimes K[H]\simeq K[G\times H]$.
\end{proposition}

\begin{proof}
	Sabemos que $\{g\otimes h:g\in G,h\in H\}$ es una base de $K[G]\otimes K[H]$ y que
	$G\times H$ es una base de $K[G\times H]$. Tenemos entonces un isomorfismo lineal 
	\[
	K[G]\otimes K[H]\to K[G\times H], 
	\quad 
	g\otimes h\mapsto (g,h),
	\]
	que además es multiplicativo. Luego $K[G]\otimes K[H]\simeq K[G\times H]$
	como álgebras.
\end{proof}

\begin{proposition}
	Si $A$ es un álgebra, entonces $A\otimes K[X]\simeq A[X]$.	
\end{proposition}

\begin{proof}
	Todo elemento de $A\otimes K[X]$ se escribe unívocamente como una suma
	finita de la forma $\sum a_i\otimes X^i$. Un cálculo sencillo muestra que
	$A\otimes K[X]\mapsto A[X]$, $\sum a_i\otimes X^i\mapsto \sum a_iX^i$, es
	un isomorfismo de álgebras.
\end{proof}

\begin{exercise}
	Demuestre que si $A$ es un álgebra, $A\otimes M_n(K)\simeq M_n(A)$. En
	particular, $M_n(K)\otimes M_m(K)\simeq M_{nm}(K)$.
\end{exercise}

\index{Extensión de escalares}
Estos últimos dos ejemplos son casos particulares de una construcción
importante que involucra productos tensoriales y se conoce como
\textbf{extensión de escalares}.

\begin{theorem}
	Sea $A$ un álgebra sobre $K$ y sea $E$ una extensión de $K$. Entonces
	$A^E=E\otimes_KA$ es un álgebra sobre $E$ con respecto a la multiplicación
	por escalares dada por
	\[
		\lambda(\mu\otimes a)=(\lambda\mu)\otimes a,
	\]
	para $\lambda,\mu\in E$ y $a\in A$.
\end{theorem}

\begin{proof}
	Sea $\lambda\in E$. Como la función $E\times A\to E\otimes_KA$,
	$(\mu,a)\mapsto (\lambda\mu)\otimes a$, es $K$-bilineal, existe una
	transformación lineal $E\otimes_KA\to E\otimes_KA$, $\mu\otimes a\mapsto
	(\lambda\mu)\otimes a$. Queda bien definida entonces la multiplicación por
	escalares y además 
	\[
	\lambda(u+v)=\lambda u+\lambda v
	\]
	para $\lambda\in E$ y $u,v\in E\otimes_KA$. Un cálculo directo muestra que además 
	\[
	(\lambda+\mu)u=\lambda u+\mu u,
	\quad
	(\lambda\mu)u=\lambda(\mu u),
	\quad
	\lambda(uv)=(\lambda u)v=u(\lambda v)
	\]
	valen para todo $u,v\in E\otimes_KA$ y $\lambda,\mu\in E$.
\end{proof}

\begin{exercise}
	Demuestre que valen las siguientes afirmaciones:
	\begin{enumerate}
		\item $1\otimes A$ es una subálgebra de $A^E$ isomorfa a $A$.
		\item Si $\{a_i:i\in I\}$ es base de $A$, entonces $\{1\otimes a_i:i\in
			I\}$ es base de $A^E$.
	\end{enumerate}
\end{exercise}

\begin{exercise}
	Demuestre que si $G$ es un grupo y $K$ es un subcuerpo de $E$, entonces
	$E\otimes_K K[G]\simeq E[G]$.
\end{exercise}

Estamos en condiciones de demostrar el teorema de Formanek:

\begin{theorem}[Formanek]
	\index{Teorema!de Formanek}
	Sea $K$ un cuerpo de característica cero y sea $G$ un grupo. Si todo
	elemento de $K[G]$ es inversible o un divisor de cero, entonces $G$ es
	localmente finito.
\end{theorem}

\begin{proof}
	Como $K$ es de característica cero, $\Q\subseteq K$ y $K[G]\simeq
	K\otimes_{\Q}\Q[G]$. Todo $\beta\in K\otimes_{\Q}\Q[Q]$ se escribe
	unívocamente como 
	\[
		\beta=1\otimes\beta_0+\sum k_i\otimes\beta_i,
	\]
	donde $\{1,k_1,k_2,\dots,\}$ es una base de $K$ como $\Q$-espacio
	vectorial. Sea $\alpha\in\Q[G]$ y sea $\beta\in K[G]$ tal que $\alpha\beta=1$. Como entonces 
	\[
	1\otimes 1=(1\otimes\alpha)\beta=1\otimes \alpha\beta_0+\sum k_i\otimes \alpha\beta_i,
	\]
	la unicidad de la escritura nos dice que $\alpha\beta_0=1$. De la misma
	forma, si $\alpha\beta=0$, entonces $\alpha\beta_j=0$ para todo $j$. Luego,
	como todo $\alpha\in\Q[G]$ es inversible o un divisor de cero, el resultado
	se obtiene al usar el teorema~\ref{thm:FormanekQ} de Formanek para $\Q$.
\end{proof}

% \section*{Rickart's theorem}

% En esta sección vamos a demostrar que para cualquier grupo $G$ el radical de
% Jacobson de $\C[G]$ es cero. Demostraremos también que el radical de Jacobson
% de $\R[G]$ es cero.

% \begin{definition}
% 	\index{Anillo!con involución}
% 	\index{Involución!de un anillo}
% 	Sea $R$ un anillo. Una \textbf{involución} del anillo $R$ es un morfismo
% 	aditivo $R\to R$, $x\mapsto x^*$, tal que $x^{**}=x$ y $(xy)^*=y^*x^*$ para
% 	todo $x,y\in R$.
% \end{definition}

% De la definición se deduce inmediatamente que si $R$ es unitario, entonces
% $1^*=1$.

% \begin{example}
% 	La conjugación $z\mapsto\overline{z}$ es una involución de $\C$.
% \end{example}

% \begin{example}
% 	La trasposición $X\mapsto X^T$ es una involución del
% 	anillo $M_n(K)$.
% \end{example}

% \begin{example}
% 	Sea $G$ un grupo. Entonces
% 	$\left(\sum_{g\in G}\alpha_gg\right)^*=\sum_{g\in G}\overline{\alpha_g}g^{-1}$ 
% 	es una involución de $\C[G]$.
% \end{example}

% Dado un grupo $G$, se define la \textbf{traza} de un elemento
% $\alpha=\sum_{g\in G}\alpha_gg\in K[G]$ como $\trace(\alpha)=\alpha_1$. Es
% fácil ver que $\trace\colon K[G]\to K$, $\alpha\mapsto\trace(\alpha)$ es una
% función $K$-lineal tal que $\trace(\alpha\beta)=\trace(\beta\alpha)$.

% \begin{exercise}
% 	Sea $G$ un grupo finito y $K$ un cuerpo tal que su característica no divide al orden de $G$.
% 	Demuestre las siguientes afirmaciones:
% 	\begin{enumerate}
% 		\item Si $\alpha\in K[G]$ es nilpotente, entonces $\trace(\alpha)=0$.
% 		\item Si $\alpha\in K[G]$ es idempotente, entonces $\trace(\alpha)=\dim
% 			K[G]\alpha/|G|$.
% 	\end{enumerate}
% \end{exercise}

% \begin{exercise}
% 	Demuestre que 
% 	$\langle\alpha,\beta\rangle=\trace(\alpha\beta^*)$, $\alpha,\beta\in\C[G]$, 
% 	define un producto interno en $\C[G]$.
% \end{exercise}

% \begin{lemma}
% 	\label{lem:algebraico}
% 	Sea $G$ un grupo. Si $J(\C[G])\ne 0$, entonces existe $\alpha\in J(\C[G])$ tal que 
% 	$\trace(\alpha^{2^m})\in\R_{\geq1}$ 
% 	para todo $m\geq1$.
% \end{lemma}

% \begin{proof}
% 	Sea $\alpha=\sum_{g\in G}\alpha_gg\in\C[G]$. Entonces	
% 	\[
% 		\trace(\alpha^*\alpha)
% 		=\sum_{g\in G}\overline{\alpha_g}\alpha_g
% 		=\sum_{g\in G}|\alpha_g|^2\geq|\alpha_1|^2
% 		=|\trace(\alpha)|^2.
% 	\]
% 	Al usar esta fórmula para algún $\alpha$ tal que $\alpha^*=\alpha$ y usar
% 	inducción se obtiene que $\trace(\alpha^{2^m})\geq|\trace(\alpha)|^{2^m}$
% 	para todo $m\geq1$. 

% 	Sea $\beta=\sum_{g\in G}\beta_gg\in J(\C[G])$ tal que $\beta\ne0$. Como
% 	$\trace(\beta^*\beta)=\sum_{g\in G}|\beta_g|^2\ne0$ y $J(\C[G])$ es un ideal, 
% 	\[
% 		\alpha=\frac{\beta^*\beta}{\trace(\beta^*\beta)}\in J(\C[G]).
% 	\]
% 	Este elemento $\alpha$ cumple que $\alpha^*=\alpha$ y $\trace(\alpha)=1$.
% 	Luego $\trace(\alpha^{2^m})\geq 1$ para todo $m\geq1$.
% \end{proof}

% El ejercicio~\ref{exa:norma} implica que $\C[G]$ con
% $\dist(\alpha,\beta)=|\alpha-\beta|$ es un espacio métrico. En este espacio
% métrico, la función $\C[G]\to\C$, $\alpha\mapsto \trace(\alpha)$, es una
% función continua.

% \begin{lemma}
% 	\label{lem:phi_diferenciable}
% 	Sea $\alpha\in J(\C[G])$. La función
% 	\[
% 		\varphi\colon\C\to\C[G],\quad
% 		\varphi(z)=(1-z\alpha)^{-1},
% 	\]
% 	es continua, diferenciable y $\varphi(z)=\sum_{n\geq0}\alpha^nz^n\in\C[G]$ si $|z|$
% 	es suficientemente pequeño.
% \end{lemma}

% \begin{proof}	
% 	Sean $y,z\in\C$. Como $\varphi(y)$ y $\varphi(z)$ conmutan, 
% 	\begin{equation}
% 		\label{eq:Rickart}
% 		\begin{aligned}
% 			\varphi(y)-\varphi(z)&=\left( (1-z\alpha)-(1-y\alpha)\right)(1-y\alpha)^{-1}(1-z\alpha)^{-1}\\
% 			&=(y-z)\alpha\varphi(y)\varphi(z).
% 		\end{aligned}
% 	\end{equation}
% 	Entonces $|\varphi(y)|\leq|\varphi(z)|+|y-z||\alpha\varphi(y)||\varphi(z)|$ y luego
% 	\[
% 		|\varphi(y)|\left( 1-|y-z||\alpha\varphi(z)|\right)\leq|\varphi(z)|.
% 	\]
% 	Fijado $z$ podemos elegir $y$ suficientemente cerca de $z$ de forma tal que
% 	se cumpla que  $1-|y-z||\alpha\varphi(z)|\geq1/2$. Luego
% 	$|\varphi(y)|\leq2|\varphi(z)|$. De la igualdad~\eqref{eq:Rickart} se
% 	obtiene entonces $|\varphi(y)-\varphi(z)|\leq2|y-z||\alpha||\varphi(z)|^2$
% 	y luego $\varphi$ es una función continua. Por la
% 	expresión~\eqref{eq:Rickart}, 
% 	\[
% 	\varphi'(z)
% 	=\lim_{y\to z}\frac{\varphi(y)-\varphi(z)}{y-z}
% 	=\lim_{y\to z}\alpha\varphi(y)\varphi(z)
% 	=\alpha\varphi(z)^2
% 	\]
% 	para todo $z\in\C$.

% 	Si $z$ es tal que $|z||\alpha|=|z\alpha|<1$, entonces 
% 	\[
% 		\varphi(z)-\sum_{n=0}^Nz^n\alpha^n
% 		=\varphi(z)\left(1-(1-z\alpha)\sum_{n=0}^Nz^n\alpha^n\right)
% 		=\varphi(z)(z\alpha)^{N+1}
% 	\]
% 	y luego
% 	\[
% 		\left|\varphi(z)-\sum_{n=0}^Nz^n\alpha^n\right|\leq|\varphi(z)||z\alpha|^{N+1}.
% 	\]
% 	Como $\varphi(z)$ está acotada cerca de $z=0$, se concluye que
% 	$\left|\varphi(z)-\sum_{n=0}^Nz^n\alpha^n\right|\to0$ si $N\to\infty$.
% \end{proof}

% Estamos en condiciones de demostrar el teorema de Rickart:

% \begin{theorem}[Rickart]
% 	\index{Teorema!de Rickart}
% 	Si $G$ es un grupo, entonces $J(\C[G])=0$.
% \end{theorem}

% \begin{proof}
% 	Sea $\alpha\in J(\C[G])$ y sea $\varphi(z)=(1-\alpha z)^{-1}$. Sea 
% 	$f\colon\C\to \C$ dada por
% 	$f(z)=\trace\varphi(z)=\trace\left((1-z\alpha)^{-1}\right)$. Por el lema~\ref{lem:phi_diferenciable},
% 	$f(z)$ es una función entera tal que $f'(z)=\trace(\alpha\varphi(z)^2)$ y
% 	\begin{equation}
% 		\label{eq:Taylor}
% 		f(z)=\sum_{n=0}^\infty z^n\trace(\alpha^n)
% 	\end{equation}
% 	si $|z|$ es suficientemente pequeño. En particular, la
% 	igualdad~\eqref{eq:Taylor} es la expansión en serie de Taylor para $f(z)$
% 	en el origen. Esto implica que esta serie tiene radio de convergencia
% 	infinito y converge a $f(z)$ para todo $z\in\C$. En particular,
% 	\begin{equation}
% 		\label{eq:limite}
% 		\lim_{n\to\infty}\trace(\alpha^n)=0.
% 	\end{equation}
% 	Por otro lado, si $\alpha\ne0$ el lema~\ref{lem:algebraico} implica que
% 	$\trace(\alpha^{2^m})\geq1$ para todo $m\geq0$, lo que contradice el límite
% 	calculado en~\eqref{eq:limite}. Luego $\alpha=0$.
% \end{proof}

% Para demostrar un corolario necesitamos dos lemas:

% \begin{lemma}[Nakayama]
% 	\label{lem:Nakayama}
% 	\index{Lema!de Nakayama}
% 	Sea $R$ un anillo unitario y sea $M$ un $R$-módulo finitamente generado. Si
% 	$J(R)M=M$, entonces $M=0$.
% \end{lemma}

% \begin{proof}
% 	Supongamos que $M$ está generado por los elementos $x_1,\dots,x_n$. Como $x_n\in M=J(R)M$, 
% 	existen $r_1,\dots,r_n\in J(R)$ tales que $x_n=r_1x_1+\cdots+r_nx_n$, es decir
% 	$(1-r_n)x_n=\sum_{j=1}^{n-1}r_jx_j$. 
% 	Como $1-r_n$ es inversible, existe $s\in R$ tal que $s(1-r_n)=1$. Luego
% 	$x_n=\sum_{j=1}^{n-1}sr_jx_j$ 
% 	y entonces $M$ está generado por $x_1,\dots,x_{n-1}$. Al repetir este
% 	procedimiento una cierta cantidad finita de veces, se obtiene que $M=0$.
% \end{proof}

% \begin{lemma}
% 	\label{lem:Rickart}
% 	Sea $\iota\colon R\to S$ un morfismo de anillos unitarios. Si 
% 	\[
% 	S=\iota(R)x_1+\cdots+\iota(R)x_n,
% 	\]
% 	donde cada $x_j$ cumple que $x_jy=yx_j$ para todo $y\in\iota(R)$, entonces
% 	$\iota(J(R))\subseteq J(S)$.
% \end{lemma}

% \begin{proof}
% 	Veamos que $J=\iota(J(R))$ actúa trivialmente en cada $S$-módulo simple $M$.
% 	Si $M$ es un $S$-módulo simple, escribimos $M=Sm$ para algún $m\ne0$. Es
% 	claro que $M$ es un $R$-módulo con $r\cdot m=\iota(r)m$. Como
% 	\[
% 		M=Sm=(\iota(R)x_1+\cdots+\iota(R)x_n)m=\iota(R)(x_1m)+\cdots+\iota(R)(x_nm),
% 	\]
% 	$M$ es finitamente generado como $\iota(R)$-módulo. Además $J(R)\cdot
% 	M=JM=\iota(J)M$ es un $S$-submódulo de $M$ pues
% 	\[
% 		x_j(JM)=(x_jJ)M=(Jx_j)M=J(x_jM)\subseteq JM.
% 	\]
% 	Como $M\ne0$, el lema de Nakayama implica que $J(R)\cdot M\subsetneq M$. Luego,
% 	como $M$ es un $S$-módulo simple, se concluye que $J(R)M=0$.
% \end{proof}

% \begin{corollary}
% 	Si $G$ es un grupo, entonces $J(\R[G])=0$. 
% \end{corollary}

% \begin{proof}
% 	Sea $\iota\colon \R[G]\to\C[G]$ la inclusión canónica. Como 
% 	\[
% 	\C[G]=\R[G]+i\R[G],
% 	\]
% 	el lema~\ref{lem:Rickart} y el teorema de Rickart implican que
% 	$\iota(J(\R[G]))\subseteq J(\C[G])=0$. Luego $J(\R[G])=0$ pues $\iota$ es
% 	inyectiva. 
% \end{proof}




%\section{Anillos semiprimitivos y semiprimos}

\begin{definition}
	\index{Anillo!semiprimitivo}
	\index{Anillo!semisimple Jacobson}
	Un anillo $R$ se dice \textbf{semiprimitivo} (o semisimple Jacobson) si
	$J(R)=0$.
\end{definition}

\begin{example}
	Si $R$ es primitivo entonces es semiprimitivo. En efecto, como $R$ es
	primitivo, $\{0\}$ es un ideal primitivo y luego, como $J(R)$ es la
	intersección de los ideales primitivos de $R$, se concluye que $J(R)=0$.
\end{example}

\begin{example}
	Si $R=\prod_{i\in I}R_i$ es producto directo de anillos semiprimitivos,
	entonces $R$ es semiprimitivo pues 
	\[
		J(R)=J\left(\prod_{i\in I}R_i\right)=J\left(\prod_{i\in I}J(R_i)\right)=0.
	\]
\end{example}

\begin{example}
	$\Z$ es semiprimitivo pues $J(\Z)=\cap_{p}\Z/p=\{0\}$.
\end{example}

\begin{example}
	Sea $R=C[a,b]$ el anillo de funciones $f\colon [a,b]\to\R$ continuas. Como
	$R$ es un anillo unitario, $J(R)$ es la intersección de los ideales
	maximales de $R$. Todo ideal maximal de $R$ es de la forma
	\[
		U_c=\{f\in C[a,b]:f(c)=0\}
	\]
	para algún $c\in[a,b]$. En efecto, es fácil ver que cada $U_c$ es un ideal;
	$U_c$ es maximal pues $C[a,b]/U_c\simeq\R$.  Luego $J(R)=\cap_{a\leq c\leq
	b}U_c=0$.
\end{example}

\begin{theorem}
	\label{thm:semiprimitivo}
	Si $R$ es un anillo, entonces $R/J(R)$ es semiprimitivo. 
\end{theorem}

\begin{proof}
	Si $R$ es un anillo radical, el resultado es trivial. Supongamos entonces
	que $J(R)\ne R$ y sea $M$ un módulo simple. Entonces $M$ es un
	$R/J(R)$-módulo simple con
	\[
		(x+J(R))m=xm,\quad
		x\in R,\,m\in M.
	\]
	Si $x+J(R)\in J(R/J(R))$ entonces $xM=(x+J(R))M=0$. Luego $x\in J(R)$ pues
	$x$ anula a cualquier módulo simple de $R$.
\end{proof}

%El teorema de densidad de Jacobson nos permite entonces obtener el siguiente resultado:
%
%\begin{theorem}
%	Sea $R$ un anillo no radical. Entonces $R/J(R)$ es isomorfo a un producto
%	subdirecto de anillos densos en espacios vectoriales sobre anillos de
%	división.	
%\end{theorem}
%
%\begin{proof}
%	Si $R$ no es radical, $J(R)\ne R$. Luego $R/J(R))$ es semiprimitivo por el
%	teorema~\ref{thm:semiprimitivo}. El teorema~\ref{thm:subdirecto} y el
%	teorema de densidad de Jacobson completan la demostración del teorema.
%\end{proof}


\begin{definition}
	\index{Producto subdirecto de anillos}
	Sea $\{R_i:i\in I\}$ una familia de anillos. Un subanillo $R$ de
	$\prod_{i\in I}R_i$ se dice un \textbf{producto subdirecto} de los $R_j$ si
	cada $\pi_j\colon R\to R_j$ es sobreyectiva. 
\end{definition}

El siguiente teorema justifica que indistintamente llamemos anillos
semiprimitivos a los anillos semisimples Jacobson:

\begin{theorem}
	\label{thm:subdirecto}
	Sea $R$ un anillo no nulo. Entonces $R$ semiprimitivo si y sólo si $R$ es
	isomorfo a un producto subdirecto de anillos primitivos.
\end{theorem}

\begin{proof}
	Supongamos que $R$ es semiprimitivo y sea $\{P_i:i\in I\}$ la familia de
	ideales primitivos de $R$. Cada $R/P_j$ es primitivo y
	$\{0\}=J(R)=\cap_{i\in I}P_i$. Para cada $j$, sean $\lambda_j\colon R\to
	R/P_j$ y $\pi_j\colon \prod_{i\in I}R/P_i\to R/P_j$ los morfismos
	canónicos. La función
	\[
		\phi\colon R\to\prod_{i\in I}R/P_i,\quad
		r\mapsto \{\lambda_i(r):i\in I\},
	\]
	es un morfismo inyectivo de anillos tal que $\pi_j\phi(R)=R/P_j$ para todo
	$j$.

	Supongamos ahora que $R$ es isomorfo a un producto subrirecto de anillos
	$R_j$ primitivos y sea $\varphi\colon R\to\prod_{i\in I}R_i$ un morfismo
	inyectivo tal que $\pi_j(\varphi(R))=R_j$ para todo $j$. Para cada $j$ sea
	$P_j=\ker\pi_j\varphi$. Como $R/P_j\simeq R_j$, cada $P_j$ es un ideal
	primitivo. Si $x\in\cap_{i\in I}P_i$ entonces $\varphi(x)=0$ y luego $x=0$.
	Luego $J(R)\subseteq\cap_{i\in I} P_i=0$. 
\end{proof}

\begin{example}
	El anillo $\Z$ es isomorfo a un producto subdirecto de los cuerpos $\Z/p$
	con $p$ primo.
\end{example}

\begin{example}
	El anillo $C[a,b]$ es isomorfo a un producto subdirecto de los cuerpos
	$C[a,b]/U_c\simeq\R$.
\end{example}

\begin{definition}
	Un anillo $R$ se dice \textbf{semiprimo} si para todo $a\in R$ tal que
	$aRa=0$ se tiene que $a=0$.
\end{definition}

\begin{lemma}
	Sea $R$ un anillo. Son equivalentes:
	\begin{enumerate}
		\item $R$ es semiprimo.
		\item Si $I$ es un ideal a izquierda tal que $I^2=0$ entonces $I=0$.
		\item Si $I$ es un ideal tal que $I^2=0$ entonces $I=0$.
		\item $R$ no tiene ideales nilpotentes no nulos. 
	\end{enumerate}
\end{lemma}

\begin{proof}
	Veamos que $(1)\implies(2)$. Si $I^2=0$ y $x\in I$, entonces $xRx\subseteq I^2=0$ y
	luego $x=0$. Las implicaciones $(2)\implies(3)$ y $(4)\implies(3)$ son triviales. Veamos que
	$(3)\implies(4)$.  Si $I$ es un ideal nilpotente no nulo, sea $n\in\N$
	minimal tal que $I^n=0$.  Como $(I^{n-1})^2=0$, $I^{n-1}=0$, una
	contradicción. Por último veamos que $(3)\implies(1)$. Sea $a\in R$ tal que
	$aRa=0$. Entonces $I=RaR$ es un ideal de $R$ tal que $I^2=0$. Por hipótesis, $RaR=I=0$. Luego
	$Ra$ y $aR$ son ideales tales que $(Ra)R=R(aR)=0$. Esto implica que $\Z a$ es un ideal de $R$
	tal que $(\Z a)R=0$ y luego $a=0$.
\end{proof}

\begin{example}
	Un anillo conmutativo es semiprimo si y sólo si no tiene elementos
	nilpotentes no nulos.
\end{example}


\begin{proposition}
	El anillo $\C[G]$ es semiprimo.
\end{proposition}

\begin{proof}
	Como $J(\C[G])=0$ por el teorema de Rickart y además el radical de Jacboson
	contiene a todo ideal nil por la proposición~\ref{pro:nilJ}, se deduce que
	$\C[G]$ no tiene ideales nil no triviales. Tampoco tiene entonces ideales
	nilpotentes no triviales y luego $\C[G]$ es semiprimo.
\end{proof}

\begin{exercise}
	Demuestre que $Z(\C[G])$ es semiprimo.
\end{exercise}

% tomar $\alpha$ tal que $\alpha^2=0$ y sea $A=K[G]\alpha$. Como $A^2=0$, $A=0$ y entonces $\alpha=0$.

\begin{example}
	Sea $D$ un anillo de división. Entonces $D[X]$ es semiprimo.
\end{example}

\begin{example}
	Sea $D$ un anillo de división. Entonces $D[[X]]$ es semiprimo y no es
	semiprimitivo.
\end{example}




%\section{Anillos semiprimitivos}
%
%\begin{lemma}
%	\label{lem:Iunitario}
%	Sea $R$ un anillo y sea $I$ un ideal de $R$ unitario. Sea $e\in I$ la
%	unidad de $I$. Entonces $e$ es un idempotente central de $R$, $I=eR$ y
%	existe un ideal $J$ de $R$ tal que $R=I\oplus J$. Además $R\simeq I\times
%	J$.
%\end{lemma}
%
%\begin{proof}
%	Como $e\in I$, $eR\subseteq I$. Luego $I=eR$ pues $I=eI\subseteq eR$. Como
%	$ex\in I$ y $xe\in I$ para todo $x\in R$, $ex=(ex)e$ y $xe=e(xe)$. Luego
%	$ex=xe$ y entonces $e$ es central e idempotente. Sea $J=\{x-ex:x\in R\}$.
%	Es fácil demostrar que $J$ es un ideal tal que $R=I\oplus J$. Además
%	$R\simeq I\times J$, via $x\mapsto (ex,x-ex)$,
%\end{proof}
%
%A continuación daremos una demostración muy sencilla del teorema de Wedderburn
%en el caso de álgebras de dimensión finita.
%
%\begin{theorem}[Artin--Wedderburn]
%	Sea $R$ un anillo artiniano a izquierda y no nulo. Entonces $R$ es
%	semiprimo si y sólo si existen $n_1,\dots,n_r\in\N$ y existen anillos de
%	división $D_1,\dots,D_r$ tales que $R\simeq M_{n_1}(D_1)\times\cdots\times
%	M_{n_r}(D_r)$.
%\end{theorem}
%
%\begin{proof}
%	Procederemos por inducción en $\dim A$. Si $\dim A=1$\dots\framebox{} 
%
%	Supongamos entonces que $\dim A>1$. Si $A$ es un álgebra prima, el
%	resultado se sigue inmediatamente del teorema de Wedderburn. Supongamos
%	entonces que existe $a\in A\setminus\{0\}$ tal que $I=\{x\in A:aAx=0\}$ es
%	no nulo. Como $I$ es un ideal de $A$, $I$ es un álgebra semiprima.
%	\framebox{?} Como $a\not\in I$, $\dim I<\dim A$, y entonces, por hipótesis
%	inductiva, existen $n_1,\dots,n_s\in\N$ y álgebras de división
%	$D_1,\dots,D_s$ tales que 
%	\[
%		I\simeq M_{n_1}(D_1)\times\cdots\times M_{n_s}(D_s).
%	\]
%	En particular, $I$ es unitario. Por el lema~\ref{lem:Iunitario}, existe un
%	ideal $J$ de $A$ tal que $A\simeq I\times J$. Como $\dim J<\dim A$, la hipótesis inductiva
%	implica que existen $n_{s+1},\dots,n_r\in\N$ y álgebras de división $D_{s+1},\dots,D_r$ tales que
%	\[
%		J\simeq M_{n_{s+1}}(D_{s+1})\times\cdots\times M_{n_r}(D_r).
%	\]
%	Luego $A\simeq I\times J\simeq \prod_{j=1}^s M_{n_j}(D_j)$.
%\end{proof}
%
%\begin{corollary}
%	Sea $A$ un álgebra no nula de dimensión finita. Si $A$ es semiprima,
%	entonces $A$ es unitaria.
%\end{corollary}
%
%%Gracias al teorema de Wedderburn se puede ir un poco más lejos:
%%\begin{corollary}
%%	Sea $A$ un álgebra unitaria. Son equivalentes:
%%	\begin{enumerate}
%%		\item $A$ es semiprima.
%%		\item Todo $A$-módulo unitario es semisimple.
%%		\item $A$ es semisimple como $A$-módulo.
%%		\item Todo ideal a izquierda de $A$ es de la forma $Ae$ para algún
%%			idempotente $e\in A$. 
%%	\end{enumerate}
%%\end{corollary}
%%
%%\begin{proof}
%%	La implicación $(1)\implies(2)$ es el teorema de Wedderburn. 
%%	
%%\end{proof}
%
%\begin{example}
%	Por el teorema de Maschke sabemos que si $G$ es un grupo finito, 
%	$\C[G]$ es un álgebra semiprimitiva y luego semisimple.
%\end{example}
%




%\section{Viejo!}
%
%\begin{theorem}[Artin--Wedderburn]
%	\index{Teorema!de Artin--Wedderburn}
%	\label{thm:ArtinWedderburn}
%	Si $R$ es un anillo, las siguientes afirmaciones son equivalentes:
%	\begin{enumerate}
%		\item $R$ es un anillo no nulo semiprimitivo y artiniano a izquierda.
%		\item Existen anillos de división $D_1,\dots,D_r$ y tales que
%			\[
%				R\simeq\prod_{i=1}^r R_i,
%			\]
%			donde $R_i=\End_{D_i}(V_i)$
%		\item Existen anillos de división $D_1,\dots,D_r$ y enteros positivos
%			$n_1,\dots,n_r$ tales que 
%			\[
%			R\simeq M_{n_1}(D_1)\times\cdots\times M_{n_r}(D_r).
%		\]
%	\end{enumerate}
%\end{theorem}
%
%\begin{proof}
%	Demostremos que $(1)\implies(2)$. Como $R\ne0$ y $J(R)=0$, $R$ admite
%	ideales primitivos. Supongamos que existe un número finito de ideales
%	primitivos distintos, digamos $P_1,\dots,P_t$. Cada $R/P_j$ es un anillo
%	primitivo y es artiniano a izquierda \framebox{?}. Entonces, por el teorema
%	de Wedderburn, para cada $j\in\{1,\dots,t\}$ existen un anillo de división
%	$D_j$ y un entero positivo $n_j$ tales que $R/P_j\simeq M_{n_j}(D_j)$. En
%	particular, cada $R/P_j$ es simple y entonces $P_j$ es un ideal maximal de
%	$R$. Como $R/P_j$ es simple, $(R/P_j)^2\ne 0$ y luego $R^2\not\subseteq
%	P_j$. Por maximalidad, $R^2+P_j=R$ y además $P_i+P_j=R$ para todo $i\ne j$.
%	Por el teorema chino del resto,
%	\[
%		R=R/0=R/J(R)=R/\cap_{j=1}^t P_j\simeq R/P_1\times\cdots\times R/P_t.
%	\]
%	Sea $\iota_k\colon R/P_k\to \prod_{j=1}^t R/P_j$ la inclusión canónica.
%	Cada $\iota_k(R/P_k)$ es un ideal simple (es decir, que como anillo es
%	simple) de $\prod_{j=1}^t R/P_j\simeq R$. Luego las imágenes, digamos
%	$I_k$, de los $\iota_k(R/P_k)$ dan ideales simples de $R$ y
%	$R=I_1\times\cdots\times I_t$.
%
%	Demostremos ahora que $(3)\implies(1)$. Para cada $j$ sea
%	$R_j=M_{n_j}(D_j)$. Como cada $R_j$ es primitivo por el teorema de
%	Wedderburn, $J(R_j)=\{0\}$ para todo $j$. Luego
%	$J(R)=\prod_{i=1}^rJ(R_j)=\{0\}$ y entonces $R$ es semiprimitivo. Además
%	$R$ es artiniano a izquierda.\framebox{?}
%\end{proof}
%
%\begin{corollary}
%	Sea $R$ un anillo semiprimitivo.
%	\begin{enumerate}
%		\item Si $R$ es artiniano a izquierda, entonces $R$ es unitario.
%		\item $R$ es artiniano a izquierda si y sólo si es artiniano a derecha.
%		\item Si $R$ es artiniano a izquierda es noetheriano.
%	\end{enumerate}
%\end{corollary}
%
%\begin{proof}
%	La primera afirmación es consecuencia inmediata del teorema de
%	Artin--Wedderburn~\ref{thm:ArtinWedderburn}.
%\end{proof}
%
%\begin{corollary}
%	Sea $R$ un anillo semiprimitivo artiniano a izquierda y sea $I$ un ideal de
%	$R$. Entonces $I=Re$ para algún idempotente $e\in R$ tal que $e\in Z(R)$.
%\end{corollary}
%
%\begin{proof}
%		
%\end{proof}
%
%\begin{proposition}
%	Sea $R$ un anillo semisimple artiniano a izquierda. 
%	\begin{enumerate}
%		\item $R=I_1\times\cdots\times I_n$ donde los $I_j$ son ideales simples.
%		\item Si $J\subseteq R$ es un ideal simple, entonces existe $k\in\{1,\dots,n\}$ tal que $J=I_k$.
%		\item Si $R=J_1\times\cdots\times J_m$ donde los $J_k$ son ideales simples, entonces $n=m$ y existe
%			$\sigma\in\Sym_n$ tal que $I_k=J_{\sigma(k)}$ para todo $k\in\{1,\dots,n\}$.
%	\end{enumerate}
%\end{proposition}
%
%\begin{proof}
%\end{proof}
%
%\begin{theorem}
%	Sea $R$ un anillo unitario no nulo. Las siguientes afirmaciones son
%	equivalentes:
%	\begin{enumerate}
%		\item $R$ es semiprimitivo y artiniano a izquierda.
%		\item Todo $R$-módulo unitario es proyectivo.
%		\item Todo $R$-módulo unitario es inyectivo.
%		\item Toda sucesión exacta de $R$-módulos unitarios se parte.
%		\item Todo $R$-módulo unitario no nulo es semisimple.
%		\item $\prescript{}{R}R$ es unitario y semisimple.
%		\item Todo ideal a izquierda de $R$ es de la forma $Re$ para algún $e\in R$ indempotente.
%		\item $\prescript{}{R}R$ es suma directa de ideales a izquierda
%			minimales $L_1,\dots,L_m$ donde cada $L_j$ es de la forma $Re_j$, y
%			los $e_j$ son idempotentes ortogonales tales que
%			$e_1+\cdots+e_m=1$. 
%	\end{enumerate}
%\end{theorem}
%
%\begin{proof}
%	Veamos que $(4)\implies(5)$. Sea $M$ un módulo unitario y sea $N$ un
%	submódulo no nulo de $M$. Como la sucesión $0\to N\to M\to M/N\to 0$ es
%	exacta, se parte. Luego $M=N\oplus X$ para algún submódulo $X$ de $N$ tal
%	que $X\simeq M/N$. Como $M$ es unitario, $Rm\ne 0$ para todo $m\in
%	M\setminus\{0\}$. Luego $M$ es semisimple por el teorema~\framebox{?}.
%
%	Veamos ahora que $(5)\implies(4)$. Sea 
%	\[
%	\begin{tikzcd}
%		0 \arrow[r]
%		& N \arrow[r]
%		& M \arrow[r]
%		& X \arrow[r]
%		& 0
%	\end{tikzcd}
%	\]
%	una sucesión exacta corta de $R$-módulos. Como $f\colon N\to f(N)$ es un
%	isomorfismo y entonces $f(N)$ es un submódulo del semisimple $M$, $f(N)$ es
%	sumando directo de $M$. Sea $\pi\colon M\to f(N)$ el morfismo canónico.
%	Entonces $\pi f=f$ y $f^{-1}\pi\colon M\to A$ es un morfismo tal que
%	$(f^{-1}\pi)f=\id_N$.\framebox{?}
%
%	Demostremos que $(5)\implies(7)$. Sea $L$ un ideal a izquierda de $R$. Como
%	los ideales a izquierda de $R$ son los submódulos de $\prescript{}{R}R$,
%	existe un ideal a izquierda $N$ de $R$ tal que $R=L\oplus N$. Existen
%	entonces $e_1\in L$ y $e_2\in N$ tales que $1=e_1+e_2$. Si $x\in L$,
%	entonces $x=xe_1+xe_2$ y luego $xe_2=x-xe_1\in L\cap N=\{0\}$. Demostramos
%	entonces que $x=xe_1$ para todo $x\in L$. En particular, $e_1e_1=e_1$ y
%	$L=Re_1$. 
%
%	Demostremos que $(7)\implies(6)$. Sea $L$ un submódulo de
%	$\prescript{}{R}R$. Como entonces $L$ es un ideal a izquierda de $R$,
%	$L=Re$ para algún idempotente $e\in R$. Como el conjunto $R(1-e)$ es un
%	ideal a izquierda de $R$ tal que $R=Re\oplus R(1-e)$, se concluye que
%	$\prescript{}{R}R$ es semisimple.\framebox{?}
%
%	Veamos que $(6)\implies(1)$. Supongamos que $\prescript{}{R}R=\sum_{i\in
%	I}N_i$, donde los $N_j$ son submódulos simples de $\prescript{}{R}R$.
%	Reordenando los $N_j$ si fuera necesario, podemos suponer que existe
%	$k\in\N$ tal que $1=e_1+\cdots+e_k$ y $e_j\in N_j$ para todo
%	$j\in\{1,\dots,k\}$. Si $r\in R$, entonces $r=re_1+\cdots+re_k\in
%	\sum_{i=1}^k N_i$. Luego $R=\sum_{i=1}^k N_i$.
%	Veamos que $J(R)=0$. Si $r\in J(R)$ entonces, como $rN_i=0$ para todo
%	$i\in\{1,\dots,k\}$, se concluye que $r=r1=re_1+\cdots+re_k=0$. Probamos
%	entonces que $R$ es semiprimitivo. Falta ver que $R$ es artiniano a
%	izquierda. Para eso basta obvervar que, como
%	\[
%		\frac{N_1\oplus\cdots\oplus N_i}{N_1\oplus\cdots\oplus N_{i-1}}\simeq N_i
%	\]
%	para cada $i\in\{1,\dots,k\}$, la serie
%	\[
%	R=N_1\oplus\cdots\oplus N_k\supsetneq N_1\oplus\cdots\oplus N_{k-1}\supsetneq\cdots\supsetneq N_1\oplus N_2\supsetneq N_1\supsetneq 0
%	\]
%	es una serie de composición.\framebox{?}
%
%	Veamos que $(1)\implies(8)$. Sin perder generalidad podemos suponer que
%	\[
%	R=\prod_{i=1}^k M_{n_j}(D_j),
%	\]
%	donde los $D_j$ son anillos de división.\framebox{?}
%
%	Veamos que $(8)\implies(5)$. Sea $M$ un módulo unitario no nulo. Si $m\in
%	M$ entonces $L_im$ es un submódulo de $M$. Los $L_jm$ generan a $M$ pues 
%	\[
%	m=1m=e_1m+\cdots+e_km\in\sum L_im.
%	\]
%	Veamos que cada $L_jm$ es simple. Fijado $i$, la función $f\colon L_i\to
%	L_im$, $x\mapsto xm$, es un morfismo sobreyectivo. Como $L_i$ es un ideal a
%	izquierda minimal, $L_i$ es un submódulo simple. Luego $m\ne0$ implica que
%	$f$ es un isomorfismo. Probamos entonces que el conjunto $\{L_jm:1\leq
%	j\leq k,m\in M,L_jm\ne 0\}$ es una familia de submódulos simples cuya suma
%	es $M$.
%\end{proof}

