\lecture{}

\topic{Artinian modules}

\begin{definition}
	Let $R$ be a ring. A module $N$ is \textbf{artinian} if every decreasing sequence 
	$N_1\supseteq N_2\supseteq\cdots$ of submodules of $N$ stabilizes, that is
	there exists $n\in\Z_{>0}$ such that 
	$N_n=N_{n+k}$ for all $k\in\Z_{>0}$.
\end{definition}

Let $X$ be a set and $\mathcal{S}$ be a set of subsets of $X$. 
We say that $A\in\mathcal{S}$ is a \textbf{minimal element} of $\mathcal{S}$
if there is no $Y\in\mathcal{S}$ such that $Y\subsetneq A$. 

\begin{proposition}
\label{pro:artinian_minimal}
	A module $N$ is artinian if and only if every non-empty subset of submodules of $N$ 
	contains a minimal element. 
\end{proposition}

\begin{proof}
	Assume that $N$ is artinian. Let $\mathcal{S}$ be the non-empty set of submodules of $N$. 
	Suppose that $\mathcal{S}$ has no minimal element and let $N_1\in\mathcal{S}$. 
	Since $N_1$ is not minimal, there exists 
	$N_2\in\mathcal{S}$ such that $N_1\supsetneq N_2$. Now assume the 
	submodules 
	\[
	N_1\supsetneq N_2\supseteq\cdots\supsetneq N_k
	\]
	we chosen. 
	Since $N_k$ is not minimal, there exists $N_{k+1}$ such that $N_k\supsetneq N_{k+1}$.
	This procedure produces a sequence $N_1\supsetneq
	N_2\supsetneq\cdots$ that cannot stabilize, a contradiction. 
	
	If $N_1\supseteq N_2\supseteq\cdots$ is a sequence of submodules, then 
	$\mathcal{S}=\{N_j:j\geq1\}$ has a minimal element, say $N_n$. Then
	$N_n=N_{n+k}$ for all $k$. 
\end{proof}

\begin{exercise}
    Prove that a ring $R$ is left artinian if every sequence of 
    left ideals $I_1\supseteq I_2\supseteq\cdots$ stabilizes. 
\end{exercise}

A module $N$ is \textbf{noetherian} if for every sequence 
$N_1\subseteq N_2\subseteq\cdots$ of submodules of $N$ there exists $n\in\Z_{>0}$ such that 
$N_n=N_{n+k}$ for all $k\in\Z_{>0}$. 

% Let $X$ be a set and $\mathcal{S}$ be a set of subsets of $X$. We say that 
% $B\in\mathcal{S}$ is a \textbf{maximal element} of $\mathcal{S}$ if
% there is no $Z\in\mathcal{S}$ such that $B\subsetneq Z$.

\begin{exercise}
    Let $M$ be a module. The following statements are equivalent:
    \begin{enumerate}
        \item $M$ is noetherian.
        \item Every submodule of $M$ is finitely generated. 
        \item Every non-empty subset $\mathcal{S}$ of submodules of $M$ contains a maximal element, that is
            an element $X\in\mathcal{S}$ such that there is no $Z\in\mathcal{S}$ such that $X\subseteq Z$.  
    \end{enumerate}
\end{exercise}

\begin{exercise}
    Prove that a ring $R$ is left noetherian if every sequence of 
    left ideals $I_1\subseteq I_2\subseteq\cdots$ stabilizes. 
\end{exercise}

\begin{exercise}
\label{xca:AN_exact}
	Let 
	\[
	\begin{tikzcd}
		0 \arrow{r}
		& A \arrow{r}{f}
		& B \arrow{r}{g}
		& C \arrow{r}
		& 0
	\end{tikzcd}
	\]
	be an exact sequence of modules. Prove that $B$ is noetherian (resp.
	artinian) if and only if $A$ and $C$ are noetherian (resp. artinian).
\end{exercise}

% \begin{definition}
% 	Un anillo $R$ se dice \textbf{noetheriano a izquierda} si el módulo 
% 	$\prescript{}{R}R$ es noetheriano.
% \end{definition}
%Similarly one defines right noetherian rings.

\begin{definition}
	A ring $R$ is \textbf{left artinian} if the module 
	$\prescript{}{R}R$ is artinian.
\end{definition}

Similarly one defines right artinian rings. 

\begin{example}
	The ring $\Z$ is noetherian. It is not artinian, as the sequence
	\[
	2\Z\supseteq
	4\Z\supseteq 8\Z\supseteq\cdots
	\]
	does not stabilize. 
\end{example}

\begin{definition}
	\label{def:serie_de_composicion}
	A \textbf{composition series} of the module $M$ is a sequence 
	\[
		\{0\}=M_0\subsetneq M_1\subsetneq M_2\subsetneq\cdots\subsetneq M_n=M
	\]
	of submodules of $M$ such that each $M_i/M_{i-1}$ is non-zero and has no non-zero 
	proper submodules. 
	In this case 
	$n$ is the length of the composition series.
\end{definition}

The previous definition makes sense also for non-unitary rings. That is why
it is required that each quotient $M_i/M_{i-1}$ has no proper submodules.

\begin{theorem}
	\label{thm:serie_de_composicion}
	A non-zero module admits a composition series if and only if it is artinian and noetherian.
\end{theorem}

\begin{proof}
	Let $M$ be a non-zero module and let $\{0\}=M_0\subsetneq
	M_1\subsetneq\cdots\subsetneq M_n=M$ be a composition series for $M$.
	We claim that each $M_i$ is artinian and noetherian. We proceed by induction on $i$. The case
	$i=0$ is trivial. Let us assume that $M_i$ is artinian and noetherian. Since 
	$M_i/M_{i+1}$ has no proper submodules and the sequence 
	\[
	\begin{tikzcd}
		0 \arrow{r}
		& M_i \arrow{r}
		& M_{i+1} \arrow{r}
		& M_{i+1}/M_i \arrow{r}
		& 0
	\end{tikzcd}
	\]
	is exact, it follows that 
	$M_{i+1}$ is artinian and noetherian, see Exercise \ref{xca:AN_exact}. 

    Conversely, let $M$ be an artinian and noetherian module. Let $M_0=\{0\}$ and 
    $M_1$ be minimal among the submodules of $M$ (it exists by Proposition \ref{pro:artinian_minimal}.
    If $M_1\ne M$, let 
	$M_2$ be minimal among those submodules of $M$ such that $M_1\subsetneq M_2$. This procedure
	produces a sequence 
	\[
		\{0\}=M_0\subsetneq M_1\subsetneq M_2\subsetneq\cdots
	\]
	of submodules of $M$, where each $M_{i+1}/M_i$ is non-zero and admits no
	proper submodules. Since $M$ is noetherian, the sequence stabilizes and
	hence it follows that $M_n=M$ for some $n$. 
\end{proof}

\begin{definition}
    Let $M$ be a module. 
	We say that the composition series
	\[
	M=V_0\supseteq V_1\supseteq\cdots\supseteq V_k=\{0\},
	\quad
	M=W_0\supseteq W_1\supseteq\cdots\supseteq W_l=\{0\},
	\]
	are \textbf{equivalent} if $k=l$ and there exists 
	$\sigma\in\Sym_n$ such that 
	$V_{i}/V_{i-1}\simeq W_{\sigma(i)}/W_{\sigma(i)-1}$
	for all $i\in\{1,\dots,k\}$.
\end{definition}

\begin{theorem}[Jordan--H\"older]
	\label{thm:JordanHolder}
	Any two composition series for a module are equivalent. 
\end{theorem}

\begin{proof}
    Let $M$ be a module and
    \[
		M=V_0\supseteq V_1\supseteq\cdots\supseteq V_k=\{0\},
		\quad
		M=W_0\supseteq W_1\supseteq\cdots\supseteq W_l=\{0\},
	\]
	be composition series of $M$. 
	We claim that these composition series are equivalent. 
	We proceed by induction on $k$. The case $k=1$ is trivial, as 
	in this case $M$ has no proper submodules and $M\supseteq\{0\}$ 
	is the only possible composition series for $M$. So
	assume the result holds for modules with composition series of length $<k$. If $V_1=W_1$, then 
	$V_1$ has composition series of lengths $k-1$ and $l-1$. The inductive hypothesis implies that 
	$k=l$ and we are done. So assume that $V_1\ne W_1$. Since $V_1$ and $W_1$ are submodules of $M$, the
	sum $V_1+W_1$ is also a submodule of $M$. Moreover, $V/V_1$ has no non-zero proper submodules
	and hence 
	$V_1+W_1=V$. Then 
	\[
		V/V_1=\frac{V_1+W_1}{V_1}\simeq\frac{V_1}{V_1\cap W_1}.
	\]
	Since $V_1$ has a composition series, $V_1$ is artinian and
	noetherian by Theorem~\ref{thm:serie_de_composicion}. The submodule $U=V_1\cap W_1$ is also 
	artinian and noetherian and hence, by Theorem \ref{thm:serie_de_composicion}, 
	it admits a composition series 
	\[
		U=U_0\supseteq U_1\supseteq\cdots\supseteq U_r=\{0\}.
	\]
    Thus
    $V_1\supseteq\cdots\supseteq V_k=\{0\}$ and  
	$V_1\supseteq U\supseteq U_1\supseteq\cdots\supseteq U_r=\{0\}$ are both composition 
	series for $V_1$. The inductive hypothesis implies that 
	$k-1=r+1$ and that these composition series are equivalent. Similarly, 
	\[
		W_1\supseteq W_1\supseteq\cdots\supseteq W_l=\{0\},
		\quad
		W_1\supseteq U\supseteq U_1\supseteq\cdots\supseteq U_{r}=\{0\},
	\]
    are both composition series for $W_1$ and hence $l-1=r+1$ and these composition 
    series are equivalent. Therefore $l=k$ and the proof is completed. 
\end{proof}

Jordan--H\"older's theorem allows us to define the 
length of modules that admit a composition series. 

\begin{definition}
    Let $M$ be a module with a composition series. 
    The \textbf{length} $\ell(M)$ of $M$ is defined as the length of any composition series of $M$. 
\end{definition}

A module is said to be of finite length if it admits a composition series. 

\begin{exercise}
	If $N$ and $Q$ are modules with composition series and  
	\[
	\begin{tikzcd}
		0 \arrow[r]
		& N \arrow{r}{f}
		& M \arrow{r}{g}
		& Q \arrow[r]
		& 0
	\end{tikzcd}
	\]
	is an exact sequence of modules, then $\ell(M)=\ell(N)+\ell(Q)$.
\end{exercise}

%\begin{proof}
%	Sean $Q=Q_0\supsetneq Q_1\supsetneq\cdots\supsetneq Q_m=0$ y
%	$N=N_0\supsetneq N_1\supseteq\cdots\supsetneq N_n=0$ series de composición
%	para $Q$ y $N$ respectivamente. Entonces
%	\[
%		M=g^{-1}(Q_0)\supsetneq g^{-1}(Q_1)\supsetneq\cdots\supsetneq g^{-1}(Q_m)=f(N_0)\supsetneq f(N_1)\supsetneq\cdots\supsetneq f(N_n)=0
%	\]
%	es una serie de composición para $M$ y luego $c(M)=c(N)+c(Q)$.
%\end{proof}

\begin{exercise}
	If $A$ and $B$ are finite-length submodules of $M$, then  
	\[
	\ell(A+B)+\ell(A\cap B)=\ell(A)+\ell(B).
	\]
\end{exercise}

\begin{theorem}
	\label{thm:Jnilpotente}
	If $R$ is a left artinian ring, then $J(R)$ is nilpotent. 
\end{theorem}

\begin{proof}
	Let $J=J(R)$. Since $R$ is a left artinian ring, the sequence 
	$(J^m)_{m\in\Z_{>0}}$ of left ideals stabilizes. There exists 
	$k\in\Z_{>0}$ such that $J^k=J^l$ for all $l\geq k$. We claim that $J^k=\{0\}$. If
	$J^k\ne\{0\}$ let $\mathcal{S}$ the set of left ideals 
	$I$ such that $J^kI\ne\{0\}$. Since 
	\[
	J^kJ^k=J^{2k}=J^k\ne\{0\},
	\]
	the set $\mathcal{S}$ is non-empty. 
	Since $R$ is left artinian, $\mathcal{S}$ has a minimal element $I_0$. Since $J^kI_0\ne\{0\}$, let $x\in
	I_0\setminus\{0\}$ be such that $J^kx\ne\{0\}$. Moreover, $J^kx$ is a left ideal of $R$ 
	contained in $I_0$ and such that $J^kx\in\mathcal{S}$, as 
	$J^k(J^kx)=J^{2k}x=J^kx\ne\{0\}$. The minimality of $I_0$ implies that, $J^kx=I_0$. In particular, 
	there exists $r\in J^k\subseteq J(R)$ such that $rx=x$. Since $-r\in
	J(R)$ is left quasi-regular, there exists $s\in R$ such that $s-r-sr=0$.
	Thus 
	\[
		x=rx=(s-sr)x=sx-s(rx)=sx-sx=0,
	\]
	a contradiction.
\end{proof}

\begin{corollary}
	Let $R$ be a left artinian ring. Each nil left ideal is nilpotent and 
	$J(R)$ is the unique maximal nilpotent ideal of $R$. 
\end{corollary}

\begin{proof}
	Let $L$ be a nil left ideal of $R$. By Proposition~\ref{pro:nilJ}, $L$
	is contained in $J(R)$. Thus $L$ is nilpotent, as $J(R)$ 
	is nilpotent by Theorem~\ref{thm:Jnilpotente}. 
\end{proof}


\topic{Semisimple modules}

In the first lectures we studied semisimple modules over finite-dimensional 
algebras. Let us now review the theory of semisimple modules over rings. 
A (finitely generated) module $M$ (over a ring $R$) is \textbf{semisimple} 
if it isomorphic to a (finite) direct sum of simple modules. 

\begin{definition}
    Let $R$ be a ring. A left ideal $L$ is said to be \textbf{minimal}
    if $L\ne\{0\}$ and there is no left ideal $L_1$
    such that $\{0\}\subsetneq L_1\subsetneq I$.
\end{definition}

The ring $\Z$ contains no minimal left ideals. If $I$ is a non-zero 
left ideal of $\Z$, then
$I=(n)$ for some $n>0$ and $I=(n)\supsetneq (2n)$. 

\begin{proposition}
    Let $R$ be a left artinian ring. 
    Then every non-zero left ideal contains a minimal left ideal. 
\end{proposition}

\begin{proof}
    Let $X$ be the family of non-zero left ideals contained in $I$. Then $X$ is non-empty, as 
    $I\in X$. Then $X$ contains a minimal element by Proposition \ref{pro:artinian_minimal}. 
\end{proof}

% \begin{proposition}
% 	Let $R$ be a unitary ring and $M$ be a unitary semisimple module. 
% 	The following statements are equivalent:
% 	\begin{enumerate}
% 		\item $M$ is noetherian.
% 		\item $M$ is artinian.
% 		\item $M$ is a direct sum of finitetely many simple modules. 
% 	\end{enumerate}
% \end{proposition}

% \begin{proof}
% 	We first prove that $3)\Longleftrightarrow1)$ and that 
% 	$3)\Longleftrightarrow2)$. Como cada submódulo simple es artiniano y
% 	noetheriano, $M$ resulta artiniano y noetheriano. Recíprocamente, si $M$ es
% 	artiniano, $I$ debe ser finito pues de lo contrario podríamos elegir
% 	elementos $i_1,i_2,i_3,\dots$ de $I$ tales que la sucesión
% 	\[
% 		\bigoplus_{i\in I}M_i\supsetneq \bigoplus_{i\in I\setminus\{i_1\}}M_i\supsetneq\bigoplus_{i\in I\setminus\{i_1,i_2\}}M_i\supsetneq\cdots
% 	\]
% 	nunca se estabiliza. Análogamente, si $M$ es noetheriano, podríamos elegir
% 	elementos $i_1,i_2,i_3,\dots\in I$ tales que la sucesión
% 	\[
% 		M_{i_1}\subsetneq M_{i_1}\oplus M_{i_2}\subsetneq\cdots
% 	\]
% 	nunca se estabiliza.
% \end{proof}

A ring $R$ with identity is \textbf{semisimple} if it is a direct sum of finitely many minimal left ideals. Note
that $\prescript{}{R}{R}$ is finitely generated by $\{1\}$. Minimal left ideals of $R$ 
are exactly the simple submodules of $\prescript{}{R}{R}$. 
This means that 
the ring $R$ is semisimple if and only if the module
$\prescript{}{R}{R}$ is semisimple.  

\begin{proposition}
    Let $R$ be a semisimple ring. Then $R$ is noetherian and artinian.
\end{proposition}

\begin{proof}
    Write $R$ as a direct sum $R=L_1\oplus\cdots\oplus L_n$ of minimal left ideals. Since 
    each $L_j$ is a simple submodule of $\prescript{}{R}{R}$, it follows that 
    \[
    L_1\oplus\cdots\oplus L_n\supsetneq L_2\oplus\cdots\oplus L_n\supsetneq\cdots\supsetneq L_n\supsetneq\{0\}
    \]
    is a composition series for $\prescript{}{R}{R}$ with composition factors
    $L_1,\dots,L_n$. Since the module $\prescript{}{R}{R}$ admits a composition
    series, it is artinian and noetherian by Theorem \ref{thm:serie_de_composicion}. It follows
    from the definitions that $R$ is left artinian and left noetherian. 
\end{proof}

Now it is possible to prove Artin--Wedderburn's theorem for rings. 
If $R$ is a semisimple ring, then
\[
R\simeq \prod_{i=1}^k M_{n_i}(D_i)
\]
for some $n_1,\dots,n_k\geq1$ and some
division rings $D_1,\dots,D_k$. 
The proof is somewhat
the same we did for finite-dimensional algebras.

\begin{theorem}
	\label{thm:SSartin=J}
	Let $R$ be a unitary ring. Then $R$ is semisimple if and only if 
	$R$ is left artinian and $J(R)=\{0\}$.
\end{theorem}

We shall need a lemma.

\begin{lemma}
	\label{lem:Jartiniano}
	Let $R$ be a unitary left artinian ring. There exists finitely many maximal ideals 
	$I_1,\dots,I_n$ of $R$ such that 
	$J(R)=I_1\cap\cdots\cap I_n$.
\end{lemma}

\begin{proof}
    The set $X$ of left ideals of the form
	$I_1\cap\cdots\cap I_n$ for finitely many maximal ideals $I_1,\dots,I_n$ of $R$
	is non-empty, as $R$ contains maximal ideals since it is a unitary ring. 
    Since $R$ is left artinian,
	Proposition~\ref{pro:artinian_minimal} implies that $X$ 
	contains a minimal element, say 
	$J=\bigcap_{i=1}^k I_i$. We claim that $J=J(R)$. 
	Since $R$ is
	unitary, $J(R)$ is the intersection of all maximal ideals of $R$ and hence 
	$J(R)\subseteq J$. Let us now prove that $J\subseteq J(R)$. 
	If not, let $x\in
	J\setminus J(R)$. Then there exists a maximal ideal $M$ such that $x\not\in
	M$. This implies that $J\cap M\subsetneq J$, a contradiction to the minimality of 
    $J$. 
\end{proof}

We now prove the theorem. 

\begin{proof}[Proof of Theorem \ref{thm:SSartin=J}]
	Assume first that $R$ is semisimple. By Artin--Wedderburn's theorem, 
	\[
		R\simeq\prod_{i=1}^kM_{n_i}(D_i)
	\]
	for some $n_1,\dots,n_k\geq1$ and some division rings $D_1,\dots,D_k$. 
	In particular, $R$ is left artinian and $J(R)=\prod_{i=1}^kJ(M_{n_i}(D_i))=\{0\}$
	because each $M_{n_i}(D_i)$ is simple. 

    Conversely, the previous lemma implies that $\{0\}=J(R)=I_1\cap\cdots\cap I_k$ for some
    maximal ideals $I_1,\dots,I_k$. Since each $R/I_i$ is simple, it follows that 
    $\prod_{i=1}^k R/I_i$ is semisimple. The map 
	\[
	R\to \prod_{i=1}^k R/I_i,\quad
	x\mapsto (x+I_1,\dots,x+I_k),
	\]
	is an ring homomorphism with 
	kernel $I_1\cap\cdots\cap I_k=\{0\}$. Thus it is injective and hence 
	it follows that $R$ 
	is also semisimple. 
\end{proof}

% Como consecuencia tenemos el siguiente resultado:

% \begin{proposition}
% 	Sea $G$ un grupo. Entonces $\C[G]$ es artiniana a izquierda si y sólo si
% 	$G$ es finito. 
% \end{proposition}

% \begin{proof}
% 	Si $G$ es finito sabemos que $\C[G]$ es artiniano a izquierda por ser de
% 	dimensión finita.  Recíprocamente, si $G$ es infinito, sabemos que
% 	$J(\C[G])=0$ (por el teorema de Rickart) y que $\C[G]$ no es semisimple
% 	(por la proposición~\ref{pro:nunca_SS}). Luego $\C[G]$ no es artiniana a izquierda por el
% 	teorema~\ref{thm:SSartin=J}.
% \end{proof}

% Concluimos la sección con el siguiente teorema:

We now present an important result that uses 
semisimplicity. 

\begin{theorem}[Hopkins--Levitszki]
	\label{thm:Hopkins-Levitski}
	Let $R$ be a unitary left artinian ring. Then $R$ is left noetherian.
\end{theorem}

\begin{proof}
	Let $J=J(R)$. Since $R$ is left artinian, $J$ is a nilpotent ideal 
	by Theorem~\ref{thm:Jnilpotente}. Let $n$ be such that $J^n=\{0\}$. Now consider the sequence 
	\[
		R\supsetneq J\supsetneq J^2\supsetneq\cdots\supsetneq J^{n-1}\supsetneq J^n=\{0\}.
	\]
	Each $J^{i}/J^{i+1}$ is a module over $R$ annihilated by $J$, 
	that is $J\cdot (J^i/J^{i+1})=\{0\}$, as 
	\[
	x\cdot (y+J^{i+1})=xy+J^{i+1}\subseteq JJ^i+J^{i+1}=J^{i+1} 
	\]
	if $x\in J$ and $y\in J^i$. 
	Thus each  
	$J^i/J^{i+1}$ is a module over $R/J$. Since $R/J$ is left artinian and 
	$J(R/J)=\{0\}$ by Theorem \ref{thm:J(R/J)=0}, it follows from the previous proposition that $R/J$ is semisimple. 
	It follows that each $J^{i}/J^{i+1}$ 
	is semisimple and hence it is left noetheriano. Inductively one proves that each 
	$J^i$ is left noetherian and therefore $R$ is left noetherian. 
\end{proof}


