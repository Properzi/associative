\chapter{}



\topic{Gilmer's theorem}

Hilbert's theorem states that 
if $R$ is a noetherian 
commutative unitary ring, then
$R[X]$ is noetherian. Following \cite{MR212007}, 
we now present the converse of Hilbert's theorem. 

\begin{theorem}[Gilmer]
\index{Gilmer's theorem}
    Let $R$ be a commutative ring. If $R[X]$ is noetherian, 
    then $R$ is unitary. 
\end{theorem}

\begin{proof}
    Let $a\in R$. For $m\geq0$, let
    \begin{align*}
    I_m&=(a,aX,aX^2,\dots,aX^m)\\
    &=R[X]a+R[X]aX+\cdots+R[X]aX^m+\Z a+\Z aX+\cdots+\Z aX^m.
    \end{align*}
    Then $I_0\subseteq I_1\subseteq\cdots I_m\subseteq I_{m+1}\subseteq\cdots$ 
    is a sequence of ideals of $R[X]$. Since $R[X]$ is noetherian,
    $I_n=I_{n+1}$ for some $n$. In particular, 
    $aX^{n+1}\in I_{n+1}=I_n$. Thus
    \[
    aX^{n+1}=\sum_{i=1}^{n+1} aX^{i-1}f_i(X)+\sum_{i=1}^{n+1} k_iaX^{i-1}
    \]
    for some $f_1(X),\dots,f_n(X)\in R[X]$ and 
    $k_1,\dots,k_n\in\Z$. Comparing the coefficient of $X^{n+1}$ 
    one gets that $a=ar$ for some $r\in R$. 
    Thus  
    \begin{equation}
        \label{eq:Gilmer}
        \text{for every $a\in R$ there exists $r\in R$ 
    such that $a=ra$.}
    \end{equation}
    
    \begin{claim}
        For every $a_1,\dots,a_n\in R$ there exists $r\in R$ 
        such that $a_i=ra_i$ for all $i$. 
    \end{claim}
    
    We proceed by induction on $n$. 
    The case $n=1$ is \eqref{eq:Gilmer}. Assume 
    that the result holds for $n-1\geq1$. By the inductive hypothesis, 
    there exists $r_1\in R$ such that $a_i=r_1a_i$ 
    for all $i\in\{1,\dots,n-1\}$. Moreover, 
    there exists $r_2\in R$ such that $a_n=ra_n$. 
    Let $r=r_1+r_2-r_1r_2$. Then
    \[
    ra_n=r_1a_n+r_2a_n-r_1r_2a_n=r_1a_n+a_n-r_1a_n=a_n.
    \]
    Moreover, for $i\in\{1,\dots,n-1\}$, 
    \[
    ra_i=r_1a_i+r_2a_i-r_1r_2a_i=a_i+r_2a_i-r_2r_1a_i=a_i+r_2a_i-r_2a_i=a_i.
    \]
    
    We now finish the proof of the theorem. 
    Let $R[X]\to R$, $f(X)\mapsto f(0)$, be an evaluation map. Since
    it is a surjective ring homomorphism, 
    $R$ is noetherian. In particular, $R$ is finitely generated, 
    say 
    \[
    R=(a_1,\dots,a_n)=Ra_1+\cdots+Ra_n+\Z a_1+\cdots+\Z a_n
    \]
    for some $a_1,\dots,a_n\in R$. 
    
    We now prove that the element $r$ from the claim we proved turns 
    $R$ into a unitary ring, that is $r=1_R$.  
    We need to show that $rb=b$ for all $b\in R$. 
    If $b\in R$, then 
    \[
    b=t_1a_1+\cdots+t_na_n+m_1a_1+\cdots+m_na_n
    \]
    for some $t_1,\dots,t_n\in R$ and $m_1,\dots,m_n\in\Z$. 
    Since $a_i=ra_i$ for all $i\in\{1,\dots,n\}$, it immediately
    follows that
    $rb=b$. 
\end{proof}

\begin{example}
    The polynomial ring $(2\Z)[X]$ is not 
    noetherian, as the ring $2\Z$ is not unitary. 
\end{example}

\topic{Artinian modules}

\begin{definition}
\index{Module!artinian}
	Let $R$ be a ring. A module $N$ is \textbf{artinian} if every decreasing sequence 
	$N_1\supseteq N_2\supseteq\cdots$ of submodules of $N$ stabilizes, that is
	there exists $n\in\Z_{>0}$ such that 
	$N_n=N_{n+k}$ for all $k\in\Z_{\geq0}$.
\end{definition}

\index{Minimal element}
Let $X$ be a set and $\mathcal{S}$ be a set of subsets of $X$. 
We say that $A\in\mathcal{S}$ is a \textbf{minimal element} of $\mathcal{S}$
if there is no $Y\in\mathcal{S}$ such that $Y\subsetneq A$. 

\begin{proposition}
\label{pro:artinian_minimal}
	A module $N$ is artinian if and only if 
	every non-empty subset of submodules of $N$ 
	contains a minimal element. 
\end{proposition}

\begin{proof}
	Assume that $N$ is artinian. Let $\mathcal{S}$ be a non-empty set of submodules of $N$. 
	Suppose that $\mathcal{S}$ has no minimal element and let $N_1\in\mathcal{S}$. 
	Since $N_1$ is not minimal, there exists 
	$N_2\in\mathcal{S}$ such that $N_1\supsetneq N_2$. Now assume the 
	submodules 
	\[
	N_1\supsetneq N_2\supsetneq\cdots\supsetneq N_k
	\]
	we chosen. 
	Since $N_k$ is not minimal, there exists $N_{k+1}$ such that $N_k\supsetneq N_{k+1}$.
	This procedure produces a sequence $N_1\supsetneq
	N_2\supsetneq\cdots$ that cannot stabilize, a contradiction. 
	
	If $N_1\supseteq N_2\supseteq\cdots$ is a sequence of submodules, then 
	$\mathcal{S}=\{N_j:j\geq1\}$ has a minimal element, say $N_n$. Then
	$N_n=N_{n+k}$ for all $k$. 
\end{proof}

\index{Module!noetherian}
A module $N$ is \textbf{noetherian} if for every sequence 
$N_1\subseteq N_2\subseteq\cdots$ of submodules of $N$ there exists $n\in\Z_{>0}$ such that 
$N_n=N_{n+k}$ for all $k\in\Z_{\geq0}$. 

% Let $X$ be a set and $\mathcal{S}$ be a set of subsets of $X$. We say that 
% $B\in\mathcal{S}$ is a \textbf{maximal element} of $\mathcal{S}$ if
% there is no $Z\in\mathcal{S}$ such that $B\subsetneq Z$.

\begin{exercise}
    Let $M$ be a module. The following statements are equivalent:
    \begin{enumerate}
        \item $M$ is noetherian.
        \item Every submodule of $M$ is finitely generated. 
        \item Every non-empty subset $\mathcal{S}$ of submodules of $M$ contains a maximal element, that is
            an element $X\in\mathcal{S}$ such that there is no $Z\in\mathcal{S}$ such that $X\subsetneq Z$.  
    \end{enumerate}
\end{exercise}

\begin{exercise}
    Prove that a ring $R$ is left noetherian if every sequence of 
    left ideals $I_1\subseteq I_2\subseteq\cdots$ stabilizes. 
\end{exercise}

\begin{exercise}
\label{xca:AN_exact}
	Let 
	\[
	\begin{tikzcd}
		0 \arrow{r}
		& A \arrow{r}{f}
		& B \arrow{r}{g}
		& C \arrow{r}
		& 0
	\end{tikzcd}
	\]
	be an exact sequence of modules. Prove that $B$ is noetherian (resp.
	artinian) if and only if $A$ and $C$ are noetherian (resp. artinian).
\end{exercise}

% \begin{definition}
% 	Un anillo $R$ se dice \textbf{noetheriano a izquierda} si el módulo 
% 	$\prescript{}{R}R$ es noetheriano.
% \end{definition}
%Similarly one defines right noetherian rings.

\begin{definition}
\index{Ring!left artinian}
	A ring $R$ is \textbf{left artinian} if the module 
	$\prescript{}{R}R$ is artinian.
\end{definition}

Similarly one defines right artinian rings. 

\begin{example}
	The ring $\Z$ is noetherian. It is not artinian, as the sequence
	\[
	2\Z\supseteq
	4\Z\supseteq 8\Z\supseteq\cdots
	\]
	does not stabilize. 
\end{example}

\begin{exercise}
    Prove that a ring $R$ is left artinian if every sequence of 
    left ideals $I_1\supseteq I_2\supseteq\cdots$ stabilizes. 
\end{exercise}


\begin{definition}
\index{Module!composition series}
	\label{def:serie_de_composicion}
	A \textbf{composition series} of the module $M$ is a sequence 
	\[
		\{0\}=M_0\subsetneq M_1\subsetneq M_2\subsetneq\cdots\subsetneq M_n=M
	\]
	of submodules of $M$ such that each $M_i/M_{i-1}$ is non-zero and has no non-zero 
	proper submodules. 
	In this case 
	$n$ is the length of the composition series.
\end{definition}

The previous definition makes sense also for non-unitary rings. That is why
it is required that each quotient $M_i/M_{i-1}$ has no proper submodules.

\begin{theorem}
	\label{thm:serie_de_composicion}
	A non-zero module admits a composition series if and only if it is artinian and noetherian.
\end{theorem}

\begin{proof}
	Let $M$ be a non-zero module and let $\{0\}=M_0\subsetneq
	M_1\subsetneq\cdots\subsetneq M_n=M$ be a composition series for $M$.
	We claim that each $M_i$ is artinian and noetherian. We proceed by induction on $i$. The case
	$i=0$ is trivial. Let us assume that $M_i$ is artinian and noetherian. Since 
	$M_i/M_{i+1}$ has no proper submodules and the sequence 
	\[
	\begin{tikzcd}
		0 \arrow{r}
		& M_i \arrow{r}
		& M_{i+1} \arrow{r}
		& M_{i+1}/M_i \arrow{r}
		& 0
	\end{tikzcd}
	\]
	is exact, it follows that 
	$M_{i+1}$ is artinian and noetherian, see Exercise \ref{xca:AN_exact}. 

    Conversely, let $M$ be a non-zero  artinian and noetherian module. Let $M_0=\{0\}$ and 
    $M_1$ be minimal among the non-zero  submodules of $M$ (it exists by Proposition \ref{pro:artinian_minimal}).
    If $M_1\ne M$, let 
	$M_2$ be minimal among those submodules of $M$ such that $M_1\subsetneq M_2$. This procedure
	produces a sequence 
	\[
		\{0\}=M_0\subsetneq M_1\subsetneq M_2\subsetneq\cdots
	\]
	of submodules of $M$, where each $M_{i+1}/M_i$ is non-zero and admits no
	proper submodules. Since $M$ is noetherian, the sequence stabilizes and
	hence it follows that $M_n=M$ for some $n$. 
\end{proof}

\begin{definition}
\index{Composition series!equivalence}
    Let $M$ be a module. 
	We say that the composition series
	\[
	M=V_0\supseteq V_1\supsetneq\cdots\supsetneq V_k=\{0\},
	\quad
	M=W_0\supsetneq W_1\supsetneq\cdots\supsetneq W_l=\{0\},
	\]
	are \textbf{equivalent} if $k=l$ and there exists 
	$\sigma\in\Sym_k$ such that 
	$V_{i}/V_{i-1}\simeq W_{\sigma(i)}/W_{\sigma(i)-1}$
	for all $i\in\{1,\dots,k\}$.
\end{definition}

\begin{exercise}
\label{xca:Z6}
    Find all composition series
    for the $\Z$-module $\Z/6$. 
\end{exercise}

\begin{theorem}[Jordan--H\"older]
	\label{thm:JordanHolder}
	\index{Jordan--H\"older theorem}
	Any two composition series for a module are equivalent. 
\end{theorem}

\begin{proof}
    Let $M$ be a module and
    \[
		M=V_0\supsetneq V_1\supsetneq\cdots\supsetneq V_k=\{0\},
		\quad
		M=W_0\supsetneq W_1\supsetneq\cdots\supsetneq W_l=\{0\},
	\]
	be composition series of $M$. 
	We claim that these composition series are equivalent. 
	We proceed by induction on $k$. The case $k=1$ is trivial, as 
	in this case $M$ has no proper submodules and $M\supseteq\{0\}$ 
	is the only possible composition series for $M$. So
	assume the result holds for modules with composition series of length $<k$. If $V_1=W_1$, then 
	$V_1$ has composition series of lengths $k-1$ and $l-1$. The inductive hypothesis implies that 
	$k=l$ and we are done. So assume that $V_1\ne W_1$. Since $V_1$ and $W_1$ are submodules of $M$, the
	sum $V_1+W_1$ is also a submodule of $M$. Moreover, $M/V_1$ has no non-zero proper submodules
	and hence 
	$V_1+W_1=V$. Then 
	\[
		M/V_1=\frac{V_1+W_1}{V_1}\simeq\frac{V_1}{V_1\cap W_1}.
	\]
	Since $V_1$ has a composition series, $V_1$ is artinian and
	noetherian by Theorem~\ref{thm:serie_de_composicion}. The submodule $U=V_1\cap W_1$ is also 
	artinian and noetherian and hence, by Theorem \ref{thm:serie_de_composicion}, 
	admits a composition series 
	\[
		U=U_0\supsetneq U_1\supsetneq\cdots\supsetneq U_r=\{0\}.
	\]
    Thus
    $V_1\supsetneq\cdots\supsetneq V_k=\{0\}$ and  
	$V_1\supseteq U\supsetneq U_1\supsetneq\cdots\supsetneq U_r=\{0\}$ are both composition 
	series for $V_1$. The inductive hypothesis implies that 
	$k-1=r+1$ and that these composition series are equivalent. Similarly, 
	\[
		W_1\supsetneq W_2\supsetneq\cdots\supsetneq W_l=\{0\},
		\quad
		W_1\supsetneq U\supsetneq U_1\supsetneq\cdots\supsetneq U_{r}=\{0\},
	\]
    are both composition series for $W_1$ and hence $l-1=r+1$ and these composition 
    series are equivalent. Therefore $l=k$ and the proof is completed. 
\end{proof}

Jordan--H\"older theorem allows us to define the 
length of modules that admit a composition series. 

\begin{definition}
\index{Module!length}
    Let $M$ be a module with a composition series. 
    The \textbf{length} $\ell(M)$ of $M$ is defined as the length of any composition series of $M$. 
\end{definition}

\index{Module!of finite length}
A module is said to be of 
finite length if it admits a composition series. 

\begin{exercise}
	If $N$ and $Q$ are modules with composition series and  
	\[
	\begin{tikzcd}
		0 \arrow[r]
		& N \arrow{r}{f}
		& M \arrow{r}{g}
		& Q \arrow[r]
		& 0
	\end{tikzcd}
	\]
	is an exact sequence of modules, then $\ell(M)=\ell(N)+\ell(Q)$.
\end{exercise}

%\begin{proof}
%	Sean $Q=Q_0\supsetneq Q_1\supsetneq\cdots\supsetneq Q_m=0$ y
%	$N=N_0\supsetneq N_1\supseteq\cdots\supsetneq N_n=0$ series de composición
%	para $Q$ y $N$ respectivamente. Entonces
%	\[
%		M=g^{-1}(Q_0)\supsetneq g^{-1}(Q_1)\supsetneq\cdots\supsetneq g^{-1}(Q_m)=f(N_0)\supsetneq f(N_1)\supsetneq\cdots\supsetneq f(N_n)=0
%	\]
%	es una serie de composición para $M$ y luego $c(M)=c(N)+c(Q)$.
%\end{proof}

\begin{exercise}
	If $A$ and $B$ are finite-length submodules of $M$, then  
	\[
	\ell(A+B)+\ell(A\cap B)=\ell(A)+\ell(B).
	\]
\end{exercise}

\begin{theorem}
	\label{thm:Jnilpotente}
	If $R$ is a left artinian ring, then $J(R)$ is nilpotent. 
\end{theorem}

\begin{proof}
	Let $J=J(R)$. Since $R$ is a left artinian ring, the sequence 
	$(J^m)_{m\in\Z_{>0}}$ of left ideals stabilizes. There exists 
	$k\in\Z_{>0}$ such that $J^k=J^l$ for all $l\geq k$. We claim that $J^k=\{0\}$. If
	$J^k\ne\{0\}$ let $\mathcal{S}$ the set of left ideals 
	$I$ such that $J^kI\ne\{0\}$. Since 
	\[
	J^kJ^k=J^{2k}=J^k\ne\{0\},
	\]
	the set $\mathcal{S}$ is non-empty. 
	Since $R$ is left artinian, $\mathcal{S}$ has a minimal element $I_0$. Since $J^kI_0\ne\{0\}$, let $x\in
	I_0\setminus\{0\}$ be such that $J^kx\ne\{0\}$. Moreover, $J^kx$ is a left ideal of $R$ 
	contained in $I_0$ and such that $J^kx\in\mathcal{S}$, as 
	$J^k(J^kx)=J^{2k}x=J^kx\ne\{0\}$. The minimality of $I_0$ implies that, $J^kx=I_0$. In particular, 
	there exists $r\in J^k\subseteq J$ such that $rx=x$. Since $-r\in
	J(R)$ is left quasi-regular, there exists $s\in R$ such that $s-r-sr=0$.
	Thus 
	\[
		x=rx=(s-sr)x=sx-s(rx)=sx-sx=0,
	\]
	a contradiction.
\end{proof}

\begin{corollary}
	Let $R$ be a left artinian ring. Each nil left ideal is nilpotent and 
	$J(R)$ is the unique maximal nilpotent ideal of $R$. 
\end{corollary}

\begin{proof}
	Let $L$ be a nil left ideal of $R$. By Proposition~\ref{pro:nilJ}, $L$
	is contained in $J(R)$. Thus $L$ is nilpotent, as $J(R)$ 
	is nilpotent by Theorem~\ref{thm:Jnilpotente}. 
\end{proof}

\topic{Akizuki's theorem}

We now prove that 
if $R$ is a unitary commutative artinian ring, 
then $R$ is noetherian. 

\begin{exercise}
\label{xca:I_fg}
    Let $R$ be a unitary commutative ring, $I$ be an ideal of $R$
    and $M$ be an $R$-module such that $I\cdot M=\{0\}$. Prove that
    if $M$ is finitely generated, then $M$ is a finitely generated
    $(R/I)$-module with
    \[
    (r+I)\cdot m=r\cdot m,\quad r\in R,m\in M.
    \]
\end{exercise}

Recall that an ideal $I$  
of a commutative ring 
$R$ is said to be \textbf{prime} if 
$xy\in I$ implies that $x\in I$ or $y\in I$. 

\begin{exercise}
    Let $R$ be an unitary commutative artinian ring. 
    \begin{enumerate}
        \item Prove that if $R$ is a domain, then $R$ is a field. 
        \item Prove that prime ideals of $R$ are maximal. 
    \end{enumerate}
\end{exercise}

% Let $x\in R\setminus\{0\}$. The sequence $(x)\supseteq (x^2)\supseteq\cdots$
% stabilizes, that is $(x^n)=(x^{n+1})$ for some $n$. There exists
% $r\in R$ such that $x^n(1-rx)=0$. Since $x^n\ne 0$, $1=rx$. 
% I prime $\implies$ $(R/I)$ domain $\implies $R/I$ field $\implies $I$ maximal 

\begin{theorem}[Akizuki]
    \index{Akizuki's theorem}
    Let $R$ be a unitary commutative ring. If $R$ is artinian, 
    then $R$ is noetherian.
\end{theorem}

\begin{proof}
    Assume that the result is not true, so there exists an ideal of $R$ 
    that is not finitely generated. 
    Let $X$ be the set of ideals of $R$ that are not finitely generated. 
    Since $X\ne\emptyset$ and $R$ is artinian, there exists a minimal 
    element $I\in X$. The minimality of $I$ implies that 
    if $J$ is an ideal of $R$ such that $J\subsetneq I$, then 
    $J$ is finitely generated. 

    \begin{claim}
        Either $RI=\{0\}$ or $RI=I$.
    \end{claim}
    
    If not, let $r\in R$ be such that $rI\ne\{0\}$ and $rI\ne I$. 
    Since $rI$ is an ideal of $R$ and
    $rI\subsetneq I$, the minimality of $I$ implies that 
    $rI$ is finitely generated. Let 
    $f\colon I\to rI$, $x\mapsto rx$. Then $f$ is a 
    surjective module homomorphism. Since $RI\ne\{0\}$, 
    $f$ is non-zero. In particular, $\ker f$ 
    is finitely generated, again by the minimality of $I$. 
    By the first isomorphism theorem, $I/\ker f\simeq rI$ as $R$-modules.
    Since $\ker f$ and $I/\ker f\simeq rI$ are finitely generated, 
    $I$ is finitely generated, a contradiction.
    
    \begin{claim}
        $M=\{r\in R:rI=\{0\}\}$ is a maximal ideal of $R$. 
    \end{claim}
    
    Routine calculations show that $M$ is an ideal. Since 
    $R$ is artinian, it is enough to show that $M$ is a prime ideal. 
    Let $rs\in M$. Then $(rs)I=\{0\}$. If $r\not\in M$, 
    then $rI\ne\{0\}$. By the previous claim, $rI=I$. Thus
    \[
    \{0\}=(rs)I=s(rI)=sI
    \]
    and hence $s\in M$.     
    
    \medskip
    Since $M$ is maximal, $K=R/M$ is a field. 
    Since $MI=\{0\}$, $I$ is an $(R/M)$-module, that is 
    $I$ is a $K$-vector space. By Exercise \ref{xca:I_fg}, 
    $\dim_KI=\infty$. Let $B$ be a basis of $I$ (as a $K$-vector space) 
    and $x_0\in B$. Let $J$ be the subspace of $I$ generated by
    $B\setminus\{x_0\}$. A direct calculation
    shows that 
    $J$ is an ideal of $R$.
    %as  
    %\[
    %J=\sum_{x\in B\setminus\{x_0\}} Kx. 
    %\]
    Since $\dim_K J=\infty$, it follows that $J$ 
    is not a finitely generated ideal of $R$ 
    (Exercise \ref{xca:I_fg}). 
    This is a contradiction, because $J$ is an ideal of $R$
    such that $J\subsetneq I$. 
\end{proof}
