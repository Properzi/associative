\section{}
\label{09}

\subsection{Local rings}

In this section, we will consider arbitrary rings with one. 

\begin{definition}
    \index{Ring!local}
    A ring is said to be \emph{local} if it contains only one maximal left ideal. 
\end{definition}

Division rings are local rings. 

% \begin{exercise}
%     Let $R$ be a commutative ring with one and $P$ be a prime ideal. Let  
%     $S=R\setminus P$. Then the localization $S^{-1}R$ is a local ring with maximal ideal $S^{-1}P$. 
% \end{exercise}

\begin{theorem}
\label{thm:local}
    Let $R$ be a ring and $I=R\setminus\mathcal{U}(R)$. The following
    statements are equivalent:
    \begin{enumerate}
        \item $R$ is local.
        \item $R/J(R)$ is a division ring.
        \item $I=J(R)$.
        \item $I$ is an ideal of $R$.
    \end{enumerate}
\end{theorem}

\begin{proof}
    We first prove $1)\implies2)$. Let $M$ be the maximal left ideal of $R$. Then $J(R)=M$. 
    Let $x\not\in M$. Then $R=Rx+M$, so $1=rx+m$ for some $r\in R$ and $m\in M$. Thus  
    $r+M$ is a left inverse of $x+M$. In particular, 
    $r\not\in M$. Since $R=Rr+M$, there exists $y\in R$ such that $1=yr$. Therefore
    $y+M$ is a left inverse of $r+M$. Thus 
    \begin{align*}
    y+M&=(y+M)(1+M)=(y+M)(r+M)(x+M)\\
    &=(yr+M)(x+M)=(1+M)(x+M)=x+M
    \end{align*}
    and hence $x+M$ is invertible. 

    Now we prove $2)\implies3)$. Clearly $J(R)\subseteq I$. 
    
    Conversely, let $x\in I$. If $x\not\in J(R)$, then
    $x+J(R)\ne J(R)$. Since $R/J(R)$ is a division ring, 
    $x+J(R)\in\mathcal{U}(R/J(R))$. In particular, $1-xy\in J(R)$ and hence 
    $xy=1-(1-xy)\in\mathcal{U}(R)$. Thus $1=(xy)z=x(yz)$ for some $z\in R$ and therefore $x\not\in I$, a contradiction. 
    
    It is trivial that $3)\implies4)$. 

    Finally, we prove $4)\implies 1)$. 
    Let $M$ be a maximal left ideal of $R$. Then $M\subseteq I$. Since $M$ 
    is maximal and $I$ is in particular a left ideal of $R$, 
    it follows that $M=I$. 
\end{proof}

\begin{definition}
    \index{Idempotent}
    An element $x$ of a ring is said to be \emph{idempotent} 
    if $x^2=x$.   
\end{definition}

\index{Trivial idempotent}
Examples of idempotents are 0 and 1. 
An idempotent $x$ is said to be \emph{non-trivial} if $x\not\in\{0,1\}$. 

\begin{exercise}
\label{xca:idempotents_modpm}
    Let $p$ be a prime number and $m>0$. 
    Prove that the only idempotents of $\Z/p^m$ are 0 and 1. 
\end{exercise}


\begin{exercise}
    \label{xca:idempotents_modn}
    How many idempotent does $\Z/n$ have?
\end{exercise}

\begin{exercise}
\label{xca:lifting_idempotents}
    Let $R$ be a ring with one and $I$ be an ideal of $R$. 
    We say that an idempotent $x\in R/I$ can be lifted if $x=e+I$ for
    some idempotent $e$ of $R$. 
    Prove that if every element of $I$ is nilpotent, then every 
    idempotent of $R/I$ can be lifted. 
\end{exercise}

The previous exercise shows that if $R$ is left artinian, 
every idempotent of $R/J(R)$ can be lifted to $R$. 

\begin{lemma}
\label{lem:J(R)_nil}
    Let $R$ be a left artinian ring. Then $J(R)$ is nil. 
\end{lemma}

\begin{proof}
    Let $x\in J(R)$. The sequence $Rx\supseteq Rx^2\supseteq\cdots$ stabilizes, so
    $Rx^n=Rx^{n+1}$ for some $n$. In particular, there exists $r\in R$ 
    such that $x^n=rx^{n+1}$. This implies that $(1-rx)x^n=0$. Since $x\in J(R)$, 
    the element $1-rx$ is invertible. Hence $x^n=0$.  
\end{proof}

\begin{theorem}
\label{thm:local_idempotent}
    Let $R$ be a left artinian ring. Then $R$ is local if and only if 
    $R$ has no non-trivial idempotents. 
\end{theorem}

\begin{proof}
    Let us first prove $\implies$. For this implication, we do not need to use that 
    $R$ is left artinian. Let $x\in R$ be an idempotent. Then $x(1-x)=0$. If $x\in\mathcal{U}(R)$, then 
    $x=1$. If $1-x\in\mathcal{U}(R)$, then $x=0$. If $x\not\in \mathcal{U}(R)$ and $1-x\not\in\mathcal{U}(R)$, 
    then, since $R\setminus\mathcal{U}(R)$ is an ideal of $R$, 
    it follows that 
    $1=x+1-x\not\in\mathcal{U}(R)$, a contradiction. 

    Now we prove $\impliedby$. By the previous lemma, $J(R)$ is nil. 
    By the previous exercise, every idempotent of $R/J(R)$ can be lifted. Thus $R/J(R)$ has 
    no non-trivial idempotents. On the other hand, by Artin--Wedderburn theorem, 
    \[
    R/J(R)\simeq\prod_{i=1}^kM_{n_i}(D_i)
    \]
    for some $n_1,\dots,n_k\geq1$ and division rings $D_1,\dots,D_k$. Then 
    $k=n_1=1$, as $R/J(R)$ has no non-trivial idempotents. Since $R/J(R)$ is a division ring, 
    $R$ is local by the previous theorem. 
\end{proof}

\begin{theorem}
    The center of a local ring is local. 
\end{theorem}

\begin{proof}
    Let $R$ be a local ring. By Theorem \ref{thm:local}, $J(R)=R\setminus\mathcal{U}(R)$.  
    We need to prove that $Z(R)\setminus\mathcal{U}(Z(R))=J(Z(R))$. 
    We first note that
    \begin{equation}
    \label{eq:U(Z(R))}
        \mathcal{U}(Z(R))=Z(R)\cap\mathcal{U}(R).
    \end{equation}
    
    We claim that $Z(R)\cap J(R)\subseteq J(Z(R))$.  
    Let $x\in Z(R)\cap J(R)$. Let $z\in Z(R)$. Since $x\in J(R)$, $1-zx\in\mathcal{U}(R)$.
    Moreover, $1-zx\in Z(R)$. Thus 
    \[
    1-zx\in Z(R)\cap\mathcal{U}(R)=\mathcal{U}(Z(R)).
    \]
    Hence $x\in J(Z(R))$. 

    To prove the theorem it is enough to show that 
    $Z(R)\setminus\mathcal{U}(Z(R))=J(Z(R))$. Let us prove the non-trivial inclusion. 
    Let $x\in Z(R)\setminus\mathcal{U}(Z(R))$.  Then 
    \eqref{eq:U(Z(R))} implies that 
    $x\not\in\mathcal{U}(R)$. 
    By Theorem \ref{thm:local}, 
    $x\in J(R)$. Then $x\in J(R)\cap Z(R)\subseteq J(Z(R))$. 
\end{proof}

\begin{exercise}
\label{eq:local_center}
    Let $R$ be a local ring. Prove that 
    $Z(R)\cap J(R)=J(Z(R))$.  
\end{exercise}

\begin{exercise}
\label{xca:local_right}
    Prove that a ring is local if and only if it contains only one maximal right ideal.
\end{exercise}

\begin{exercise}
\label{xca:non_local1}
    Find a non-local ring with a unique maximal ideal. 
\end{exercise}

\begin{exercise}
\label{xca:non_local2}
    Let $R$ be a ring with at least three elements. 
    If $|\mathcal{U}(R)|=1$, then $R$ is not local. 
\end{exercise}

\index{Ring!Von Neumann regular}
A ring $R$ is said to be \emph{Von Neumann regular} if  
for every non-zero $r\in R$, $r=rxr$ for some $x\in R$. 

\begin{exercise}
\label{xca:VonNeumann_local}
    Prove that a local Von Neumann ring is a division ring. 
\end{exercise}

\begin{exercise}
\label{xca:nilp_or_unit}
    Let $R$ be a ring such that every element of $R$ is either 
    nilpotent or a unit. Prove that $R$ is local. 
\end{exercise}

\index{Ring!semilocal}
A ring $R$ is said to be \emph{semilocal} if $R/J(R)$ is left artinian. 

\begin{exercise}
\label{xca:semilocal}
    Prove the following statements:
    \begin{enumerate}
        \item Every local ring is semilocal.
        \item $R$ is semilocal if and only if $R/J(R)$ is semisimple.
        \item If $R$ has finitely many maximal left ideals, then $R$ is semilocal. 
        \item If $R_1,\dots,R_k$ are rings, then $\oplus_{i=1}^k R_i$ is semilocal
            if and only if each $R_i$ is semilocal. 
    \end{enumerate}
\end{exercise}

\begin{example}
\label{xca:semilocal_commutative}
    Let $R$ be a ring such that $R/J(R)$ is commutative. Prove
    that $R$ is semilocal if and only if $R$ has finitely many maximal ideals. 
\end{example}


% \subsection{*Hurewicz' theorem}

% \begin{theorem}[Hurewicz]
%     \label{thm:Hurewicz}
%     \index{Hurewicz' theorem}
%     Let $G$ be a group and $I$ be the augmentation ideal of $\Z[G]$. 
%     Then $G/[G,G]\simeq I/I^2$ as (abelian) groups. 
% \end{theorem}

% \begin{proof}
%     Let $\varphi\colon G\to I/I^2$, $g\mapsto g-1_G+I^2$. Since $g-1_G\in I$ for all $g\in G$, $\varphi$ is well-defined. The map $\varphi$ is a group homomorphism. Since 
%     $(g-1_G)(h-1_G)\in I^2$, 
%     \begin{align*}
%     \varphi(gh) &= gh-1_G+I^2\\
%     &=gh-1_G-(g-1_G)(h-1_G)+I^2+I^2\\
%     &=g-1_G+h-1_G+I^2\\
%     &=\varphi(g)+\varphi(h)
%     \end{align*}
%     holds for all $g,h\in G$. 

%     Since $[G,G]\subseteq\ker\varphi$, there exists a group homomorphism
%     \[
%     \overline{\varphi}\colon G/[G,G]\to I/I^2,\quad 
%     g[G,G]\mapsto g-1_G+I^2.
%     \]
%     We claim that $\overline{\varphi}$ is an isomorphism. 
%     Let us construct the inverse of $\overline{\varphi}$. Let 
%     \[
%     \psi\colon I\to G/[G,G],\quad 
%     \sum_{g\in G}m_g(g-1_G)\mapsto \left(\prod_{g\in G}g^{m_g}\right)[G,G].
%     \]
%     Since $G/[G,G]$ is abelian, the map $\psi$ is well-defined, that is
%     the order of the factors in $\prod_{g\in G}g^{m_g}$ does not matter. Note that 
%     $I^2\subseteq\ker\psi$, as 
%     $\{(g-1_G)(h-1_G):g,h\in G\}$ generates the additive group $I^2$ 
%     and 
%     \begin{align*}
%         \psi((g-1_G)(h-1_G))&=\psi( (gh-1_G)-(g-1_G)-(h-1_G))\\
%         &=(ghg^{-1}h^{-1})[G,G]\\
%         &=[G,G].
%     \end{align*}
%     Therefore there exists a group homomorphism
%     \[
%     \overline{\psi}\colon I/I^2\to G/[G,G],\quad 
%     \sum_{g\in G}m_g(g-1_G)+I^2\mapsto \left(\prod_{g\in G}g^{m_g}\right)[G,G].
%     \]
%     A direct calculation shows that $\overline{\psi}$ is the inverse 
%     of $\overline{\varphi}$. 
% \end{proof}


% \subsection{*When a group algebra is reduced?}
% \
% \begin{exercise}
%     Is the ring $\C[\Z/2]$ reduced? 
% \end{exercise}

% % Let $G=\langle g:g^2=1\rangle$. If $(a+bg)^2=0$, then
% % $(a^2+b^2)+(2ab)g=0$. Thus $a=b=0$. 

% \begin{problem}
% \label{prob:reduced}
%     Let $G$ be a torsion-free group. Is
%     $K[G]$ is reduced?
% \end{problem}

% Problem \ref{prob:reduced} is related to other important
% open problems about group algebras 
% (e.g. zero-divisors, units, 
% indempotents and semisimplicity of group
% rings).

% \begin{exercise}
% \label{xca:reduced_central}
%     Prove that idempotents of reduced rings are central. 
% \end{exercise}

% The previous exercise is used to solve the following problem.

% \begin{exercise}
% \label{xca:x^3=x}
%     Let $R$ be a ring such that $x^3=x$ for all $x\in R$. Prove that
%     $R$ is commutative. 
% \end{exercise}

% Exercise \ref{xca:x^3=x} is hard. 
% Even harder is the following exercise:

% \begin{exercise}
% \label{xca:x^4=x}
%     Let $R$ be a ring such that $x^4=x$ for all $x\in R$. Prove
%     that $R$ is commutative. 
% \end{exercise}

% %Other exercises about reduced rings. 

% %\begin{exercise}
% %\label{xca:reduced}
% %    Prove that a ring is reduced if 
% %    and only it has no non-zero nilpotent elements. 
% %\end{exercise}

% \begin{exercise}
% \label{xca:reduced=>semiprime}
%     Reduced rings are semiprime.
% \end{exercise}
 
% \begin{theorem}
% \label{thm:reduced}
%     Let $K$ be a field and $G$ be a group. If $K[G]$
%     is reduced, then every finite subgroup of $G$ is normal. 
% \end{theorem}

% \begin{proof}
%     Let $H=\{h_1,\dots,h_n\}$ be a finite normal subgroup of $G$. 
%     We claim that $n=|H|$ is invertible in $K$. If $\ch K=0$, this 
%     is clear. If $\ch K=p>0$ and $n$ is not invertible in $K$, 
%     then $p$ divides $n=|H|$. By Cauchy's theorem, 
%     there exists an element $h\in H$ of order $n$, that is 
%     $|h|=n$. Since $(1-h)^p=1-h^p=0$ and $K[G]$ is reduced,
%     $h=1$, a contradiction. 
    
%     Let $\alpha=\frac{1}{n}\sum_{i=1}^nh_i\in K[G]$. Then
%     \[
%     \alpha^2=\frac{1}{n^2}\sum_{i=1}^n\sum_{j=1}^nh_ih_j
%     =\frac{1}{n^2}\sum_{i=1}^nn\alpha=\alpha.
%     \]
%     Thus $\alpha$ is idempotent. As idempotent 
%     element of reduced rings are central (Exercise \ref{xca:reduced_central}), 
%     $g\alpha g^{-1}=\alpha$ for all $g\in G$. If $g\in G$, 
%     then 
%     \[
%     \sum_{i=1}^n gh_ig^{-1}=\sum_{i=1}^n h_i.
%     \]
%     It follows that $H$ is normal in $G$, 
%     as for each $i\in\{1,\dots,n\}$ 
%     there exists $j\in\{1,\dots,n\}$ such that 
%     $gh_ig^{-1}=h_j\in H$. 
% \end{proof}

% \begin{example}
%     If $K$ is a field, then $K[\Sym_3]$ is not reduced. 
%     In fact, 
%     if 
%     \[
%     \alpha=(12)+(123)-(132)-(13),
%     \]
%     then 
%     $\alpha^2=0$. 
% \end{example}

% \begin{exercise}
%     Prove that the converse of Theorem \ref{thm:reduced} 
%     does not hold. 
% \end{exercise}

