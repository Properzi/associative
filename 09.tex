\chapter{}

\topic{Rickart's theorem}

Let $K$ be a field and $G$ be a group. The \textbf{group algebra} $K[G]$ 
is the vector space (over $K$) with basis $\{g:g\in G\}$ 
and the algebra structure given by the multiplication
\[
	\left(\sum_{g\in G}\lambda_gg\right)\left(\sum_{h\in G}\mu_hh\right)
	=\sum_{g,h\in G}\lambda_g\mu_h(gh).
\]
Note that every element of $K[G]$ is a finite sum of the form $\sum_{g\in G}\lambda_gg$.

\begin{exercise}
\label{xc:K[G]notsimple}
    If $G$ is non-trivial, then $K[G]$ is not simple. 
\end{exercise}

\begin{exercise}
	Let $G=C_n$ be the (multiplicative) cyclic group of order $n$. Prove that 
	$K[G]\simeq K[X]/(X^n-1)$. 
\end{exercise}

\begin{exercise}
	Let $G$ be a finitely-generated torsion-free abelian group. Prove that 
	$K[G]$ is a domain. 
\end{exercise}

\begin{exercise}
	Let $G$ be a group and $H$ be a subgroup of $G$. Let $\alpha\in K[H]$. Prove that 
    $\alpha$ is invertible (resp. left zero divisor) in $K[H]$ if and only if 
	$\alpha$ is invertible (resp. left zero divisor) in
	$K[G]$.
\end{exercise}

\begin{exercise}
	Let $G$ be a group and $\alpha=\sum_{g\in G}\lambda_gg\in K[G]$.  
	The \textbf{support} of $\alpha$ is the set 
	\[
		\supp\alpha=\{g\in G:\lambda_g\ne 0\}.
	\]
	Prove that if $g\in G$, then 
	$\supp(g\alpha)=g(\supp\alpha)$ and $\supp(\alpha g)=(\supp\alpha)g$.
\end{exercise}

% El objetivo de esta sección es calcular el radical de Jacobson del álgebra de
% grupo de un grupo finito. Comenzamos con un ejemplo:

\begin{exercise}
	Let $G=C_2=\langle g\rangle\simeq\Z/2$ the (multiplicative) 
	group with two elements. Note that every element of $K[G]$ is of the form
	$a1+bg$ for some $a,b\in K$. Prove the following statements:
	\begin{enumerate}
	    \item If the characteristic of $K$ is different from two, then 
	    \[
		K[G]\to K\times K,
		\quad
		a1+bg\mapsto (a+b,a-b),
	\]
	is an algebra isomorhism. 
	\item If the characteristic of $K$ is two, then 
	\[
	K[G]\to \begin{pmatrix}
			K & K\\
			0 & K
		\end{pmatrix},
		\quad
		a1+bg\mapsto\begin{pmatrix}
			a+b & b\\
			0 & a+b
		\end{pmatrix},
	\]
	is an algebra isomorphism. 
	\end{enumerate}
\end{exercise}

If $A$ is an algebra over $K$ and $\rho\colon G\to \mathcal{U}(A)$
is a group homomorphism, where $\mathcal{U}(A)$ is the group of units of $A$, then 
the map \[
	K[G]\to A,\quad 
\sum_{g\in G}\lambda_gg\mapsto\sum_{g\in G}\lambda_g\rho(g),
\]
is an algebra homomorphism. 

\begin{exercise}
	Let $G=C_3$ be the (multiplicative) group of three elements. Prove that
	$\R[G]\simeq\R\times\C$.
% 	Escribamos $G=\langle g:g^3=1\rangle$ y sea 
% 	\[
% 		\varphi\colon\R[G]\to\R\times\C,
% 		\quad
% 		g\mapsto (1,\omega),
% 	\]
% 	donde $\omega$ es una raíz cúbica primitiva de la unidad. Entonces
% 	$\varphi$ es inyectivo pues
% 	$0=\varphi(a1+bg+cg^2)=(a+b+c,a+b\omega+c\omega^2)$ implica que $a=b=c=0$.
% 	Luego $\varphi$ es un isomorfismo pues
% 	$\dim_\R\R[G]=\dim_\R(\R\times\C)=3$. 
\end{exercise}

\begin{exercise}
	Let $G=\langle r,s:r^3=s^2=1,\,srs=r^{-1}\rangle$ be the dihedral group of six elements. 
	Prove the following statements:
	\begin{enumerate}
	    \item $\C[G]\simeq\C\times\C\times M_2(\C)$.
	    \item $\Q[G]\simeq\Q\times\Q\times M_2(\Q)$.
	\end{enumerate}  
% 	Sea $\omega$ una raíz cúbica de la unidad y sean  
% 	\[
% 		R=\begin{pmatrix}
% 			\omega & 0\\
% 			0 & \omega^2
% 		\end{pmatrix},
% 		\quad
% 		S=\begin{pmatrix}
% 			0 & 1\\
% 			1 & 0
% 		\end{pmatrix}.
% 	\]
% 	Un cálculo sencillo muestra que $R^2=S^2=I$ y que $SRS=R^{-1}$. Sea
% 	\[
% 		\varphi\colon\C[G]\to\C\times\C\times M_2(\C),\quad
% 		r\mapsto (1,1,R),\quad
% 		s\mapsto (1,-1,S).
% 	\]
% 	Es fácil ver que $\varphi$ es un morfismo de álgebras. Veamos que es
% 	biyectivo. Como $\dim_{\C}\C[G]=\dim_{\C}(\C\times\C\times M_2(\C))=6$,
% 	basta ver que $\varphi$ es inyectivo. Si 
% 	\[
% 		\alpha=a_0+a_1r+a_2r^2+(b_0+b_1r+b_2r^2)s\in\ker\varphi,
% 	\]
% 	entonces 
% 	\[
% 		0=\varphi(\alpha)=\left(u,v,\begin{pmatrix} \alpha_{11} & \alpha_{12}\\\alpha_{21}&\alpha_{22}\end{pmatrix}\right), 
% 	\]
% 	donde
% 	\begin{align*}
% 		&u = a_0+a_1+a_2+b_0+b_1+b_2, && v = a_0+a_1+a_2-b_0-b_1-b_2,\\
% 		&\alpha_{11}=a_0+a_1\omega+a_2\omega^2, && \alpha_{12}=b_0+b_1\omega+b_2\omega^2,\\
% 		&\alpha_{21}=b_0+b_2\omega+b_1\omega^2, && \alpha_{22}=a_0+a_2\omega+a_1\omega^2.
% 	\end{align*}
% 	Un cálculo sencillo muestra que estas ecuaciones implican que
% 	$\alpha=0$ y luego $\varphi$ es inyectiva.  
\end{exercise}

We now consider the following problem. It is known as Jacobson's semisimplicity problem. 

\begin{openproblem}
\label{Jacobson's semisimplicity problem}
Let $G$ be a group and $K$ be a field. When $J(K[G])=\{0\}$?
\end{openproblem}

As an application of Amitsur's theorem we prove that 
complex group algebras have null Jacobson radical.
This is known as 
Rickart's theorem. The original proof found by Rickart 
uses complex analysis. Here, however, 
we present an algebraic proof. 


\begin{theorem}[Rickart]
\index{Rickart's theorem}
\label{thm:Rickart}
    Let $G$ be a group. Then $J(\C[G])=\{0\}$.
\end{theorem}

To prove the theorem we need a lemma.

\begin{lemma}
Let $G$ be a group. Then $J(\C[G])$ is nil.        
\end{lemma}

\begin{proof}
    We need to show that every element of $J(\C[G])$ is nilpotent. 
    If $G$ is countable, then the result follows from Amitsur's theorem. So assume that 
    $G$ is not countable. Let $\alpha\in J(\C[G])$, say
    \[
    \alpha=\sum_{i=1}^n\lambda_ig_i,
    \]
    where $\lambda_1,\dots,\lambda_n\in\C$ and $g_1,\dots,g_n\in G$. Let $H=\langle g_1,\dots,g_n\rangle$.
    Then $\alpha\in \C[H]$ and $H$ is countable. We claim that $\alpha\in J(\C[H])$. Decompose
    $G$ as a disjoint union 
    \[
    G=\bigcup_\lambda x_\lambda H
    \]
    of cosets of $H$ in $G$. Then $\C[G]=\bigoplus_\lambda x_\lambda\C[H]$ and
    hence $\C[G]=\C[H]\oplus K$ for some right module $K$ over $\C[H]$ (this follows
    from the fact that one of the cosets is that of $H$). Since $\alpha\in J(\C[G])$, for each 
    $\beta\in\C[H]$ there exists $\gamma\in\C[G]$ such that 
    $\gamma(1-\beta\alpha)=1$. Write $\gamma=\gamma_1+\kappa$ for $\gamma_1\in\C[H]$ and $\kappa\in K$. Then
    \[
    1=\gamma(1-\beta\alpha)=\gamma_1(1-\beta\alpha)+\kappa(1-\beta\alpha)
    \]
    and hence $\kappa(1-\beta\alpha)\in K\cap \C[H]=\{0\}$, as $\beta\in\mathbb{C}[H]$. 
    Since $1=\gamma_1(1-\beta\alpha)$, it follows that
    $\alpha\in J(\C[H])$ and the lemma follows from Amitsur's theorem.  
\end{proof}

We now prove the theorem. 

\begin{proof}[Proof of Theorem \ref{thm:Rickart}]
    For $\alpha=\sum_{i=1}^n\lambda_ig_i\in\C[G]$ let 
    \[
    \alpha^*=\sum_{i=1}^n\overline{\lambda_i}g_i^{-1}.
    \]
    Then $\alpha\alpha^*=0$ if and only if $\alpha=0$ and, moreover, 
    $(\alpha\beta)^*=\beta^*\alpha^*$ for all $\beta\in\C[G]$. 
    Assume that $J(\C[G])\ne\{0\}$ and let $\alpha\in J(\C[G])\setminus\{0\}$. Then
    $\beta=\alpha\alpha^*\in J(\C[G])$, as $J(\C[G])$ is an ideal of $\C[G]$. Moreover, the previous 
    lemma implies that $\beta$ is nilpotent. Note that $\beta\ne 0$, as $\alpha\ne0$. Now  
    \[
    (\beta^m)^*=(\beta^*)^m=\beta^m
    \]
    for all $m\geq1$. If there exists $k\geq2$ such that $\beta^k=0$ and $\beta^{k-1}\ne 0$, then
    \[
    \beta^{k-1}\left(\beta^{k-1}\right)^*=\beta^{2k-2}=0
    \]
    and hence $\beta^{k-1}=0$, a contradiction. Thus $\beta=0$ and therefore $\alpha=0$. 
\end{proof}

\begin{exercise}
	If $G$ is a group, then $J(\R[G])=0$. 
\end{exercise}

% To obtain a consequence of Rickart's theorem we need two lemmas. 

% \begin{lemma}[Nakayama]
% 	\label{lem:Nakayama}
% 	\index{Nakayama's lemma}
% 	Let $R$ be a unitary ring and $M$ be a finitely generated module. If 
% 	$J(R)\cdot M=M$, then $M=\{0\}$.
% \end{lemma}

% \begin{proof}
%     Since $M$ is finitely generated, we may assume that 
% 	$M=(x_1,\dots,x_n)$. Since $x_n\in M=J(R)\cdot M$, 
% 	there exist $r_1,\dots,r_n\in J(R)$ such that $x_n=r_1\cdot x_1+\cdots+r_n\cdot x_n$, that is 
% 	$(1-r_n)\cdot x_n=\sum_{j=1}^{n-1}r_j\cdot x_j$. 
% 	Since $1-r_n$ is invertible, there exists $s\in R$ such that $s(1-r_n)=1$. Thus 
% 	$x_n=\sum_{j=1}^{n-1}(sr_j)\cdot x_j$ 
% 	and hence $M=(x_1,\dots,x_{n-1})$. Repeating this procedure several times 
% 	one obtains $M=\{0\}$.
% \end{proof}

% \begin{lemma}
% 	\label{lem:Rickart}
% 	Let $\iota\colon R\to S$ be a homomorphism of unitary rings. If	
% 	\[
% 	S=\iota(R)x_1+\cdots+\iota(R)x_n,
% 	\]
% 	where each $x_j$ is such that $x_jy=yx_j$ for all $y\in\iota(R)$, then 
% 	$\iota(J(R))\subseteq J(S)$.
% \end{lemma}

% \begin{proof}
% 	We claim that $J=\iota(J(R))$ acts trivially on each simple $S$-module $M$.
% 	If is $M$ is a simple module over $S$, then, in particular, $M=S\cdot m$ for some $m\ne0$. 
% 	Now $M$ is a module over $R$ with $r\cdot m=\iota(r)\cdot m$. Since 
% 	\[
% 		M=S\cdot m=(\iota(R)x_1+\cdots+\iota(R)x_n)\cdot m=\iota(R)\cdot (x_1\cdot m)+\cdots+\iota(R)\cdot (x_n\cdot m),
% 	\]
% 	it follows that 
% 	$M$ is finitely generated as a module over $\iota(R)$. Moreover, 
% 	\[
% 	J(R)\cdot
% 	M=J\cdot M=\iota(J)\cdot M
% 	\]
% 	is an $S$-submodule of $M$, as 
% 	\[
% 		x_j\cdot (J\cdot M)=(x_j J)\cdot M=(J x_j)\cdot M=J\cdot (x_j\cdot M)\subseteq J\cdot M.
% 	\]
% 	Since $M\ne\{0\}$, Nakayama's lemma implies that $J(R)\cdot M\subsetneq M$. The simplicity of 
% 	the $S$-module $M$ implies that $J(R)\cdot M=\{0\}$.
% \end{proof}

% We now obtain the following consequence of Rickart's theorem. 

% \begin{theorem}
% 	If $G$ is a group, then $J(\R[G])=0$. 
% \end{theorem}

% \begin{proof}
% 	Let $\iota\colon \R[G]\to\C[G]$ be the canonical inclusion. Since 
% 	\[
% 	\C[G]=\R[G]+i\R[G],
% 	\]
% 	Lemma~\ref{lem:Rickart} and Rickart's theorem imply that 
% 	$\iota(J(\R[G]))\subseteq J(\C[G])=0$. Thus $J(\R[G])=0$, as $\iota$ is injective. 
% \end{proof}

We now characterize when complex group algebras 
are left artinian. For that purpose
we need a lemma. This is similar to one of the implications proved in Proposition \ref{pro:semisimple}. However,
in the arbitrary setting we are considering, we need to use Zorn's lemma. 

\begin{lemma}
    Let $M$ be a semisimple module and $N$ be a submodule. 
    Then $N$ is a direct summand.
\end{lemma}

\begin{proof}[Sketch of the proof]
    Let $M=\oplus_{i\in I}M_i$ be a direct sum of simple modules  
    and let $i\in I$. 
    Since $N\cap M_i$ is a submodule of $M_i$ and $M_i$ is simple, it follows
    that $N\cap M_i=\{0\}$ or $N\cap M_i=M_i$. If
    $N\cap M_i=M_i$ for all $i\in I$, then $N=M$ and the lemma is proved. So we may assume
    that there exists $i\in I$ such that $N\cap M_i=\{0\}$. Let $X$ be the set
    of subsets $J$ of $I$ such that $N\cap (\oplus_{j\in J}M_j)=\{0\}$. Our assumptions
    imply that $X$ is non-empty. Zorn's lemma implies the existence of 
    a maximal element $K$. Let $N_1=\oplus_{k\in K}M_k$. We claim that
    $N\oplus N_1=M$. If not, there exists $i\in I$ such that
    $M_i\not\subseteq N\oplus N_1$. The simplicity of $M_i$ implies that
    $M_i\cap (N\cap N_1)=\{0\}$, which contradicts the maximality of $K$. 
\end{proof}

A direct application of the lemma proves that
complex group algebras of infinite groups are never semisimple. 

\begin{proposition}
    \label{pro:KGsemisimple}
    If $G$ is an infinite group, then $\C[G]$ is not semisimple. 
\end{proposition}

\begin{proof}
	Assume that $R=\C[G]$ is semisimple.  Let $I$ 
	be the augmentation ideal of $R$, that is
	\[
	I=\left\{\alpha=\sum_{g\in G}\lambda_gg\in R:\sum_{g\in G}\lambda_g=0\right\}.
	\]
	By the previous lemma, 
	there exists exists a non-zero ideal $J$ such that 
	$R=I\oplus J$. Since $R$ is unitary, there exist $e\in I$ and $f\in J$ such that
	$1=e+f$. If
	$x\in I$, then $x=xe+xf$ and hence $xf=x-xe\in I\cap J=\{0\}$. Since 
	$x=xe$ for all $x\in I$, it follows that $e=e^2$. Similarly one proves
	that $f^2=f$. Moreover, $ef=0$, as $ef\in I\cap J=\{0\}$.  Since $I$ 
	is the augmentation ideal of $R$ and $If=(Re)f=R(ef)=\{0\}$, we conclude that
	$(g-1)f=0$
	for all $g\in G$, as $g-1\in I$. If $f=\sum_{h\in
	G}\lambda_hh$ (finite sum), then  
	\[
	f=gf=\sum_{h\in G}\lambda_h(gh)=\sum_{h\in
	G}\lambda_{g^{-1}h}h.
	\]
	Thus $\lambda_h=\lambda_{g^{-1}h}$ for all $g,h\in G$. Since $G$ 
	is infinite, some $\lambda_g=0$ and hence $f=0$. Thus $e=1$ and $I=\C[G]$, a contradiction. 
% 	If $f=0$, then $e=1$ and $I=\C[G]$, a contradiction.  
% 	, a contradiction because 
% 	$f\ne 0$ implies that the sum that defines $f$ should be an infinite sum.
\end{proof}

% The ideal $I(G)$ used in the proof of the previous proposition 
% is known as the \textbf{augmentation ideal} 
% of $\C[G]$.

\begin{theorem}
	Let $G$ be a group. Then $\C[G]$ 
	is left artinian if and only if 
	$G$ is finite. 
\end{theorem}

\begin{proof}
    If $G$ is finite, then $\C[G]$ is left artinian because $\dim\C[G]=|G|<\infty$. So assume that 
    $G$ is infinite. By Rickart's theorem,   
	$J(\C[G])=0$. Moreover, $\C[G]$
	is not semisimple by the previous proposition. Thus
	$\C[G]$ is not left artinian by Theorem~\ref{thm:SSartin=J}.
\end{proof}

\topic{Maschke's theorem}

We now present another instance of the Jacobson semisimplicity problem.
In this case, our result is for finite groups. 

\begin{theorem}[Maschke]
\index{Maschke's theorem}
	Let $G$ be a finite group. Then $J(K[G])=0$ if and only 
	if the characteristic of $K$ is zero 
	or does not divide the order of $G$. 
\end{theorem}

\begin{proof}
	Assume that $G=\{g_1,\dots,g_n\}$, where $g_1=1$. Let 
	\[
	\rho\colon K[G]\to K,
	\quad
	\alpha\mapsto\trace(L_{\alpha}),
	\]
	where 
	$L_{\alpha}(\beta)=\alpha\beta$. Then 
	\[
	\rho(g_i)=\begin{cases}
	    n & \text{if $i=1$,}\\
	    0 & \text{if $2\leq i\leq n$},
	\end{cases}
	\]
	as $L_{g_i}(g_j)=g_{i}g_j\ne g_j$, the matrix of 
	$L_{g_i}$ in the basis $\{g_1,\dots,g_n\}$ contains zeros in the main diagonal. 

	Assume that $J=J(K[G])$ is non-zero and let 
	$\alpha=\sum_{i=1}^n\lambda_ig_i\in J\setminus\{0\}$. Without loss of generality
	we may assume that $\lambda_1\ne 0$ (if $\lambda_1=0$ there exists some 
	$\lambda_i\ne 0$ and we need to take $g_i^{-1}\alpha\in J$). Then 
	\[
		\rho(\alpha)=\sum_{i=1}^n \lambda_i\rho(g_i)=n\lambda_1.
	\]
	Since $G$ is finite, $K[G]$ is a finite-dimensional algebra and hence 
	$K[G]$ is left artinian. Since $J$ is a nilpotent ideal, 
	in particular, $\alpha$ is a nilpotent element. Then 
	$L_{\alpha}$ is nilpotent and hence $0=\rho(\alpha)=n\lambda_1$. This implies that
	the characteristic of the field $K$ divides $n$. 

	Conversely, let $K$ be a field of prime characteristic and that this primes divides 
	$n$. Let $\alpha=\sum_{i=1}^ng_i$. Since $\alpha
	g_j=g_j\alpha=\alpha$ for all $j\in\{1,\dots,n\}$, the set 
	$I=K[G]\alpha$ is an ideal of $K[G]$. Since, moreover,   
	\[
		\alpha^2=\sum_{i=1}^n g_i\alpha=n\alpha=0
	\]
	in the field $K$, it follows that $I$ is a nilpotent non-zero ideal. Thus $J(K[G])\ne\{0\}$, 
	as Proposition~\ref{pro:nilJ} yields $I\subseteq J(K[G])$.
\end{proof}

Since the Jacobson radical of a group algebra of a finite group contains 
every nil left ideal, the following consequence of the theorem follows immediately:

\begin{corollary}
	\label{cor:GfinitoNOnil}
	Let $G$ be a finite group. Then $K[G]$ does not contain non-zero nil left ideals. 
\end{corollary}


% \begin{proof}
% 	Es consecuencia inmediata del teorema de Maschke ya que $J(K[G])$ contiene a
% 	todo ideal a izquierda nil.	
% \end{proof}

%\index{Anillo!semisimple}
%Recordemos que un anillo unitario $R$ se dice \textbf{semisimple} si para cada
%ideal $I$ de $R$ existe un ideal $J$ de $R$ tal que $R=I\oplus J$.
%
%%\begin{corollary}
%%	Sea $G$ un grupo finito y $K$ un cuerpo de característica coprima con el
%%	orden de $G$. Entonces $K[G]$ es semisimple.
%%\end{corollary}
%%
%%\begin{proof}
%%	
%%\end{proof}
%
%\begin{theorem}
%	Si $G$ es un grupo infinito, entonces $K[G]$ nunca es semisimple.
%\end{theorem}
%
%\begin{proof}
%	Sea $R=K[G]$ y supongamos que $R$ es semisimple.  Si $I$ es el ideal de
%	aumentación de $R$, existe un ideal no nulo $J$ de $R$ tal que $R=I\oplus
%	J$. Como $R$ es unitario, existen $e\in I$, $f\in J$ tales que $1=e+f$. Si
%	$x\in I$, entonces $x=xe+xf$ y luego $xf=x-xe\in I\cap J=\{0\}$. Como
%	entonces $x=xe$ para todo $x\in I$, en particular $e_1=e_1^2$. Análogamente
%	vemos que $e_2^2=e_2$. Además $ef=0$ pues $ef\in I\cap J=\{0\}$.  Como $I$
%	es el ideal de aumentación y $If=(Re)f=R(ef)=0$, se concluye que $(g-1)f=0$
%	para todo $g\in G$ pues $g-1\in I$. Si suponemos que $f=\sum_{h\in
%	G}\lambda_hh$, entonces 
%	\[
%	f=gf=\sum_{h\in G}\lambda_h(gh)=\sum_{h\in
%	G}\lambda_{g^{-1}h}h.
%	\]
%	Luego $\lambda_h=\lambda_{g^{-1}h}$ para todo $g,h\in G$, una contradicción
%	pues como $f\ne 0$ la suma que define a $f$ es infinita. 
%\end{proof}


