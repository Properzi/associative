\chapter{}

\topic{Anillos semiprimitivos y semiprimos}

\begin{definition}
	\index{Anillo!semiprimitivo}
	\index{Anillo!semisimple Jacobson}
	Un anillo $R$ se dice \textbf{semiprimitivo} (o semisimple Jacobson) si
	$J(R)=0$.
\end{definition}

\begin{example}
	Si $R$ es primitivo entonces es semiprimitivo. En efecto, como $R$ es
	primitivo, $\{0\}$ es un ideal primitivo y luego, como $J(R)$ es la
	intersección de los ideales primitivos de $R$, se concluye que $J(R)=0$.
\end{example}

\begin{example}
	Si $R=\prod_{i\in I}R_i$ es producto directo de anillos semiprimitivos,
	entonces $R$ es semiprimitivo pues 
	\[
		J(R)=J\left(\prod_{i\in I}R_i\right)=J\left(\prod_{i\in I}J(R_i)\right)=0.
	\]
\end{example}

\begin{example}
	$\Z$ es semiprimitivo pues $J(\Z)=\cap_{p}\Z/p=\{0\}$.
\end{example}

\begin{example}
	Sea $R=C[a,b]$ el anillo de funciones $f\colon [a,b]\to\R$ continuas. Como
	$R$ es un anillo unitario, $J(R)$ es la intersección de los ideales
	maximales de $R$. Todo ideal maximal de $R$ es de la forma
	\[
		U_c=\{f\in C[a,b]:f(c)=0\}
	\]
	para algún $c\in[a,b]$. En efecto, es fácil ver que cada $U_c$ es un ideal;
	$U_c$ es maximal pues $C[a,b]/U_c\simeq\R$.  Luego $J(R)=\cap_{a\leq c\leq
	b}U_c=0$.
\end{example}

\begin{theorem}
	\label{thm:semiprimitivo}
	Si $R$ es un anillo, entonces $R/J(R)$ es semiprimitivo. 
\end{theorem}

\begin{proof}
	Si $R$ es un anillo radical, el resultado es trivial. Supongamos entonces
	que $J(R)\ne R$ y sea $M$ un módulo simple. Entonces $M$ es un
	$R/J(R)$-módulo simple con
	\[
		(x+J(R))m=xm,\quad
		x\in R,\,m\in M.
	\]
	Si $x+J(R)\in J(R/J(R))$ entonces $xM=(x+J(R))M=0$. Luego $x\in J(R)$ pues
	$x$ anula a cualquier módulo simple de $R$.
\end{proof}

%El teorema de densidad de Jacobson nos permite entonces obtener el siguiente resultado:
%
%\begin{theorem}
%	Sea $R$ un anillo no radical. Entonces $R/J(R)$ es isomorfo a un producto
%	subdirecto de anillos densos en espacios vectoriales sobre anillos de
%	división.	
%\end{theorem}
%
%\begin{proof}
%	Si $R$ no es radical, $J(R)\ne R$. Luego $R/J(R))$ es semiprimitivo por el
%	teorema~\ref{thm:semiprimitivo}. El teorema~\ref{thm:subdirecto} y el
%	teorema de densidad de Jacobson completan la demostración del teorema.
%\end{proof}


\begin{definition}
	\index{Producto subdirecto de anillos}
	Sea $\{R_i:i\in I\}$ una familia de anillos. Un subanillo $R$ de
	$\prod_{i\in I}R_i$ se dice un \textbf{producto subdirecto} de los $R_j$ si
	cada $\pi_j\colon R\to R_j$ es sobreyectiva. 
\end{definition}

El siguiente teorema justifica que indistintamente llamemos anillos
semiprimitivos a los anillos semisimples Jacobson:

\begin{theorem}
	\label{thm:subdirecto}
	Sea $R$ un anillo no nulo. Entonces $R$ semiprimitivo si y sólo si $R$ es
	isomorfo a un producto subdirecto de anillos primitivos.
\end{theorem}

\begin{proof}
	Supongamos que $R$ es semiprimitivo y sea $\{P_i:i\in I\}$ la familia de
	ideales primitivos de $R$. Cada $R/P_j$ es primitivo y
	$\{0\}=J(R)=\cap_{i\in I}P_i$. Para cada $j$, sean $\lambda_j\colon R\to
	R/P_j$ y $\pi_j\colon \prod_{i\in I}R/P_i\to R/P_j$ los morfismos
	canónicos. La función
	\[
		\phi\colon R\to\prod_{i\in I}R/P_i,\quad
		r\mapsto \{\lambda_i(r):i\in I\},
	\]
	es un morfismo inyectivo de anillos tal que $\pi_j\phi(R)=R/P_j$ para todo
	$j$.

	Supongamos ahora que $R$ es isomorfo a un producto subrirecto de anillos
	$R_j$ primitivos y sea $\varphi\colon R\to\prod_{i\in I}R_i$ un morfismo
	inyectivo tal que $\pi_j(\varphi(R))=R_j$ para todo $j$. Para cada $j$ sea
	$P_j=\ker\pi_j\varphi$. Como $R/P_j\simeq R_j$, cada $P_j$ es un ideal
	primitivo. Si $x\in\cap_{i\in I}P_i$ entonces $\varphi(x)=0$ y luego $x=0$.
	Luego $J(R)\subseteq\cap_{i\in I} P_i=0$. 
\end{proof}

\begin{example}
	El anillo $\Z$ es isomorfo a un producto subdirecto de los cuerpos $\Z/p$
	con $p$ primo.
\end{example}

\begin{example}
	El anillo $C[a,b]$ es isomorfo a un producto subdirecto de los cuerpos
	$C[a,b]/U_c\simeq\R$.
\end{example}

\begin{definition}
	Un anillo $R$ se dice \textbf{semiprimo} si para todo $a\in R$ tal que
	$aRa=0$ se tiene que $a=0$.
\end{definition}

\begin{lemma}
	Sea $R$ un anillo. Son equivalentes:
	\begin{enumerate}
		\item $R$ es semiprimo.
		\item Si $I$ es un ideal a izquierda tal que $I^2=0$ entonces $I=0$.
		\item Si $I$ es un ideal tal que $I^2=0$ entonces $I=0$.
		\item $R$ no tiene ideales nilpotentes no nulos. 
	\end{enumerate}
\end{lemma}

\begin{proof}
	Veamos que $(1)\implies(2)$. Si $I^2=0$ y $x\in I$, entonces $xRx\subseteq I^2=0$ y
	luego $x=0$. Las implicaciones $(2)\implies(3)$ y $(4)\implies(3)$ son triviales. Veamos que
	$(3)\implies(4)$.  Si $I$ es un ideal nilpotente no nulo, sea $n\in\Z_{>0}$ 
	minimal tal que $I^n=0$.  Como $(I^{n-1})^2=0$, $I^{n-1}=0$, una
	contradicción. Por último veamos que $(3)\implies(1)$. Sea $a\in R$ tal que
	$aRa=0$. Entonces $I=RaR$ es un ideal de $R$ tal que $I^2=0$. Por hipótesis, $RaR=I=0$. Luego
	$Ra$ y $aR$ son ideales tales que $(Ra)R=R(aR)=0$. Esto implica que $\Z a$ es un ideal de $R$
	tal que $(\Z a)R=0$ y luego $a=0$.
\end{proof}

\begin{example}
	Un anillo conmutativo es semiprimo si y sólo si no tiene elementos
	nilpotentes no nulos.
\end{example}


\begin{proposition}
	El anillo $\C[G]$ es semiprimo.
\end{proposition}

\begin{proof}
	Como $J(\C[G])=0$ por el teorema de Rickart y además el radical de Jacboson
	contiene a todo ideal nil por la proposición~\ref{pro:nilJ}, se deduce que
	$\C[G]$ no tiene ideales nil no triviales. Tampoco tiene entonces ideales
	nilpotentes no triviales y luego $\C[G]$ es semiprimo.
\end{proof}

\begin{exercise}
	Demuestre que $Z(\C[G])$ es semiprimo.
\end{exercise}

% tomar $\alpha$ tal que $\alpha^2=0$ y sea $A=K[G]\alpha$. Como $A^2=0$, $A=0$ y entonces $\alpha=0$.

\begin{example}
	Sea $D$ un anillo de división. Entonces $D[X]$ es semiprimo.
\end{example}

\begin{example}
	Sea $D$ un anillo de división. Entonces $D[[X]]$ es semiprimo y no es
	semiprimitivo.
\end{example}




%\section{Anillos semiprimitivos}
%
%\begin{lemma}
%	\label{lem:Iunitario}
%	Sea $R$ un anillo y sea $I$ un ideal de $R$ unitario. Sea $e\in I$ la
%	unidad de $I$. Entonces $e$ es un idempotente central de $R$, $I=eR$ y
%	existe un ideal $J$ de $R$ tal que $R=I\oplus J$. Además $R\simeq I\times
%	J$.
%\end{lemma}
%
%\begin{proof}
%	Como $e\in I$, $eR\subseteq I$. Luego $I=eR$ pues $I=eI\subseteq eR$. Como
%	$ex\in I$ y $xe\in I$ para todo $x\in R$, $ex=(ex)e$ y $xe=e(xe)$. Luego
%	$ex=xe$ y entonces $e$ es central e idempotente. Sea $J=\{x-ex:x\in R\}$.
%	Es fácil demostrar que $J$ es un ideal tal que $R=I\oplus J$. Además
%	$R\simeq I\times J$, via $x\mapsto (ex,x-ex)$,
%\end{proof}
%
%A continuación daremos una demostración muy sencilla del teorema de Wedderburn
%en el caso de álgebras de dimensión finita.
%
%\begin{theorem}[Artin--Wedderburn]
%	Sea $R$ un anillo artiniano a izquierda y no nulo. Entonces $R$ es
%	semiprimo si y sólo si existen $n_1,\dots,n_r\in\N$ y existen anillos de
%	división $D_1,\dots,D_r$ tales que $R\simeq M_{n_1}(D_1)\times\cdots\times
%	M_{n_r}(D_r)$.
%\end{theorem}
%
%\begin{proof}
%	Procederemos por inducción en $\dim A$. Si $\dim A=1$\dots\framebox{} 
%
%	Supongamos entonces que $\dim A>1$. Si $A$ es un álgebra prima, el
%	resultado se sigue inmediatamente del teorema de Wedderburn. Supongamos
%	entonces que existe $a\in A\setminus\{0\}$ tal que $I=\{x\in A:aAx=0\}$ es
%	no nulo. Como $I$ es un ideal de $A$, $I$ es un álgebra semiprima.
%	\framebox{?} Como $a\not\in I$, $\dim I<\dim A$, y entonces, por hipótesis
%	inductiva, existen $n_1,\dots,n_s\in\N$ y álgebras de división
%	$D_1,\dots,D_s$ tales que 
%	\[
%		I\simeq M_{n_1}(D_1)\times\cdots\times M_{n_s}(D_s).
%	\]
%	En particular, $I$ es unitario. Por el lema~\ref{lem:Iunitario}, existe un
%	ideal $J$ de $A$ tal que $A\simeq I\times J$. Como $\dim J<\dim A$, la hipótesis inductiva
%	implica que existen $n_{s+1},\dots,n_r\in\N$ y álgebras de división $D_{s+1},\dots,D_r$ tales que
%	\[
%		J\simeq M_{n_{s+1}}(D_{s+1})\times\cdots\times M_{n_r}(D_r).
%	\]
%	Luego $A\simeq I\times J\simeq \prod_{j=1}^s M_{n_j}(D_j)$.
%\end{proof}
%
%\begin{corollary}
%	Sea $A$ un álgebra no nula de dimensión finita. Si $A$ es semiprima,
%	entonces $A$ es unitaria.
%\end{corollary}
%
%%Gracias al teorema de Wedderburn se puede ir un poco más lejos:
%%\begin{corollary}
%%	Sea $A$ un álgebra unitaria. Son equivalentes:
%%	\begin{enumerate}
%%		\item $A$ es semiprima.
%%		\item Todo $A$-módulo unitario es semisimple.
%%		\item $A$ es semisimple como $A$-módulo.
%%		\item Todo ideal a izquierda de $A$ es de la forma $Ae$ para algún
%%			idempotente $e\in A$. 
%%	\end{enumerate}
%%\end{corollary}
%%
%%\begin{proof}
%%	La implicación $(1)\implies(2)$ es el teorema de Wedderburn. 
%%	
%%\end{proof}
%
%\begin{example}
%	Por el teorema de Maschke sabemos que si $G$ es un grupo finito, 
%	$\C[G]$ es un álgebra semiprimitiva y luego semisimple.
%\end{example}
%




%\section{Viejo!}
%
%\begin{theorem}[Artin--Wedderburn]
%	\index{Teorema!de Artin--Wedderburn}
%	\label{thm:ArtinWedderburn}
%	Si $R$ es un anillo, las siguientes afirmaciones son equivalentes:
%	\begin{enumerate}
%		\item $R$ es un anillo no nulo semiprimitivo y artiniano a izquierda.
%		\item Existen anillos de división $D_1,\dots,D_r$ y tales que
%			\[
%				R\simeq\prod_{i=1}^r R_i,
%			\]
%			donde $R_i=\End_{D_i}(V_i)$
%		\item Existen anillos de división $D_1,\dots,D_r$ y enteros positivos
%			$n_1,\dots,n_r$ tales que 
%			\[
%			R\simeq M_{n_1}(D_1)\times\cdots\times M_{n_r}(D_r).
%		\]
%	\end{enumerate}
%\end{theorem}
%
%\begin{proof}
%	Demostremos que $(1)\implies(2)$. Como $R\ne0$ y $J(R)=0$, $R$ admite
%	ideales primitivos. Supongamos que existe un número finito de ideales
%	primitivos distintos, digamos $P_1,\dots,P_t$. Cada $R/P_j$ es un anillo
%	primitivo y es artiniano a izquierda \framebox{?}. Entonces, por el teorema
%	de Wedderburn, para cada $j\in\{1,\dots,t\}$ existen un anillo de división
%	$D_j$ y un entero positivo $n_j$ tales que $R/P_j\simeq M_{n_j}(D_j)$. En
%	particular, cada $R/P_j$ es simple y entonces $P_j$ es un ideal maximal de
%	$R$. Como $R/P_j$ es simple, $(R/P_j)^2\ne 0$ y luego $R^2\not\subseteq
%	P_j$. Por maximalidad, $R^2+P_j=R$ y además $P_i+P_j=R$ para todo $i\ne j$.
%	Por el teorema chino del resto,
%	\[
%		R=R/0=R/J(R)=R/\cap_{j=1}^t P_j\simeq R/P_1\times\cdots\times R/P_t.
%	\]
%	Sea $\iota_k\colon R/P_k\to \prod_{j=1}^t R/P_j$ la inclusión canónica.
%	Cada $\iota_k(R/P_k)$ es un ideal simple (es decir, que como anillo es
%	simple) de $\prod_{j=1}^t R/P_j\simeq R$. Luego las imágenes, digamos
%	$I_k$, de los $\iota_k(R/P_k)$ dan ideales simples de $R$ y
%	$R=I_1\times\cdots\times I_t$.
%
%	Demostremos ahora que $(3)\implies(1)$. Para cada $j$ sea
%	$R_j=M_{n_j}(D_j)$. Como cada $R_j$ es primitivo por el teorema de
%	Wedderburn, $J(R_j)=\{0\}$ para todo $j$. Luego
%	$J(R)=\prod_{i=1}^rJ(R_j)=\{0\}$ y entonces $R$ es semiprimitivo. Además
%	$R$ es artiniano a izquierda.\framebox{?}
%\end{proof}
%
%\begin{corollary}
%	Sea $R$ un anillo semiprimitivo.
%	\begin{enumerate}
%		\item Si $R$ es artiniano a izquierda, entonces $R$ es unitario.
%		\item $R$ es artiniano a izquierda si y sólo si es artiniano a derecha.
%		\item Si $R$ es artiniano a izquierda es noetheriano.
%	\end{enumerate}
%\end{corollary}
%
%\begin{proof}
%	La primera afirmación es consecuencia inmediata del teorema de
%	Artin--Wedderburn~\ref{thm:ArtinWedderburn}.
%\end{proof}
%
%\begin{corollary}
%	Sea $R$ un anillo semiprimitivo artiniano a izquierda y sea $I$ un ideal de
%	$R$. Entonces $I=Re$ para algún idempotente $e\in R$ tal que $e\in Z(R)$.
%\end{corollary}
%
%\begin{proof}
%		
%\end{proof}
%
%\begin{proposition}
%	Sea $R$ un anillo semisimple artiniano a izquierda. 
%	\begin{enumerate}
%		\item $R=I_1\times\cdots\times I_n$ donde los $I_j$ son ideales simples.
%		\item Si $J\subseteq R$ es un ideal simple, entonces existe $k\in\{1,\dots,n\}$ tal que $J=I_k$.
%		\item Si $R=J_1\times\cdots\times J_m$ donde los $J_k$ son ideales simples, entonces $n=m$ y existe
%			$\sigma\in\Sym_n$ tal que $I_k=J_{\sigma(k)}$ para todo $k\in\{1,\dots,n\}$.
%	\end{enumerate}
%\end{proposition}
%
%\begin{proof}
%\end{proof}
%
%\begin{theorem}
%	Sea $R$ un anillo unitario no nulo. Las siguientes afirmaciones son
%	equivalentes:
%	\begin{enumerate}
%		\item $R$ es semiprimitivo y artiniano a izquierda.
%		\item Todo $R$-módulo unitario es proyectivo.
%		\item Todo $R$-módulo unitario es inyectivo.
%		\item Toda sucesión exacta de $R$-módulos unitarios se parte.
%		\item Todo $R$-módulo unitario no nulo es semisimple.
%		\item $\prescript{}{R}R$ es unitario y semisimple.
%		\item Todo ideal a izquierda de $R$ es de la forma $Re$ para algún $e\in R$ indempotente.
%		\item $\prescript{}{R}R$ es suma directa de ideales a izquierda
%			minimales $L_1,\dots,L_m$ donde cada $L_j$ es de la forma $Re_j$, y
%			los $e_j$ son idempotentes ortogonales tales que
%			$e_1+\cdots+e_m=1$. 
%	\end{enumerate}
%\end{theorem}
%
%\begin{proof}
%	Veamos que $(4)\implies(5)$. Sea $M$ un módulo unitario y sea $N$ un
%	submódulo no nulo de $M$. Como la sucesión $0\to N\to M\to M/N\to 0$ es
%	exacta, se parte. Luego $M=N\oplus X$ para algún submódulo $X$ de $N$ tal
%	que $X\simeq M/N$. Como $M$ es unitario, $Rm\ne 0$ para todo $m\in
%	M\setminus\{0\}$. Luego $M$ es semisimple por el teorema~\framebox{?}.
%
%	Veamos ahora que $(5)\implies(4)$. Sea 
%	\[
%	\begin{tikzcd}
%		0 \arrow[r]
%		& N \arrow[r]
%		& M \arrow[r]
%		& X \arrow[r]
%		& 0
%	\end{tikzcd}
%	\]
%	una sucesión exacta corta de $R$-módulos. Como $f\colon N\to f(N)$ es un
%	isomorfismo y entonces $f(N)$ es un submódulo del semisimple $M$, $f(N)$ es
%	sumando directo de $M$. Sea $\pi\colon M\to f(N)$ el morfismo canónico.
%	Entonces $\pi f=f$ y $f^{-1}\pi\colon M\to A$ es un morfismo tal que
%	$(f^{-1}\pi)f=\id_N$.\framebox{?}
%
%	Demostremos que $(5)\implies(7)$. Sea $L$ un ideal a izquierda de $R$. Como
%	los ideales a izquierda de $R$ son los submódulos de $\prescript{}{R}R$,
%	existe un ideal a izquierda $N$ de $R$ tal que $R=L\oplus N$. Existen
%	entonces $e_1\in L$ y $e_2\in N$ tales que $1=e_1+e_2$. Si $x\in L$,
%	entonces $x=xe_1+xe_2$ y luego $xe_2=x-xe_1\in L\cap N=\{0\}$. Demostramos
%	entonces que $x=xe_1$ para todo $x\in L$. En particular, $e_1e_1=e_1$ y
%	$L=Re_1$. 
%
%	Demostremos que $(7)\implies(6)$. Sea $L$ un submódulo de
%	$\prescript{}{R}R$. Como entonces $L$ es un ideal a izquierda de $R$,
%	$L=Re$ para algún idempotente $e\in R$. Como el conjunto $R(1-e)$ es un
%	ideal a izquierda de $R$ tal que $R=Re\oplus R(1-e)$, se concluye que
%	$\prescript{}{R}R$ es semisimple.\framebox{?}
%
%	Veamos que $(6)\implies(1)$. Supongamos que $\prescript{}{R}R=\sum_{i\in
%	I}N_i$, donde los $N_j$ son submódulos simples de $\prescript{}{R}R$.
%	Reordenando los $N_j$ si fuera necesario, podemos suponer que existe
%	$k\in\N$ tal que $1=e_1+\cdots+e_k$ y $e_j\in N_j$ para todo
%	$j\in\{1,\dots,k\}$. Si $r\in R$, entonces $r=re_1+\cdots+re_k\in
%	\sum_{i=1}^k N_i$. Luego $R=\sum_{i=1}^k N_i$.
%	Veamos que $J(R)=0$. Si $r\in J(R)$ entonces, como $rN_i=0$ para todo
%	$i\in\{1,\dots,k\}$, se concluye que $r=r1=re_1+\cdots+re_k=0$. Probamos
%	entonces que $R$ es semiprimitivo. Falta ver que $R$ es artiniano a
%	izquierda. Para eso basta obvervar que, como
%	\[
%		\frac{N_1\oplus\cdots\oplus N_i}{N_1\oplus\cdots\oplus N_{i-1}}\simeq N_i
%	\]
%	para cada $i\in\{1,\dots,k\}$, la serie
%	\[
%	R=N_1\oplus\cdots\oplus N_k\supsetneq N_1\oplus\cdots\oplus N_{k-1}\supsetneq\cdots\supsetneq N_1\oplus N_2\supsetneq N_1\supsetneq 0
%	\]
%	es una serie de composición.\framebox{?}
%
%	Veamos que $(1)\implies(8)$. Sin perder generalidad podemos suponer que
%	\[
%	R=\prod_{i=1}^k M_{n_j}(D_j),
%	\]
%	donde los $D_j$ son anillos de división.\framebox{?}
%
%	Veamos que $(8)\implies(5)$. Sea $M$ un módulo unitario no nulo. Si $m\in
%	M$ entonces $L_im$ es un submódulo de $M$. Los $L_jm$ generan a $M$ pues 
%	\[
%	m=1m=e_1m+\cdots+e_km\in\sum L_im.
%	\]
%	Veamos que cada $L_jm$ es simple. Fijado $i$, la función $f\colon L_i\to
%	L_im$, $x\mapsto xm$, es un morfismo sobreyectivo. Como $L_i$ es un ideal a
%	izquierda minimal, $L_i$ es un submódulo simple. Luego $m\ne0$ implica que
%	$f$ es un isomorfismo. Probamos entonces que el conjunto $\{L_jm:1\leq
%	j\leq k,m\in M,L_jm\ne 0\}$ es una familia de submódulos simples cuya suma
%	es $M$.
%\end{proof}


\topic{Jacobson's density theorem}

\begin{definition}
	\index{Anillo!denso de operadores lineales}
	Sean $D$ un anillo de división y $V$ un espacio vectorial sobre $D$. Un
	subanillo $R\subseteq\End_D(V)$ se dice \textbf{denso} en $V$ si para cada
	$n\in\Z_{>0}$, cada $\{u_1,\dots,u_n\}\subseteq V$ linealmente independiente de
	$V$ y cada conjunto $\{v_1,\dots,v_n\}\subseteq V$ (no necesariamente
	linealmente independiente) existe $f\in R$ tal que $f(u_j)=v_j$ para todo
	$j\in\{1,\dots,n\}$.
\end{definition}

%\begin{lemma}
%	Sea $R$ un subanillo de $\End_D(V)$. Entonces $R$ es denso en $V$ si y sólo
%	si para todo $g\in\End_D(V)$ y todo subespacio $U$ de $V$ de dimensión
%	finita existe $f\in R$ tal que $f|_U=g|_U$.
%\end{lemma}
%
%\begin{proof}
%	Supngamos que $R$ es denso en $V$. Sean $g\in\End_D(V)$ y $U\subseteq V$ un
%	subespacio de dimensión finita. Sea $\{u_1,\dots,u_n\}$ una base de $U$.
%	Como $R$ es denso, existe $f\in R$ tal que $f(u_j)=g(u_j)$ para todo
%	$j\in\{1,\dots,n\}$ y luego $f|_U=g|_U$.
%
%	Recíprocamente, sea $n\in\N$ y sean $\{u_1,\dots,u_n\}\subseteq V$ un
%	conjunto linealmente independiente y $\{v_1,\dots,v_n\}\subseteq V$. Sean
%	$g\in\End_D(V)$ tal que $g(u_j)=v_j$ para todo $j\in\{1,\dots,n\}$ y $U$ el
%	subespacio de $V$ generado por $u_1,\dots,u_n$. Por hipótesis existe $f\in
%	R$ tal que $f|_U=g|_U$ y luego $f(u_j)=g(u_j)=v_j$ para todo
%	$j\in\{1,\dots,n\}$.
%\end{proof}

\begin{lemma}
	\label{lem:unico_denso}
	Sea $D$ un anillo de división 
	$V$ un $D$-espacio vectorial de dimensión finita. Entonces $\End_D(V)$ es
	el único anillo denso en $V$.
\end{lemma}

\begin{proof}
	Sea $R$ denso en $V$ y sea $\{v_1,\dots,v_n\}$ una base de $V$. Por
	definición, $R\subseteq\End_D(V)$. Si $g\in\End_D(V)$ entonces, como $R$ es
	denso en $V$, existe $f\in R$ tal que $f(v_j)=g(v_j)$ para todo
	$j\in\{1,\dots,n\}$. Luego $g=f\in R$.
\end{proof}

\begin{lemma}
	\label{lem:ideal_denso}
	Sea $R$ un anillo denso en $V$ y sea $I$ un ideal no nulo de $R$. Entonces
	$I$ es denso en $V$.
\end{lemma}

\begin{proof}
	Sea $I$ un ideal no nulo de $R$. Sean $h\in I\setminus\{0\}$ y $u\in V$
	tales que $h(u)=v\ne0$. Sea $\{u_1,\dots,u_n\}\subseteq V$ un conjunto
	linealmente independiente y sea $\{v_1,\dots,v_n\}\subseteq V$. Como $R$ es
	denso en $V$, existen $g_1,\dots,g_n\in R$ tales que $g_i(u_i)=u$ y
	$g_i(u_j)=0$ si $i\ne j$. Existen además $f_1,\dots,f_n\in R$ tales que
	$f_i(v)=v_i$. Entonces $\gamma=\sum_{i=1}^n f_ihg_i\in I$ cumple que
	$\gamma(u_j)=v_j$ para todo $j\in\{1,\dots,n\}$.
\end{proof}

%Ahora demostraremos el teorema de densidad de Jacobson. 
%Necesitaremos el siguiente
%lema:
%\begin{lemma}
%	\label{lem:densidad}
%	Sea $M$ un $R$-módulo simple y $D=\End_R(M)$.  Si $N$ es un subespacio de
%	$M$ tal que $\dim_DN<\infty$ y $m\in M\setminus N$, entonces existe $r\in
%	R$ tal que $rm\ne 0$ y $rN=0$.
%\end{lemma}
%
%\begin{proof}
%	Supongamos que la afirmación no es cierta y sea $N$ un contraejemplo de la
%	mínima dimensión posible. Entonces $\dim_DN\geq1$ (pues el resultado es
%	verdadero en el caso $N=0$). Sea $N_0$ un subespacio de $N$ tal que $\dim
%	N_0=\dim N-1$ y sea
%	\[
%		L=\{r\in R:rN_0=0\}.
%	\]
%	Como por la minimalidad de $N$ nuestra afirmación es cierta para $N_0$,
%	para cualquier $x\in N\setminus N_0$ se tiene que $Lx=N$ (pues existe $r\in
%	L$ tal que $rx=\ne 0$). Como $L$ es ideal a izquierda de $R$ y
%	$Lx\subseteq N$ es un submódulo, $Lx=N$ pues $N$ es simple.
%
%	Sea $w\in V\setminus U$ tal que nuestra afirmación no es cierta y sea $u\in
%	U\setminus U_0$.  La función 
%	\[
%		\delta\colon V\to V,\quad
%		v\mapsto lw,
%	\]
%	donde $v=lu\in Lu=V$ (que depende de $u$ y $w$) 
%	está bien definida: si $l_1,l_2\in L$ son tales que $v=l_1u=l_2u$ entonces $(l_1-l_2)u=0$ y luego
%	\[
%		0=\delta(0)=\delta((l_1-l_2)u)=(l_1-l_2)w=l_1w-l_2w. 
%	\]
%	Además $\delta$ es morfismo de $R$-módulos pues si $l\in L$ es tal que $v=lu$ entonces
%	\[
%		\delta(rv)=\delta(r(lu))=\delta( (rl)u)=(rl)w=r(lw)=r\delta(v)
%	\]
%	para todo $r\in R$.
%
%
%\end{proof}

\begin{theorem}[densidad de Jacobson]
	\label{thm:densidad}
	\index{Teorema!de densidad de Jacobson}
	\index{Jacobson!densidad de}
	Un anillo $R$ es primitivo si y sólo si es isomorfo a un anillo denso en un
	espacio vectorial sobre un anillo de división.
\end{theorem}

\begin{proof}
	Si $R$ es isomorfo a un anillo
	denso en un $D$-módulo $V$ donde $D$ es un anillo de división, entonces $R$
	es primitivo pues $V$ es un módulo simple y fiel. Es fiel: si
	$f\in\Ann_R(V)$ entonces $f=0$ pues $f(v)=0$ para todo $v\in V$. Es simple
	pues si $W\subseteq V$ es un submódulo no nulo, $v\in V$ y $w\in
	W\setminus\{0\}$ entonces existe $f\in R$ tal que $v=f(w)\in W$. 

	Supongamos ahora que $R$ es primitivo y sea $V$ un módulo simple y fiel.
	Por el lema de Schur, $D=\End_R(V)$ es un anillo de división. Luego $V$ es
	un $D$-espacio vectorial con las operaciones
	\[
	\delta v=\delta(v),\quad
	\delta(rv)=r(\delta v),\quad
	v\in V,r\in R,\delta\in D.
	\]
	Para $r\in R$ definimos 
	\[
		\gamma_r\colon V\to V,\quad
		v\mapsto rv.
	\]
	Es fácil ver que $\gamma_r\in\End_D(V)$ y que la función $R\to\End_D(V)$,
	$r\mapsto\gamma_r$, es un morfismo de anillos. Como $V$ es fiel,
	$R\simeq\gamma(R)=\{\gamma_r:r\in R\}$ (si $\gamma_r=\gamma_s$ entonces
	$rv=\gamma_r(v)=\gamma_s(v)=sv$ para todo $v\in V$ y luego $r=s$ pues
	$(r-s)v=0$ para todo $v\in V$).

	\begin{claim}
		Si $U$ es un subespacio de $V$
		de dimensión finita, para cada $w\in V\setminus U$ existe $r\in R$ tal que
		$\gamma_r(U)=0$ y $\gamma_r(w)\ne0$.
	\end{claim}

	Supongamos que la afirmación no es cierta y sea $U$ un contraejemplo de la
	mínima dimensión posible. Entonces $\dim_DU\geq1$ (pues el resultado es
	cierto para el subespacio nulo). Sea $U_0$ un subespacio de $U$ tal que
	$\dim U_0=\dim U-1$ y sea
	\[
		L=\{l\in R:\gamma_l(U_0)=0\}.
	\]
	Como por la minimalidad de $U$ nuestra afirmación es cierta para $U_0$,
	para cualquier $v\in V\setminus U_0$ se tiene que $Lv=V$ (pues existe $l\in
	L$ tal que $lv=\gamma_l(v)\ne 0$, y como $L$ es ideal a izquierda de $R$ sabemos
	que $Lv\subseteq V$ es un submódulo y $V$ es simple).

	Sea $w\in V\setminus U$ tal que nuestra afirmación no es cierta y sea $u\in
	U\setminus U_0$.  La función 
	\[
		\delta\colon V\to V,\quad
		v\mapsto lw,
	\]
	donde $v=lu\in Lu=V$ (que depende de $u$ y $w$) 
	está bien definida: si $l_1,l_2\in L$ son tales que $v=l_1u=l_2u$ entonces $(l_1-l_2)u=0$ y luego
	\[
		0=\delta(0)=\delta((l_1-l_2)u)=(l_1-l_2)w=l_1w-l_2w. 
	\]
	Además $\delta$ es morfismo de $R$-módulos pues si $l\in L$ es tal que $v=lu$ entonces
	\[
		\delta(rv)=\delta(r(lu))=\delta( (rl)u)=(rl)w=r(lw)=r\delta(v)
	\]
	para todo $r\in R$.

	Para todo $l\in L$ se tiene que 
	\[
		l(\delta(u)-w)=l\delta(u)-lw=\delta(lu)-lw=0,
	\]
	y entonces $L(\delta(u)-w)=0$. Pero esto implica que $\delta(u)-w\not\in V\setminus U_0$, es
	decir $\delta(u)-w\in U_0$. Luego 
	\[
		w=xu-(xu-w)\in Du+U_0=U,
	\]
	una contradicción. 

	Esta afirmación alcanza para demostrar el teorema. En efecto, sean
	$u_1,\dots,u_n\in V$ vectores linealmente independientes y sean
	$v_1,\dots,v_n\in V$ vectores arbitrarios. Si fijamos $i\in\{1,\dots,n\}$, la afirmación anterior con
	\[
		U=\langle u_1,\dots,u_{i-1},u_{i+1},\dots,u_n\rangle
	\]
	y $w=u_i$ nos dice que existe $r_i\in R$ tal que $\gamma_{r_i}(u_j)=0$ si
	$j\ne i$ y $\gamma_{r_i}(u_i)\ne 0$. Como además existe $s_i\in R$ tal que
	$\gamma_{s_i}\gamma_{r_i}(u_i)=v_i$, se concluye que el elemento
	$r=\sum_{j=1}^n s_jr_j\in R$ es tal que $\gamma_r(u_i)=v_i$ para todo
	$i\in\{1,\dots,n\}$.
\end{proof}

%\begin{exercise}
%	Sea $R$ un anillo denso en $V$. Demuestre que $R$ es artiniano a izquierda
%	si y sólo si $V$ es de dimensión finita.
%\end{exercise}
% si V es de dimensión finita, es fácil por el lema~\ref{lem:unico_denso}
% hungerford pag 419

\begin{corollary}
	Si $R$ es un anillo primitivo, entonces existe un anillo de división $D$
	tal que $R\simeq\End_D(V)$ para algún $D$-espacio vectorial $V$ de
	dimensión finita, o bien para todo $m\in\Z_{>0}$ existe un subanillo $R_m$ de
	$R$ y un morfismo de anillos sobreyectivo $R_m\to\End_D(V_m)$ para algún $D$-espacio
	vectorial $V_m$ tal que $\dim_DV_m=m$.
\end{corollary}

\begin{proof}
	Sabemos que $R$ admite un módulo $V$ simple y fiel. Además, como $R$ es
	primitivo, por el teorema~\ref{thm:densidad} podemos suponer que existe un
	anillo de división $D$ tal que $R$ es denso en un $D$-espacio vectorial
	$V$.  Sea $\gamma\colon R\to\End_D(V)$, $r\mapsto\gamma_r$, donde
	$\gamma_r(v)=rv$. Como $V$ es fiel, $\gamma$ es inyectiva. Luego
	$R\simeq\gamma(R)$. 

	Si $V$ es de dimensión finita, el resultado se obtiene del
	lema~\ref{lem:unico_denso}.  Supongamos entonces que $V$ es de dimensión
	infinita y sea $\{u_1,u_2,\dots\}$ un conjunto linealmente independiente.
	Para cada $m\in\Z_{>0}$ sea $V_m$ el subespacio generado por $\{u_1,\dots,u_m\}$
	y sea $R_m=\{r\in R:rV_m\subseteq V_m\}$. Es fácil ver que $R_m$ es un
	subanillo de $R$. Como $R$ es denso en $V$, la función
	\[
		R_m\to \End_D(V_m),\quad
		r\mapsto\gamma_r|_{V_m}
	\]
	es un morfismo sobreyectivo de anillos. 
\end{proof}

%\section{El teorema de Wedderburn}

En álgebra conmutativa los dominios juegan un papel fundamental. En álgebra no
conmutativa las cosas no son tan similares ya que el anillo $M_n(K)$ no es un
dominio. Nos interesa entonces encontrar un concepto similar al de dominio que
funcione en el contexto no conmutativo.

\begin{definition}
	\index{Anillo!primo} 
	Sea $R$ un anillo (no necesariamente con unidad). Diremos que $R$ es
	\textbf{primo} si dados $x,y\in R$ tales que $xRy=0$ entonces $x=0$ o bien
	$y=0$.
\end{definition}

\begin{example}
	Recordemos que un anillo $R$ es un \textbf{dominio} si $xy=0$ implica que
	$x=0$ o bien $y=0$.  Todo dominio es trivialmente un anillo primo.
\end{example}

\begin{example}
	Un anillo conmutativo es primo si y sólo si es un dominio pues $ab=0$ si y
	sólo si $aRb=0$.
\end{example}

\begin{example}
	Un ideal no nulo de un anillo primo es un anillo primo.
\end{example}

\begin{lemma}
	Sea $R$ un anillo. Son equivalentes:
	\begin{enumerate}
		\item $R$ es primo.
		\item Si $I$ y $J$ son ideales a izquierda tales que $IJ=0$ entonces
			$I=0$ o bien $J=0$.
		\item Si $I$ y $J$ son ideales tales que $IJ=0$ entonces $I=0$ o bien
			$J=0$.
	\end{enumerate}
\end{lemma}

\begin{proof}
	Veamos primero que $(1)\implies(2)$. Sean $I$ y $J$ ideales a izquierda
	tales que $IJ=0$. Entonces $IRJ=I(RJ)\subseteq IJ=0$. Supongamos que $J\ne
	0$. Si $u\in I$ y $v\in J\setminus\{0\}$, entonces $uRv\in IRJ=0$ y luego
	$u=0$.

	La implicación $(2)\implies(3)$ es trivial. 

	Veamos entonces que $(3)\implies(1)$. Sean $x,y\in R$ tales que $xRy=0$.
	Sean $I=RxR$ y $J=RyR$. Como $IJ=(RxR)(RyR)=R(xRy)R=0$, por hipótesis,
	podemos suponer que entonces $I=0$. En particular $Rx$ y $xR$ son ideales
	pues $R(xR)=(Rx)R=0$. Pero entonces $\Z x$ es un ideal de $R$ tal que $(\Z x)R=0$. Luego $x=0$. 
\end{proof}

\begin{example}
	Todo anillo simple es trivialmente primo. La afirmación recíproca no es
	cierta: $\Z$ es un anillo primo (por ser un dominio) pero no es simple.
\end{example}

\begin{example}
	Si $R_1$ y $R_2$ son anillos, $R=R_1\times R_2$ no es primo pues
	$I=R_1\times 0$ y $J=0\times R_2$ son ideales no nulos tales que $IJ=0$.
\end{example}

\begin{lemma}
	\label{lem:primoizqmin=>prim}
	Sea $R$ un anillo primo y sea $L$ un ideal a izquierda minimal de $R$.
	Entonces $R$ es primitivo.
\end{lemma}

\begin{proof}
	Como $L$ es ideal a izquierda minimal, es simple como $R$-módulo. Veamos
	que como $R$ es primo, $L$ es fiel. Sea $y\in L\setminus\{0\}$ y sea
	$x\in\Ann_R(L)$.  Entonces, como $xRy\in xRL\subseteq xL=0$, se concluye
	que $x=0$.
\end{proof}

\begin{lemma}
	\label{lem:denso_artiniano}
	Sea $D$ un anillo de división y sea $R$ un anillo denso en un
	$D$-espacio vectorial $V$. Si $R$ es artiniano a izquierda, 
	entonces $V$ es de dimensión finita.
\end{lemma}

\begin{proof}
	Supongamos que $V$ tiene dimensión infinita y sea $\{u_1,u_2,\dots,\}$ un
	subconjunto de $V$ linealmente independiente. Como $R\subseteq\End_D(V)$,
	$V$ es un $R$-módulo con $f\cdot v=f(v)$, donde $f\in R$ y $v\in V$. Para
	cada $n\in\Z_{>0}$ sea 
	\[
		I_n=\Ann_R(\{u_1,\dots,u_n\}.
	\]
	Los $I_j$ son ideales a izquierda de $R$ tales que $I_1\supseteq
	I_2\supseteq\cdots\supseteq I_n\supseteq\cdots$. Veamos que esta sucesión
	no se estabiliza: Sean $n\in\Z_{>0}$ y $v\in V\setminus\{0\}$. Como $R$ es denso
	en $V$, existe $f\in R$ tal que $f(u_j)=0$ para todo $j\in\{1,\dots,n\}$ y
	$f(u_{n+1})=v\ne0$. Luego $I_1\supsetneq I_2\supsetneq\cdots\supsetneq
	I_n\supsetneq\cdots$, una contradicción pues $R$ es artiniano a izquierda.
\end{proof}

\begin{theorem}[Wedderburn]
	\index{Teorema!de Wedderburn}
	Sea $R$ un anillo artiniano a izquierda. Las siguientes afirmaciones son
	equivalentes:
	\begin{enumerate}
		\item $R$ es simple.
		\item $R$ es primo.
		\item $R$ es primitivo.
		\item $R\simeq M_n(D)$ para algún $n$ y algún anillo de división $D$.
	\end{enumerate}
\end{theorem}

\begin{proof}
	La implicación $(1)\implies(2)$ es trivial. 
	
	Para demostrar que $(2)\implies(3)$ basta observar que como $R$ es
	artiniano, $R$ tiene un ideal a izquierda minimal.  Por el
	lema~\ref{lem:primoizqmin=>prim}, $R$ es primitivo. 

	Veamos que $(3)\implies(4)$. Si $R$ es primitivo, por el teorema de
	densidad de Jacbonson, existe un anillo de división $D$ tal que 
	$R$ es isomorfo a un anillo $S$ que es denso en un $D$-espacio vectorial
	$V$. Como $R$ es artiniano a izquierda, el lema~\ref{lem:denso_artiniano} implica que 
	$R=\End_D(V)\simeq M_n(D)$ pues $\dim_DV<\infty$. 

	Por último, $(4)\implies(1)$ es trivial pues $M_n(D)$ es simple. 
%	$D$-espacio vectorial $V$ de dimensión finita. Si $u_1,\dots,u_m\in V$ son
%	linealmente independientes sobre $D$, existen $f_1,\dots,f_m\in S$ tales
%	que $f_i(u_i)\ne0$ y $f_i(u_j)=0$ si $i\ne j$. Como los $f_j$ son
%	linealmente independientes sobre $k$, $\dim_DV\leq \dim A$. Luego $A=\End_DV\simeq M_n(D^{\op})$ 
%	por la proposición\dots
\end{proof}

Para completar nuestra presentación del teorema de Wedderburn, veremos
que la descomposición es única. Necesitaremos dos lemas previos:

\begin{lemma}
	\label{lem:wedderburn_unididad}
	Sea $D$ un anillo de división. Entonces 
	\[
		D^{\op}\simeq\End_{M_n(D)}(D^{n}). 
	\]
\end{lemma}

\begin{proof}
	Sea  
	\begin{align*}
		&\phi\colon D^{\op}\to\End_{M_n(D)}(D^{n}), & d\mapsto \phi(d)\colon &D^{n}\to D^{n},
	\end{align*}
	donde $\phi(d)(x)=xd$. Es evidente que $\phi$ es lineal; 
	es morfismo pues además 
	\begin{align*}
		\phi(d_1\cdot_{\op} d_2)(x)&=\phi(d_2d_1)(x)=x(d_2d_1)=(xd_2)d_1=\phi(d_1)\phi(d_2)(x).
	\end{align*}
	Como $\phi$ es no nulo y $D^{\op}$ es es simple por ser de división, se concluye que 
	$\phi$ es inyectivo. Veamos que
	$\phi$ es sobreyectivo: sean $f\in\End_{M_n(D)}(D^{n})$ y 
	\[
		e_1=\colvec{4}{1}{0}{\vdots}{0},\quad
		\colvec{4}{d_1}{d_2}{\vdots}{d_n}=
		f(e_1),
		\quad
		A=\begin{pmatrix} 
			a_1 & 0 & \cdots & 0\\
			\vdots & \vdots & \ddots & \vdots\\
			a_n & 0 & \cdots & 0
		\end{pmatrix}.
	\]
	Entonces
	\[
		f\colvec{4}{a_1}{a_2}{\vdots}{a_n}
		=f(Ae_1)=Af(e_1)=
		\colvec{4}{a_1d_1}{a_2d_2}{\vdots}{a_nd_1}=\phi(d_1)\colvec{3}{a_1}{\vdots}{a_n}.
	\]
\end{proof}

%%\begin{lemma}
%	\label{lem:Wedderburn:unicidad1}
%	Sea $D$ un anillo de división y sea $V$ un $D$-espacio vectorial de dimensión finita. 
%	Sea $R=\End_D(V)$. Si $M$ y $N$ son $R$-módulos 
%	simples y fieles, entonces $M\simeq N$.
%\end{lemma}
%
%\begin{proof}
%	Como $R=\End_D(V)\simeq M_n(D)$ es artiniano a izquierda, $R$ contiene un
%	ideal a izquierda minimal. Como $M$ es fiel, existe $m\in M$ tal que $Lm\ne
%	0$ (si $Lm=0$ para todo $m\in M$, $L\subseteq\Ann_R(M)=0$). Como $Lm$ es un
%	submódulo no nulo de $M$ y $M$ es simple, $Lm=M$. La función
%	\[
%		\theta\colon L\to Lm=M,\quad
%		x\mapsto xm
%	\]
%	es morfismo sobreyectivo. Como $L$ es simple como módulo, por el lema de
%	Schur, $\theta$ es inyectivo. Luego $L\simeq M$. Similarmente se demuestra
%	que $L\simeq N$.
%\end{proof}
%
%\begin{lemma}
%	\label{lem:Wedderburn:unicidad2}
%	Sea $D$ un anillo de división y sea $V$ un $D$-espacio vectorial. Sea
%	$g\colon V\to V$ un morfismo de grupos abelianos tal que $fg=gf$ para todo
%	$f\in\End_D(V)$. Entonces existe $d\in D$ tal que $g(v)=dv$ para todo $v\in
%	V$.
%\end{lemma}
%
%\begin{proof}
%	Sea $u\in V\setminus\{0\}$. Vamos a demostrar que $\{u,g(u)\}$ es
%	linealmente dependiente sobre $D$. Supongamos que $\{u,g(u)\}$ fuera linealmente
%	independiente. Entonces $\dim_DV\geq2$. 
%	Como $\End_D(V)$ es denso en $V$, existe $f\in\End_D(V)$ tal
%	que $f(u)=0$ y $f(g(u))\ne0$. Luego
%	\[
%		0\ne 0=f(g(u))=g(f(u))=g(0)=0,
%	\]
%	una contradicción.  Como $\{u,g(u)\}$ es linealmente dependiente sobre $D$,
%	existe $d\in D$ tal que $g(u)=du$. Si $v\in V$, por densidad existe
%	$h\in\End_D(V)$ tal que $h(u)=v$. Luego
%	\[
%		g(v)=g(h(u))=h(g(u))=h(du)=dh(u)=dv.
%	\]
%\end{proof}
%
%\begin{proposition}
%	Para cada $j\in\{1,2\}$ sea $V_j$ un espacio vectorial sobre el anillo de división $D_j$ de dimensión finita $n_j$. 
%	Si $\End_{D_1}(V_1)\simeq\End_{D_2}(V_2)$ entonces $n_1=n_2$ y $D_1\simeq D_2$.
%%	Sea $V$ es un espacio vectorial sobre el anillo de división $D$ y sea $W$
%%	un espacio vectorial sobre el anillo de división $E$. Supongamos que $V$ y
%%	$W$ son de dimensión finita. Si $\End_D(V)\simeq\End_E(W)$ entonces
%%	$\dim_DV=\dim_EW$ y $D\simeq E$.
%\end{proposition}
%
%\begin{proof}	
%	Cada $V_j$ es un $\End_{D_j}(V_j)$-módulo fiel. Sea $R=\End_{D_1}(V_1)$. Como existe un isomorfismo
%	$\sigma\colon R\to\End_{D_2}(V_2)$, $V_2$ es un
%	$R$-módulo simple y fiel con
%	\[
%		fv_2=\sigma(f)v_2,\quad
%		f\in R,\,v_2\in V_2.
%	\]
%	Por el lema~\ref{lem:Wedderburn:unicidad1}, existe un isomorfismo
%	$\phi\colon V_1\to V_2$ de $R$-módulos; en particular, 
%	\[
%		\phi(fv_1)=f(\phi(v_1))=(\sigma(f))\phi(v_1)
%	\]
%	para todo $v_1\in V_1$ y $f\in R$. Luego 
%	$\sigma(f)=\phi f\phi^{-1}\colon V_2\to V_2$ es un morfismo de grupos
%	abelianos. Para $j\in\{1,2\}$ y $d_j\in D_j$ sea 
%	\[
%		\gamma_{d_1}\colon V_j\to V_j,\quad x\mapsto d_jx.
%	\]
%	Como cada $V_j$ es fiel, $\gamma_{d_j}=0$ si y sólo si $d_j=0$. Si $f\in R$ y $d_1\in D_1$ entonces $f\gamma_{d_1}=\gamma_{d_1}f$. Luego
%	\[
%		(\phi\gamma_{d_1}\phi^{-1})\sigma(f)
%		=(\phi\gamma_{d_1}\phi^{-1})(\phi f\phi^{-1})
%		=\phi f\gamma_{d_1}\phi^{-1}
%		=\sigma(f)\phi\gamma_{d_1}\phi^{-1}.
%	\]
%	Como $\sigma$ es sobreyectiva, el lema~\ref{lem:Wedderburn:unicidad2} con
%	$V=V_2$ y $g=\phi\gamma_{d_1}\phi^{-1}$ implica que existe $d_2\in D_2$ tal
%	que $\phi\gamma_{d_1}\phi^{-1}=\gamma_{d_2}$. Sea $\tau\colon D_1\to D_2$,
%	$d_1\mapsto d_2$. Entonces $\phi\gamma_{d}\phi^{-1}=\gamma_{\tau(d_2)}$ 
%	\framebox{?}
%	Además 
%	\[
%		\phi(d_1v)=\phi\gamma_{d_1}(v)=\gamma_{\¿au(d_1)}\phi(v)=\tau_{d_1}\phi(v)
%	\]
%	Como $\{u_1,\dots,u_m\}$ es linealmente independiente sobre $D_1$ si y sólo si $\{\phi(u_1),\dots,\phi(u_m)\}$ es linealmente
%	independiente sobre $D_2$, se concluye que $\dim_{D_1}(V_1)=\dim_{D_2}(V_2)$. 
%\end{proof}

\begin{lemma}
	\label{lem:simple_izqminimal}
	Sea $R$ un anillo simple con un ideal a izquierda $L$ minimal. Entonces
	todo $R$-módulo simple es isomorfo a $L$. 
\end{lemma}

\begin{proof}
	Sea $M$ un módulo simple. Como $LR$ es un ideal de $R$ y el anillo $R$ es
	simple, $LR=R$.  Como
	\[
		0\ne RM=(LR)M=L(RM)\subseteq LM,
	\]
	existe $m\in M$ tal que $Lm\ne 0$. Luego $Lm$ es un submódulo no nulo del simple $M$ y entonces
	$Lm=M$. El morfismo $\gamma\colon L\to M$, $l\mapsto lm$, es sobreyectivo e
	inyectiva (pues $\ker\gamma$ es un ideal a izquierda propiamente
	contenido en $L$). Luego $L\simeq M$. 
\end{proof}

\begin{theorem}
	Si $D$ y $E$ son anillos de división tales que Si $M_n(D)\simeq M_m(E)$
	entonces $n=m$ y $D\simeq E$.
\end{theorem}

\begin{proof}
	Como $M_n(D)$ es artiniano a izquierda, existe 
	un ideal a izquierda $L$ minimal. Como
	$D^{n}\simeq E^{m}\simeq L$ como $M_n(D)$-módulos (ver ejemplo~\ref{exa:I_k}), 
	el lema~\ref{lem:simple_izqminimal} implica que
	\begin{align*}
		D^{\op}\simeq\End_{M_n(D)}(D^{n})\simeq\End_{M_n(D)}(L)\simeq\End_{M_m(E)}(L)\simeq\End_{M_m(E)}(E^{m})\simeq E^{\op}.
	\end{align*}
	Luego $D\simeq E$ y entonces $n=m$ pues $\dim M_n(D)=\dim M_m(E)$.
\end{proof}

%\section{El teorema de Connel}

Una pregunta surge naturalmente: ¿Cuándo el anillo de grupo $K[G]$ es primo?
Obtendremos una respuesta completa en el caso en que $K$ sea un cuerpo de
característica cero. 

%\begin{lemma}
%	\label{lemma:Dfg}
%	Sea $H$ un subgrupo finitamente generado de $\Delta(G)$.
%	\begin{enumerate}
%		\item $(G:C_G(H))$ es finito.
%		\item $(H:Z(H))$ es finito.
%		\item $[H,H]$ es finito.
%		\item Si $H_0$ es el conjunto de elementos de torsión de $H$, $H_0$ es
%			un subgrupo normal finito de $H$ y $H/H_0$ es finitamente generado,
%			abeliano y libre de torsión.
%	\end{enumerate}
%\end{lemma}
%
%\begin{proof}
%	Veamos la primera afirmación: Si $H=\langle
%	h_1,\dots,h_n\rangle\subseteq\Delta(G)$, entonces $(G:C_G(h_i))$ es finito
%	para todo $i\in\{1,\dots,n\}$. Como $C_G(H)=\cap_{i=1}^nC_G(h_i)$, se
%	concluye que $(G:C_G(H))$ es finito.
%
%	Para demostrar la segunda afirmación basta observar que $Z(H)=H\cap C_G(H)$
%	y luego $(H:Z(H))\leq(G:C_G(H)<\infty$. % necesito dos lemas
%
%	La tercera afirmación es consecuencia de la segunda gracias a un teorema de
%	Schur.
%
%	Por último, demostremos la cuarta afirmación.  El grupo $H/[H,H]$ es
%	abeliano y finitamente generado y luego, sus elementos de torsión forman un
%	grupo finito. Como $[H,H]$ es finito, $[H,H]$ es un subgrupo normal de
%	$H_0$. Vamos a demostrar que la torsión de $H/[H,H]$ es igual a
%	$H_0/[H,H]$. La inclusión $\supseteq$ es trivial. Veamos entonces que vale
%	$\subseteq$: so $(x[H,H])^k=1$, entonces $x^k\in[H,H]$. Luego $(x^k)^m=1$ y
%	luego $x\in H_0$. Tenemos entonces que 
%	\[
%		H/[H,H]\simeq\Z^r\times\operatorname{tor}(H/[H,H])\simeq\Z^r\times H_0/[H,H]
%	\]
%	y luego $H/H_0$ es finitamente generado, abeliano y libre de torsión.
%
%\end{proof}
%
%\begin{lemma}
%	\label{lemma:K[abelian]}
%	Si $G$ un grupo abeliano finitamente generado y sin torsión, entonces
%	$K[G]$ es un dominio. 
%\end{lemma}
%
%\begin{proof}
%	Por el teorema
%	de estructura de grupos abelianos finitamente generados podemos escribir
%	$G=\langle x_1\rangle\times\cdots\langle x_n\rangle$, donde
%	$\langle x_j\rangle\simeq\Z$ para todo $j\in\{1,\dots,n\}$. Todo elemento
%	de $G$ se escribe unívocamente como $x_1^{m_1}\cdots x_n^{m_n}$ y
%	luego la función 
%	\[
%		\iota\colon K[X_1,\dots,X_n]\to K[G],\quad
%		X_j\mapsto x_j,
%	\]
%	es un
%	morfismo de anillos inyectivo. Si $\alpha\in K[G]$, entonces existe
%	$m\in\N$ suficientemente grande tal que $\iota((X_1\cdots X_n)^m)\alpha\in
%	\iota(K[X_1,\dots,X_n])\simeq K[X_1,\dots,X_n]$. Luego $K[G]\subseteq
%	K(X_1,\dots,X_n)$ y $K[G]$ es un dominio.
%\end{proof}

%\begin{lemma}
%	Si $G$ es un grupo, entonces
%	$\Delta(G)/\Delta^+(G)$ es abeliano y libre de torsión.
%%	Valen las siguientes afirmaciones:
%%	\begin{enumerate}
%%		%\item $\Delta^+(G)$ está generado por los subgrupos normales finitos de $G$.
%%		\item 
%%		\item Si $\Delta^+(G)=1$, entonces $K[\Delta(G)]$ es un dominio.
%%	\end{enumerate}
%\end{lemma}
%
%\begin{proof}
%%	Demostremos la primera afirmación. 
%	Sean $y_1,\dots,y_n\in\Delta(G)$ y sea $L=\langle y_1,\dots,y_n\rangle$.
%	Como $[L,L]$ es finito por el lema~\ref{lemma:Dfg}, $[L,L]\subseteq\Delta^+(G)$. Luego
%	$\Delta(G)/\Delta^+(G)$ es abeliano y libre de torsión.
%%
%%	Para demostrar la segunda afirmación basta observar que si $\Delta^+(G)=1$
%%	entonces, por el primer ítem, $\Delta(G)$ es abeliano, finitamente generado
%%	y libre de torsión. Luego $K[\Delta(G)]$ es un dominio por el
%%	lema~\ref{lemma:K[abelian]}. 
%\end{proof}

Si $S$ es un subconjunto finito de un grupo $G$ se define
$\widehat{S}=\sum_{x\in S}x$. 

\begin{lemma}
	\label{lemma:sumN}
	Sea $N$ un subgrupo normal finito de $G$. Entonces $\widehat{N}$ es central
	en $K[G]$ y además $\widehat{N}(\widehat{N}-|N|1)=0$.
\end{lemma}

\begin{proof}
	Supongamos que $N=\{n_1,\dots,n_k\}$ y 
	sea $g\in G$. Como la función $N\to N$, $n\mapsto gng^{-1}$, es una biyección, 
	\[
		g\widehat{N}g^{-1}=g(n_1+\cdots+n_k)g^{-1}=gn_1g^{-1}+\cdots+gn_kg^{-1}=\widehat{N}.
	\]
	Como $nN=N$ si $n\in N$, se tiene que $n\widehat{N}=\widehat{N}$. Luego
	$\widehat{N}\widehat{N}=\sum_{j=1}^k n_j\widehat{N}=|N|\widehat{N}$.
\end{proof}

Necesitamos el siguiente teorema:

\begin{theorem}[Dietzmann]
	\index{Teorema!de Dietzmann}
	\label{theorem:Dietzmann} 
	Sea $G$ un grupo y sea $X\subseteq G$ un
	subconjunto finito de $G$ cerrado por conjugación. Si existe $n$ tal
	que $x^n=1$ para todo $x\in X$, entonces $\langle X\rangle$ es un subgrupo
	finito de $G$.
\end{theorem}

\begin{proof}
	Sea $S=\langle X\rangle$. Como $x^{-1}=x^{n-1}$, todo elemento de $S$ puede
	escribirse como producto (finito) de elementos de $X$. 
	
	Fijemos $x\in X$. Vamos a demostrar que si $x\in X$ aparece $k\geq 1$ veces
	en la representación de una palabra $s$, podemos escribir a $s$ como producto de $m$
	elementos de $X$ donde los primeros $k$ son iguales a $x$.  Supongamos que
	\[
	s=x_1x_2\cdots x_{t-1}xx_{t+1}\cdots x_m,
	\]
	donde cada $x_j\ne x$ para todo $j\in\{1,\dots,t-1\}$. Entonces
	\[
		s=x(x^{-1}x_1x)(x^{-1}x_2x)\cdots (x^{-1}x_{t-1}x)x_{t+1}\cdots x_m
	\]
	es producto de $m$ elementos de $X$ pues $X$ es cerrado por conjugación, y
	el primer elemento es nuestro $x$. Este mismo argumento implica que $s$
	puede escribirse como
	\[
		s=x^ky_{k+1}\cdots y_m,
	\]
	donde los $y_j$ son elementos de $X\setminus\{x\}$.

	Sea ahora $s\in S$ y escribamos a $s$ como producto de $m$ elementos de $X$,
	donde $m$ es el mínimo posible.  Para ver que $S$ es finito basta ver que 
	$m\leq (n-1)|X|$. 
	
	Si suponemos que $m>(n-1)|X|$, 
	al menos un $x\in X$ aparecería $n$ veces en la
	representación de $s$. Sin pérdida de generalidad, podríamos escribir 
	\[
		s=x^nx_{n+1}\cdots x_m=x_{n+1}\cdots x_m,
	\]
	una contradicción a la minimalidad de $m$. 
\end{proof}

Antes de seguir hacia nuestro objetivo demostraremos un teorema de Schur:

\begin{theorem}[Schur]
\index{Teorema!de Schur}
\label{thm:Schur}
	Si $Z(G)$ tiene índice finito en $G$ entonces $[G,G]$ es finito.
\end{theorem}

\begin{proof}
	Supongamos que $(G:Z(G))=n$. 
	Sea $X$ el conjunto de conmutadores de $G$. El conjunto $X$ es finito pues como la función
	\[
		\varphi\colon X\to G/Z(G)\times G/Z(G),\quad [x,y]\mapsto (xZ(G),yZ(G)),
	\]
	es inyectiva, se tiene que $|X|\leq n^2$. Para ver que $\varphi$ es
	inyectiva supongamos que $(xZ(G),yZ(G))=(uZ(G),vZ(G))$. Entonces $u^{-1}x\in Z(G)$, 
	$v^{-1}y\in Z(G)$ y luego 
	\begin{align*}
		[u,v]&=uvu^{-1}v^{-1}=uv(u^{-1}x)x^{-1}v^{-1}=xvx^{-1}(v^{-1}y)y^{-1}=xyx^{-1}y^{-1}=[x,y].
	\end{align*}
	Además $X$ es cerrado por conjugación pues
	\[
		g[x,y]g^{-1}=[gxg^{-1},gyg^{-1}]
	\]
	para todo $g,x,y\in G$. Como $g\mapsto g^n$ es un morfismo de grupos $G\to
	Z(G)$, lema~\ref{lem:center} implica que $[x,y]^n=[x^n,y^n]=1$ para todo
	$[x,y]\in X$.  Luego el teorema queda demostrado al aplicar el
	teorema~\ref{theorem:Dietzmann} de Dietzmann.
\end{proof}


Si $G$ es un grupo, consideramos el subconjunto %los siguientes subconjuntos:
\begin{align*}
%	&\Delta(G)=\{x\in G:(G:C_G(x))<\infty\},\\
	&\Delta^+(G)=\{x\in \Delta(G):\text{$x$ tiene orden finito}\}.
\end{align*}

\begin{lemma}
	\label{lem:DcharG}
	Si $G$ es un grupo, entonces $\Delta^+(G)$ es un subgrupo
	característico de $G$.
\end{lemma}

\begin{proof}
	Claramente $1\in\Delta^+(G)$. 
	Sean $x,y\in\Delta^+(G)$ y sea $H$ el subgrupo de $G$ generado por el
	conjunto $C$ formado por los finitos conjugados de $x$ e $y$. Si $|x|=n$ y
	$|y|=m$, entonces $c^{nm}=1$ para todo $c\in C$. Como $C$ es 
	finito y cerrado por conjugación, el teorema de Dietzmann implica que $H$ es
	finito. Luego $H\subseteq\Delta^+(G)$ y en particular $xy^{-1}\in\Delta^+(G)$.  Es
	evidente que $\Delta^+(G)$ es un subgrupo característico pues para todo
	$f\in\Aut(G)$ se tiene que $f(x)\in\Delta^+(G)$ si $x\in\Delta^+(G)$.
%	Primero veamos que $\Delta(G)$ es un subgrupo de $G$. Si $x,y\in\Delta(G)$
%	y $g\in G$, entonces $g(xy^{-1})g^{-1}=(gxg^{-1})(gyg^{-1})^{-1}$. Además
%	$1\in\Delta(G)$. Veamos ahora que $\Delta(G)$ es característico en $G$. Si
%	$f\in\Aut(G)$ y $x\in G$, entonces, como $f(gxg^{-1})=f(g)f(x)f(g)^{-1}$,
%	se concluye que $f(x)\in\Delta(G)$.
%	Para ver que $\Delta^+(G)$ es un subgrupo, 
%	Sean
%	$x_1,\dots,x_n\in\Delta^+(G)$ y $H=\langle x_1,\dots,x_n\rangle$. Como
%	$H$ es finito, $H\subseteq\Delta^+(G)$ y luego $\Delta^+(G)$ es un
%	subgrupo. Es evidente que es un subgrupo característico pues para todo
%	$f\in\Aut(G)$ se tiene que $f(x)\in\Delta^+(G)$ si $x\in\Delta^+(G)$.
\end{proof}

La segunda aplicación del teorema de Dietzmann es el siguiente resultado:

\begin{lemma}
	\label{lem:Connel}
	Sea $G$ un grupo y sea  $x\in\Delta^+(G)$.  Existe entonces un subgrupo
	finito $H$ normal en $G$ tal que $x\in H$.
\end{lemma}

Dejamos la demostración como ejercicio ya que el muy similar a lo que hicimos
en la demostración del lema~\ref{lem:DcharG}.

%\begin{proof}
%	Sea $H$ el subgrupo generado por los conjugados de $x$. Como $x$ tiene
%	finitos conjugados, $H$ es finitamente generado. Además $H$ es claramente
%	normal en $G$ y está generado por elementos de torsión. Todos los finitos
%	generadores de $H$ tienen el mismo orden, digamos $n$. Por el teorema de
%	Dietzmann, $H$ resulta ser un grupo finito.
%\end{proof}

\begin{theorem}[Connell]
	\label{thm:Connel}
	\index{Teorema!de Connel}
	Supongamos que el cuerpo $K$ es de característica cero. 
	Sea $G$ un grupo. Las siguientes afirmaciones son equivalentes:
	\begin{enumerate}
		\item $K[G]$ es primo.
		\item $Z(K[G])$ es primo.
		\item $G$ no tiene subgrupos finitos normales no triviales.
		\item $\Delta^+(G)=1$.
	\end{enumerate}
\end{theorem}

\begin{proof}
	Demostremos que $(1)\implies(2)$. Como $Z(K[G])$ es un anillo conmutativo,
	probar que es primo es equivalente a probar que no existen divisores de
	cero no triviales. Sean $\alpha,\beta\in Z(K[G])$ tales que
	$\alpha\beta=0$. Sean $A=\alpha K[G]$ y $B=\beta K[G]$. Como $\alpha$ y
	$\beta$ son centrales, $A$ y $B$ son ideales de $K[G]$. Como $AB=0$,
	entonces $A=\{0\}$ o $B=\{0\}$ pues $K[G]$ es primo.  Luego $\alpha=0$ o
	$\beta=0$.

	Demostremos ahora que $(2)\implies(3)$. Sea $N$ un subgrupo normal finito.
	Por el lema~\ref{lemma:sumN}, $\widehat{N}=\sum_{x\in N}x$ es central en
	$K[G]$ y $\widehat{N}(\widehat{N}-|N|1)=0$. Como $\widehat{N}\ne 0$ (pues
	$K$ tiene característica cero) y $Z(K[G])$ es un dominio,
	$\widehat{N}=|N|1$, es decir: $N=\{1\}$.

	Demostremos que $(3)\implies(4)$. Sea $x\in\Delta^+(G)$. Por el
	lema~\ref{lem:Connel} sabemos que existe un subgrupo finito $H$ normal en
	$G$ que contiene a $x$. Como por hipótesis $H$ es trivial, se concluye que
	$x=1$.

	Finalmente demostramos que $(4)\implies(1)$. Sean $A$ y $B$ ideales de
	$K[G]$ tales que $AB=0$. Supongamos que $B\ne 0$ y sea $\beta\in
	B\setminus\{0\}$.  Si $\alpha\in A$, entonces, como $\alpha
	K[G]\beta\subseteq \alpha B\subseteq AB=0$, el lema~\ref{lem:Passman} de
	Passman implica que $\pi_{\Delta(G)}(\alpha)\pi_{\Delta(G)}(\beta)=0$.
	Como por hipótesis $\Delta^+(G)$ es trivial, sabemos que $\Delta(G)$ es 
	libre de torsión y luego $\Delta(G)$ es abeliano por el
	lema~\ref{lem:FCabeliano}. Esto nos dice que $K[\Delta(G)]$ no tiene
	divisores de cero y luego $\alpha=0$. Demostramos entonces que $B\ne0$
	implica que $A=0$.
\end{proof}

% necesito: 
% pag 376 del Hungerford: un módulo no nulo admite una serie de composición
% si y sólo si es noetheriano y artiniano
% agregar además el teorema de Hopkins--Levitzky que dice


\begin{theorem}[Connel]
	Sea $K$ un cuerpo de característica cero y sea $G$ un grupo. Entonces
	$K[G]$ es artiniano a izquierda si y sólo si $G$ es finito.
\end{theorem}

\begin{proof}
	Si $G$ es finito, $K[G]$ es un álgebra de dimensión finita y luego
	es artiniano a izquierda. Supongamos entonces que $K[G]$ es artiniano
	a izquierda. 
	
	Primero observemos que si $K[G]$ es un álgebra prima, entonces por el
	teorema de Wedderburn $K[G]$ es simple y luego
	$G$ es el grupo trivial (pues si $G$ no es trivial, $K[G]$ no es simple ya
	que el ideal de aumentación es un ideal no nulo de $K[G]$).

	Como $K[G]$ es artiniano a izquierda, es noetheriano a izquierda por
	Hopkins--Levitzky y entonces, $K[G]$ admite una serie de composición por el
	teorema~\ref{thm:serie_de_composicion}.  Para demostrar el teorema
	procederemos por inducción en la longitud de la serie de composición de
	$K[G]$. Si la longitud es uno, $\{0\}$ es el único ideal de $K[G]$ y luego
	$K[G]$ es prima y el resultado está demostrado. Si suponemos que el
	resultado vale para longitud $n$ y además $K[G]$ no es prima, entonces, por
	el teorema de Connel, $G$ posee un subgrupo normal $H$ finito y no trivial. Al
	considerar el morfismo canónico $K[G]\to K[G/H]$ vemos que $K[G/H]$ es
	artiniano a izquierda y tiene longitud $<n$. Por hipótesis inductiva, $G/H$
	es un grupo finito y luego, como $H$ también es finito, $G$ es finito.
\end{proof}
