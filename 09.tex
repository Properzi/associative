\chapter{}
\label{09}

\topic{Andrunakevic--Rjabuhin's theorem}
%https://ysharifi.wordpress.com/2010/06/04/about-reduced-rings-1/
%\begin{exercise}
%\label{xca:reduced}
%     Let $R$ be a ring and $I$ be an ideal of $R$.
%     Prove that $I$ is prime if and only if $xRy\subseteq I$ implies
%     either $x\in I$ or $y\in I$. 
% \end{exercise}

% \begin{sol}{xca:reduced}
%     Let $A$ and $B$ be ideals such that $AB\subseteq I$. If 
%     $A\not\subseteq I$ and $B\not\subseteq J$, let 
%     $x\in A\setminus P$ and $y\in B\setminus P$. Then
%     $xRy\subseteq AB\subseteq I$, a contradiction. 
%     Conversely, if $xRy\subseteq I$ and $x\not I$ and $y\not\in I$, 
%     then $A=(x)\not\subseteq I$ and $B=(y)\not\subseteq P$.
% \end{sol}

\begin{definition}
\index{Ring!reduced}
     A ring $R$ is \textbf{reduced} if 
     has no non-zero nilpotent elements. 
\end{definition}

Every commutative domain is reduced. 

\begin{example}
    The ring $\Z\times\Z$ with the usual operations 
    is reduced but not a domain. 
\end{example}

\begin{example}
    The ring $\Z/6$ is reduced. However, $\Z/4$ is not reduced. 
\end{example}

\begin{exercise}
\label{xca:reduced}
    Prove that a ring $R$ is \textbf{reduced} if and only 
    if for all $r\in R$ such that $r^2=0$ one has $r=0$.
\end{exercise}

%\begin{exercise}
%\label{xca:reduced_Zn}
%    Let $n\geq2$. Then $\Z/n$ is reduced 
%    but not a domain if and only if $n$ is square-free 
%    but not prime.
%\end{exercise}

\begin{exercise}
    \label{xca:reduced_RX}
    Let $R$ be a commutative ring that is reduced but not a domain.
    Prove that $R[X]$ is reduced but not a domain. 
\end{exercise}

The previous exercise and induction 
shows that if $R$ is reduced but not a domain, 
then so is $R[X_1,\dots,X_n]$. 
% https://ysharifi.wordpress.com/2010/11/01/zero-divisors-and-nilpotent-elements-in-polynomial-rings/

\begin{example}
    Let $R=\Z/3\times\Z/3$ with
    operations $(a,b)+(c,d)=(a+c,b+d)$ and 
    $(a,b)(c,d)=(ac,ad+bc)$. Then $R$ is a commutative
    ring with identity $(1,0)$. Since 
    $(0,1)$ is a non-zero nilpotent element, $R$ is not reduced. 
\end{example}

\begin{definition}
\index{Ideal!reduced}
    Let $R$ be a ring and $I$ be an ideal of $R$. 
    Then $I$ is \textbf{reduced} if $R/I$ is a reduced ring. 
\end{definition}

Let $R$ be a ring and 
$I$ be a reduced ideal of $R$. If $ab\in I$, then 
$ba\in I$. In fact, since $ab\in I$, 
$(ba)^2=b(ab)a\in I$.
Since $R/I$ is reduced, $ba\in I$. 


 \begin{theorem}[Andrunakevic--Rjabuhin]
 \label{thm:AndrunakevicRjabuhin}
 \index{Andrunakevic--Rjabuhin's theorem}
 	Let $R$ be a non-zero ring. If $R$ is reduced, there exists
 	an ideal $I$ of $R$ such that 
 	then $R/I$ has no non-zero zero-divisors. 
 \end{theorem}

 Let $R$ be a ring and $I$ be an ideal of $R$. If $S$ 
 is a subset of $R$, the \emph{left annihilator} of $S$
 modulo $I$ is the set $\{r\in R:rS\subseteq I\}$.  

 \begin{lemma}
    Let $R$ be a ring and $I$ be a reduced ideal. 
    If $S\subseteq R$ is a subset, then 
    the left annihilator of $S$ modulo $I$ 
    is a reduced ideal. 
 \end{lemma}

\begin{proof}
    We need to show that $A=\{r\in R:rS\subseteq I\}$ 
    is a reduced ideal. 
    A straightforward calculation shows that $A$ is 
    a left ideal. We claim that $A$ is a right ideal. Let $r\in R$
    and $a\in A$. Then 
    $as\in I$ for all $s\in S$. Since $I$ is reduced, $sa\in I$ for all $s\in S$. Since 
    $I$ is an ideal of $R$, $sar\in I$
    for all $s\in S$. Using again 
    that $I$ is reduced, 
    $ars\in I$ for all $s\in S$. Thus 
    $ar\in A$. 
    
    We now claim that $A$ is reduced. If $a^2\in A$, then 
    $aas=a^2s\in I$ for all $s\in S$. 
    Since $I$ is reduced, $asa\in I$ for
    all $s\in S$. Thus $(as)^2=(asa)s\in I$ for all $s\in S$. Since $I$ 
    is reduced, $as\in I$ for all $s\in S$. Hence $a\in A$. 
\end{proof}

\begin{exercise}
    Let $R$ be a ring and $I$ be a reduced ideal. 
    If $S\subseteq R$ is a subset, then 
    the right annihilator of $S$ modulo $I$ 
    is a reduced ideal. 
\end{exercise}

\begin{proof}[Proof of Theorem \ref{thm:AndrunakevicRjabuhin}]
    Let $x\in R\setminus\{0\}$. Let $X$ 
    be the set of reduced ideals $I$ such that 
    $x\not\in I$. Since $R$ is reduced, $\{0\}$ 
    is a reduced ideal and hence $X\ne\emptyset$. 
    A standard application of Zorn's lemma shows that
    there exists a maximal element $M\in X$. 
    
    We claim that $R/M$ has no non-zero divisors. If not, 
    there exist $a,b\in R$ such that $ab\in M$, $a\not\in M$ 
    and $b\not\in M$. Let $A$ be the left annihilator of $\{b\}$ 
    modulo $M$ and $B$ be the right annihilator of $\{a\}$ 
    modulo $M$. By the previous lemma, $A$ and $B$ 
    are reduced ideals of $R$. Since  
    $a\in A$, $M\subsetneq A$. Similarly, since 
    $b\in B$, $M\subsetneq B$. Moreover, $AB\subseteq M$. 
    Since $x\in A\cap B$, $x^2\in AB\subseteq M$. Since 
    $M$ is reduced, $x\in M$, a contradiction. 
\end{proof}

\begin{exercise}
    Prove that a reduced ring is a subdirect product
    of rings without no non-zero divisors. 
\end{exercise}

\begin{exercise}
    Is the ring $\C[\Z/2]$ reduced? 
\end{exercise}

% Let $G=\langle g:g^2=1\rangle$. If $(a+bg)^2=0$, then
% $(a^2+b^2)+(2ab)g=0$. Thus $a=b=0$. 

\begin{problem}
\label{prob:reduced}
    Let $G$ be a torsion-free group. Is
    $K[G]$ is reduced?
\end{problem}

Problem \ref{prob:reduced} is related to other important
open problems about group algebras 
(e.g. zero-divisors, units, 
indempotents and semisimplicity of group
rings).

\begin{exercise}
\label{xca:reduced_central}
    Prove that idempotents of reduced rings are central. 
\end{exercise}

The previous exercise is used to solve the following problem.

\begin{exercise}
\label{xca:x^3=x}
    Let $R$ be a ring such that $x^3=x$ for all $x\in R$. Prove that
    $R$ is commutative. 
\end{exercise}

Exercise \ref{xca:x^3=x} is hard. 
Even harder is the following exercise:

\begin{exercise}
\label{xca:x^4=x}
    Let $R$ be a ring such that $x^4=x$ for all $x\in R$. Prove
    that $R$ is commutative. 
\end{exercise}

%Other exercises about reduced rings. 

%\begin{exercise}
%\label{xca:reduced}
%    Prove that a ring is reduced if 
%    and only it has no non-zero nilpotent elements. 
%\end{exercise}

\begin{exercise}
\label{xca:reduced=>semiprime}
    Reduced rings are semiprime.
\end{exercise}
 
\begin{theorem}
\label{thm:reduced}
    Let $K$ be a field and $G$ be a group. If $K[G]$
    is reduced, then every finite subgroup of $G$ is normal. 
\end{theorem}

\begin{proof}
    Let $H=\{h_1,\dots,h_n\}$ be a finite normal subgroup of $G$. 
    We claim that $n=|H|$ is invertible in $K$. If $\ch K=0$, this 
    is clear. If $\ch K=p>0$ and $n$ is not invertible in $K$, 
    then $p$ divides $n=|H|$. By Cauchy's theorem, 
    there exists an element $h\in H$ of order $n$, that is 
    $|h|=n$. Since $(1-h)^p=1-h^p=0$ and $K[G]$ is reduced,
    $h=1$, a contradiction. 
    
    Let $\alpha=\frac{1}{n}\sum_{i=1}^nh_i\in K[G]$. Then
    \[
    \alpha^2=\frac{1}{n^2}\sum_{i=1}^n\sum_{j=1}^nh_ih_j
    =\frac{1}{n^2}\sum_{i=1}^nn\alpha=\alpha.
    \]
    Thus $\alpha$ is idempotent. As idempotent 
    element of reduced rings are central (Exercise \ref{xca:reduced_central}), 
    $g\alpha g^{-1}=\alpha$ for all $g\in G$. If $g\in G$, 
    then 
    \[
    \sum_{i=1}^n gh_ig^{-1}=\sum_{i=1}^n h_i.
    \]
    It follows that $H$ is normal in $G$, 
    as for each $i\in\{1,\dots,n\}$ 
    there exists $j\in\{1,\dots,n\}$ such that 
    $gh_ig^{-1}=h_j\in H$. 
\end{proof}

\begin{example}
    If $K$ is a field, then $K[\Sym_3]$ is not reduced. 
    In fact, 
    if 
    \[
    \alpha=(12)+(123)-(132)-(13),
    \]
    then 
    $\alpha^2=0$. 
\end{example}

\begin{exercise}
    Prove that the converse of Theorem \ref{thm:reduced} 
    does not hold. 
\end{exercise}

\topic{Rickart's theorem}


We now consider Jacobson's semisimplicity problem. 

\begin{openproblem}
\label{Jacobson's semisimplicity problem}
Let $G$ be a group and $K$ be a field. When $J(K[G])=\{0\}$?
\end{openproblem}

As an application of Amitsur's theorem \ref{thm:Amitsur}, 
we prove that 
complex group algebras have null Jacobson radical.
This is known as 
Rickart's theorem. The original proof found by Rickart 
uses complex analysis. Here, however, 
we present an algebraic proof. 

\begin{theorem}[Rickart]
\index{Rickart's theorem}
\label{thm:Rickart}
    Let $G$ be a group. Then $J(\C[G])=\{0\}$.
\end{theorem}

To prove the theorem, we need a lemma.

\begin{lemma}
Let $G$ be a group. Then $J(\C[G])$ is nil.        
\end{lemma}

\begin{proof}
    We need to show that every element of $J(\C[G])$ is nilpotent. 
    If $G$ is countable, then the result follows from Amitsur's theorem \ref{thm:Amitsur}. So assume that 
    $G$ is not countable. Let $\alpha\in J(\C[G])$, say
    \[
    \alpha=\sum_{i=1}^n\lambda_ig_i,
    \]
    where $\lambda_1,\dots,\lambda_n\in\C$ and $g_1,\dots,g_n\in G$. Let $H=\langle g_1,\dots,g_n\rangle$.
    Then $\alpha\in \C[H]$ and $H$ is countable. We claim that $\alpha\in J(\C[H])$. Decompose
    $G$ as a disjoint union 
    \[
    G=\bigcup_\lambda x_\lambda H
    \]
    of cosets of $H$ in $G$. Then $\C[G]=\bigoplus_\lambda x_\lambda\C[H]$ and
    hence $\C[G]=\C[H]\oplus K$ for some right module $K$ over $\C[H]$ (this follows
    from the fact that one of the cosets is that of $H$). Since $\alpha\in J(\C[G])$, for each 
    $\beta\in\C[H]$ there exists $\gamma\in\C[G]$ such that 
    $\gamma(1-\beta\alpha)=1$. Write $\gamma=\gamma_1+\kappa$ for $\gamma_1\in\C[H]$ and $\kappa\in K$. Then
    \[
    1=\gamma(1-\beta\alpha)=\gamma_1(1-\beta\alpha)+\kappa(1-\beta\alpha)
    \]
    and hence $\kappa(1-\beta\alpha)\in K\cap \C[H]=\{0\}$, as $\beta\in\mathbb{C}[H]$. 
    Since $1=\gamma_1(1-\beta\alpha)$, it follows that
    $\alpha\in J(\C[H])$ and the lemma follows from Amitsur's theorem \ref{thm:Amitsur}.  
\end{proof}

We now prove the theorem. 

\begin{proof}[Proof of Theorem \ref{thm:Rickart}]
    For $\alpha=\sum_{i=1}^n\lambda_ig_i\in\C[G]$ let 
    \[
    \alpha^*=\sum_{i=1}^n\overline{\lambda_i}g_i^{-1}.
    \]
    Then $\alpha\alpha^*=0$ if and only if $\alpha=0$ and, moreover, 
    $(\alpha\beta)^*=\beta^*\alpha^*$ for all $\beta\in\C[G]$. 
    Assume that $J(\C[G])\ne\{0\}$ and let $\alpha\in J(\C[G])\setminus\{0\}$. Then
    $\beta=\alpha\alpha^*\in J(\C[G])$, as $J(\C[G])$ is an ideal of $\C[G]$. Moreover, the previous 
    lemma implies that $\beta$ is nilpotent. Note that $\beta\ne 0$, as $\alpha\ne0$. Now  
    \[
    (\beta^m)^*=(\beta^*)^m=\beta^m
    \]
    for all $m\geq1$. If there exists $k\geq2$ such that $\beta^k=0$ and $\beta^{k-1}\ne 0$, then
    \[
    \beta^{k-1}\left(\beta^{k-1}\right)^*=\beta^{2k-2}=0
    \]
    and hence $\beta^{k-1}=0$, a contradiction. Thus $\beta=0$ and therefore $\alpha=0$. 
\end{proof}

\begin{exercise}
	If $G$ is a group, then $J(\R[G])=0$. 
\end{exercise}

% To obtain a consequence of Rickart's theorem we need two lemmas. 

% \begin{lemma}[Nakayama]
% 	\label{lem:Nakayama}
% 	\index{Nakayama's lemma}
% 	Let $R$ be a unitary ring and $M$ be a finitely generated module. If 
% 	$J(R)\cdot M=M$, then $M=\{0\}$.
% \end{lemma}

% \begin{proof}
%     Since $M$ is finitely generated, we may assume that 
% 	$M=(x_1,\dots,x_n)$. Since $x_n\in M=J(R)\cdot M$, 
% 	there exist $r_1,\dots,r_n\in J(R)$ such that $x_n=r_1\cdot x_1+\cdots+r_n\cdot x_n$, that is 
% 	$(1-r_n)\cdot x_n=\sum_{j=1}^{n-1}r_j\cdot x_j$. 
% 	Since $1-r_n$ is invertible, there exists $s\in R$ such that $s(1-r_n)=1$. Thus 
% 	$x_n=\sum_{j=1}^{n-1}(sr_j)\cdot x_j$ 
% 	and hence $M=(x_1,\dots,x_{n-1})$. Repeating this procedure several times 
% 	one obtains $M=\{0\}$.
% \end{proof}

% \begin{lemma}
% 	\label{lem:Rickart}
% 	Let $\iota\colon R\to S$ be a homomorphism of unitary rings. If	
% 	\[
% 	S=\iota(R)x_1+\cdots+\iota(R)x_n,
% 	\]
% 	where each $x_j$ is such that $x_jy=yx_j$ for all $y\in\iota(R)$, then 
% 	$\iota(J(R))\subseteq J(S)$.
% \end{lemma}

% \begin{proof}
% 	We claim that $J=\iota(J(R))$ acts trivially on each simple $S$-module $M$.
% 	If is $M$ is a simple module over $S$, then, in particular, $M=S\cdot m$ for some $m\ne0$. 
% 	Now $M$ is a module over $R$ with $r\cdot m=\iota(r)\cdot m$. Since 
% 	\[
% 		M=S\cdot m=(\iota(R)x_1+\cdots+\iota(R)x_n)\cdot m=\iota(R)\cdot (x_1\cdot m)+\cdots+\iota(R)\cdot (x_n\cdot m),
% 	\]
% 	it follows that 
% 	$M$ is finitely generated as a module over $\iota(R)$. Moreover, 
% 	\[
% 	J(R)\cdot
% 	M=J\cdot M=\iota(J)\cdot M
% 	\]
% 	is an $S$-submodule of $M$, as 
% 	\[
% 		x_j\cdot (J\cdot M)=(x_j J)\cdot M=(J x_j)\cdot M=J\cdot (x_j\cdot M)\subseteq J\cdot M.
% 	\]
% 	Since $M\ne\{0\}$, Nakayama's lemma implies that $J(R)\cdot M\subsetneq M$. The simplicity of 
% 	the $S$-module $M$ implies that $J(R)\cdot M=\{0\}$.
% \end{proof}

% We now obtain the following consequence of Rickart's theorem. 

% \begin{theorem}
% 	If $G$ is a group, then $J(\R[G])=0$. 
% \end{theorem}

% \begin{proof}
% 	Let $\iota\colon \R[G]\to\C[G]$ be the canonical inclusion. Since 
% 	\[
% 	\C[G]=\R[G]+i\R[G],
% 	\]
% 	Lemma~\ref{lem:Rickart} and Rickart's theorem imply that 
% 	$\iota(J(\R[G]))\subseteq J(\C[G])=0$. Thus $J(\R[G])=0$, as $\iota$ is injective. 
% \end{proof}

\begin{corollary}
\label{cor:C[G]_semiprime}
	The ring $\C[G]$ is semiprime.
\end{corollary}

\begin{proof}
	Since $J(\C[G])=\{0\}$ by Rickart's theorem and the Jacobson radical
	contains every nil ideal by Proposition~\ref{pro:nilJ}, it follows that
	$\C[G]$ does not contain non-trivial nil ideals. Thus $\C[G]$ does not
	contain non-trivial nilpotent ideals and hence $\C[G]$ is semiprime.
\end{proof}

\begin{exercise}
	Prove that $Z(\C[G])$ is semiprime.
\end{exercise}

We now characterize when complex group algebras 
are left artinian. For that purpose,
we need a lemma. This is similar to one of the implications proved in Proposition \ref{pro:semisimple}. However,
in the arbitrary setting we are considering, we need to use Zorn's lemma. 

\begin{lemma}
    Let $M$ be a semisimple module and $N$ be a submodule. 
    Then $N$ is a direct summand.
\end{lemma}

\begin{proof}[Sketch of the proof]
    Let $M=\oplus_{i\in I}M_i$ be a direct sum of simple modules  
    and let $i\in I$. 
    Since $N\cap M_i$ is a submodule of $M_i$ and $M_i$ is simple, it follows
    that $N\cap M_i=\{0\}$ or $N\cap M_i=M_i$. If
    $N\cap M_i=M_i$ for all $i\in I$, then $N=M$ and the lemma is proved. So we may assume
    that there exists $i\in I$ such that $N\cap M_i=\{0\}$. Let $X$ be the set
    of subsets $J$ of $I$ such that $N\cap (\oplus_{j\in J}M_j)=\{0\}$. Our assumptions
    imply that $X$ is non-empty. Zorn's lemma implies the existence of 
    a maximal element $K$. Let $N_1=\oplus_{k\in K}M_k$. We claim that
    $N\oplus N_1=M$. If not, there exists $i\in I$ such that
    $M_i\not\subseteq N\oplus N_1$. The simplicity of $M_i$ implies that
    $M_i\cap (N\oplus N_1)=\{0\}$, which contradicts the maximality of $K$. 
\end{proof}

A direct application of the lemma proves that
complex group algebras of infinite groups are never semisimple. 

\begin{proposition}
    \label{pro:KGsemisimple}
    If $G$ is an infinite group, then $\C[G]$ is not semisimple. 
\end{proposition}

\begin{proof}
	Assume that $R=\C[G]$ is semisimple.  Let $I$ 
	be the augmentation ideal of $R$, that is
	\[
	I=\left\{\alpha=\sum_{g\in G}\lambda_gg\in R:\sum_{g\in G}\lambda_g=0\right\}.
	\]
	By the previous lemma, 
	there exists a non-zero ideal $J$ such that 
	$R=I\oplus J$. Since $R$ is unitary, there exist $e\in I$ and $f\in J$ such that
	$1=e+f$. If
	$x\in I$, then $x=xe+xf$ and hence $xf=x-xe\in I\cap J=\{0\}$. Since 
	$x=xe$ for all $x\in I$, it follows that $e=e^2$. Similarly, one proves
	that $f^2=f$. Moreover, $ef=0$, as $ef\in I\cap J=\{0\}$.  Since $I$ 
	is the augmentation ideal of $R$ and $If=(Re)f=R(ef)=\{0\}$ (note that $I=Re$ because $x=xe$ for all $x\in I$), we conclude that
	$(g-1)f=0$
	for all $g\in G$, as $g-1\in I$. If $f=\sum_{h\in
	G}\lambda_hh$ (finite sum), then  
	\[
	f=gf=\sum_{h\in G}\lambda_h(gh)=\sum_{h\in
	G}\lambda_{g^{-1}h}h.
	\]
	Thus $\lambda_h=\lambda_{g^{-1}h}$ for all $g,h\in G$. Since $G$ 
	is infinite, some $\lambda_g=0$ and hence $f=0$. Thus $e=1$ and $I=\C[G]$, a contradiction. 
% 	If $f=0$, then $e=1$ and $I=\C[G]$, a contradiction.  
% 	, a contradiction because 
% 	$f\ne 0$ implies that the sum that defines $f$ should be an infinite sum.
\end{proof}

% The ideal $I(G)$ used in the proof of the previous proposition 
% is known as the \textbf{augmentation ideal} 
% of $\C[G]$.

\begin{theorem}
	Let $G$ be a group. Then $\C[G]$ 
	is left artinian if and only if 
	$G$ is finite. 
\end{theorem}

\begin{proof}
    If $G$ is finite, then $\C[G]$ is left artinian because $\dim\C[G]=|G|<\infty$. So assume that 
    $G$ is infinite. By Rickart's theorem,   
	$J(\C[G])=0$. Moreover, $\C[G]$
	is not semisimple by the previous proposition. Thus
	$\C[G]$ is not left artinian by Theorem~\ref{thm:SSartin=J}.
\end{proof}


