\chapter{}
\label{09}

\topic{Local rings}

In this section, we will consider arbitrary rings with one. 

\begin{definition}
    \index{Ring!local}
    A ring is said to be \textbf{local} if it contains only one maximal left ideal. 
\end{definition}

Division rings are local rings. 

% \begin{exercise}
%     Let $R$ be a commutative ring with one and $P$ be a prime ideal. Let  
%     $S=R\setminus P$. Then the localization $S^{-1}R$ is a local ring with maximal ideal $S^{-1}P$. 
% \end{exercise}

\begin{theorem}
\label{thm:local}
    Let $R$ be a ring and $I=R\setminus\mathcal{U}(R)$. The following
    statements are equivalent:
    \begin{enumerate}
        \item $R$ is local.
        \item $R/J(R)$ is a division ring.
        \item $I=J(R)$.
        \item $I$ is an ideal of $R$.
    \end{enumerate}
\end{theorem}

\begin{proof}
    We first prove $1)\implies2)$. Let $M$ be the maximal left ideal of $R$. Then $J(R)=M$. 
    Let $x\not\in M$. Then $R=Rx+M$, so $1=rx+m$ for some $r\in R$ and $m\in M$. Thus  
    $r+M$ is a left inverse of $x+M$. In particular, 
    $r\not\in M$. Since $R=Rr+M$, there exists $y\in R$ such that $1=yr$. Therefore
    $y+M$ is a left inverse of $r+M$. Thus 
    \begin{align*}
    y+M&=(y+M)(1+M)=(y+M)(r+M)(x+M)\\
    &=(yr+M)(x+M)=(1+M)(x+M)=x+M
    \end{align*}
    and hence $x+M$ is invertible. 

    Now we prove $2)\implies3)$. Clearly $J(R)\subseteq I$. 
    Conversely, let $x\in I$. If $x\not\in J(R)$, then
    $x+J(R)\ne J(R)$. Since $R/J(R)$ is a division ring, 
    $x+J(R)\in\mathcal{U}(R/J(R))$. In particular, $1-xy\in J(R)$ and hence 
    $xy=1-(1-xy)\in\mathcal{U}(R)$. Thus $1=(xy)z=x(yz)$ for some $z\in R$ and therefore $x\not\in I$, a contradiction. 
    
    It is trivial that $3)\implies4)$. 

    Finally, we prove $4)\implies 1)$. 
    Let $M$ be a maximal left ideal of $R$. Then $M\subseteq I$. Since $M$ 
    is maximal and $I$ is in particular a left ideal of $R$, 
    it follows that $M=I$. 
\end{proof}

\begin{definition}
    \index{Idempotent}
    An element $x$ of a ring is said to be \textbf{idempotent} 
    if $x^2=x$.   
\end{definition}

\index{Trivial idempotent}
Examples of idempotents are 0 and 1. 
An idempotent $x$ is said to be \textbf{non-trivial} if $x\not\in\{0,1\}$. 

\begin{exercise}
\label{xca:idempotents_modpm}
    Let $p$ be a prime number and $m>0$. 
    Prove that the only idempotents of $\Z/p^m$ are 0 and 1. 
\end{exercise}


\begin{exercise}
    \label{xca:idempotents_modn}
    How many idempotent does $\Z/n$ have?
\end{exercise}

\begin{exercise}
\label{xca:lifting_idempotents}
    Let $R$ be a ring with one and $I$ be an ideal of $R$. 
    We say that an idempotent $x\in R/I$ can be lifted if $x=e+I$ for
    some idempotent $e$ of $R$. 
    Prove that if every element of $I$ is nilpotent, then every 
    idempotent of $R/I$ can be lifted. 
\end{exercise}

The previous exercise shows that if $R$ is left artinian, 
every idempotent of $R/J(R)$ can be lifted to $R$. 

\begin{lemma}
\label{lem:J(R)_nil}
    Let $R$ be a left artinian ring. Then $J(R)$ is nil. 
\end{lemma}

\begin{proof}
    Let $x\in J(R)$. The sequence $Rx\supseteq Rx^2\supseteq\cdots$ stabilizes, so
    $Rx^n=Rx^{n+1}$ for some $n$. In particular, there exists $r\in R$ 
    such that $x^n=rx^{n+1}$. This implies that $(1-rx)x^n=0$. Since $x\in J(R)$, 
    the element $1-rx$ is invertible. Hence $x^n=0$.  
\end{proof}

\begin{theorem}
\label{thm:local_idempotent}
    Let $R$ be a left artinian ring. Then $R$ is local if and only if 
    $R$ has no non-trivial idempotents. 
\end{theorem}

\begin{proof}
    Let us first prove $\implies$. For this implication, we do not need to use that 
    $R$ is left artinian. Let $x\in R$ be an idempotent. Then $x(1-x)=0$. If $x\in\mathcal{U}(R)$, then 
    $x=1$. If $1-x\in\mathcal{U}(R)$, then $x=0$. If $x\not\in \mathcal{U}(R)$ and $1-x\not\in\mathcal{U}(R)$, 
    then, since $R\setminus\mathcal{U}(R)$ is an ideal of $R$, 
    it follows that 
    $1=x+1-x\not\in\mathcal{U}(R)$, a contradiction. 

    Now we prove $\impliedby$. By the previous lemma, $J(R)$ is nil. 
    By the previous exercise, every idempotent of $R/J(R)$ can be lifted. Thus $R/J(R)$ has 
    no non-trivial idempotents. On the other hand, by Artin--Wedderburn, 
    \[
    R/J(R)\simeq\prod_{i=1}^kM_{n_i}(D_i)
    \]
    for some $n_1,\dots,n_k\geq1$ and division rings $D_1,\dots,D_k$. Then 
    $k=n_1=1$, as $R/J(R)$ has no non-trivial idempotents. Since $R/J(R)$ is a division ring, 
    $R$ is local by the previous theorem. 
\end{proof}

\begin{theorem}
    The center of a local ring is local. 
\end{theorem}

\begin{proof}
    Let $R$ be a local ring. By Theorem \ref{thm:local}, $J(R)=R\setminus\mathcal{U}(R)$.  
    We need to prove that $Z(R)\setminus\mathcal{U}(Z(R))=J(Z(R))$. 
    We first note that
    \begin{equation}
    \label{eq:U(Z(R))}
        \mathcal{U}(Z(R))=Z(R)\cap\mathcal{U}(R).
    \end{equation}
    
    We claim that $Z(R)\cap J(R)\subseteq J(Z(R))$.  
    Let $x\in Z(R)\cap J(R)$. Let $z\in Z(R)$. Since $x\in J(R)$, $1-zx\in\mathcal{U}(R)$.
    Moreover, $1-zx\in Z(R)$. Thus 
    \[
    1-zx\in Z(R)\cap\mathcal{U}(R)=\mathcal{U}(Z(R)).
    \]
    Hence $x\in J(Z(R))$. 

    To prove the theorem it is enough to show that 
    $Z(R)\setminus\mathcal{U}(Z(R))=J(Z(R))$. Let us prove the non-trivial inclusion. 
    Let $x\in Z(R)\setminus\mathcal{U}(Z(R))$.  Then 
    \eqref{eq:U(Z(R))} implies that 
    $x\not\in\mathcal{U}(R)$. 
    By Theorem \ref{thm:local}, 
    $x\in J(R)$. Then $x\in J(R)\cap Z(R)\subseteq J(Z(R))$. 
\end{proof}

\begin{exercise}
\label{eq:local_center}
    Let $R$ be a local ring. Prove that 
    $Z(R)=J(R)\subseteq J(Z(R))$.  
\end{exercise}

\begin{exercise}
\label{xca:local_right}
    Prove that a ring is local if and only if it contains only one maximal right ideal.
\end{exercise}

\begin{exercise}
\label{xca:non_local1}
    Find a non-local ring with a unique maximal ideal. 
\end{exercise}

\begin{exercise}
\label{xca:non_local2}
    Let $R$ be a ring with at least three elements. 
    If $|\mathcal{U}(R)|=1$, then $R$ is not local. 
\end{exercise}

\index{Ring!Von Neumann regular}
A ring $R$ is said to be \textbf{Von Neumann regular} if  
for every non-zero $r\in R$, $r=rxr$ for some $x\in R$. 

\begin{exercise}
\label{xca:VonNeumann_local}
    Prove that a ring $R$ is local if and only if $R$ is a division ring. 
\end{exercise}

\begin{exercise}
\label{xca:nilp_or_unit}
    Let $R$ be a ring such that every element of $R$ is either 
    nilpotent or a unit. Prove that $R$ is local. 
\end{exercise}

\index{Ring!semilocal}
A ring $R$ is said to be \textbf{semilocal} if $R/J(R)$ is left artinian. 

\begin{exercise}
\label{xca:semilocal}
    Prove the following statements:
    \begin{enumerate}
        \item Every local ring is semilocal.
        \item $R$ is semilocal if and only if $R/J(R)$ is semisimple.
        \item If $R$ has finitely many maximal ideals, then $R$ is semilocal. 
        \item If $R_1,\dots,R_k$ are rings, then $\oplus_{i=1}^k R_i$ is semilocal
            if and only if each $R_i$ is semilocal. 
    \end{enumerate}
\end{exercise}

\begin{example}
\label{xca:semilocal_commutative}
    Let $R$ be a ring such that $R/J(R)$ is commutative. Prove
    that $R$ is semilocal if and only if $R$ has finitely many maximal ideals. 
\end{example}

\topic{*When is group algebra local?}

\begin{proposition}
    \label{pro:augmentation}
    Let $R$ be a commutative ring with one. 
    Let $f\colon G\to H$ be a group homomorphism with kernel $K$. Then
    \[
    \varphi\colon R[G]\to R[H],
    \quad 
    \sum\lambda_ig_i\mapsto \sum\lambda_if(g),
    \]
    is a ring homomorphism with kernel the ideal 
    of $R[G]$ generated by $\{k-1:k\in K\}$. 
\end{proposition}

\begin{proof}
    A direct calculation shows that the map $\varphi$ is a well-defined ring homomorphism. Let 
    $S=\{k-1:k\in K\}$. Then $(S)\subseteq \ker\varphi$. 
    
    Let us show that 
    $\ker\varphi\subseteq (S)$. Let $\alpha=\sum r_ig_i\in\ker\varphi$. Then 
    \[
    \varphi(\alpha)=\sum r_if(g_i)=0.
    \]
    Let 
    $\{Kg_{i_1},\dots,Kg_{i_k}\}$ be the subset of pairwise distinct cosets 
    of $Kg_1,\dots,Kg_n$. Write  
    \[
    \alpha=\sum\sum s_{ij}k_{ij}g_{i_j}
    \]
    for some $s_{ij}\in R$ and $k_{ij}\in K$. Then 
    \begin{equation}
    \label{eq:0=s_ij}
    0=\varphi(\alpha)=\sum\sum s_{ij}\varphi(k_{ij}g_{i_j})
    =\sum\sum s_{ij}f(g_{i_j}),
    \end{equation}
    as $K=\ker f$. Note that 
    \[
    f(g_{i_j})=f(g_{i_k})\implies 
    g_{i_j}g_{i_k}^{-1}\in K\implies 
    g_{i_j}K=g_{i_k}K.
    \]
    Thus $f(g_{i_j})\ne f(g_{i_k})$ for $j\ne k$. Since $R[H]$ is a free $R$-module 
    with basis $\{h:h\in H\}$, Equality
    \eqref{eq:0=s_ij}
    implies that $\sum_i s_{ij}=0$ for all $j$. Thus
    \[
    \alpha=\sum\sum s_{ij}k_{ij}g_{i_j}=\sum\sum s_{ij}(k_{ij}-1)g_{i_j}\in (S).\qedhere
    \]
\end{proof}

\begin{corollary}
\label{cor:R[G/N]}
    Let $R$ be a commutative ring with one. If 
    $G$ is a group and $N$ is a normal subgroup of $G$, then
    \[
    R[G/N]\simeq R[G]/I,
    \]
    where $I$ is the ideal of $R[G]$ generated by $\{n-1:n\in N\}$. 
\end{corollary}

\begin{proof}
    Apply the previous proposition to the canonical map $\pi\colon G\to G/N$ to get
    a ring homomorphism $\varphi\colon R[G]\to R[G/N]$. The kernel of $\varphi$ is the ideal $I$ 
    generated by the set $\{g-1:g\in\ker\pi=N\}$. Since 
    $\pi$ is surjective, $\varphi$ is surjective. Then 
    the claim follows from the first isomorphism theorem. 
\end{proof}

Let $K$ be a field and $G$ be a group. We write $A(K[G])$ to denote
the ideal of $K[G]$ generated by the set $\{g-1:g\in G\}$. This ideal is known as the
\textbf{augmentation ideal} of $K[G]$. 

\begin{corollary}
\label{cor:local_K[N]_and_K[G/N]}
    Let $K$ be a field. 
    Let $G$ be a group and $N$ be a central subgroup of $G$. If $K[N]$ and $K[G/N]$ are local, 
    then $K[G]$ is local. 
\end{corollary}

\begin{proof}
    By Corollary \ref{cor:R[G/N]}, $K[G/N]\simeq K[G]/I$, where 
    $I$ is the ideal of $K[G]$ generated by $\{n-1:n\in N\}$. Since $N\subseteq Z(G)$, 
    $I$ is central in $K[G]$. Note that 
    \[
    I=A(K[N])K[G].
    \]
    Let $\alpha\in A(K[G])$. 
    Since $K[G/N]$ is local, $A(K[G/N])$ is nil by Theorem \ref{thm:local}. Since $K[G]/I\simeq K[G/N]$, this implies that 
    there exists $m$ such that $\alpha^m\in I$. Since $K[N]$ is local,  
    $A(K[N])$ is nil by Theorem \ref{thm:local}. Moreover, $K[N]$ is central in $K[G]$, because $N\subseteq Z(G)$. This implies that $I=A(K[N])K[G]$ is also nil. In particular, 
    $\alpha$ is nil. Hence 
    $K[G]$ is nil and therefore $K[G]$ is local by Theorem \ref{thm:local}. 
\end{proof}


\begin{exercise}
\label{xca:augmentation}
    Let $R$ be a commutative ring with one and $G$ be a group. 
    Prove that 
    the map $R[G]\to R$, $\sum_{g\in G}r_gg\mapsto\sum_{g\in G}r_g$, is a surjective
    ring homomorphism with kernel $A(R[G])$.  
\end{exercise}

\begin{lemma}
    Let $K$ be a field and $G$ be a finite group. 
    The following statements are equivalent: 
    \begin{enumerate}
        \item $K[G]$ is local. 
        \item $A(K[G])\subseteq J(K[G])$. 
        \item $A(K[G])$ is nil.
        \item $A(K[G])=J(K[G])$. 
    \end{enumerate}
\end{lemma}

\begin{proof}
    Let us prove that $1)\implies 2)$. Since $K[G]$ is local, 
    $R\setminus\mathcal{U}(K[G])=J(K[G])$ by Theorem \ref{thm:local}. Since $K[G]\setminus\mathcal{U}(K[G])$ contains 
    every proper ideal of $K[G]$, $A(K[G])\subseteq J(K[G])$. 

    We now prove that $2)\implies 3)$. Since $G$ is finite, $K[G]$ is artinian. By Lemma \ref{lem:J(R)_nil}, 
    $J(K[G])$ is nil. Hence $A(K[G])$ is nil. 

    We now prove that $3)\implies 4)$. Since $J(K[G])$ contains every nil ideal (see Proposition~\ref{pro:nilJ}), 
    $A(K[G])\subseteq J(K[G])$. On the other hand, $K[G]/A(K[G])\simeq K$. Since $K$ is a field, the correspondence theorem
    implies that $A(K[G])=J(K[G])$. 

    Finally, we prove that $4)\implies 1)$. Since $A(K[G])=J(K[G])$, Exercise \ref{xca:augmentation} implies that 
    $K[G]/J(K[G])\simeq K$. Since $K$ is a field, 
    it is, in particular, a division ring. Thus $K[G]$ is local by Theorem \ref{thm:local}.     
\end{proof}


\begin{exercise}
\label{xca:C_p:local}
    Let $p$ be a prime number, $K$ be a field of characteristic $p$ and 
    $G$ be a cyclic group of order $p$. Prove that $K[G]$ is local. 
\end{exercise}


\begin{exercise}
\label{xca:K[G]_domain_easy}
    Let $K$ be a field and $G$ be a finite group. Then 
    $K[G]$ is a domain if and only if $|G|=1$. 
\end{exercise}


\begin{theorem}
    Let $K$ be a field and $G$ be a non-trivial finite group. 
    Then $K[G]$ is local if and only if $K$ is of characteristic $p>0$ and $G$ is a $p$-group. 
\end{theorem}

\begin{proof}
    Let us first prove $\implies$. Assume first that $K$ is a field of characteristic zero. 
    By Maschke's theorem, $J(K[G])=\{0\}$. By Theorem \ref{thm:local}, 
    $K[G]$ is a division ring. In particular, $K[G]$ is a domain, 
    a contradiction (see Exercise \ref{xca:K[G]_domain_easy}).

    Assume now that $K$ is of characteristic $p>0$. 
    Let $q$ be a prime divisor of $|G|$ and $g\in G$ an element of order $q$. 
    Since 
    \[
    (1-g)(1+\cdots+g^{q-1})=1-g^q=0,
    \]
    $1-g\not\in\mathcal{U}(K[G])$ and $1+\cdots+g^{q-1}\not\in\mathcal{U}(K[G])$. It follows
    that $1-g^m\not\in\mathcal{U}(K[G])$ for all $m\geq0$. By Theorem \ref{thm:local}, 
    $K[G]\setminus J(K[G])$ is an ideal. Thus  
    \[
    q1_G=1+\cdots+g^{q-1}+\sum_{m=1}^{q-1}(1-g^m)\not\in\mathcal{U}(K[G])
    \]
    If $q\ne 0$ in $K$, then $q1_G\in\mathcal{U}(K[G])$. Hence $q=0$ in $K$ and
    therefore $p$ divides $q$. We conclude that $G$ is a $p$-group. 

    We now prove $\impliedby$. Let $G$ be a $p$-group and $K$ be a field of characteristic $p>0$. We proceed
    by induction on $|G|$. 
    If $|G|=p$, $K[G]$ is a local ring (see Exercise \ref{xca:C_p:local}).
    % \[
    % K[G]\simeq K[X]/(X^p-1)\simeq K[X]/((X-1)^p), 
    % \]
    % as $X^p-1=(X-1)^p$. But $K[X]/((X-1)^p)$ is a commutative local ring  
    If $|G|>p$, let $Z=Z(G)$. Since $G$ is a $p$-group, $|Z|\geq p$. Let $N$ be a subgroup of $Z$ of order $p$. 
    Then $|N|<|G|$ and $|G/N|<|G|$. By the inductive hypothesis, both 
    $K[N]$ and $K[G/N]$ are local. By Corollary \ref{cor:local_K[N]_and_K[G/N]}, $K[G]$ is local too. 
\end{proof}

\topic{*Hurewitz' theorem}

\begin{theorem}[Hurewicz]
    \label{thm:Hurewicz}
    \index{Hurewicz' theorem}
    Let $G$ be a group and $I$ be the augmentation ideal of $\Z[G]$. 
    Then $G/[G,G]\simeq I/I^2$ as (abelian) groups. 
\end{theorem}

\begin{proof}
    Let $\varphi\colon G\to I/I^2$, $g\mapsto g-1_G+I^2$. Since $g-1_G\in I$ for all $g\in G$, $\varphi$ is well-defined. The map $\varphi$ is a group homomorphism. Since 
    $(g-1_G)(h-1_G)\in I^2$, 
    \begin{align*}
    \varphi(gh) &= gh-1_G+I^2\\
    &=gh-1_G-(g-1_G)(h-1_G)+I^2+I^2\\
    &=g-1_G+h-1_G+I^2\\
    &=\varphi(g)+\varphi(h)
    \end{align*}
    holds for all $g,h\in G$. 

    Since $[G,G]\subseteq\ker\varphi$, there exists a group homomorphism
    \[
    \overline{\varphi}\colon G/[G,G]\to I/I^2,\quad 
    g[G,G]\mapsto g-1_G+I^2.
    \]
    We claim that $\overline{\varphi}$ is an isomorphism. 
    Let us construct the inverse of $\overline{\varphi}$. Let 
    \[
    \psi\colon I\to G/[G,G],\quad 
    \sum_{g\in G}m_g(g-1_G)\mapsto \left(\prod_{g\in G}g^{m_g}\right)[G,G].
    \]
    Since $G/[G,G]$ is abelian, the map $\psi$ is well-defined, that is
    the order of the factors in $\prod_{g\in G}g^{m_g}$ does not matter. Note that 
    $I^2\subseteq\ker\psi$, as 
    $\{(g-1_G)(h-1_G):g,h\in G\}$ generates the additive group $I^2$ 
    and 
    \begin{align*}
        \psi((g-1_G)(h-1_G))&=\psi( (gh-1_G)-(g-1_G)-(h-1_G))\\
        &=(ghg^{-1}h^{-1})[G,G]\\
        &=[G,G].
    \end{align*}
    Therefore there exists a group homomorphism
    \[
    \overline{\psi}\colon I/I^2\to G/[G,G],\quad 
    \sum_{g\in G}m_g(g-1_G)+I^2\mapsto \left(\prod_{g\in G}g^{m_g}\right)[G,G].
    \]
    A direct calculation shows that $\overline{\psi}$ is the inverse 
    of $\overline{\varphi}$. 
\end{proof}

\topic{Andrunakevic--Rjabuhin's theorem}
%\begin{exercise}
%\label{xca:reduced}
%     Let $R$ be a ring and $I$ be an ideal of $R$.
%     Prove that $I$ is prime if and only if $xRy\subseteq I$ implies
%     either $x\in I$ or $y\in I$. 
% \end{exercise}

% \begin{sol}{xca:reduced}
%     Let $A$ and $B$ be ideals such that $AB\subseteq I$. If 
%     $A\not\subseteq I$ and $B\not\subseteq J$, let 
%     $x\in A\setminus P$ and $y\in B\setminus P$. Then
%     $xRy\subseteq AB\subseteq I$, a contradiction. 
%     Conversely, if $xRy\subseteq I$ and $x\not I$ and $y\not\in I$, 
%     then $A=(x)\not\subseteq I$ and $B=(y)\not\subseteq P$.
% \end{sol}

\begin{definition}
\index{Ring!reduced}
     A ring $R$ is \textbf{reduced} if 
     has no non-zero nilpotent elements. 
\end{definition}

Every commutative domain is reduced. 

\begin{example}
    The ring $\Z\times\Z$ with the usual operations 
    is reduced but not a domain. 
\end{example}

\begin{example}
    The ring $\Z/6$ is reduced. However, $\Z/4$ is not reduced. 
\end{example}

\begin{exercise}
\label{xca:reduced}
    Prove that a ring $R$ is \textbf{reduced} if and only 
    if for all $r\in R$ such that $r^2=0$ one has $r=0$.
\end{exercise}

%\begin{exercise}
%\label{xca:reduced_Zn}
%    Let $n\geq2$. Then $\Z/n$ is reduced 
%    but not a domain if and only if $n$ is square-free 
%    but not prime.
%\end{exercise}

\begin{exercise}
    \label{xca:reduced_RX}
    Let $R$ be a commutative ring that is reduced but not a domain.
    Prove that $R[X]$ is reduced but not a domain. 
\end{exercise}

The previous exercise and induction 
shows that if $R$ is reduced but not a domain, 
then so is $R[X_1,\dots,X_n]$. 

\begin{example}
    Let $R=\Z/3\times\Z/3$ with
    operations $(a,b)+(c,d)=(a+c,b+d)$ and 
    $(a,b)(c,d)=(ac,ad+bc)$. Then $R$ is a commutative
    ring with identity $(1,0)$. Since 
    $(0,1)$ is a non-zero nilpotent element, $R$ is not reduced. 
\end{example}

\begin{definition}
\index{Ideal!reduced}
    Let $R$ be a ring and $I$ be an ideal of $R$. 
    Then $I$ is \textbf{reduced} if $R/I$ is a reduced ring. 
\end{definition}

Let $R$ be a ring and 
$I$ be a reduced ideal of $R$. If $ab\in I$, then 
$ba\in I$. In fact, since $ab\in I$, 
$(ba)^2=b(ab)a\in I$.
Since $R/I$ is reduced, $ba\in I$. 

 \begin{theorem}[Andrunakevic--Rjabuhin]
 \label{thm:AndrunakevicRjabuhin}
 \index{Andrunakevic--Rjabuhin's theorem}
 	Let $R$ be a non-zero ring. If $R$ is reduced, there exists
 	an ideal $I$ of $R$ such that 
 	then $R/I$ has no non-zero zero-divisors. 
 \end{theorem}

 Let $R$ be a ring and $I$ be an ideal of $R$. If $S$ 
 is a subset of $R$, the \emph{left annihilator} of $S$
 modulo $I$ is the set $\{r\in R:rS\subseteq I\}$.  

 \begin{lemma}
    Let $R$ be a ring and $I$ be a reduced ideal. 
    If $S\subseteq R$ is a subset, then 
    the left annihilator of $S$ modulo $I$ 
    is a reduced ideal. 
 \end{lemma}

\begin{proof}
    We need to show that $A=\{r\in R:rS\subseteq I\}$ 
    is a reduced ideal. 
    A straightforward calculation shows that $A$ is 
    a left ideal. We claim that $A$ is a right ideal. Let $r\in R$
    and $a\in A$. Then 
    $as\in I$ for all $s\in S$. Since $I$ is reduced, $sa\in I$ for all $s\in S$. Since 
    $I$ is an ideal of $R$, $sar\in I$
    for all $s\in S$. Using again 
    that $I$ is reduced, 
    $ars\in I$ for all $s\in S$. Thus 
    $ar\in A$. 
    
    We now claim that $A$ is reduced. If $a^2\in A$, then 
    $aas=a^2s\in I$ for all $s\in S$. 
    Since $I$ is reduced, $asa\in I$ for
    all $s\in S$. Thus $(as)^2=(asa)s\in I$ for all $s\in S$. 
    Since $I$ is reduced, $as\in I$ for all $s\in S$. Hence $a\in A$. 
\end{proof}

Similarly, if $S$ is a subset of a ring $R$, then 
the \emph{right annihilator} 
$\{r\in R:Sr\subseteq I\}$ 
of $S$ modulo $I$ 
is a reduced ideal. 

\begin{proof}[Proof of Theorem \ref{thm:AndrunakevicRjabuhin}]
    Let $x\in R\setminus\{0\}$. Let $X$ 
    be the set of reduced ideals $I$ such that 
    $x\not\in I$. Since $R$ is reduced, $\{0\}$ 
    is a reduced ideal and hence $X\ne\emptyset$. 
    A standard application of Zorn's lemma shows that
    there exists a maximal element $M\in X$. 
    
    We claim that $R/M$ has no non-zero divisors. If not, 
    there exist $a,b\in R$ such that $ab\in M$, $a\not\in M$ 
    and $b\not\in M$. Let $A$ be the left annihilator of $\{b\}$ 
    modulo $M$ and $B$ be the right annihilator of $\{a\}$ 
    modulo $M$. By the previous lemma, $A$ and $B$ 
    are reduced ideals of $R$. Since  
    $a\in A$, $M\subsetneq A$. Similarly, since 
    $b\in B$, $M\subsetneq B$. Moreover, $AB\subseteq M$. 
    Since $x\in A\cap B$, $x^2\in AB\subseteq M$. Since 
    $M$ is reduced, $x\in M$, a contradiction. 
\end{proof}

\begin{exercise}
    Prove that a reduced ring is a subdirect product
    of rings without no non-zero divisors. 
\end{exercise}

\begin{exercise}
    Is the ring $\C[\Z/2]$ reduced? 
\end{exercise}

% Let $G=\langle g:g^2=1\rangle$. If $(a+bg)^2=0$, then
% $(a^2+b^2)+(2ab)g=0$. Thus $a=b=0$. 

\begin{problem}
\label{prob:reduced}
    Let $G$ be a torsion-free group. Is
    $K[G]$ is reduced?
\end{problem}

Problem \ref{prob:reduced} is related to other important
open problems about group algebras 
(e.g. zero-divisors, units, 
indempotents and semisimplicity of group
rings).

\begin{exercise}
\label{xca:reduced_central}
    Prove that idempotents of reduced rings are central. 
\end{exercise}

The previous exercise is used to solve the following problem.

\begin{exercise}
\label{xca:x^3=x}
    Let $R$ be a ring such that $x^3=x$ for all $x\in R$. Prove that
    $R$ is commutative. 
\end{exercise}

Exercise \ref{xca:x^3=x} is hard. 
Even harder is the following exercise:

\begin{exercise}
\label{xca:x^4=x}
    Let $R$ be a ring such that $x^4=x$ for all $x\in R$. Prove
    that $R$ is commutative. 
\end{exercise}

%Other exercises about reduced rings. 

%\begin{exercise}
%\label{xca:reduced}
%    Prove that a ring is reduced if 
%    and only it has no non-zero nilpotent elements. 
%\end{exercise}

\begin{exercise}
\label{xca:reduced=>semiprime}
    Reduced rings are semiprime.
\end{exercise}
 
\begin{theorem}
\label{thm:reduced}
    Let $K$ be a field and $G$ be a group. If $K[G]$
    is reduced, then every finite subgroup of $G$ is normal. 
\end{theorem}

\begin{proof}
    Let $H=\{h_1,\dots,h_n\}$ be a finite normal subgroup of $G$. 
    We claim that $n=|H|$ is invertible in $K$. If $\ch K=0$, this 
    is clear. If $\ch K=p>0$ and $n$ is not invertible in $K$, 
    then $p$ divides $n=|H|$. By Cauchy's theorem, 
    there exists an element $h\in H$ of order $n$, that is 
    $|h|=n$. Since $(1-h)^p=1-h^p=0$ and $K[G]$ is reduced,
    $h=1$, a contradiction. 
    
    Let $\alpha=\frac{1}{n}\sum_{i=1}^nh_i\in K[G]$. Then
    \[
    \alpha^2=\frac{1}{n^2}\sum_{i=1}^n\sum_{j=1}^nh_ih_j
    =\frac{1}{n^2}\sum_{i=1}^nn\alpha=\alpha.
    \]
    Thus $\alpha$ is idempotent. As idempotent 
    element of reduced rings are central (Exercise \ref{xca:reduced_central}), 
    $g\alpha g^{-1}=\alpha$ for all $g\in G$. If $g\in G$, 
    then 
    \[
    \sum_{i=1}^n gh_ig^{-1}=\sum_{i=1}^n h_i.
    \]
    It follows that $H$ is normal in $G$, 
    as for each $i\in\{1,\dots,n\}$ 
    there exists $j\in\{1,\dots,n\}$ such that 
    $gh_ig^{-1}=h_j\in H$. 
\end{proof}

\begin{example}
    If $K$ is a field, then $K[\Sym_3]$ is not reduced. 
    In fact, 
    if 
    \[
    \alpha=(12)+(123)-(132)-(13),
    \]
    then 
    $\alpha^2=0$. 
\end{example}

\begin{exercise}
    Prove that the converse of Theorem \ref{thm:reduced} 
    does not hold. 
\end{exercise}

\topic{Rickart's theorem}

We now consider Jacobson's semisimplicity problem. 

\begin{openproblem}
\label{Jacobson's semisimplicity problem}
Let $G$ be a group and $K$ be a field. When $J(K[G])=\{0\}$?
\end{openproblem}

As an application of Amitsur's theorem \ref{thm:Amitsur}, 
we prove that 
complex group algebras have null Jacobson radical.
This is known as 
Rickart's theorem. The original proof found by Rickart 
uses complex analysis. Here, however, 
we present an algebraic proof. 

\begin{theorem}[Rickart]
\index{Rickart's theorem}
\label{thm:Rickart}
    Let $G$ be a group. Then $J(\C[G])=\{0\}$.
\end{theorem}

To prove the theorem, we need a lemma.

\begin{lemma}
Let $G$ be a group. Then $J(\C[G])$ is nil.        
\end{lemma}

\begin{proof}
    We need to show that every element of $J(\C[G])$ is nilpotent. 
    If $G$ is countable, then the result follows from Amitsur's theorem \ref{thm:Amitsur}. So assume that 
    $G$ is not countable. Let $\alpha\in J(\C[G])$, say
    \[
    \alpha=\sum_{i=1}^n\lambda_ig_i,
    \]
    where $\lambda_1,\dots,\lambda_n\in\C$ and $g_1,\dots,g_n\in G$. Let $H=\langle g_1,\dots,g_n\rangle$.
    Then $\alpha\in \C[H]$ and $H$ is countable. We claim that $\alpha\in J(\C[H])$. Decompose
    $G$ as a disjoint union 
    \[
    G=\bigcup_\lambda x_\lambda H
    \]
    of cosets of $H$ in $G$. Then $\C[G]=\bigoplus_\lambda x_\lambda\C[H]$ and
    hence $\C[G]=\C[H]\oplus K$ for some right module $K$ over $\C[H]$ (this follows
    from the fact that one of the cosets is that of $H$). Since $\alpha\in J(\C[G])$, for each 
    $\beta\in\C[H]$ there exists $\gamma\in\C[G]$ such that 
    $\gamma(1-\beta\alpha)=1$. Write $\gamma=\gamma_1+\kappa$ for $\gamma_1\in\C[H]$ and $\kappa\in K$. Then
    \[
    1=\gamma(1-\beta\alpha)=\gamma_1(1-\beta\alpha)+\kappa(1-\beta\alpha)
    \]
    and hence $\kappa(1-\beta\alpha)\in K\cap \C[H]=\{0\}$, as $\beta\in\mathbb{C}[H]$. 
    Since $1=\gamma_1(1-\beta\alpha)$, it follows that
    $\alpha\in J(\C[H])$ and the lemma follows from Amitsur's theorem \ref{thm:Amitsur}.  
\end{proof}

We now prove the theorem. 

\begin{proof}[Proof of Theorem \ref{thm:Rickart}]
    For $\alpha=\sum_{i=1}^n\lambda_ig_i\in\C[G]$ let 
    \[
    \alpha^*=\sum_{i=1}^n\overline{\lambda_i}g_i^{-1}.
    \]
    Then $\alpha\alpha^*=0$ if and only if $\alpha=0$ and, moreover, 
    $(\alpha\beta)^*=\beta^*\alpha^*$ for all $\beta\in\C[G]$. 
    Assume that $J(\C[G])\ne\{0\}$ and let $\alpha\in J(\C[G])\setminus\{0\}$. Then
    $\beta=\alpha\alpha^*\in J(\C[G])$, as $J(\C[G])$ is an ideal of $\C[G]$. Moreover, the previous 
    lemma implies that $\beta$ is nilpotent. Note that $\beta\ne 0$, as $\alpha\ne0$. Now  
    \[
    (\beta^m)^*=(\beta^*)^m=\beta^m
    \]
    for all $m\geq1$. If there exists $k\geq2$ such that $\beta^k=0$ and $\beta^{k-1}\ne 0$, then
    \[
    \beta^{k-1}\left(\beta^{k-1}\right)^*=\beta^{2k-2}=0
    \]
    and hence $\beta^{k-1}=0$, a contradiction. Thus $\beta=0$ and therefore $\alpha=0$. 
\end{proof}

\begin{exercise}
	If $G$ is a group, then $J(\R[G])=0$. 
\end{exercise}

% To obtain a consequence of Rickart's theorem we need two lemmas. 

% \begin{lemma}[Nakayama]
% 	\label{lem:Nakayama}
% 	\index{Nakayama's lemma}
% 	Let $R$ be a unitary ring and $M$ be a finitely generated module. If 
% 	$J(R)\cdot M=M$, then $M=\{0\}$.
% \end{lemma}

% \begin{proof}
%     Since $M$ is finitely generated, we may assume that 
% 	$M=(x_1,\dots,x_n)$. Since $x_n\in M=J(R)\cdot M$, 
% 	there exist $r_1,\dots,r_n\in J(R)$ such that $x_n=r_1\cdot x_1+\cdots+r_n\cdot x_n$, that is 
% 	$(1-r_n)\cdot x_n=\sum_{j=1}^{n-1}r_j\cdot x_j$. 
% 	Since $1-r_n$ is invertible, there exists $s\in R$ such that $s(1-r_n)=1$. Thus 
% 	$x_n=\sum_{j=1}^{n-1}(sr_j)\cdot x_j$ 
% 	and hence $M=(x_1,\dots,x_{n-1})$. Repeating this procedure several times 
% 	one obtains $M=\{0\}$.
% \end{proof}

% \begin{lemma}
% 	\label{lem:Rickart}
% 	Let $\iota\colon R\to S$ be a homomorphism of unitary rings. If	
% 	\[
% 	S=\iota(R)x_1+\cdots+\iota(R)x_n,
% 	\]
% 	where each $x_j$ is such that $x_jy=yx_j$ for all $y\in\iota(R)$, then 
% 	$\iota(J(R))\subseteq J(S)$.
% \end{lemma}

% \begin{proof}
% 	We claim that $J=\iota(J(R))$ acts trivially on each simple $S$-module $M$.
% 	If is $M$ is a simple module over $S$, then, in particular, $M=S\cdot m$ for some $m\ne0$. 
% 	Now $M$ is a module over $R$ with $r\cdot m=\iota(r)\cdot m$. Since 
% 	\[
% 		M=S\cdot m=(\iota(R)x_1+\cdots+\iota(R)x_n)\cdot m=\iota(R)\cdot (x_1\cdot m)+\cdots+\iota(R)\cdot (x_n\cdot m),
% 	\]
% 	it follows that 
% 	$M$ is finitely generated as a module over $\iota(R)$. Moreover, 
% 	\[
% 	J(R)\cdot
% 	M=J\cdot M=\iota(J)\cdot M
% 	\]
% 	is an $S$-submodule of $M$, as 
% 	\[
% 		x_j\cdot (J\cdot M)=(x_j J)\cdot M=(J x_j)\cdot M=J\cdot (x_j\cdot M)\subseteq J\cdot M.
% 	\]
% 	Since $M\ne\{0\}$, Nakayama's lemma implies that $J(R)\cdot M\subsetneq M$. The simplicity of 
% 	the $S$-module $M$ implies that $J(R)\cdot M=\{0\}$.
% \end{proof}

% We now obtain the following consequence of Rickart's theorem. 

% \begin{theorem}
% 	If $G$ is a group, then $J(\R[G])=0$. 
% \end{theorem}

% \begin{proof}
% 	Let $\iota\colon \R[G]\to\C[G]$ be the canonical inclusion. Since 
% 	\[
% 	\C[G]=\R[G]+i\R[G],
% 	\]
% 	Lemma~\ref{lem:Rickart} and Rickart's theorem imply that 
% 	$\iota(J(\R[G]))\subseteq J(\C[G])=0$. Thus $J(\R[G])=0$, as $\iota$ is injective. 
% \end{proof}


\begin{definition}
	\index{Ring!semiprime}
	A ring $R$ \textbf{semiprime} if 
	$aRa=\{0\}$ implies $a=0$.
\end{definition}

\begin{proposition}
	Let $R$ be a ring. The following statements are equivalent: 
	\begin{enumerate}
		\item $R$ is semiprime.
		\item If $I$ is a left ideal such that $I^2=\{0\}$, then $I=\{0\}$.
		\item If $I$ is an ideal such that $I^2=\{0\}$, then $I=\{0\}$.
		\item $R$ does not contain non-zero nilpotent ideals.
	\end{enumerate}
\end{proposition}

\begin{proof}
	We first prove that $1)\implies2)$. If $I^2=\{0\}$ y $x\in I$, then
	$xRx\subseteq I^2=\{0\}$ and thus $x=0$. The implications $2)\implies3)$
	and $4)\implies3)$ are both trivial. Let us prove that $3)\implies4)$.  If
	$I$ is a non-zero nilpotent ideal, let $n\in\Z_{>0}$ be minimal such that
	$I^n=\{0\}$.  Since $(I^{n-1})^2=\{0\}$, it follows that $I^{n-1}=\{0\}$, a
	contradiction.  Finally, we prove that $3)\implies1)$. Let $a\in R$ be such
	that $aRa=\{0\}$. Then $I=RaR$ is an ideal of $R$ such that $I^2=\{0\}$. Thus 
	$RaR=\{0\}$. This means that $Ra$ and $aR$ are ideals such that
	$(Ra)R=R(aR)=\{0\}$ (for example, $R(aR)\subseteq RaR=\{0\}\subseteq aR$). 
	Moreover, since $(Ra)(Ra)=\{0\}$ and $(aR)(aR)=\{0\}$, it follows that
	$aR=Ra=\{0\}$. 
	This implies that $\Z a$ is an ideal of $R$, as $R(\Z a)\subseteq \Z(Ra)=\{0\}$ and 
	$(\Z a)R\subseteq aR=\{0\}$. Now $(\Z a)(\Z a)\subseteq (\Z a)R=\{0\}$ and hence
	$a=0$, as $\Z a=\{0\}$. 
\end{proof}

Two consequences:

\begin{exercise}
	A commutative ring is semiprime if and only if it does not contain non-zero
	nilpotent elements. 
\end{exercise}

% tomar $\alpha$ tal que $\alpha^2=0$ y sea $A=K[G]\alpha$. Como $A^2=0$, $A=0$ y entonces $\alpha=0$.

\begin{exercise}
\label{xca:D_semiprime_semiprimitive}
	Let $D$ be a division ring. 
	\begin{enumerate}
		\item $D[X]$ is semiprime and semiprimitive.
		\item $D[\![X]\!]$ is semiprime and it is not semiprimitive.
	\end{enumerate}
\end{exercise}

% We will prove 
% in Lecture \ref{09} (Corollary \ref{cor:C[G]_semiprime}) 
% that the if $G$ is a group, 
% then the ring $\C[G]$ is semiprime. 

\begin{corollary}
\label{cor:C[G]_semiprime}
	The ring $\C[G]$ is semiprime.
\end{corollary}

\begin{proof}
	Since $J(\C[G])=\{0\}$ by Rickart's theorem and the Jacobson radical
	contains every nil ideal by Proposition~\ref{pro:nilJ}, it follows that
	$\C[G]$ does not contain non-trivial nil ideals. Thus $\C[G]$ does not
	contain non-trivial nilpotent ideals and hence $\C[G]$ is semiprime.
\end{proof}

\begin{exercise}
	Prove that $Z(\C[G])$ is semiprime.
\end{exercise}

We now characterize when complex group algebras 
are left artinian. For that purpose,
we need a lemma. This is similar to one of the implications proved in Proposition \ref{pro:semisimple}. However,
in the arbitrary setting we are considering, we need to use Zorn's lemma. 

\begin{lemma}
    Let $M$ be a semisimple module and $N$ be a submodule. 
    Then $N$ is a direct summand.
\end{lemma}

\begin{proof}[Sketch of the proof]
    Let $M=\oplus_{i\in I}M_i$ be a direct sum of simple modules  
    and let $i\in I$. 
    Since $N\cap M_i$ is a submodule of $M_i$ and $M_i$ is simple, it follows
    that $N\cap M_i=\{0\}$ or $N\cap M_i=M_i$. If
    $N\cap M_i=M_i$ for all $i\in I$, then $N=M$ and the lemma is proved. So we may assume
    that there exists $i\in I$ such that $N\cap M_i=\{0\}$. Let $X$ be the set
    of subsets $J$ of $I$ such that $N\cap (\oplus_{j\in J}M_j)=\{0\}$. Our assumptions
    imply that $X$ is non-empty. Zorn's lemma implies the existence of 
    a maximal element $K$. Let $N_1=\oplus_{k\in K}M_k$. We claim that
    $N\oplus N_1=M$. If not, there exists $i\in I$ such that
    $M_i\not\subseteq N\oplus N_1$. The simplicity of $M_i$ implies that
    $M_i\cap (N\oplus N_1)=\{0\}$, which contradicts the maximality of $K$. 
\end{proof}

A direct application of the lemma proves that
complex group algebras of infinite groups are never semisimple. 

\begin{proposition}
    \label{pro:KGsemisimple}
    If $G$ is an infinite group, then $\C[G]$ is not semisimple. 
\end{proposition}

\begin{proof}
	Assume that $R=\C[G]$ is semisimple.  Let $I$ 
	be the augmentation ideal of $R$, that is
	\[
	I=\left\{\alpha=\sum_{g\in G}\lambda_gg\in R:\sum_{g\in G}\lambda_g=0\right\}.
	\]
	By the previous lemma, 
	there exists a non-zero ideal $J$ such that 
	$R=I\oplus J$. Since $R$ is unitary, there exist $e\in I$ and $f\in J$ such that
	$1=e+f$. If
	$x\in I$, then $x=xe+xf$ and hence $xf=x-xe\in I\cap J=\{0\}$. Since 
	$x=xe$ for all $x\in I$, it follows that $e=e^2$. Similarly, one proves
	that $f^2=f$. Moreover, $ef=0$, as $ef\in I\cap J=\{0\}$.  Since $I$ 
	is the augmentation ideal of $R$ and $If=(Re)f=R(ef)=\{0\}$ (note that $I=Re$ because $x=xe$ for all $x\in I$), we conclude that
	$(g-1)f=0$
	for all $g\in G$, as $g-1\in I$. If $f=\sum_{h\in
	G}\lambda_hh$ (finite sum), then  
	\[
	f=gf=\sum_{h\in G}\lambda_h(gh)=\sum_{h\in
	G}\lambda_{g^{-1}h}h.
	\]
	Thus $\lambda_h=\lambda_{g^{-1}h}$ for all $g,h\in G$. Since $G$ 
	is infinite, some $\lambda_g=0$ and hence $f=0$. Thus $e=1$ and $I=\C[G]$, a contradiction. 
% 	If $f=0$, then $e=1$ and $I=\C[G]$, a contradiction.  
% 	, a contradiction because 
% 	$f\ne 0$ implies that the sum that defines $f$ should be an infinite sum.
\end{proof}

% The ideal $I(G)$ used in the proof of the previous proposition 
% is known as the \textbf{augmentation ideal} 
% of $\C[G]$.

\begin{theorem}
	Let $G$ be a group. Then $\C[G]$ 
	is left artinian if and only if 
	$G$ is finite. 
\end{theorem}

\begin{proof}
    If $G$ is finite, then $\C[G]$ is left artinian because $\dim\C[G]=|G|<\infty$. So assume that 
    $G$ is infinite. By Rickart's theorem,   
	$J(\C[G])=0$. Moreover, $\C[G]$
	is not semisimple by the previous proposition. Thus
	$\C[G]$ is not left artinian by Theorem~\ref{thm:SSartin=J}.
\end{proof}


