\section{Connel's theorem}

When $K[G]$ is prime? Connel's theorem gives a full answer to this natural
question in the case where $K$ is of characteristic zero. 

%\begin{lemma}
%	\label{lemma:Dfg}
%	Sea $H$ un subgrupo finitamente generado de $\Delta(G)$.
%	\begin{enumerate}
%		\item $(G:C_G(H))$ es finito.
%		\item $(H:Z(H))$ es finito.
%		\item $[H,H]$ es finito.
%		\item Si $H_0$ es el conjunto de elementos de torsión de $H$, $H_0$ es
%			un subgrupo normal finito de $H$ y $H/H_0$ es finitamente generado,
%			abeliano y libre de torsión.
%	\end{enumerate}
%\end{lemma}
%
%\begin{proof}
%	Veamos la primera afirmación: Si $H=\langle
%	h_1,\dots,h_n\rangle\subseteq\Delta(G)$, entonces $(G:C_G(h_i))$ es finito
%	para todo $i\in\{1,\dots,n\}$. Como $C_G(H)=\cap_{i=1}^nC_G(h_i)$, se
%	concluye que $(G:C_G(H))$ es finito.
%
%	Para demostrar la segunda afirmación basta observar que $Z(H)=H\cap C_G(H)$
%	y luego $(H:Z(H))\leq(G:C_G(H)<\infty$. % necesito dos lemas
%
%	La tercera afirmación es consecuencia de la segunda gracias a un teorema de
%	Schur.
%
%	Por último, demostremos la cuarta afirmación.  El grupo $H/[H,H]$ es
%	abeliano y finitamente generado y luego, sus elementos de torsión forman un
%	grupo finito. Como $[H,H]$ es finito, $[H,H]$ es un subgrupo normal de
%	$H_0$. Vamos a demostrar que la torsión de $H/[H,H]$ es igual a
%	$H_0/[H,H]$. La inclusión $\supseteq$ es trivial. Veamos entonces que vale
%	$\subseteq$: so $(x[H,H])^k=1$, entonces $x^k\in[H,H]$. Luego $(x^k)^m=1$ y
%	luego $x\in H_0$. Tenemos entonces que 
%	\[
%		H/[H,H]\simeq\Z^r\times\operatorname{tor}(H/[H,H])\simeq\Z^r\times H_0/[H,H]
%	\]
%	y luego $H/H_0$ es finitamente generado, abeliano y libre de torsión.
%
%\end{proof}
%
%\begin{lemma}
%	\label{lemma:K[abelian]}
%	Si $G$ un grupo abeliano finitamente generado y sin torsión, entonces
%	$K[G]$ es un dominio. 
%\end{lemma}
%
%\begin{proof}
%	Por el teorema
%	de estructura de grupos abelianos finitamente generados podemos escribir
%	$G=\langle x_1\rangle\times\cdots\langle x_n\rangle$, donde
%	$\langle x_j\rangle\simeq\Z$ para todo $j\in\{1,\dots,n\}$. Todo elemento
%	de $G$ se escribe unívocamente como $x_1^{m_1}\cdots x_n^{m_n}$ y
%	luego la función 
%	\[
%		\iota\colon K[X_1,\dots,X_n]\to K[G],\quad
%		X_j\mapsto x_j,
%	\]
%	es un
%	morfismo de anillos inyectivo. Si $\alpha\in K[G]$, entonces existe
%	$m\in\N$ suficientemente grande tal que $\iota((X_1\cdots X_n)^m)\alpha\in
%	\iota(K[X_1,\dots,X_n])\simeq K[X_1,\dots,X_n]$. Luego $K[G]\subseteq
%	K(X_1,\dots,X_n)$ y $K[G]$ es un dominio.
%\end{proof}

%\begin{lemma}
%	Si $G$ es un grupo, entonces
%	$\Delta(G)/\Delta^+(G)$ es abeliano y libre de torsión.
%%	Valen las siguientes afirmaciones:
%%	\begin{enumerate}
%%		%\item $\Delta^+(G)$ está generado por los subgrupos normales finitos de $G$.
%%		\item 
%%		\item Si $\Delta^+(G)=1$, entonces $K[\Delta(G)]$ es un dominio.
%%	\end{enumerate}
%\end{lemma}
%
%\begin{proof}
%%	Demostremos la primera afirmación. 
%	Sean $y_1,\dots,y_n\in\Delta(G)$ y sea $L=\langle y_1,\dots,y_n\rangle$.
%	Como $[L,L]$ es finito por el lema~\ref{lemma:Dfg}, $[L,L]\subseteq\Delta^+(G)$. Luego
%	$\Delta(G)/\Delta^+(G)$ es abeliano y libre de torsión.
%%
%%	Para demostrar la segunda afirmación basta observar que si $\Delta^+(G)=1$
%%	entonces, por el primer ítem, $\Delta(G)$ es abeliano, finitamente generado
%%	y libre de torsión. Luego $K[\Delta(G)]$ es un dominio por el
%%	lema~\ref{lemma:K[abelian]}. 
%\end{proof}

If $S$ is a finite subset of a group $G$, then we define 
$\widehat{S}=\sum_{x\in S}x$. 

\begin{lemma}
	\label{lemma:sumN}
	Let $N$ be a finite normal subgroup of $G$. Then $\widehat{N}=\sum_{x\in N}x$ is central
	in $K[G]$ and $\widehat{N}(\widehat{N}-|N|1)=0$.
\end{lemma}

\begin{proof}
	Assume that $N=\{n_1,\dots,n_k\}$. Let 
	$g\in G$. Since $N\to N$, $n\mapsto gng^{-1}$, is bijective, 
	\[
		g\widehat{N}g^{-1}=g(n_1+\cdots+n_k)g^{-1}=gn_1g^{-1}+\cdots+gn_kg^{-1}=\widehat{N}.
	\]
	Since $nN=N$ if $n\in N$, it follows that $n\widehat{N}=\widehat{N}$. Thus 
	$\widehat{N}\widehat{N}=\sum_{j=1}^k n_j\widehat{N}=|N|\widehat{N}$.
\end{proof}

If $G$ is a group, let 
\begin{align*}
	&\Delta^+(G)=\{x\in \Delta(G):\text{$x$ has finite order}\}.
\end{align*}
An application of Dietzmann's theorem:

\begin{proposition}
	\label{lem:DcharG}
	If $G$ is a group, then $\Delta^+(G)$ is a characteristic subgroup of $G$.
\end{proposition}

\begin{proof}
	Clearly, $1\in\Delta^+(G)$. 
	Let $x,y\in\Delta^+(G)$ and $H$ be the subgroup of $G$ generated by the set 
	$C$ formed by all finite conjugates of $x$ and $y$. If $|x|=n$ and 
	$|y|=m$, then $c^{nm}=1$ for all $c\in C$. 
	Since $C$ is finite and closed under conjugation, Dietzmann's theorem 
	implies that $H$ is finite and hence 
	$H\subseteq\Delta^+(G)$. In particular, $xy^{-1}\in\Delta^+(G)$. It is now clear
	that $\Delta^+(G)$ is a characteristic subgroup, as for 
	every $f\in\Aut(G)$ and $x\in\Delta^+(G)$ it follows that $f(x)\in\Delta^+(G)$. 
\end{proof}

To prove Connel's theorem we need a lemma. 

\begin{lemma}
	\label{lem:Connel}
	Let $G$ be a group and  $x\in\Delta^+(G)$. There exists a finite normal subgroup
	$H$ of $G$ such that $x\in H$.
\end{lemma}

\begin{proof}
	Let $H$ be the subgroup generated by the conjugates of $x$. Since $x$ has finitely many conjugates, 
	$H$ is finitely generated. Moreover, $H$ is normal in $G$ and it is generated by torsion elements. 
	All these generators of $H$ have the same order, say $n$. By Dietzmann's theorem, 
	$H$ is finite. 
\end{proof}

\index{Ring!prime}
Recall that a ring $R$ is said to be \textbf{prime} 
if for $x,y\in R$ such that $xRy=\{0\}$ it follows that $x=0$ or $y=0$. Prime rings
are non-commutative analogs of domains. 

\begin{theorem}[Connell]
	\label{thm:Connel}
	\index{Connel's theorem}
	Let $K$ be a field of characteristic zero. Let 
	$G$ be a group. The following statements are equivalent: 
	\begin{enumerate}
		\item $K[G]$ is prime.
		\item $Z(K[G])$ is prime.
		\item $G$ does not contain non-trivial finite normal subgroups. 
		\item $\Delta^+(G)=\{1\}$.
	\end{enumerate}
\end{theorem}

\begin{proof}
	We first prove that $1)\implies2)$. Since $Z(K[G])$ is commutative, we need to prove that 
	there are no non-trivial zero divisors. Let $\alpha,\beta\in Z(K[G])$ be such that 
	$\alpha\beta=0$. Let $A=\alpha K[G]$ and $B=\beta K[G]$. Since both $\alpha$ and 
	$\beta$ are central, both $A$ and $B$ are ideals of $K[G]$. Since $AB=\{0\}$,
	it follows that either $A=\{0\}$ or $B=\{0\}$, as $K[G]$ is prime by assumption. 
	Thus either $\alpha=0$ or 
	$\beta=0$.

	We now prove that $2)\implies3)$. Let $N$ be a normal finite subgroup of $G$. 
	By Lemma~\ref{lemma:sumN}, $\widehat{N}=\sum_{x\in N}x$ is central in 
	$K[G]$ and $\widehat{N}(\widehat{N}-|N|1)=0$. Since $\widehat{N}\ne 0$ (recall that 
	$K$ has characteristic zero) and $Z(K[G])$ is a domain,
	$\widehat{N}=|N|1$, that is $N=\{1\}$.

	Let us prove that $3)\implies4)$. Let $x\in\Delta^+(G)$. By Lemma~\ref{lem:Connel},
	there exists a finite normal subgroup $H$ of $G$ that contains $x$. By assumption, $H$ is trivial. 
	Hence $x=1$.

	Finally, let us prove that $4)\implies1)$. Let $A$ and $B$ be ideals of 
	$K[G]$ such that $AB=\{0\}$. Assume that $B\ne\{ 0\}$ and let $\beta\in
	B\setminus\{0\}$.  If $\alpha\in A$, then, since 
	\[
	\alpha K[G]\beta\subseteq \alpha B\subseteq AB=\{0\},
	\]
	Passman's lemma ~\ref{lem:Passman}
	implies that $\pi_{\Delta(G)}(\alpha)\pi_{\Delta(G)}(\beta)=\{0\}$.
	By assumption, $\Delta^+(G)$ is trivial. 
	Thus $\Delta(G)$ is torsion-free. 
	As $\Delta(\Delta(G))=\Delta(G)$, it follows from 
	Proposition~\ref{pro:FCabeliano} that  
	$\Delta(G)$ is abelian. Therefore 
	$K[\Delta(G)]$ has no zero divisors and therefore $\alpha=0$. 
\end{proof}

\index{Hopkins--Levitzky's theorem}
We now need to recall Hopkins--Levitzky's theorem. The theorem
states that unitary left artinian rings are left noetherian.

\begin{theorem}[Connel]
\index{Connel's theorem}
	Let $K$ be a field of characteristic zero. If $G$ is a group, then 
	$K[G]$ is left artinian if and only if $G$ is finite.
\end{theorem}

\begin{proof}
	If $G$ is finite, $K[G]$ is left artinian, as it is a finite-dimensional algebra.
	
	Let us assume that $K[G]$ is left artinian. 
	If $K[G]$ is prime, Wedderburn's theorem implies that 
	$K[G]$ is simple and hence 
	$G$ is trivial (otherwise, $K[G]$ is not simple as the augmentation ideal 
	is a non-zero ideal of $K[G]$).

	Since $K[G]$ is left artinian, it is left noetherian by Hopkins--Levitzky's theorem. Thus $K[G]$
	admits a composition series. We proceed by induction on the length of this composition series of 
	$K[G]$. If the length is one, $\{0\}$ is the only ideal of $K[G]$ and hence the result follows as 
	$K[G]$ is prime. If we assume the result holds for length $n$ and $K[G]$ is not prime, then, 
	Connel's theorem implies that $G$ contains a finite non-trivial normal subgroup $H$. The canonical map
	$K[G]\to K[G/H]$ implies that $K[G/H]$ is left artinian and has length 
	$<n$. By using the inductive hypothesis, $G/H$
	is a finite group. Since $H$ is also finite, it follows that $G$ is finite. 
\end{proof}

