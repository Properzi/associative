\section*{Some topics for final projects}

\pagestyle{plain}
\fancyhf{}
\fancyhead[LE,RO]{Associative algebras}
\fancyhead[RE,LO]{Final projects}
\fancyfoot[CE,CO]{\leftmark}
\fancyfoot[LE,RO]{\thepage}

%\addcontentsline{toc}{chapter}{Some topics for final projects}

We collect here some topics for final presentations. Some topics
can also be used as bachelor's or master's theses. 

\subsection*{Rickart's theorem}

In Lecture \ref{09} we presented an algebraic proof of Rickart's theorem. 
The original proof uses analysis; see \cite[(6.4) of Chapter II]{MR1838439}. 

\subsection*{Connel's theorem}

In Lecture \ref{11} we presented the statement of Connel's theorem, which
characterizes prime group rings over fields of characteristic zero 
(see Theorem \ref{thm:Connel}); the proof of this  
result appears for example in \cite[Theorem 2.10 of Chapter 4]{MR798076}. 
As a corollary, one obtains 
that, if $K$ is a field of characteristic zero,
then the group ring $K[G]$ is left artinian if and only if the group
$G$ is finite; see 
\cite[Theorem 1.1 of Chapter 10]{MR798076} for a proof. 

\subsection*{Kolchin's theorem}

Let $U_n(\C)$ be the subgroup of $\GL_n(\C)$ 
of matrices $(u_{ij})$ such that 
\[
u_{ij}=\begin{cases}
1&\text{if $i=j$},\\
0&\text{if $i>j$}.\end{cases}
\]

A matrix $a\in\GL_n(\C)$ is said to be \emph{unipotent} 
if its characteristic polynomial is of the form $(X-1)^n$. 
A subgroup $G$ of $\GL_n(\C)$ is said to be \emph{unipotent} if
each $g\in G$ is unipotent. 

An important theorem of Kolchin states that 
every unipotent subgroup of $\GL_n(\C)$ is conjugate
of some subgroup of $U_n(\C)$. The theorem and its proof 
appear, for example, 
in the 
VUB course \href{https://github.com/vendramin/representation}{Representation theory of algebras}.

\subsection*{Dedekind-finite rings}

The idea is to develop the basic aspects of Dedekind-finite rings.
A standard reference is Lam's book \cite{MR2278849}. 

%\subsection*{Frobenius algebras}

% definition
% some properties. The group algebra of a finite group is Frobenius
% reference?

%\subsection*{Gelfand--Kirillov dimension of algebras}

% definition
% some properties. 
% reference?

%\subsection*{Hopfian algebras}

% reference?

%\subsection*{von Neumann regular rings}

\subsection*{The Skolem--Noether theorem}

Any automorphism of the full $n\times n$ matrix algebra 
is conjugation by some invertible $n\times n$ matrix. This is an elementary 
instance of the celebrated Skolem--Noether theorem. We refer to 
\cite[Chapter 4]{MR3308118} for the theorem 
and its proof (in a more general context).

\subsection*{The double centralizer theorem}

Let $R$ be a ring. 
The centralizer of a subring $S$ of $R$ 
is 
\[
C_R(S)=\{r\in R: rs=sr\text{ for all $s\in S$}\}.
\]
Clearly, $C_R(C_R(S))\supseteq S$, but equality does not always hold. 
The double centralizer theorems give conditions under which one can conclude that the equality occurs; see \cite[Chapter 4]{MR3308118}. 

\subsection*{The Amitsur--Levitzki theorem}

The theorem states that 
if $A$ is a commutative algebra, then 
the matrix algebra 
$M_n(K)$ satisfies the identity 
\[
s_{2n}(a_{1},\dots ,a_{2n})=0,
\]
where 
\[
s_{n}(X_1,\dots,X_n)=\sum_{\sigma\in\Sym_n}\sgn(\sigma)X_{\sigma(1)}\cdots X_{\sigma(n)}.
\]
See \cite[Theorem 6.39]{MR3308118} for the beautiful 
proof found by Rosset. 

\subsection*{Non-commutative Hilbert's basis theorem}

There exists a non-commutative version of the celebrated
Hilbert's basis theorem. It is based on the theory of Ore's extensions (also known as \emph{skew polynomial rings}). The theorem
appears in \cite[I.8.3]{MR1321145}; see \cite[I.7]{MR1321145} 
for the basic theory of Ore's extensions. 

\subsection*{Bi-ordered or left-ordered groups}

Basic notions about ordered groups appear in the book
of Passman \cite{MR798076}, where the motivation is based on 
algebraic properties of group algebras. 

\subsection*{The Golod--Shafarevich theorem}

This is an important theorem of non-commutative algebra
with several interesting applications, for example, in group theory. 
A quick proof (and some applications) can be found in the book \cite{MR1449137} of Herstein. 

\subsection*{The Brauer group}

The Brauer group is a helpful tool to classify division algebras over fields. It can also be defined in terms of Galois cohomology. 
See \cite{MR1233388} for the definition and some properties. 

\subsection*{The Weyl algebra}

The Weyl algebra is the quotient of the free algebra on two generators
$X$ and $Y$ by the ideal generated by the element
$YX-XY-1$. The Weyl algebra is a simple ring that is 
not a matrix ring over a division ring. It is also a non-commutative domain and an Ore extension. See \cite{MR1838439} for more information. 
In 1968, Dixmier conjectured that any 
endomorphism of a Weyl algebra is an automorphism; the conjecture
is still open. 

\subsection*{Gardam's theorem}

Let $K$ be a field and $G$ be a torsion-free group. 
What do the units of $K[G]$ look like? The conjecture
is that units of $K[G]$ are of the form $\lambda g$ for
some $0\ne\lambda\in K$ and $g\in G$. Recently, 
Gardam \cite{MR4334981} found a counterexample 
in the case that 
$K$ is the field of two elements. The problem is still
open for fields of characteristic zero. 





% PI algebras
% kaplansky
% left orderable
% gardam
% promislow
% y algunos de los temas que puse allá también
% braces 
% 
% https://ysharifi.wordpress.com/category/noncommutative-ring-theory-notes/frobenius-algebras/