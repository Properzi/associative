\chapter*{Some topics for final projects}

\pagestyle{plain}
\fancyhf{}
\fancyhead[LE,RO]{Representation theory of groups}
\fancyhead[RE,LO]{Final projects}
\fancyfoot[CE,CO]{\leftmark}
\fancyfoot[LE,RO]{\thepage}

\addcontentsline{toc}{chapter}{Some topics for final projects}

We collect here some topics for final presentations. Some topics
can also be used as bachelor or master theses. 

\subsection*{Rickart's theorem}

In Lecture \ref{09} we presented an algebraic proof of Rickart's theorem. 
The original proof uses analysis; see \cite[(6.4) of Chapter II]{MR1838439}. 

\subsection*{Connel's theorem}

In Lecture \ref{11} we presented the statement of Connel's theorem, which
characterizes prime group rings over fields of characteristic zero 
(see Theorem \ref{thm:Connel}); the proof of this  
result appears for example in \cite[Theorem 2.10 of Chapter 4]{MR798076}. 
As a corollary, one obtains 
that, if $K$ is a field of characteristic zero,
then the group ring $K[G]$ is left artinian if and only if the group
$G$ is finite (see Corollary \ref{cor:Connel}); see 
\cite[Theorem 1.1 of Chapter 10]{MR798076} for a proof. 

\subsection*{Kolchin's theorem}

Let $U_n(\C)$ be the subgroup of $\GL_n(\C)$ 
of matrices $(u_{ij})$ such that 
\[
u_{ij}=\begin{cases}
1&\text{if $i=j$},\\
0&\text{if $i>j$}.\end{cases}
\]

A matrix $a\in\GL_n(\C)$ is said to be \textbf{unipotent} 
if its characteristic polynomial is of the form $(X-1)^n$. 
A subgroup $G$ of $\GL_n(\C)$ is said to be \textbf{unipotent} if
each $g\in G$ is unipotent. 

An important theorem of Kolchin states that 
every unipotent subgroup of $\GL_n(\C)$ is conjugate
of some subgroup of $U_n(\C)$. The theorem and its proof 
appear, for example, 
in the 
VUB course \href{https://github.com/vendramin/representation}{Representation theory of algebras}.

% kaplansky
% left orderable
% gardam
% promislow
% y algunos de los temas que puse allá también
% braces 