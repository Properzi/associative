\chapter*{Some topics for final projects}

\pagestyle{plain}
\fancyhf{}
\fancyhead[LE,RO]{Representation theory of groups}
\fancyhead[RE,LO]{Final projects}
\fancyfoot[CE,CO]{\leftmark}
\fancyfoot[LE,RO]{\thepage}

\addcontentsline{toc}{chapter}{Some topics for final projects}

We collect here some topics for final presentations. Some topics
can also be used as bachelor or master theses. 

\subsection*{Rickart's theorem}

In Lecture \ref{09} we presented an algebraic proof of Rickart's theorem. 
The original proof uses analysis; see \cite[(6.4) of Chapter II]{MR1838439}. 

\subsection*{Connel's theorem}

In Lecture \ref{11} we presented the statement of Connel's theorem, which
characterizes prime group rings over fields of characteristic zero 
(see Theorem \ref{thm:Connel}); the proof of this  
result appears for example in \cite[Theorem 2.10 of Chapter 4]{MR798076}. 
As a corollary, one obtains 
that, if $K$ is a field of characteristic zero,
then the group ring $K[G]$ is left artinian if and only if the group
$G$ is finite (see Corollary \ref{cor:Connel}); see 
\cite[Theorem 1.1 of Chapter 10]{MR798076} for a proof. 

\subsection*{Kolchin's theorem}

Let $U_n(\C)$ be the subgroup of $\GL_n(\C)$ 
of matrices $(u_{ij})$ such that 
\[
u_{ij}=\begin{cases}
1&\text{if $i=j$},\\
0&\text{if $i>j$}.\end{cases}
\]

A matrix $a\in\GL_n(\C)$ is said to be \textbf{unipotent} 
if its characteristic polynomial is of the form $(X-1)^n$. 
A subgroup $G$ of $\GL_n(\C)$ is said to be \textbf{unipotent} if
each $g\in G$ is unipotent. 

An important theorem of Kolchin states that 
every unipotent subgroup of $\GL_n(\C)$ is conjugate
of some subgroup of $U_n(\C)$. The theorem and its proof 
appear, for example, 
in the 
VUB course \href{https://github.com/vendramin/representation}{Representation theory of algebras}.

\subsection*{Dedekind-finite rings}

The idea is to develop basic aspects of Dedekind-finite rings.
A standard reference is Lam's book \cite{MR2278849}. 

%\subsection*{Frobenius algebras}

% definition
% some properties. The group algebra of a finite group is Frobenius
% reference?

%\subsection*{Gelfand--Kirillov dimension of algebras}

% definition
% some properties. 
% reference?

%\subsection*{Hopfian algebras}

% reference?

%\subsection*{von Neumann regular rings}

\subsection*{Skolem--Noether theorem}

Any automorphism of the full $n\times n$ matrix algebra 
is conjugation by some invertible $n\times n$ matrix. This is an elementary 
instance of the celebrated Skolem--Noether theorem. We refer to 
\cite[Chapter 4]{MR3308118} for the theorem 
and its proof (in a more general context).

\subsection*{Double centralizer theorem}

Let $R$ be a ring. 
The centralizer of a subring $S$ of $R$ 
is 
\[
C_R(S)=\{r\in R: rs=sr\text{ for all $s\in S$}\}.
\]
Clearly $C_R(C_R(S))\supseteq S$, but equality not always holds. 
The double centralizer theorems give conditions under which one can conclude that equality occurs; see for example \cite[Chapter 4]{MR3308118}. 


% PI algebras
% kaplansky
% left orderable
% gardam
% promislow
% y algunos de los temas que puse allá también
% braces 
% skolam-noether
% 
% https://ysharifi.wordpress.com/category/noncommutative-ring-theory-notes/frobenius-algebras/