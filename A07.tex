\section*{Project: Gardam's theorem}
\label{section:Gardam}

\index{Promislow's group}
The unit problem is still open for fields of characteristic zero. 
However, it was recently solved by Gardam \cite{MR4334981} 
in the case of 
$K$ the field of two elements. We will present Gardam's theorem 
as a computer calculation. We will use \textsf{GAP} \cite{GAP4}.  

\begin{lemma}
    The group $G=\langle a,b:a^{-1}b^2a=b^{-2},b^{-1}a^2b=a^{-2}\rangle$
    is torsion-free. Moreover, the subgroup $N=\langle a^2,b^2,(ab)^2\rangle$ is normal in $G$, 
    free-abelian of rank three and $G/N\simeq C_2\times C_2$.
\end{lemma}

\begin{proof}
    We first construct the group.
\begin{lstlisting}
gap> F := FreeGroup(2);;
gap> A := F.1;;
gap> B := F.2;;
gap> rels := [(B^2)^A*B^2,(A^2)^B*A^2];;
gap> G := F/rels;;
gap> a := G.1;;
gap> b := G.2;;
\end{lstlisting}
    Now we construct the subgroup $N$ generated by $a^2$, $b^2$ and $(ab)^2$. 
    It is easy to check that $N$ is normal 
    in $G$ and that $G/N\simeq C_2\times C_2$. It is even easier 
    to do this with the computer.
\begin{lstlisting}
gap> N := Subgroup(G, [a^2,b^2,(a*b)^2]);;
gap> IsNormal(G,N);
true
gap> StructureDescription(G/N);
"C2 x C2"
\end{lstlisting}
    It is easy to check by hand that $N$ is abelian, and not so easy to do it with
    the computer. For example, 
    \[
    b^{-2}a^2b^{-2}=b^{-1}a^{-2}b=(b^{-1}a^2b)^{-1}=(a^{-2})^{-1}=a^2.
    \]
    We use the computer to show that
    $N$ is free abelian of rank three. 
\begin{lstlisting}
gap> AbelianInvariants(N);
[ 0, 0, 0 ]
\end{lstlisting}

    Let us prove that $G$ is torsion-free. Let 
    $x=a^2$, $y=b^2$ and $z=(ab)^2$. 
    Since $(G:N)=4$, the group $G$ decomposes as a disjoint union
    $G=N\cup aN\cup bN\cup (ab)N$. Let $g\in G$ be a non-trivial element of finite order. 
    Since $N$ is torsion-free, $g\in aN\cup bN\cup (ab)N$. Without loss of generality 
    we may assume that $g\in aN$, so $g=an$ for some $n\in N$. 
    Let $\pi\colon G\to G/N$ be the canonical map. 
    Since $g\not\in N$ and $\pi(g)\in G/N\simeq C_2\times C_2$, 
    \[
    \pi(g^2)=\pi(g)^2=1
    \]
    so $g^2\in N$ and hence $g^2=1$, as $N$ is torsion-free.  Thus
    \[
    1=g^2=(an)^2=(an)(an)=a^2(a^{-1}na)n=x(a^{-1}na)n.
    \]
    Write $n=x^iy^jz^k$ for some $i,j,k\in\Z$. Then
    \[
    a^{-1}na=(a^{-1}x^ia)(a^{-1}y^ja)(a^{-1}z^ka)=x^it^{-j}z^{-k}
    \]
    and hence $(a^{-1}na)n=x^{2i}$. Then 
    $1=g^2=x(a^{-1}na)n=x^{2i+1}$, a contradiction. 
\end{proof}

Let $P$ be the group
generated by
\[
a=\begin{pmatrix}
1 & 0 & 0 & 1/2\\
0 & -1 & 0 & 1/2\\
0 & 0 & -1 & 0\\
0 & 0 & 0 & 1
\end{pmatrix},
\quad
b=\begin{pmatrix}
-1 & 0 & 0 & 0\\
0 & 1 & 0 & 1/2\\
0 & 0 & -1 & 1/2\\
0 & 0 & 0 & 1
\end{pmatrix}. 
\]
The group $P$ appears in the literature with various names. 
For us $P$ will be the \emph{Promislow group}. 
It is easy to check
that there exists a surjective group homomorphism $G\to P$. 

\begin{exercise}
\label{xca:Promislow}
    Prove that $G\simeq P$.   
\end{exercise}

A solution to Exercise \ref{xca:Promislow} appears 
in \cite{MR798076}. 

\begin{theorem}[Gardam]
\label{thm:Gardam}
\index{Gardam's theorem}
    Let $\F_2$ be the field of two elements. Consider the elements 
    $x=a^2$, $y=b^2$ and $z=(ab)^2$ of $P$ and let 
    \begin{align*}
        &p=(1+x)(1+y)(1+z^{-1}), 
        &&q = x^{-1}y^{-1}+x+y^{-1}z+z,\\
        &r = 1+x+y^{-1}z+xyz,
        &&s=1+(x+x^{-1}+y+y^{-1})z^{-1}.
    \end{align*}
    Then $u=p+qa+rb+sab$ is a non-trivial unit in $\F_2[P]$. 
\end{theorem}

\begin{proof}
    We claim that the inverse of $u$ 
    is the element $v=p_1+q_1a+r_1b+s_1ab$, where
    \begin{align*}
        p_1=x^{-1}(a^{-1}pa), 
        && q_1=-x^{-1}q,
        && r_1=-y^{-1}r,
        && s_1=z^{-1}(a^{-1}sa).
    \end{align*}
    We only need to show that $uv=vu=1$. We will perform this
    calculation with \textsf{GAP}. 
    We first need to create the group $P=\langle a,b\rangle$. 
\begin{lstlisting} 
gap> a := [[1,0,0,1/2],[0,-1,0,1/2],[0,0,-1,0],[0,0,0,1]];;
gap> b := [[-1,0,0,0],[0,1,0,1/2],[0,0,-1,1/2],[0,0,0,1]];;
gap> P := Group([a,b]);
\end{lstlisting}
    Now we create the group algebra $F[P]$ and 
    the embedding $P\hookrightarrow F[P]$.  
    The field $\F_2$ will be \lstinline{GF(2)} 
    and the embedding will be denoted by \lstinline{f}. 
\begin{lstlisting}
gap> R := GroupRing(GF(2),P);;
gap> f := Embedding(P, R);;
\end{lstlisting}
    We first need the elements $x$, $y$ and $z$ that were defined in the
    statement.
\begin{lstlisting}
gap> x := Image(f, a^2);;
gap> y := Image(f, b^2);;
gap> z := Image(f, (a*b)^2);;
\end{lstlisting}
    Now we define the elements $p$, $q$, $r$ and $s$. Note that
    the identity of the group algebra \lstinline{R} 
    is \lstinline{One(R)}. 
\begin{lstlisting}
gap> p := (One(R)+x)*(One(R)+y)*(One(R)+Inverse(z));;
gap> r := One(R)+x+Inverse(y)*z+x*y*z;;
gap> q := Inverse(x)*Inverse(y)+x+Inverse(y)*z+z;;
gap> s := One(R)+(x+Inverse(x)+y+Inverse(y))*Inverse(z);
\end{lstlisting}
    Rather than trying 
    to compute the inverse of $u$ we will show that 
    $uv=vu=1$. For that purpose we need to define
    $p_1$, $q_1$, $r_1$ and $s_1$.
\begin{lstlisting}
gap> p1 := Inverse(x)*p^Image(f, a);;
gap> q1 := -Inverse(x)*q;;
gap> r1 := -Inverse(y)*r;;
gap> s1 := Inverse(z)*s^Image(f, a);;
\end{lstlisting}
Now it is time to prove the theorem. 
\begin{lstlisting}
gap> u := p+q*a+r*b+s*a*b;;
gap> v := p1+q1*a+r1*b+s1*a*b;;
gap> IsOne(u*v);
true
gap> IsOne(v*u);
true
\end{lstlisting}
This completes the proof of the theorem. 
\end{proof}

Our proof of Theorem \ref{thm:Gardam} is exactly 
as that of \cite{MR4334981}. 

\begin{bonus}
     Let $p$ be a prime number and $\F_p$ be the field of size $p$. 
     Use the technique 
     for proving Gardam's theorem to prove Murray's theorem
     on the existence 
     on non-trivial units in $\F_p[P]$. 
     Reference: \href{https://arxiv.org/abs/2106.02147}{arXiv:2106.02147}. 
\end{bonus}



