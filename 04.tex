\chapter{}

We now review some basic results useful to compute radicals. 

\begin{proposition}
	Let $\{R_i:i\in I\}$ be a family of rings. Then 
	\[
	J\left(\prod_{i\in I}R_i\right)=\prod_{i\in I}J(R_i).
	\]
\end{proposition}

\begin{proof}
	Let $R=\prod_{i\in I}R_i$ and $x=(x_i)_{i\in I}\in R$.  The left ideal 
    $Rx$ is quasi-regular if and only if each left ideal $R_ix_i$
	is quasi-regular in $R_i$, as $x$ is quasi-regular in $R$ if and only if each 
	$x_i$ is quasi-regular in $R_i$. Thus $x\in J(R)$ if and only if $x_i\in
	J(R_i)$ for all $i\in I$.
\end{proof}

For the next result we shall need a lemma.

\begin{lemma}
	\label{lemma:trickJ1}
	Let $R$ be a ring and $x\in R$. 
	If $-x^2$ is a left quasi-regular element, then $x$ también. 
\end{lemma}

\begin{proof}
	Sea $r\in R$ tal que $r+(-x^2)+r(-x^2)=0$ y sea $s=r-x-rx$. Entonces $x$ es
	casi-regular a izquierda pues 
	\begin{align*}
		s+x+sx&=(r-x-rx)+x+(r-x-rx)x\\
		&=r-x-rx+x+rx-x^2-rx^2=r-x^2-rx^2=0.\qedhere 
\end{align*}
\end{proof}

%\begin{lemma}
%	\label{lemma:trickJ2}
%	Sea $R$ un anillo. Entonces $x\in J(R)$ si y sólo si $Rx$ es un ideal a
%	izquierda casi-regular a izquierda.
%\end{lemma}
%
%\begin{proof}
%	Si $x\in J(R)$ entonces $Rx\subseteq J(R)$ y luego todo elemento de $Rx$ es
%	casi-regular a izquierda. Recíprocamente, si $Rx$ es casi-regular a
%	izquierda, $Rx+\Z x$ es un ideal a izquierda de $R$. Si $s=rx+nx\in Rx+\Z
%	x$, entonces $-s^2$ es casi-regular a izquierda (pues $-s^2\in Rx$). Por el
%	lema~\ref{lemma:trickJ1}, $s$ es casi-regular a izquierda; en particular,
%	$x$ es casi-regular a izquierda y luego $x\in J(R)$. 
%\end{proof}

\begin{proposition}
	\label{proposition:J(I)}
	If $I$ is an ideal of $R$, then $J(I)=I\cap J(R)$. 
\end{proposition}

\begin{proof}
	Since $I\cap J(R)$ if an ideal of $I$, if $x\in I\cap J(R)$, then $x$ is
	left quasi-regular in $R$. Let $r\in R$ be such that $r+x+rx=0$. 
	Since $r=-x-rx\in I$, $x$ is left quasi-regular 
	in $I$. Thus $I\cap J(R)\subseteq J(I)$. 

	Let $x\in J(I)$ and $r\in R$. Since $-(rx)^2=(-rxr)x\in
	I(J(I))\subseteq J(I)$, the element $-(rx)^2$ is left quasi-regular a izquierda
	en $I$. Thus $rx$ is left quasi-regular by
	Lemma~\ref{lemma:trickJ1}.
\end{proof}

\begin{definition}
\index{Ring!radical}
A ring $R$ is said to be \textbf{radical} if $J(R)=R$. 
\end{definition}

\begin{example}
	If $R$ is a ring, then $J(R)$ is a radical ring, by Proposition~\ref{proposition:J(I)}.
\end{example}

\begin{example}
	The Jacobson radical of $\Z/8$ is $\{0,2,4,6\}$. 
\end{example}

There are several characterizations of radical rings. 

\begin{theorem}
	\label{theorem:anillo_radical}
	Let $R$ be ring. The following statements are equivalent: 
	\begin{enumerate}
		\item $R$ is radical.
		\item $R$ admits no simple $R$-modules. 
		\item $R$ no tiene ideales a izquierda maximales y regulares.
		\item $R$ no tiene ideales a izquierda primitivos.
		\item Every element of $R$ is quasi-regular. 
		\item $(R,\circ)$ is a group. 
	\end{enumerate}
\end{theorem}

\begin{proof}
	The equivalence $(1)\Longleftrightarrow(5)$ follows from 
	Theorem~\ref{thm:casireg_eq}. 
    
    The equivalence $(5)\Longleftrightarrow(6)$ is left as an exercise. 

	Let us prove that $(1)\implies(2)$. Assume that there exists a simple $R$-module $N$. Since 
	$R=J(R)\subseteq\Ann_R(N)$, $R=\Ann_S(N)$. 
	Hence $RN=\{0\}$, a contradiction to the simplicity of $N$.
	
	To prove $(2)\implies(3)$ we note that for each regular and maximal left ideal 
	$I$, the quotient $R/I$ is a simple $R$-module by
	Proposición~\ref{proposition:R/I}. 
	
	To prove $(3)\implies(4)$ assume that there is a primitive left ideal 
	$I=\Ann_R(M)$, where $M$ is some simple $R$-module. Since $R=J(R)\subseteq I$, it follows that  
    $I=R$, a contradiction to the simplicity of $M$.
    %$RM=\{0\}$.

	Finally we prove $(4)\implies(2)$. If $M$ is a simple $R$-module, then 
	$\Ann_R(M)$ is a primitive left ideal.
\end{proof}

\begin{example}
	Let 
	\[
	A=\left\{\frac{2x}{2y+1}:x,y\in\Z\right\}.
	\]
	Then $A$ is a radical ring, as the inverse of the element $\frac{2x}{2y+1}$
	with respect to the circle operation 
	$\circ$ is 
	\[
	\left(\frac{2x}{2y+1}\right)'=\frac{-2x}{2(x+y)+1}.
	\]
\end{example}

\begin{definition}
\index{Ring!nil}
A ring $R$ is said to be \textbf{nil} if for every $x\in R$ there
exists $n=n(x)$ such that $x^n=0$. 
\end{definition}

\begin{exercise}
    Prove that a nil ring is a radical ring. 
\end{exercise}

\begin{exercise}
    Let $\R[X]$ be the ring of power series with real coefficients. Prove that the ideal 
    $X\R[X]$ consisting of power series with zero constant term is a radical ring
    that is not nil. 
\end{exercise}


\begin{theorem}
	\label{thm:Jnilpotente}
	If $R$ is a left artinian ring, then $J(R)$ is nilpotent. 
\end{theorem}

\begin{proof}
	Let $J=J(R)$. Since $R$ is a left artinian ring, the sequence 
	$(J^m)_{m\in\N}$ of left ideals stabilizes. There exists 
	$k\in\N$ such that $J^k=J^l$ for all $l\geq k$. We claim that $J^k=\{0\}$. If
	$J^k\ne\{0\}$ let $\mathcal{S}$ the set of left ideals 
	$I$ such that $J^kI\ne\{0\}$. Since 
	\[
	J^kJ^k=J^{2k}=J^k\ne\{0\},
	\]
	the set $\mathcal{S}$ is non-empty. 
	Since $R$ is left artinian, $\mathcal{S}$ has a minimal element $I_0$. Since $J^kI_0\ne\{0\}$, let $x\in
	I_0\setminus\{0\}$ be such that $J^kx\ne\{0\}$. Moreover, $J^kx$ is a left ideal of $R$ 
	contained in $I_0$ and such that $J^kx\in\mathcal{S}$, as 
	$J^k(J^kx)=J^{2k}x=J^kx\ne\{0\}$. The minimality of $I_0$ implies that, $J^kx=I_0$. In particular, 
	there exists $r\in J^k\subseteq J(R)$ such that $rx=x$. Since $-r\in
	J(R)$ is left quasi-regular, there exists $s\in R$ such that $s-r-sr=0$.
	Thus 
	\[
		x=rx=(s-sr)x=sx-s(rx)=sx-sx=0,
	\]
	a contradiction.
\end{proof}

\begin{corollary}
	Let $R$ be a left artinian ring. Each nil left ideal is nilpotent and 
	$J(R)$ is the unique maximal nilpotent ideal of $R$. 
\end{corollary}

\begin{proof}
	Let $L$ be a nil left ideal of $R$. By Proposition~\ref{pro:nilJ}, $L$
	is contained in $J(R)$. Thus $L$ is nilpotent, as $J(R)$ 
	is nilpotent by Theorem~\ref{thm:Jnilpotente}. 
\end{proof}

\begin{theorem}
	Let $R$ be a ring and $n\in\N$. Then $J(M_n(R))=M_n(J(R))$. 
\end{theorem}

\begin{proof}
	We first prove that $J(M_n(R))\subseteq M_n(J(R))$. 
	If $J(R)=R$, the theorem is clear. Let us assume that $J(R)\ne R$ and let  
	$J=J(R)$. 
	If $M$ is a simple $R$-module, then $M^n$ is a simple $M_n(R)$-module with the usual multiplication. 
	Let $x=(x_{ij})\in J(M_n(R))$ and $m_1,\dots,m_n\in M$. Then
	\[
		x\colvec{3}{m_1}{\vdots}{m_n}=0.
	\]
	In particular, $x_{ij}\in\Ann_R(M)$ for all $i,j\in\{1,\dots,n\}$. Hence 
	$x\in M_n(J)$. 

	We now prove that $M_n(J)\subseteq J(M_n(R))$. Let 
	\[
		J_1=\begin{pmatrix}
			J & 0 & \cdots & 0\\
			J & 0 & \cdots & 0\\
			\vdots & \vdots & \ddots & \vdots\\
			J & 0 & \cdots & 0
		\end{pmatrix}
		\quad\text{and}\quad
		x=\begin{pmatrix}
			x_1 & 0 & \cdots & 0\\
			x_2 & 0 & \cdots & 0\\
			\vdots & \vdots & \ddots & \vdots\\
			x_n & 0 & \cdots & 0
		\end{pmatrix}\in J_1.
	\]
	Since $x_1$ es quasi-regular, there exists $y_1\in R$ such that $x_1+y_1+x_1y_1=0$.
	If
	\[
		y=\begin{pmatrix}
			y_1 & 0 & \cdots & 0\\
			0 & 0 & \cdots & 0\\
			\vdots & \vdots & \ddots & \vdots\\
			0 & 0 & \cdots & 0
		\end{pmatrix}, 
	\]
	then $u=x+y+xy$ is lower triangular, as  
	\[
		u=\begin{pmatrix}
			0 & 0 & \cdots & 0\\
			x_2y_1 & 0 & \cdots & 0\\
			x_3y_1 & 0 & \cdots & 0\\
			\vdots & \vdots & \ddots & \vdots\\
			x_ny_1 & 0 & \cdots & 0
		\end{pmatrix}.
	\]
	Since  
	$u^n=0$, the element
	\[
	v=-u+u^2-u^3+\cdots+(-1)^{n-1} u^{n-1}
	\]
	is such that 
	$u+v+uv=0$. Thus $x$ is right quasi-regular, as  
	\begin{align*}
		x+(y+v+yv)+x(y+v+yv)&=0,
	\end{align*}
	and therefore $J_1$ is right quasi-regular. Similarly one proves that 
	each $J_i$ is right quasi-regular and hence $J_i\subseteq J(M_n(R))$ for all 
	$i\in\{1,\dots,n\}$. In conclusion, 
	\[
	J_1+\cdots+J_n\subseteq J(M_n(R))
	\]
	and therefore $M_n(J)\subseteq J(M_n(R))$.
\end{proof}

\begin{exercise}
	Let $R$ be a unitary ring. Then  
	\[
	J(R)=\bigcap\{M:\text{$M$ is a left maximal ideal}\}.
	\]
\end{exercise}

\begin{exercise}
	Let $R$ be a unitary ring. The
	following statements are equivalent: 
	\begin{enumerate}
		\item $x\in J(R)$.
		\item $xM=0$ for all simple $R$-module $M$.
		\item $x\in P$ for all primitive left ideal $P$.
		\item $1+rx$ is invertible for all $r\in R$.
		\item $1+\sum_{i=1}^n r_ixs_i$ is invertible for all $n\in\N$ and all $r_i,s_i\in R$.
		\item $x$ belongs to every left maximal ideal maximal. 
	\end{enumerate}
\end{exercise}

The following exercises are optional. They somewhat show a recent new application of radical rings
to solutions of the celebrated Yang--Baxter equation. 

\begin{exercise}
A pair $(X,r)$ is a \textbf{solution} to the 
Yang--Baxter equation if $X$ is a set and
$r\colon X\times X\to X\times X$ is a bijective map such that  
\[
	(r\times\id)\circ (\id\times r)\circ (r\times\id)
	=(\id\times r)\circ (r\times\id)\circ (\id\times r)
\]
The solution $(X,r)$ is said to be \textbf{involutive} 
if $r^2=\id$. By convention we write 
\[
	r(x,y)=(\sigma_x(y),\tau_y(x)).
\]
The solution $(X,r)$ is said to be \textbf{non-degenerate}  
$\sigma_x\colon X\to X$ and 
$\tau_x\colon X\to X$ are bijective for all $x\in X$.

\begin{enumerate}
    \item Let $X$ be a set and $\sigma\colon X\to X$ be a bijective map. Prove that  
          the pair $(X,r)$, where 
          $r(x,y)=(\sigma(y),\sigma^{-1}(x))$, is an involutive non-degenerate solution. 
\end{enumerate}
Let $R$ be a radical ring. For $x,y\in R$ let 
\begin{align*}
	&\lambda_x(y)=-x+x\circ y=xy+y,\\
	&\mu_y(x)=\lambda_x(y)'\circ x\circ y=(xy+y)'x+x
\end{align*}
Prove the following statements:
\begin{enumerate}
    \setcounter{enumi}{1}
		\item $\lambda\colon (R,\circ)\to\Aut(R,+)$, $x\mapsto
			\lambda_x$, is wis a group homomorphism.
		\item $\mu\colon (R,\circ)\to\Aut(R,+)$, $y\mapsto\mu_y$,
    		is a group antihomomorphism.
	    \item The map 
    	\[
	        r\colon R\times R\to R\times R,\quad
	        r(x,y)=(\lambda_x(y),\mu_y(x)),
	    \]
	is an involutive non-degenerate solution. 
\end{enumerate}
\end{exercise}

%\begin{exercise}
%	Sea $A$ un anillo radical. Para $a,b\in A$ se define 
%	\[
%		\mu_b(a)=\lambda_a(b)'\circ a\circ b=(ab+b)'a+a.
%	\]
%	Demuestre que la función $\mu\colon (A,\circ)\to\Aut(A,+)$,
%	$b\mapsto\mu_b$, está bien definida y es un antimorfismo de grupos.
%\end{exercise}

\begin{exercise}
    If $D$ is a division ring and $R=D[X_1,\dots,X_n]$, then
    $J(R)=\{0\}$. 
%     Como las unidades de $R$ son los elementos no nulos de $D$,
% 	$J(R)$ es un ideal de $D$. Como $D$ es simple, $J(R)\in\{0,D\}$. Si
% 	$J(R)=D$, entonces existe  $f\in R$ tal que $-1+f+(-1)f=0$ y luego $-1=0$,
% 	una contradicción. Luego $J(R)=0$.
\end{exercise}


\begin{example}
\index{Ring!local}
    A commutative and unitary ring $R$ is \textbf{local} if it contains
    only one maximal ideal. 
	If $R$ is a local ring and $M$ be its maximal ideal, then $J(R)=M$. Some particular cases: 
	\begin{enumerate}
		\item If $K$ is a field and $R=K\left[ [X] \right]$, then $J(R)=(X)$. 
		\item If $p$ is a prime number and $R=\Z/p^n$, then $J(R)=(p)$. 
	\end{enumerate}
\end{example}

We finish the discussion on the Jacobson radical with 
some results in the case of unitary algebras. 

\begin{theorem}
	Let $A$ be a $K$-algebra and $I$ be a subset of $A$. Then $I$ is 
	a left regular maximal ideal ideal of the algebra $A$ if and only if $I$ is 
	a left regular maximal ideal of the ring $A$.
\end{theorem}

\begin{proof}
	Sea $I$ un ideal a izquierda maximal y regular del anillo $A$.  Queremos
	demostrar que $\lambda I\subseteq I$ para todo $\lambda\in K$. Si suponemos
	que $\lambda I\not\subseteq I$ para algún $\lambda$, entonces $I+\lambda I$
	es un ideal a izquierda del anillo $A$ que contiene a $I$ pues
	\[
	a(I+\lambda I)=aI+a(\lambda I)\subseteq I+\lambda (aI)\subseteq I+\lambda I.
	\]
	Como $I$ es maximal, $I+\lambda I=A$. Por la regularidad de $I$, existe $e\in R$ tal que
	$a-ae\in I$ para todo $a\in A$. Si escribimos $e=x+\lambda y$ para $x,y\in
	I$, entonces
	\[
		e^2=e(x+\lambda y)=ex+e(\lambda y)=ex+(\lambda e)y\in I.
	\]
	Como $e^2-e\in I$ y $e^2\in I$, se concluye que $e\in I$. Luego $A=I$ pues
	$a-ae\in I$ para todo $a\in A$, una contradicción.

	Recíprocamente, si $I$ es un ideal a izquierda maximal y regular del
	álgebra $A$, entonces $I$ es ideal a izquierda regular del anillo $A$.
	Falta ver entonces que $I$ es maximal. Por el
	ejercicio~\ref{exa:Zorn:regular} sabemos que existe un ideal a izquierda
	maximal $L$ del anillo $A$ que contiene a $I$. Como $L$ es regular, la
	implicación demostrada nos dice que $L$ es un ideal a izquierda maximal y
	regular del anillo $A$. Luego $L=I$ por la maximalidad de $I$.
\end{proof}

\begin{corollary}
	Sea $A$ un álgebra. El radical de Jacobson del anillo $A$ coincide con el
	radical de Jacobson del álgebra $A$.
\end{corollary}

\begin{proof}
	Es consecuencia del teorema anterior y de que el radical de Jacobson es la
	intersección de los ideales a izquierda maximales y regulares.
\end{proof}

\begin{lemma}
	\label{lemma:algebraico=nil}
	Sea $A$ un álgebra unitaria y sea $x\in J(A)$. 
	Entonces $x$ es algebraico si y sólo si $x$ es nil.
\end{lemma}

\begin{proof}
	Demostremos la implicación no trivial. Como $x$ es algebraico, 
	existen $a_0,\dots,a_n\in K$ no todos cero tales que
	\[
		a_0+a_1x+\cdots+a_nx^n=0.
	\]
	Sea $r$ el menor entero tal que $a_r\ne 0$. Podemos escribir entonces
	\[
		x^r(1+b_1x+\cdots+b_mx^m)=0,
	\]
	donde $b_1,\dots,b_m\in K$.  Como $1+b_1x+\cdots+b_mx^m$ es una unidad por
	el corolario~\ref{cor:Jcon1}, entonces $x^r=0$.
\end{proof}

Como aplicación tenemos el siguiente resultado:

\begin{theorem}
	\label{thm:algebraica=>Jnil}
	Si $A$ es un álgebra algebraica, $J(A)$ es el mayor ideal nil de $A$.
\end{theorem}

\begin{proof}
	Por el lema~\ref{lemma:algebraico=nil}, $J(A)$ es un ideal nil. Por la
	propoisición~\ref{pro:nilJ}, $J(A)$ es el mayor ideal nil de $A$.
\end{proof}

\begin{theorem}[Amitsur]
	\label{thm:Amitsur}
	Si $A$ es una $K$-álgebra unitaria tal que $\dim_KA<|K|$ (como cardinales),
	entonces $J(A)$ es el mayor ideal nil de $A$.
\end{theorem}

\begin{proof}
	Si $K$ es un cuerpo finito, entonces $A$ es un álgebra de dimensión finita.
	Como entonces $A$ es algebraica, $J(A)$ es un ideal nil por el
	teorema~\ref{thm:algebraica=>Jnil}.

	Supongamos entonces que $K$ es infinito. Sea $a\in J(A)$. El
	corolario~\ref{cor:Jcon1} implica que todo elemento de la forma
	$1-\lambda^{-1}a$, $\lambda\in K\setminus\{0\}$, es inversible. Entonces 
	\[
		a-\lambda=-\lambda(1-\lambda^{-1}a)
	\]
	es inversible para todo $\lambda\in K\setminus\{0\}$. Sea
	$S=\{(a-\lambda)^{-1}:\lambda\in K\setminus\{0\}\}$. Como
	\[
	(a-\lambda)^{-1}=(a-\mu)^{-1}\Longleftrightarrow\lambda=\mu,
	\]
	entonces $|S|=|K\setminus\{0\}|=|K|>\dim_KA$. Como entonces $S$ 
	es linealmente dependiente, 
	existen escalares no
	nulos $\beta_1,\dots,\beta_n\in K$ y elementos distintos $\lambda_1,\dots,\lambda_n\in K$ tales que
	\begin{equation}
		\label{eq:Amitsur}
		\sum_{i=1}^n \beta_i(a-\lambda_i)^{-1}=0.
	\end{equation}
	Si multiplicamos~\eqref{eq:Amitsur} por $\prod_{i=1}^n(a-\lambda_i)$, obtenemos
	\[
		\sum_{i=1}^n\beta_i\prod_{j\ne i}(a-\lambda_j)=0.
	\]
	Afirmamos que $a$ es algebraico sobre $K$. En efecto, 
	\[
		f=\sum_{i=1}^n\beta_i\prod_{j\ne i}(X-\lambda_j)
	\]
	es no nulo pues
	$f(\lambda_1)=\beta_1(\lambda_1-\lambda_2)\cdots(\lambda_1-\lambda_n)\ne0$
	y cumple que $f(a)=0$. Como $a\in J(A)$ es algebraico, $a$ es nil por el
	lema~\ref{lemma:algebraico=nil}.
\end{proof}

Amitsur's theorem implies the following result. 

\begin{corollary}
	Sea $K$ un cuerpo no numerable y $A$ una $K$-álgebra con base numerable.
	Entonces $J(A)$ es el mayor ideal nil de $A$.
\end{corollary}

% \begin{proof}
% 	Es consecuencia del teorema de Amitsur pues $\dim_KA<|K|$. 
% \end{proof}



The following problem is maybe the most important open 
problem in non-commutative ring theory. 

\begin{openproblem}[K\"othe]
\label{prob:Koethe}
Let $R$ be a ring. Is the sum 
of two arbitrary nil left ideals of $R$ is nil?
\end{openproblem}

Open problem~\ref{prob:Koethe} is the well-known K\"othe's conjecture. 
The conjecture was first formulated in 1930, see \cite{MR1545158}. It is known to be true
in several cases. In full generality, the problem is still open. In~\cite{MR306251} 
Krempa proved that
the following statements are equivalent:
\begin{enumerate}
    \item K\"othe's conjecture is true.  
    \item If $R$ is a nil ring, then $R[X]$ is a radical ring. 
    \item If $R$ is a nil ring, then $M_2(R)$ is a nil ring. 
    \item Let $n\geq2$. If $R$ is a nil ring, then $M_n(R)$ is a nil ring. 
\end{enumerate}

In 1956 Amitsur formulated the following conjecture, see for example
\cite{MR0347873}: If $R$ is a nil ring, then $R[X]$ is a nil ring. In~\cite{MR1793911} 
Smoktunowicz found a counterexample to Amitsur's conjecture. 
This counterexample suggests that K\"othe's conjecture might be false. 
A simplification of Smoktunowicz's example
appears in~\cite{MR3169522}. See \cite{MR1879880,MR2275597} for more
information on K\"othe's conjecture and related topics. 
