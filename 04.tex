\chapter{}
\label{04}


\topic{Jacobson's radical}
%Let us discuss the Jacobson radical and radical rings. 

\begin{definition}
\index{Jacobson radical}
Let $R$ be a ring. The \textbf{Jacobson radical} $J(R)$
is the intersection of all the annihilators of simple left $R$-modules. If $R$ does not
have simple left $R$-modules, then $J(R)=R$. 
\end{definition}

From the definition, it follows
that $J(R)$ is an ideal. Moreover, 
	\[
		J(R)=\bigcap\{P:\text{$P$ left primitive ideal}\}.
	\]

	If $I$ is an ideal of $R$ and $n\in\Z_{>0}$, $I^n$ is the additive subgroup of $R$ 
generated by the set $\{y_1\dots y_n:y_j\in I\}$. 

\begin{definition}
\index{Ideal!nilpotent}
\index{Left ideal!nilpotent}
An ideal $I$ of $R$ is \textbf{nilpotent} 
if $I^n=\{0\}$ for some $n\in\Z_{>0}$.
\end{definition}

Similarly, one defines right or left nilpotent ideals. 
Note that an ideal $I$ is nilpotent if and only if there exists $n\in\Z_{>0}$ such that 
$x_1x_2\cdots x_n=0$ for all $x_1,\dots,x_n\in I$.  

\begin{definition}
	An element $x$ of a ring is said to be \textbf{nil} (or nilpotent) if $x^n=0$ for some $n\in\Z_{>0}$. 
\end{definition}

\begin{definition}
\index{Ideal!nil}
\index{Left ideal!nil}
An ideal $I$ of a ring is said to be \text{nil} if every element of $I$ is nil. 
\end{definition}

Similarly, one defines right or left nil ideals. 
Note that every nilpotent ideal is nil, as $I^n=0$ implies $x^n=0$ for all 
$x\in I$.

\begin{example}
	Let $R=\C[X_1,X_2,\dots]/(X_1,X_2^2,X_3^3,\dots)$. The ideal 
	$I=(X_1,X_2,X_3,\dots)$ is nil in $R$, as it is generated by nilpotent element. However, it is not nilpotent. Indeed, if $I$ is nilpotent, then there exists $k\in\Z_{>0}$ such that 
	$I^k=0$ and hence $x_i^k=0$ for all $i$, a contradiction since 
	$x_{k+1}^k\ne0$. 	
\end{example}

\begin{proposition}
	\label{pro:nilJ}
	Let $R$ be a ring. Then every nil left ideal (resp. right ideal) is contained in $J(R)$.
\end{proposition}

\begin{proof}
	Assume that there is a nil left ideal (resp. right ideal) $I$ such that 
	$I\not\subseteq J(R)$. There exists a simple $R$-module $M$ such that 
	$n=xm\ne 0$ for some $x\in I$ and some $m\in M$. Since $M$ is simple,
	$Rn=M$ and hence there exists $r\in R$ such that 
	\[
	(rx)\cdot m=r\cdot (x\cdot m)=r\cdot n=m\quad\text{(resp.
	$(xr)\cdot n=x\cdot (r\cdot n)=x\cdot m=n$).}
	\]
	Thus $(rx)^k\cdot m=m$ (resp. $(xr)^k\cdot n=n$) for all 
	$k\geq1$, a contradiction since $rx\in I$ (resp. $xr\in I$) is a nilpotent element. 
\end{proof}

\begin{definition}
\index{Element!left quasi-regular}
\index{Element!quasi-regular}
Let $R$ be a ring. An element $a\in R$ is said to be 
\textbf{left quasi-regular} if there exists $r\in R$ such that $r+a+ra=0$. Similarly, 
$a$ is said to be \textbf{right quasi-regular} if there exists $r\in R$ such that $a+r+ar=0$. 
\end{definition}

% \begin{exercise}
% 	\label{exercise:circ}
Let $R$ be a ring. A direct calculation shows that
\[
R\times R\to R,
\quad
(r,s)\mapsto r\circ s=r+s+rs,
\]
is an associative operation with neutral element $0$.
To show an explicit example let $R=\Z/3=\{0,1,2\}$. 
The multiplication table for the circle 
operation is  
	\begin{table}[ht]
		\centering
		\begin{tabular}{c|ccc}
			$\circ$ & 0 & 1 & 2\tabularnewline
			\hline
			0 & 0 & 1 & 2\tabularnewline
			1 & 1 & 0 & 2\tabularnewline
			2 & 2 & 2 & 2\tabularnewline
		\end{tabular}
% 		\caption{The multiplication table of the radical ring $\Z/3$.}
 	\end{table}
%\end{exercise}

% \begin{exercise}
% 	Let $R=\Z/3=\{0,1,2\}$. Compute the multiplication table with respect to the circle 
%  	operation given by the previous exercise.  
%  	%is then 
% % 	\begin{table}[ht]
% % 		\centering
% % 		\begin{tabular}{c|ccc}
% % 			$\circ$ & 0 & 1 & 2\tabularnewline
% % 			\hline
% % 			0 & 0 & 1 & 2\tabularnewline
% % 			1 & 1 & 0 & 2\tabularnewline
% % 			2 & 2 & 2 & 2\tabularnewline
% % 		\end{tabular}
% % 		\caption{The multiplication table of the radical ring $\Z/3$.}
% % 	\end{table}
% \end{exercise}

If $R$ is unitary, an element $x\in R$ is left quasi-regular (resp. right quasi-regular)
if and only if $1+x$ is left invertible (resp. right invertible). In fact, 
if $r\in R$ is such that $r+x+rx=0$, then $(1+r)(1+x)=1+r+x+rx=1$.
Conversely, if there exists $y\in R$ such that $y(1+x)=1$, then  
\[
(y-1)\circ x=y-1+x+(y-1)x=0.
\]

\begin{example}
	If $x\in R$ is a nilpotent element, 
    then $y=\sum_{n\geq1}x^n\in R$ is left quasi-regular. 
	In fact, if there exists $N$ such that $x^N=0$, 
    then the sum defining $y$ is finite 
    and $y+(-x)+y(-x)=0$.  Is right quasi-regular?
%	En efecto, si existe $N$ tal que $x^N=0$, la suma que
%	define al elemento $y$ es finita y cumple que $y+(-x)+y(-x)=0$.  
\end{example}

\begin{definition}
A left ideal $I$ of $R$ is said to be 
\textbf{left quasi-regular} (resp. right quasi-regular) if every element of $I$ is
left quasi-regular (resp. right quasi-regular). A left ideal 
is said to be \textbf{quasi-regular} if it is left and right quasi-regular. 
\end{definition}

Similarly 
one defines right quasi-regular ideals and quasi-regular ideals. 

\begin{lemma}
	\label{lemma:casiregular}
	Let $I$ be a left ideal of $R$. If $I$ is left quasi-regular, then 
	$I$ is quasi-regular.
\end{lemma}

\begin{proof}
	Let $x\in I$. Let us prove that $x$ is right quasi-regular. Since $I$ is
	left quasi-regular, there exists $r\in R$ such that $r\circ x=r+x+rx=0$. Since 
	$r=-x-rx\in I$, there exists $s\in R$ such that $s\circ
	r=s+r+sr=0$. Then $s$ is right quasi-regular and  
	\[
	x=0\circ x=(s\circ r)\circ x=s\circ (r\circ x)=s\circ 0=s.\qedhere
	\]
\end{proof}

% \index{Lemma!Zorn}
% Let $(A,\leq)$ be a \textbf{partially order set}, this means that $A$ is a set together with a 
% reflexive, transitive, and anti-symmetric binary relation
% $R$ en $A\times A$, where $a\leq b$ if and only if $(a,b)\in R$. 
% Recall that the relation is reflexive if $a\leq a$ for all $a\in A$, the relation is transitive if 
% $a\leq b$ and $b\leq c$ imply that 
% $a\leq c$ and the relation is anti-symmetric if $a\leq b$ and $b\leq a$ imply $a=b$.
% The elements $a,b\in A$ are said to be \textbf{comparable} if $a\leq b$ or $b\leq
% a$. An element $a\in A$ is said to be \textbf{maximal} if 
% $c\leq a$ 
% for all $c\in A$
% that is comparable with $a$. 
% An \textbf{upper bound} for a non-empty subset $B\subseteq A$ is an element $d\in
% A$ such that $b\leq d$ for all $b\in B$. A \textbf{chain} in $A$ is a subset 
% $B$ such that every pair of elements of $B$ are comparable. 
% \textbf{Zorn's lemma} states the following property: 
% \begin{quote}
% If $A$ is a non-empty partially ordered set such that every chain in 
% $A$ contains an upper bound in $A$, then $A$ contains a maximal element. 
% \end{quote}

% Our application of Zorn's lemma:
The following result uses Zorn's lemma. 

\begin{lemma}
	\label{lemma:maxreg}
	Let $R$ be a ring, and $x\in R$ be an element that is not left quasi-regular Then there
	exists a maximal left ideal $M$ such that 
	$x\not\in M$. Moreover, $R/M$ is a simple $R$-module and 
	$x\not\in\Ann_R(R/M)$.
\end{lemma}

\begin{proof}
	Let $T=\{r+rx:r\in R\}$. A straightforward calculation shows that $T$ is a left ideal of 
	$R$ such that $x\not\in T$ (if $x\in T$, then $r+rx=-x$ for some 
	$r\in R$, a contradiction since $x$ is not left quasi-regular). 

	The only left ideal of $R$ containing 
	$T\cup\{x\}$ is $R$. Indeed, if there exists a left ideal $U$ containing $T$, then 
    $x\not\in U$, since otherwise every $r\in R$ could be written as 
	$r=(r+rx)+r(-x)\in U$. 

	Let $\mathcal{S}$ be the set of proper left ideals of $R$ containing 
	$T$ partially ordered by inclusion. If $\{K_i:i\in I\}$ is a chain in 
	$\mathcal{S}$, then $K=\cup_{i\in I}K_i$ is an upper bound for the chain 
	($K$ is a proper, as $x\not\in K$). Zorn's lemma implies that 
	$\mathcal{S}$ admits a maximal element $M$. Thus $M$
	is a maximal left ideal such that $x\not\in M$. Moreover, $M$ is regular
	since $r-r(-x)\in T\subseteq M$ for all $r\in R$. Therefore $R/M$ is a simple 
	$R$-module by Proposition~\ref{proposition:R/I}. Since $x\cdot (x+M)\ne
	0$ (if $x^2\in M$, then  $x\in M$, as $x+x^2\in
	T\subseteq M$), it follows that $x\not\in\Ann_R(R/M)$.
\end{proof}

If $x\in R$ is not left quasi-regular, the lemma implies that there exists 
a simple $R$-module $M$ such $x\not\in\Ann_R(M)$. Thus 
$x\not\in J(R)$.

\begin{theorem}
	\label{thm:casireg_eq}
	Let $R$ be a ring and $x\in R$. The following statements are equivalent: 
	\begin{enumerate}
		\item The left ideal generated by $x$ is quasi-regular.
		\item $Rx$ is quasi-regular.
		\item $x\in J(R)$.
	\end{enumerate}
\end{theorem}

\begin{proof}
	The implication $(1)\implies(2)$ is trivial, as $Rx$ is included in the left ideal 
	generated by $x$.  
	
	We now prove $(2)\implies(3)$. If
	$x\not\in J(R)$, then Lemma~\ref{lemma:maxreg} implies that there exists a simple 
	$R$-module $M$ such that $xm\ne 0$ for some $m\in M$. The simplicity of $M$ implies
	that $R(xm)=M$. Thus there exists $r\in R$ such that $rxm=-m$. There is an element 
	$s\in R$ such that $s+rx+s(rx)=0$ and hence 
	\[
	-m=rxm=(-s-srx)m=-sm+sm=0,
	\]
	a contradiction. 
	
	Finally, to prove $(3)\implies(1)$, it is enough to note that 
	$x$ is left quasi-regular. If $x\in J(R)$, 
	then $x$ is left quasi-regular by 
	the previous lemma. 
	Thus the left ideal generated by 
	$x$ is quasi-regular by Lemma~\ref{lemma:casiregular}.
\end{proof}

The theorem immediately implies the following corollary. 

\begin{corollary}
	If $R$ is a ring, then $J(R)$ is a quasi-regular ideal that contains every 
	left quasi-regular ideal. 
\end{corollary}

The following result is somewhat what we all had in mind. 

% \begin{proof}
% 	Es consecuencia inmediata del teorema~\ref{thm:casireg_eq}. 
%%	Como \[
%%	J(R)=\bigcap\{\Ann_R(R/I):I\text{ ideal a izquierda maximal y
%%	regular}\}
%%	\]
%%	por el teorema~\ref{thm:J(R)}, 
%%	todo $x\in J(R)$ es casi-regular gracias al
%%	lema~\ref{lemma:K}. Si $I$ es un ideal casi-regular a izquierda, entonces
%%	$I\subseteq J(R)$ por el lema~\ref{lemma:JsupsetCR}.
%\end{proof}

%\begin{lemma}
%%	\label{lemma:maxreg}
%	Sea $R$ un anillo y sea $I\ne R$ un ideal a izquierda regular. Entonces $I$
%	está contenido en algún ideal a izquierda maximal y regular.
%\end{lemma}
%
%\begin{proof}
%	
%\end{proof}

%\begin{lemma}
%	\label{lemma:K}
%	Sean $R$ un anillo y $K=\bigcap\{I:\text{$I$ ideal a izquierda maximal y
%	regular}\}$.  Entonces $K$ es un ideal a izquierda casi-regular.
%\end{lemma}
%
%\begin{proof}
%	Gracias al lema~\ref{lemma:casiregular}, basta ver que $K$ es casi-regular
%	a izquierda.  Sea $a\in K$ y sea $T=\{r+ra:r\in R\}$.  Es claro que $T$ es
%	un ideal a izquierda regular con $e=-a$. Como $a\in K$, existe $s\in R$ tal
%	que $s+a+sa=0$. Entonces, como $-a=s+sa$, 
%	\[
%	r+re=r+r(-a)=r+r(s+sa)=r+rs+rsa\in r+T
%	\]
%	para todo $r\in R$. 
%	Si $T\ne R$, por el lema~\ref{lemma:maxreg} existe un ideal a izquierda
%	$J$, maximal y regular.  Como $a\in K\subseteq J$, $J$ es ideal a izquierda
%	y $r+ra\in T\subseteq J$ para todo $r\in R$, se concluye que $R=J$, una
%	contradicción.  Luego $T=R$ y entonces existe $r\in R$ tal que $r+ra=-a$.
%	Esto implica que $a$ es casi-regular a izquierda. 
%\end{proof}

%\begin{lemma}
%	\label{lemma:JsupsetCR}
%	Sea $R$ un anillo que admite un $R$-módulo simple. Si $I$ es un ideal a
%	izquierda casi-regular a izquierda, $I\subseteq J(R)$.
%\end{lemma}
%
%\begin{proof}
%	Supongamos que $I\not\subseteq J(R)$. Existe entonces un $R$-módulo simple
%	$N$ tal que $IN\ne 0$. Entonces $In\ne 0$ para algún $0\ne n\in N$. Como
%	$I$ es un ideal a izquierda, $In\subseteq N$ es un submódulo no nulo del
%	simple $N$. Luego $In=N$. Existe entonces $x\in I$ tal que $xn=-n$. Como
%	$I$ es casi-regular a izquierda, existe $r\in R$ tal que $r+x+rx=0$.
%	Entonces
%	\[
%		0=0n=(r+x+rx)n=rn+xn+rxn=rn-n-rn=-n,
%	\]
%	una contradicción.
%\end{proof}

\begin{theorem}
	\label{thm:J(R)}
	Let $R$ be a ring such that $J(R)\ne R$. Then 
	\begin{align*}
		J(R)&=\bigcap\{I:\text{$I$ regular maximal left ideal of $R$}\}.
	\end{align*}
\end{theorem}

\begin{proof}
    We only prove the non-trivial inclusion. 
	Let 
	\[
	K=\bigcap\{I:\text{$I$ regular maximal left ideal of $R$}\}.
	\]
	By
	Proposition~\ref{proposition:R/I}, 
	\[
		J(R)=\bigcap\{\Ann_R(R/I):I\text{ regular maximal left ideal of $R$}\}.
	\]
	Let $I$ be a regular maximal left ideal. If $r\in J(R)\subseteq
	\Ann_R(R/I)$, then, since $I$ is regular, there exists $e\in R$ such that
	$r-re\in I$. Since 
	\[
	re+I=r(e+I)=\{0\},
	\]
	$re\in I$ and hence $r\in I$. Thus $J(R)\subseteq K$. 
\end{proof}

\begin{example}
	Each maximal ideals of $\Z$ is of the form $p\Z=\{pm:m\in\Z\}$ for some prime number $p$. 
	Thus $J(\Z)=\cap_p p\Z=\{0\}$.
\end{example}

%\begin{example}
%	Sea $D$ un anillo de división y sea $R=D[x_1,\dots,x_n]$. Si $f\in J(R)$
%	entonces\dots Luego $J(R)=0$. 
%\end{example}



We now review some basic results useful to compute radicals. 

\begin{proposition}
	Let $\{R_i:i\in I\}$ be a family of rings. Then 
	\[
	J\left(\prod_{i\in I}R_i\right)=\prod_{i\in I}J(R_i).
	\]
\end{proposition}

\begin{proof}
	Let $R=\prod_{i\in I}R_i$ and $x=(x_i)_{i\in I}\in R$.  The left ideal 
    $Rx$ is quasi-regular if and only if each left ideal $R_ix_i$
	is quasi-regular in $R_i$, as $x$ is quasi-regular in $R$ if and only if each 
	$x_i$ is quasi-regular in $R_i$. Thus $x\in J(R)$ if and only if $x_i\in
	J(R_i)$ for all $i\in I$.
\end{proof}

For the next result, we shall need a lemma.

\begin{lemma}
	\label{lemma:trickJ1}
	Let $R$ be a ring and $x\in R$. 
	If $-x^2$ is a left quasi-regular element, then so is $x$. 
\end{lemma}

\begin{proof}
	Let $r\in R$ be such that $r+(-x^2)+r(-x^2)=0$ and $s=r-x-rx$. Then
    $x$ is left quasi-regular, as 
    \begin{align*}
		s+x+sx&=(r-x-rx)+x+(r-x-rx)x\\
		&=r-x-rx+x+rx-x^2-rx^2=r-x^2-rx^2=0.\qedhere 
\end{align*}
\end{proof}

%\begin{lemma}
%	\label{lemma:trickJ2}
%	Sea $R$ un anillo. Entonces $x\in J(R)$ si y sólo si $Rx$ es un ideal a
%	izquierda casi-regular a izquierda.
%\end{lemma}
%
%\begin{proof}
%	Si $x\in J(R)$ entonces $Rx\subseteq J(R)$ y luego todo elemento de $Rx$ es
%	casi-regular a izquierda. Recíprocamente, si $Rx$ es casi-regular a
%	izquierda, $Rx+\Z x$ es un ideal a izquierda de $R$. Si $s=rx+nx\in Rx+\Z
%	x$, entonces $-s^2$ es casi-regular a izquierda (pues $-s^2\in Rx$). Por el
%	lema~\ref{lemma:trickJ1}, $s$ es casi-regular a izquierda; en particular,
%	$x$ es casi-regular a izquierda y luego $x\in J(R)$. 
%\end{proof}

\begin{proposition}
	\label{proposition:J(I)}
	If $I$ is an ideal of $R$, then $J(I)=I\cap J(R)$. 
\end{proposition}

\begin{proof}
	Since $I\cap J(R)$ if an ideal of $I$, if $x\in I\cap J(R)$, then $x$ is
	left quasi-regular in $R$. Let $r\in R$ be such that $r+x+rx=0$. 
	Since $r=-x-rx\in I$, $x$ is left quasi-regular 
	in $I$. Thus $I\cap J(R)\subseteq J(I)$. 

	Let $x\in J(I)$ and $r\in R$. Since $-(rx)^2=(-rxr)x\in
	I(J(I))\subseteq J(I)$, the element $-(rx)^2$ is left quasi-regular 
	in $I$. Thus $rx$ is left quasi-regular by
	Lemma~\ref{lemma:trickJ1}.
\end{proof}

