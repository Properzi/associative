\chapter{}

\begin{definition}
\index{Ideal!primitive}
An ideal $P$ of a ring $R$ is said to be \textbf{primitive} if $P=\Ann_R(M)$
for some simple $R$-module $M$. 
\end{definition}

\begin{lemma}
	\label{lemma:primitivo}
	Let $R$ be a ring and $P$ be an ideal of $R$. Then $P$ is primitive if and only if 
	$R/P$ is a primitive ring.
\end{lemma}

\begin{proof}
	Assume that $P=\Ann_R(M)$ for some $R$-module $M$. Then $M$ is a simple 
	$(R/P)$-module with $(r+P)\cdot m=r\cdot m$, $r\in R$, $m\in M$. This is well-defined, as 
	$P=\Ann_R(M)$. Since $M$ is a simple $R$-module, it follows that $M$ is 
	a simple $(R/P)$-module. Moreover, $\Ann_{R/P}M=\{0\}$. Indeed, if 
	$(r+P)\cdot M=\{0\}$, then $r\in\Ann_RM=P$ and hence $r+P=P$.

	Assume now that $R/P$ is primitive. Let $M$ be a faithful simple $(R/P)$-module. 
	Then $r\cdot m=(r+P)\cdot m$, $r\in R$,
	$m\in M$, turns $M$ into an $R$-module. It follows that $M$ is simple and that $P=\Ann_R(M)$. 
\end{proof}

%\begin{example}
%	Si $I$ es un ideal maximal de un anillo unitario $R$, entonces $I$ es
%	primitivo. Como $I$ es ideal maximal y regular (pues $1\in R$), el cociente
%	$R/I$ es un anillo unitario simple y luego $R/I$ es primitivo por la
%	proposición~\ref{proposition:simple=>prim}. 
%\end{example}

%\begin{example}
%	Si $I$ es un ideal primitivo de un anillo conmutativo $R$, entonces $I$ es
%	maximal pues $R/I$ es un cuerpo (por ser primitivo y conmutativo), ver
%	proposición~\ref{proposition:prim+conm=cuerpo}.
%\end{example}

\begin{example}
	Let $R_1,\dots,R_n$ be primitive rings and $R=R_1\times\cdots\times
	R_n$. Then each $P_i=R_1\times\cdots\times R_{i-1}\times\{0\}\times
	R_{i+1}\times\cdots\times R_n$ is a primitive ideal of $R$ since 
	$R/P_i\simeq R_i$.
\end{example}

%Recordemos que un ideal a izquierda $L$ de $R$ se dice \textbf{minimal} si
%$L\ne0$ y $L$ no contiene propiamenete a otros ideales a izquierda no nulos de
%$R$.
%
%\begin{example}
%	Sea $L$ un ideal a izquierda de $R$ tal que $RL\ne0$. Entonoces $L$ es
%	simple si y sólo si $L$ es minimal.
%\end{example}

\begin{lemma}
	\label{lemma:maxprim}
	Let $R$ be a ring. If $P$ is a primitive ideal, there exists a maximal 
	left ideal $I$ such that $P=\{x\in R:xR\subseteq I\}$.
	Conversely, if $I$ is a maximal regular left ideal, then 
	$\{x\in R:xR\subseteq L\}$ is a primitive ideal. 
\end{lemma}

\begin{proof}
	Assume that $P=\Ann_R(M)$ for some simple $R$-module $M$. By
	Proposition~\ref{proposition:R/I}, there exists a regular maximal 
	left ideal 
	$I$ such that $M\simeq R/I$. Then $P=\Ann_R(R/I)=\{x\in
	R:xR\subseteq I\}$. 

	Conversely, let $I$ a regular maximal left ideal. By
	Proposition~\ref{proposition:R/I}, $R/I$ is a simple $R$-module. Then
	\[
	\Ann_R(R/L)=\{x\in R:xR\subseteq I\}
	\]
	if a primitive ideal.
\end{proof}

%\begin{remark}
%	Una consecuencia trivial del lema~\ref{lemma:maxprim} es la siguiente: en
%	un anillo unitario, todo ideal a izquierda maximal contiene un ideal
%	primitivo.
%\end{remark}

\begin{exercise}
\label{xca:maximal=>prim}
    Maximal ideals of unitary rings are primitive.  
\end{exercise}

\begin{exercise}
	Prove that every primitive ideal of a commutative ring is maximal.
\end{exercise}

\begin{exercise}
    Prove that $M_n(R)$ is primitive if and only if $R$ is primitive.
\end{exercise}

% Si $P$ es primitivo, entonces $R/P$ es un cuerpo(por ser primitivo y conmutativo) y luego $P$ es maximal

Let us discuss the Jacobson radical and radical rings. 

\begin{definition}

Let $R$ be a ring. The \textbf{Jacobson radical} $J(R)$
is the intersection of all the annihilators of simple left $R$-modules. If $R$ does not
have simple left $R$-modules, then $J(R)=R$. 
\end{definition}

From the definition it follows
that $J(R)$ is an ideal. Moreover, 
	\[
		J(R)=\bigcap\{P:\text{$P$ left primitive ideal}\}.
	\]

If $I$ is an ideal of $R$ and $n\in\N$, $I^n$ is the additive subgroup of $R$ 
generated by the set $\{y_1\dots y_n:y_j\in I\}$. 

\begin{definition}
An ideal $I$ of $R$ is \textbf{nilpotent} 
if $I^n=\{0\}$ for some $n\in\N$.
\end{definition}

Similarly one defines right or left nil ideals. 
Note that an ideal $I$ is nilpotent if and only if there exists $n\in\N$ such that 
$x_1x_2\cdots x_n=0$ for all $x_1,\dots,x_n\in I$.  

\begin{definition}
An element $x$ of a ring is said to be \textbf{nil} (or nilpotent) if $x^n=0$ for some $n\in\N$. 
\end{definition}

\begin{definition}
An ideal $I$ of a ring is said to be \text{nil} if every element of $I$ is nil. 
\end{definition}

Every nilpotent ideal is nil, as $I^n=0$ implies $x^n=0$ for all 
$x\in I$.

\begin{example}
	Let $R=\C[x_1,x_2,\dots]/(x_1,x_2^2,x_3^3,\dots)$. The ideal 
	$I=(x_1,x_2,x_3,\dots)$ is nil in $R$, as it is generated by nilpotent element. However, it is not nilpotente. Indeed, if $I$ is nilpotent, then there exists $k\in\N$ such that 
	$I^k=0$ and hence $x_i^k=0$ for all $i$, a contradiction since 
	$x_{k+1}^k\ne0$. 	
\end{example}

\begin{proposition}
	\label{pro:nilJ}
	Let $R$ be a ring. Then every nil left ideal (resp. right ideal) is contained in $J(R)$.
\end{proposition}

\begin{proof}
	Assume that there is a nil left ideal (resp. right ideal) $I$ such that 
	$I\not\subseteq J(R)$. There exists a simple $R$-module $M$ such that 
	$n=xm\ne 0$ for some $x\in I$ and some $m\in M$. Since $M$ is simple,
	$Rn=M$ and hence there exists $r\in R$ such that 
	\[
	(rx)m=r(xm)=rn=m\quad\text{(resp.
	$(xr)n=x(rn)=xm=n$).}
	\]
	Thus $(rx)^km=m$ (resp. $(xr)^kn=n$) for all 
	$k\geq1$, a contradiction since $rx\in I$ (resp. $xr\in I$) is a nilpotent element. 
\end{proof}

\begin{definition}
Let $R$ be a ring. An element $a\in R$ is said to be 
\textbf{left quasi-regular} if there exists $r\in R$ such that $r+a+ra=0$. Similarly, 
$a$ is said to be \textbf{right quasi-regular} if there exists $r\in R$ such that $a+r+ar=0$. 
\end{definition}

\begin{exercise}
	\label{exercise:circ}
	Let $R$ be a ring. Prove that $R\times R\to R$,
	$(r,s)\mapsto r\circ s=r+s+rs$, is an associative operation with neutral element $0$.
\end{exercise}

\begin{exercise}
	Let $R=\Z/3=\{0,1,2\}$. Compute the multiplication table with respect to the circle 
 	operation given by the previous exercise.  
 	%is then 
% 	\begin{table}[ht]
% 		\centering
% 		\begin{tabular}{c|ccc}
% 			$\circ$ & 0 & 1 & 2\tabularnewline
% 			\hline
% 			0 & 0 & 1 & 2\tabularnewline
% 			1 & 1 & 0 & 2\tabularnewline
% 			2 & 2 & 2 & 2\tabularnewline
% 		\end{tabular}
% 		\caption{The multiplication table of the radical ring $\Z/3$.}
% 	\end{table}
\end{exercise}

If $R$ is unitary, an element $x\in R$ is left quasi-regular (resp. right quasi-regular)
if and only if $1+x$ is left invertible (resp. right invertible). In fact, 
if $r\in R$ is such that $r+x+rx=0$, then $(1+r)(1+x)=1+r+x+rx=1$.
Conversely, if there exists $y\in R$ such that $y(1+x)=1$, then  
\[
(y-1)\circ x=y-1+x+(y-1)x=0.
\]

\begin{example}
	If $x\in R$ is a nilpotent element, then $y=\sum_{n\geq1}x^n\in R$ is quasi-regular. 
	En efecto, si existe $N$ tal que $x^N=0$, la suma que
	define al elemento $y$ es finita y cumple que $y+(-x)+y(-x)=0$.  
\end{example}

\begin{definition}
A left ideal $I$ of $R$ is said to be 
\textbf{left quasi-regular} (resp. right quasi-regular) if every element of $I$ is
left quasi-regular (resp. right quasi-regular). A left ideal 
is said to be \textbf{quasi-regular} if it is left and right quasi-regular. 
\end{definition}

Similarly 
one defines right quasi-regular ideals and quasi-regular ideals. 

\begin{lemma}
	\label{lemma:casiregular}
	Let $I$ be a left ideal of $R$. If $I$ is left quasi-regular, then 
	$I$ is quasi-regular.
\end{lemma}

\begin{proof}
	Let $x\in I$. Let us prove that $x$ is right quasi-regular. Since $I$ is
	left quasi-regular, there exists $r\in R$ such that $r\circ x=r+x+rx=0$. Since 
	$r=-x-rx\in I$, there exists $s\in R$ tal que $s\circ
	r=s+r+sr=0$. Then $s$ is right quasi-regular and  
	\[
	x=0\circ x=(s\circ r)\circ x=s\circ (r\circ x)=s\circ 0=s.\qedhere
	\]
\end{proof}

\index{Lemma!Zorn}
Let $(A,\leq)$ be a \textbf{partially order set}, this means that $A$ is a set together with a 
reflexive, transitive and anti-symmetric binary relation
$R$ en $A\times A$, where $a\leq b$ if and only if $(a,b)\in R$. 
Recall that the relation is reflexive if $a\leq a$ for all $a\in A$, the relation is transitive if 
$a\leq b$ and $b\leq c$ imply that 
$a\leq c$ and the relation is anti-symmetric if $a\leq b$ and $b\leq a$ imply $a=b$.
The elements $a,b\in A$ are said to be \textbf{comparable} if $a\leq b$ or $b\leq
a$. An element $a\in A$ is said to be \textbf{maximal} if 
$c\leq a$ 
for all $c\in A$
that is comparable with $a$. 
An \textbf{upper bound} for a non-empty subset $B\subseteq A$ is an element $d\in
A$ such that $b\leq d$ for all $b\in B$. A \textbf{chain} in $A$ is a subset 
$B$ such that every pair of elements of $B$ are comparable. 
\textbf{Zorn's lemma} states the following property: 
\begin{quote}
If $A$ is a non-empty partially ordered set such that every chain in 
$A$ contains an upper bound in $A$, then $A$ contains a maximal element. 
\end{quote}

Our application of Zorn's lemma:

\begin{lemma}
	\label{lemma:maxreg}
	Let $R$ be a ring and $x\in R$ be an element that is not left quasi-regular Then there
	exists a maximal left ideal $M$ such that 
	$x\not\in M$. Moreover, $R/M$ is a simple $R$-module and  
	$x\not\in\Ann_R(R/M)$.
\end{lemma}

\begin{proof}
	Let $T=\{r+rx:r\in R\}$. A straightforward calculation shows that $T$ is a left ideal of 
	$R$ such that $x\not\in T$ (if $x\in T$, then $r+rx=-x$ for some 
	$r\in R$, a contradiction since $x$ is not left quasi-regular). 

	The only left ideal of $R$ containing 
	$T\cup\{x\}$ is $R$. Indeed, if there exists a left ideal $U$ containing $T$, then 
    $x\not\in U$, since otherwise every $r\in R$ could be written as 
	$r=(r+rx)+r(-x)\in U$. 

	Let $\mathcal{S}$ be the set of proper left ideals of $R$ containing 
	$T$ partially ordered by inclusion. If $\{K_i:i\in I\}$ is a chain in 
	$\mathcal{S}$, then $K=\cup_{i\in I}K_i$ is an upper bound for the chain 
	($K$ is a proper, as $x\not\in K$). Zorn's lemma implies that 
	$\mathcal{S}$ admits a maximal element $M$. Thus $M$
	is a maximal left ideal such that $x\not\in M$. Moreover, $M$ is regular
	since $r+r(-x)\in T\subseteq M$ for all $r\in R$. Therefore $R/M$ is a simple 
	$R$-module by Proposition~\ref{proposition:R/I}. Since $x(x+M)\ne
	0$ (if $x^2\in M$, then  $x\in M$, as $x+x^2\in
	T\subseteq M$), it follows that $x\not\in\Ann_R(R/M)$.
\end{proof}

If $x\in R$ is not left quasi-regular, the lemma implies that there exists 
a simple $R$-module $M$ such $x\not\in\Ann_R(M)$. Thus 
$x\not\in J(R)$.

\begin{theorem}
	\label{thm:casireg_eq}
	Let $R$ be a ring and $x\in R$. The following statements are equivalent: 
	\begin{enumerate}
		\item The left ideal generated by $x$ is quasi-regular.
		\item $Rx$ is quasi-regular.
		\item $x\in J(R)$.
	\end{enumerate}
\end{theorem}

\begin{proof}
	The implication $(1)\implies(2)$ is trivial, as $Rx$ is included in the left ideal 
	generated by $x$.  
	
	We now prove $(2)\implies(3)$. If
	$x\not\in J(R)$, then Lemma~\ref{lemma:maxreg} implies that there exists a simple 
	$R$-module $M$ such that $xm\ne 0$ for some $m\in M$. The simplicity of $M$ implies
	that $R(xm)=M$. Thus there exists $r\in R$ such that $rxm=-m$. There is an element 
	$s\in R$ such that $s+rx+s(rx)=0$ and hence 
	\[
	-m=rxm=(-s-srx)m=-sm+sm=0,
	\]
	a contradiction. 
	
	Finally, to prove $(3)\implies(1)$ it is enough to note that 
	$x$ is left quasi-regular. Thus the left ideal generated by 
	$x$ is quasi-regular by Lemma~\ref{lemma:casiregular}.
\end{proof}

The theorem immediately implies the following corollary. 

\begin{corollary}
	If $R$ is a ring, then $J(R)$ if a quasi-regular ideal that contains every 
	left quasi-regular ideal. 
\end{corollary}

The following result is somewhat what we all had in mind. 

% \begin{proof}
% 	Es consecuencia inmediata del teorema~\ref{thm:casireg_eq}. 
%%	Como \[
%%	J(R)=\bigcap\{\Ann_R(R/I):I\text{ ideal a izquierda maximal y
%%	regular}\}
%%	\]
%%	por el teorema~\ref{thm:J(R)}, 
%%	todo $x\in J(R)$ es casi-regular gracias al
%%	lema~\ref{lemma:K}. Si $I$ es un ideal casi-regular a izquierda, entonces
%%	$I\subseteq J(R)$ por el lema~\ref{lemma:JsupsetCR}.
%\end{proof}

%\begin{lemma}
%%	\label{lemma:maxreg}
%	Sea $R$ un anillo y sea $I\ne R$ un ideal a izquierda regular. Entonces $I$
%	está contenido en algún ideal a izquierda maximal y regular.
%\end{lemma}
%
%\begin{proof}
%	
%\end{proof}

%\begin{lemma}
%	\label{lemma:K}
%	Sean $R$ un anillo y $K=\bigcap\{I:\text{$I$ ideal a izquierda maximal y
%	regular}\}$.  Entonces $K$ es un ideal a izquierda casi-regular.
%\end{lemma}
%
%\begin{proof}
%	Gracias al lema~\ref{lemma:casiregular}, basta ver que $K$ es casi-regular
%	a izquierda.  Sea $a\in K$ y sea $T=\{r+ra:r\in R\}$.  Es claro que $T$ es
%	un ideal a izquierda regular con $e=-a$. Como $a\in K$, existe $s\in R$ tal
%	que $s+a+sa=0$. Entonces, como $-a=s+sa$, 
%	\[
%	r+re=r+r(-a)=r+r(s+sa)=r+rs+rsa\in r+T
%	\]
%	para todo $r\in R$. 
%	Si $T\ne R$, por el lema~\ref{lemma:maxreg} existe un ideal a izquierda
%	$J$, maximal y regular.  Como $a\in K\subseteq J$, $J$ es ideal a izquierda
%	y $r+ra\in T\subseteq J$ para todo $r\in R$, se concluye que $R=J$, una
%	contradicción.  Luego $T=R$ y entonces existe $r\in R$ tal que $r+ra=-a$.
%	Esto implica que $a$ es casi-regular a izquierda. 
%\end{proof}

%\begin{lemma}
%	\label{lemma:JsupsetCR}
%	Sea $R$ un anillo que admite un $R$-módulo simple. Si $I$ es un ideal a
%	izquierda casi-regular a izquierda, $I\subseteq J(R)$.
%\end{lemma}
%
%\begin{proof}
%	Supongamos que $I\not\subseteq J(R)$. Existe entonces un $R$-módulo simple
%	$N$ tal que $IN\ne 0$. Entonces $In\ne 0$ para algún $0\ne n\in N$. Como
%	$I$ es un ideal a izquierda, $In\subseteq N$ es un submódulo no nulo del
%	simple $N$. Luego $In=N$. Existe entonces $x\in I$ tal que $xn=-n$. Como
%	$I$ es casi-regular a izquierda, existe $r\in R$ tal que $r+x+rx=0$.
%	Entonces
%	\[
%		0=0n=(r+x+rx)n=rn+xn+rxn=rn-n-rn=-n,
%	\]
%	una contradicción.
%\end{proof}

\begin{theorem}
	\label{thm:J(R)}
	Let $R$ be a ring such that $J(R)\ne R$. Then 
	\begin{align*}
		J(R)&=\bigcap\{I:\text{$I$ regular maximal left ideal of $R$}\}.
	\end{align*}
\end{theorem}

\begin{proof}
    We only prove the non-trivial inclusion. 
	Let 
	\[
	K=\bigcap\{I:\text{$I$ regular maximal left ideal of $R$}\}.
	\]
	By
	Proposition~\ref{proposition:R/I}, 
	\[
		J(R)=\bigcap\{\Ann_R(R/I):I\text{ regular maximal left ideal of $R$}\}.
	\]
	Let $I$ be a regular maximal left ideal. If $r\in J(R)\subseteq
	\Ann_R(R/I)$, then, since $I$ is regular, there exists $e\in R$ such that
	$r-re\in I$. Since 
	\[
	re+I=r(e+I)=0,
	\]
	$re\in I$ and hence $r\in I$. Thus $J(R)\subseteq K$. 
\end{proof}

\begin{example}
	Each maximal ideals of $\Z$ is of the form $p\Z=\{pm:m\in\Z\}$ for some prime number $p$. 
	Thus $J(\Z)=\cap_p p\Z=\{0\}$.
\end{example}

%\begin{example}
%	Sea $D$ un anillo de división y sea $R=D[x_1,\dots,x_n]$. Si $f\in J(R)$
%	entonces\dots Luego $J(R)=0$. 
%\end{example}



We now review some basic results useful to compute radicals. 

\begin{proposition}
	Let $\{R_i:i\in I\}$ be a family of rings. Then 
	\[
	J\left(\prod_{i\in I}R_i\right)=\prod_{i\in I}J(R_i).
	\]
\end{proposition}

\begin{proof}
	Let $R=\prod_{i\in I}R_i$ and $x=(x_i)_{i\in I}\in R$.  The left ideal 
    $Rx$ is quasi-regular if and only if each left ideal $R_ix_i$
	is quasi-regular in $R_i$, as $x$ is quasi-regular in $R$ if and only if each 
	$x_i$ is quasi-regular in $R_i$. Thus $x\in J(R)$ if and only if $x_i\in
	J(R_i)$ for all $i\in I$.
\end{proof}

For the next result we shall need a lemma.

\begin{lemma}
	\label{lemma:trickJ1}
	Let $R$ be a ring and $x\in R$. 
	If $-x^2$ is a left quasi-regular element, then $x$ también. 
\end{lemma}

\begin{proof}
	Sea $r\in R$ tal que $r+(-x^2)+r(-x^2)=0$ y sea $s=r-x-rx$. Entonces $x$ es
	casi-regular a izquierda pues 
	\begin{align*}
		s+x+sx&=(r-x-rx)+x+(r-x-rx)x\\
		&=r-x-rx+x+rx-x^2-rx^2=r-x^2-rx^2=0.\qedhere 
\end{align*}
\end{proof}

%\begin{lemma}
%	\label{lemma:trickJ2}
%	Sea $R$ un anillo. Entonces $x\in J(R)$ si y sólo si $Rx$ es un ideal a
%	izquierda casi-regular a izquierda.
%\end{lemma}
%
%\begin{proof}
%	Si $x\in J(R)$ entonces $Rx\subseteq J(R)$ y luego todo elemento de $Rx$ es
%	casi-regular a izquierda. Recíprocamente, si $Rx$ es casi-regular a
%	izquierda, $Rx+\Z x$ es un ideal a izquierda de $R$. Si $s=rx+nx\in Rx+\Z
%	x$, entonces $-s^2$ es casi-regular a izquierda (pues $-s^2\in Rx$). Por el
%	lema~\ref{lemma:trickJ1}, $s$ es casi-regular a izquierda; en particular,
%	$x$ es casi-regular a izquierda y luego $x\in J(R)$. 
%\end{proof}

\begin{proposition}
	\label{proposition:J(I)}
	If $I$ is an ideal of $R$, then $J(I)=I\cap J(R)$. 
\end{proposition}

\begin{proof}
	Since $I\cap J(R)$ if an ideal of $I$, if $x\in I\cap J(R)$, then $x$ is
	left quasi-regular in $R$. Let $r\in R$ be such that $r+x+rx=0$. 
	Since $r=-x-rx\in I$, $x$ is left quasi-regular 
	in $I$. Thus $I\cap J(R)\subseteq J(I)$. 

	Let $x\in J(I)$ and $r\in R$. Since $-(rx)^2=(-rxr)x\in
	I(J(I))\subseteq J(I)$, the element $-(rx)^2$ is left quasi-regular a izquierda
	en $I$. Thus $rx$ is left quasi-regular by
	Lemma~\ref{lemma:trickJ1}.
\end{proof}

\begin{definition}
\index{Ring!radical}
A ring $R$ is said to be \textbf{radical} if $J(R)=R$. 
\end{definition}

\begin{example}
	If $R$ is a ring, then $J(R)$ is a radical ring, by Proposition~\ref{proposition:J(I)}.
\end{example}

\begin{example}
	The Jacobson radical of $\Z/8$ is $\{0,2,4,6\}$. 
\end{example}

There are several characterizations of radical rings. 

\begin{theorem}
	\label{theorem:anillo_radical}
	Let $R$ be ring. The following statements are equivalent: 
	\begin{enumerate}
		\item $R$ is radical.
		\item $R$ admits no simple $R$-modules. 
		\item $R$ no tiene ideales a izquierda maximales y regulares.
		\item $R$ no tiene ideales a izquierda primitivos.
		\item Every element of $R$ is quasi-regular. 
		\item $(R,\circ)$ is a group. 
	\end{enumerate}
\end{theorem}

\begin{proof}
	The equivalence $(1)\Longleftrightarrow(5)$ follows from 
	Theorem~\ref{thm:casireg_eq}. 
    
    The equivalence $(5)\Longleftrightarrow(6)$ is left as an exercise. 

	Let us prove that $(1)\implies(2)$. Assume that there exists a simple $R$-module $N$. Since 
	$R=J(R)\subseteq\Ann_R(N)$, $R=\Ann_S(N)$. 
	Hence $RN=\{0\}$, a contradiction to the simplicity of $N$.
	
	To prove $(2)\implies(3)$ we note that for each regular and maximal left ideal 
	$I$, the quotient $R/I$ is a simple $R$-module by
	Proposición~\ref{proposition:R/I}. 
	
	To prove $(3)\implies(4)$ assume that there is a primitive left ideal 
	$I=\Ann_R(M)$, where $M$ is some simple $R$-module. Since $R=J(R)\subseteq I$, it follows that  
    $I=R$, a contradiction to the simplicity of $M$.
    %$RM=\{0\}$.

	Finally we prove $(4)\implies(2)$. If $M$ is a simple $R$-module, then 
	$\Ann_R(M)$ is a primitive left ideal.
\end{proof}

\begin{example}
	Let 
	\[
	A=\left\{\frac{2x}{2y+1}:x,y\in\Z\right\}.
	\]
	Then $A$ is a radical ring, as the inverse of the element $\frac{2x}{2y+1}$
	with respect to the circle operation 
	$\circ$ is 
	\[
	\left(\frac{2x}{2y+1}\right)'=\frac{-2x}{2(x+y)+1}.
	\]
\end{example}

\begin{definition}
\index{Ring!nil}
A ring $R$ is said to be \textbf{nil} if for every $x\in R$ there
exists $n=n(x)$ such that $x^n=0$. 
\end{definition}

\begin{exercise}
    Prove that a nil ring is a radical ring. 
\end{exercise}

\begin{exercise}
    Let $\R[X]$ be the ring of power series with real coefficients. Prove that the ideal 
    $X\R[X]$ consisting of power series with zero constant term is a radical ring
    that is not nil. 
\end{exercise}


\begin{theorem}
	\label{thm:Jnilpotente}
	If $R$ is a left artinian ring, then $J(R)$ is nilpotent. 
\end{theorem}

\begin{proof}
	Let $J=J(R)$. Since $R$ is a left artinian ring, the sequence 
	$(J^m)_{m\in\N}$ of left ideals stabilizes. There exists 
	$k\in\N$ such that $J^k=J^l$ for all $l\geq k$. We claim that $J^k=\{0\}$. If
	$J^k\ne\{0\}$ let $\mathcal{S}$ the set of left ideals 
	$I$ such that $J^kI\ne\{0\}$. Since 
	\[
	J^kJ^k=J^{2k}=J^k\ne\{0\},
	\]
	the set $\mathcal{S}$ is non-empty. 
	Since $R$ is left artinian, $\mathcal{S}$ has a minimal element $I_0$. Since $J^kI_0\ne\{0\}$, let $x\in
	I_0\setminus\{0\}$ be such that $J^kx\ne\{0\}$. Moreover, $J^kx$ is a left ideal of $R$ 
	contained in $I_0$ and such that $J^kx\in\mathcal{S}$, as 
	$J^k(J^kx)=J^{2k}x=J^kx\ne\{0\}$. The minimality of $I_0$ implies that, $J^kx=I_0$. In particular, 
	there exists $r\in J^k\subseteq J(R)$ such that $rx=x$. Since $-r\in
	J(R)$ is left quasi-regular, there exists $s\in R$ such that $s-r-sr=0$.
	Thus 
	\[
		x=rx=(s-sr)x=sx-s(rx)=sx-sx=0,
	\]
	a contradiction.
\end{proof}

\begin{corollary}
	Let $R$ be a left artinian ring. Each nil left ideal is nilpotent and 
	$J(R)$ is the unique maximal nilpotent ideal of $R$. 
\end{corollary}

\begin{proof}
	Let $L$ be a nil left ideal of $R$. By Proposition~\ref{pro:nilJ}, $L$
	is contained in $J(R)$. Thus $L$ is nilpotent, as $J(R)$ 
	is nilpotent by Theorem~\ref{thm:Jnilpotente}. 
\end{proof}

\begin{theorem}
	Let $R$ be a ring and $n\in\N$. Then $J(M_n(R))=M_n(J(R))$. 
\end{theorem}

\begin{proof}
	We first prove that $J(M_n(R))\subseteq M_n(J(R))$. 
	If $J(R)=R$, the theorem is clear. Let us assume that $J(R)\ne R$ and let  
	$J=J(R)$. 
	If $M$ is a simple $R$-module, then $M^n$ is a simple $M_n(R)$-module with the usual multiplication. 
	Let $x=(x_{ij})\in J(M_n(R))$ and $m_1,\dots,m_n\in M$. Then
	\[
		x\colvec{3}{m_1}{\vdots}{m_n}=0.
	\]
	In particular, $x_{ij}\in\Ann_R(M)$ for all $i,j\in\{1,\dots,n\}$. Hence 
	$x\in M_n(J)$. 

	We now prove that $M_n(J)\subseteq J(M_n(R))$. Let 
	\[
		J_1=\begin{pmatrix}
			J & 0 & \cdots & 0\\
			J & 0 & \cdots & 0\\
			\vdots & \vdots & \ddots & \vdots\\
			J & 0 & \cdots & 0
		\end{pmatrix}
		\quad\text{and}\quad
		x=\begin{pmatrix}
			x_1 & 0 & \cdots & 0\\
			x_2 & 0 & \cdots & 0\\
			\vdots & \vdots & \ddots & \vdots\\
			x_n & 0 & \cdots & 0
		\end{pmatrix}\in J_1.
	\]
	Since $x_1$ es quasi-regular, there exists $y_1\in R$ such that $x_1+y_1+x_1y_1=0$.
	If
	\[
		y=\begin{pmatrix}
			y_1 & 0 & \cdots & 0\\
			0 & 0 & \cdots & 0\\
			\vdots & \vdots & \ddots & \vdots\\
			0 & 0 & \cdots & 0
		\end{pmatrix}, 
	\]
	then $u=x+y+xy$ is lower triangular, as  
	\[
		u=\begin{pmatrix}
			0 & 0 & \cdots & 0\\
			x_2y_1 & 0 & \cdots & 0\\
			x_3y_1 & 0 & \cdots & 0\\
			\vdots & \vdots & \ddots & \vdots\\
			x_ny_1 & 0 & \cdots & 0
		\end{pmatrix}.
	\]
	Since  
	$u^n=0$, the element
	\[
	v=-u+u^2-u^3+\cdots+(-1)^{n-1} u^{n-1}
	\]
	is such that 
	$u+v+uv=0$. Thus $x$ is right quasi-regular, as  
	\begin{align*}
		x+(y+v+yv)+x(y+v+yv)&=0,
	\end{align*}
	and therefore $J_1$ is right quasi-regular. Similarly one proves that 
	each $J_i$ is right quasi-regular and hence $J_i\subseteq J(M_n(R))$ for all 
	$i\in\{1,\dots,n\}$. In conclusion, 
	\[
	J_1+\cdots+J_n\subseteq J(M_n(R))
	\]
	and therefore $M_n(J)\subseteq J(M_n(R))$.
\end{proof}

\begin{exercise}
	Let $R$ be a unitary ring. Then  
	\[
	J(R)=\bigcap\{M:\text{$M$ is a left maximal ideal}\}.
	\]
\end{exercise}

\begin{exercise}
	Let $R$ be a unitary ring. The
	following statements are equivalent: 
	\begin{enumerate}
		\item $x\in J(R)$.
		\item $xM=0$ for all simple $R$-module $M$.
		\item $x\in P$ for all primitive left ideal $P$.
		\item $1+rx$ is invertible for all $r\in R$.
		\item $1+\sum_{i=1}^n r_ixs_i$ is invertible for all $n\in\N$ and all $r_i,s_i\in R$.
		\item $x$ belongs to every left maximal ideal maximal. 
	\end{enumerate}
\end{exercise}

The following exercises are optional. They somewhat show a recent new application of radical rings
to solutions of the celebrated Yang--Baxter equation. 

\begin{exercise}
A pair $(X,r)$ is a \textbf{solution} to the 
Yang--Baxter equation if $X$ is a set and
$r\colon X\times X\to X\times X$ is a bijective map such that  
\[
	(r\times\id)\circ (\id\times r)\circ (r\times\id)
	=(\id\times r)\circ (r\times\id)\circ (\id\times r)
\]
The solution $(X,r)$ is said to be \textbf{involutive} 
if $r^2=\id$. By convention we write 
\[
	r(x,y)=(\sigma_x(y),\tau_y(x)).
\]
The solution $(X,r)$ is said to be \textbf{non-degenerate}  
$\sigma_x\colon X\to X$ and 
$\tau_x\colon X\to X$ are bijective for all $x\in X$.

\begin{enumerate}
    \item Let $X$ be a set and $\sigma\colon X\to X$ be a bijective map. Prove that  
          the pair $(X,r)$, where 
          $r(x,y)=(\sigma(y),\sigma^{-1}(x))$, is an involutive non-degenerate solution. 
\end{enumerate}
Let $R$ be a radical ring. For $x,y\in R$ let 
\begin{align*}
	&\lambda_x(y)=-x+x\circ y=xy+y,\\
	&\mu_y(x)=\lambda_x(y)'\circ x\circ y=(xy+y)'x+x
\end{align*}
Prove the following statements:
\begin{enumerate}
    \setcounter{enumi}{1}
		\item $\lambda\colon (R,\circ)\to\Aut(R,+)$, $x\mapsto
			\lambda_x$, is a group homomorphism.
		\item $\mu\colon (R,\circ)\to\Aut(R,+)$, $y\mapsto\mu_y$,
    		is a group antihomomorphism.
	    \item The map 
    	\[
	        r\colon R\times R\to R\times R,\quad
	        r(x,y)=(\lambda_x(y),\mu_y(x)),
	    \]
	is an involutive non-degenerate solution. 
\end{enumerate}
\end{exercise}

%\begin{exercise}
%	Sea $A$ un anillo radical. Para $a,b\in A$ se define 
%	\[
%		\mu_b(a)=\lambda_a(b)'\circ a\circ b=(ab+b)'a+a.
%	\]
%	Demuestre que la función $\mu\colon (A,\circ)\to\Aut(A,+)$,
%	$b\mapsto\mu_b$, está bien definida y es un antimorfismo de grupos.
%\end{exercise}

\begin{exercise}
    If $D$ is a division ring and $R=D[X_1,\dots,X_n]$, then
    $J(R)=\{0\}$. 
%     Como las unidades de $R$ son los elementos no nulos de $D$,
% 	$J(R)$ es un ideal de $D$. Como $D$ es simple, $J(R)\in\{0,D\}$. Si
% 	$J(R)=D$, entonces existe  $f\in R$ tal que $-1+f+(-1)f=0$ y luego $-1=0$,
% 	una contradicción. Luego $J(R)=0$.
\end{exercise}


\begin{example}
\index{Ring!local}
    A commutative and unitary ring $R$ is \textbf{local} if it contains
    only one maximal ideal. 
	If $R$ is a local ring and $M$ be its maximal ideal, then $J(R)=M$. Some particular cases: 
	\begin{enumerate}
		\item If $K$ is a field and $R=K\left[ [X] \right]$, then $J(R)=(X)$. 
		\item If $p$ is a prime number and $R=\Z/p^n$, then $J(R)=(p)$. 
	\end{enumerate}
\end{example}

We finish the discussion on the Jacobson radical with 
some results in the case of unitary algebras. 

\begin{theorem}
	Let $A$ be a $K$-algebra and $I$ be a subset of $A$. Then $I$ is 
	a left regular maximal ideal ideal of the algebra $A$ if and only if $I$ is 
	a left regular maximal ideal of the ring $A$.
\end{theorem}

\begin{proof}
	Sea $I$ un ideal a izquierda maximal y regular del anillo $A$.  Queremos
	demostrar que $\lambda I\subseteq I$ para todo $\lambda\in K$. Si suponemos
	que $\lambda I\not\subseteq I$ para algún $\lambda$, entonces $I+\lambda I$
	es un ideal a izquierda del anillo $A$ que contiene a $I$ pues
	\[
	a(I+\lambda I)=aI+a(\lambda I)\subseteq I+\lambda (aI)\subseteq I+\lambda I.
	\]
	Como $I$ es maximal, $I+\lambda I=A$. Por la regularidad de $I$, existe $e\in R$ tal que
	$a-ae\in I$ para todo $a\in A$. Si escribimos $e=x+\lambda y$ para $x,y\in
	I$, entonces
	\[
		e^2=e(x+\lambda y)=ex+e(\lambda y)=ex+(\lambda e)y\in I.
	\]
	Como $e^2-e\in I$ y $e^2\in I$, se concluye que $e\in I$. Luego $A=I$ pues
	$a-ae\in I$ para todo $a\in A$, una contradicción.

	Recíprocamente, si $I$ es un ideal a izquierda maximal y regular del
	álgebra $A$, entonces $I$ es ideal a izquierda regular del anillo $A$.
	Falta ver entonces que $I$ es maximal. Por el
	ejercicio~\ref{exa:Zorn:regular} sabemos que existe un ideal a izquierda
	maximal $L$ del anillo $A$ que contiene a $I$. Como $L$ es regular, la
	implicación demostrada nos dice que $L$ es un ideal a izquierda maximal y
	regular del anillo $A$. Luego $L=I$ por la maximalidad de $I$.
\end{proof}

\begin{corollary}
	Sea $A$ un álgebra. El radical de Jacobson del anillo $A$ coincide con el
	radical de Jacobson del álgebra $A$.
\end{corollary}

\begin{proof}
	Es consecuencia del teorema anterior y de que el radical de Jacobson es la
	intersección de los ideales a izquierda maximales y regulares.
\end{proof}

\begin{lemma}
	\label{lemma:algebraico=nil}
	Sea $A$ un álgebra unitaria y sea $x\in J(A)$. 
	Entonces $x$ es algebraico si y sólo si $x$ es nil.
\end{lemma}

\begin{proof}
	Demostremos la implicación no trivial. Como $x$ es algebraico, 
	existen $a_0,\dots,a_n\in K$ no todos cero tales que
	\[
		a_0+a_1x+\cdots+a_nx^n=0.
	\]
	Sea $r$ el menor entero tal que $a_r\ne 0$. Podemos escribir entonces
	\[
		x^r(1+b_1x+\cdots+b_mx^m)=0,
	\]
	donde $b_1,\dots,b_m\in K$.  Como $1+b_1x+\cdots+b_mx^m$ es una unidad por
	el corolario~\ref{cor:Jcon1}, entonces $x^r=0$.
\end{proof}

Como aplicación tenemos el siguiente resultado:

\begin{theorem}
	\label{thm:algebraica=>Jnil}
	Si $A$ es un álgebra algebraica, $J(A)$ es el mayor ideal nil de $A$.
\end{theorem}

\begin{proof}
	Por el lema~\ref{lemma:algebraico=nil}, $J(A)$ es un ideal nil. Por la
	propoisición~\ref{pro:nilJ}, $J(A)$ es el mayor ideal nil de $A$.
\end{proof}

\begin{theorem}[Amitsur]
	\label{thm:Amitsur}
	Si $A$ es una $K$-álgebra unitaria tal que $\dim_KA<|K|$ (como cardinales),
	entonces $J(A)$ es el mayor ideal nil de $A$.
\end{theorem}

\begin{proof}
	Si $K$ es un cuerpo finito, entonces $A$ es un álgebra de dimensión finita.
	Como entonces $A$ es algebraica, $J(A)$ es un ideal nil por el
	teorema~\ref{thm:algebraica=>Jnil}.

	Supongamos entonces que $K$ es infinito. Sea $a\in J(A)$. El
	corolario~\ref{cor:Jcon1} implica que todo elemento de la forma
	$1-\lambda^{-1}a$, $\lambda\in K\setminus\{0\}$, es inversible. Entonces 
	\[
		a-\lambda=-\lambda(1-\lambda^{-1}a)
	\]
	es inversible para todo $\lambda\in K\setminus\{0\}$. Sea
	$S=\{(a-\lambda)^{-1}:\lambda\in K\setminus\{0\}\}$. Como
	\[
	(a-\lambda)^{-1}=(a-\mu)^{-1}\Longleftrightarrow\lambda=\mu,
	\]
	entonces $|S|=|K\setminus\{0\}|=|K|>\dim_KA$. Como entonces $S$ 
	es linealmente dependiente, 
	existen escalares no
	nulos $\beta_1,\dots,\beta_n\in K$ y elementos distintos $\lambda_1,\dots,\lambda_n\in K$ tales que
	\begin{equation}
		\label{eq:Amitsur}
		\sum_{i=1}^n \beta_i(a-\lambda_i)^{-1}=0.
	\end{equation}
	Si multiplicamos~\eqref{eq:Amitsur} por $\prod_{i=1}^n(a-\lambda_i)$, obtenemos
	\[
		\sum_{i=1}^n\beta_i\prod_{j\ne i}(a-\lambda_j)=0.
	\]
	Afirmamos que $a$ es algebraico sobre $K$. En efecto, 
	\[
		f=\sum_{i=1}^n\beta_i\prod_{j\ne i}(X-\lambda_j)
	\]
	es no nulo pues
	$f(\lambda_1)=\beta_1(\lambda_1-\lambda_2)\cdots(\lambda_1-\lambda_n)\ne0$
	y cumple que $f(a)=0$. Como $a\in J(A)$ es algebraico, $a$ es nil por el
	lema~\ref{lemma:algebraico=nil}.
\end{proof}

Amitsur's theorem implies the following result. 

\begin{corollary}
	Sea $K$ un cuerpo no numerable y $A$ una $K$-álgebra con base numerable.
	Entonces $J(A)$ es el mayor ideal nil de $A$.
\end{corollary}

% \begin{proof}
% 	Es consecuencia del teorema de Amitsur pues $\dim_KA<|K|$. 
% \end{proof}

\begin{openproblem}[Jacobson--Herstein]
\label{prob:Jacobson}
\index{Jacobson conjecture}
\index{Jacobson--Herstein conjecture}
Let $R$ be a noetherian ring. Is then 
\[
\bigcap_{n\geq1}J(R)^n=\{0\}?
\]
\end{openproblem}

Open problem \ref{prob:Jacobson} was originally formulated by Jacobson in 1956 \cite{MR0222106} 
for one-sided noetherian rings. In 1965 Herstein \cite{MR188253} found a counterexample
and reformulated the conjecture as it appears here. 

%%% exercise

The following problem is maybe the most important open 
problem in non-commutative ring theory. 

\begin{openproblem}[K\"othe]
\label{prob:Koethe}
\index{K\"othe conjecture}
Let $R$ be a ring. Is the sum 
of two arbitrary nil left ideals of $R$ is nil?
\end{openproblem}

Open problem~\ref{prob:Koethe} is the well-known K\"othe's conjecture. 
The conjecture was first formulated in 1930, see \cite{MR1545158}. It is known to be true
in several cases. In full generality, the problem is still open. In~\cite{MR306251} 
Krempa proved that
the following statements are equivalent:
\begin{enumerate}
    \item K\"othe's conjecture is true.  
    \item If $R$ is a nil ring, then $R[X]$ is a radical ring. 
    \item If $R$ is a nil ring, then $M_2(R)$ is a nil ring. 
    \item Let $n\geq2$. If $R$ is a nil ring, then $M_n(R)$ is a nil ring. 
\end{enumerate}

In 1956 Amitsur formulated the following conjecture, see for example
\cite{MR0347873}: If $R$ is a nil ring, then $R[X]$ is a nil ring. In~\cite{MR1793911} 
Smoktunowicz found a counterexample to Amitsur's conjecture. 
This counterexample suggests that K\"othe's conjecture might be false. 
A simplification of Smoktunowicz's example
appears in~\cite{MR3169522}. See \cite{MR1879880,MR2275597} for more
information on K\"othe's conjecture and related topics. 
