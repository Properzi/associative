\RequirePackage{amsmath} 

\documentclass[graybox,envcountsect]{svmono}

\usepackage[T1]{fontenc}
\usepackage[utf8]{inputenc}

\usepackage{amsmath}
\usepackage[notref,notcite]{showkeys}
\usepackage{float}
\usepackage{amssymb}
\usepackage{amstext}
\usepackage{mathtools}
\usepackage{xcolor} 
\usepackage{centernot}
\usepackage{listings}
\usepackage{multicol}
\usepackage{mathptmx}
%\usepackage{newtxtext,newtxmath}
%\usepackage{txfonts}
\usepackage{datetime}
\usepackage{tikz-cd}

\usepackage{helvet}
\usepackage{courier}
\usepackage{type1cm}         
\usepackage{makeidx}        
\usepackage{graphicx}        
\usepackage{multicol}        
\usepackage[all]{xy}
\usepackage{hyperref} 
%\usepackage{tikz-cd}

\usepackage[small,bf]{caption}

\usepackage{tikz}
\usetikzlibrary{braids}
	
\usepackage[bottom]{footmisc}

% for QED
\let\proof\relax\let\endproof\relax
\usepackage{amsthm}

\overfullrule=1mm

%%% for Spanish
% \def\abstractname{Resumen}%
% \def\ackname{Agradecimientos}%
% \def\andname{y}%
% \def\bibname{Referencias}%
% \def\lastandname{, y}%
% \def\appendixname{Apéndice}%
% \def\chaptername{Capítulo}%
% \def\claimname{Afirmación}%
% \def\conjecturename{Conjetura}%
% \def\contentsname{Contenidos}%
% \def\corollaryname{Corolario}%
% \def\definitionname{Definici\'on}%
% \def\emailname{e-mail}%
% \def\examplename{Ejemplo}%
% \def\examplesname{Ejemplos}%
% \def\exercisename{Ejercicio}%
\def\figurename{Figure}%
% \def\forewordname{Foreword}%
% \def\keywordname{{\bf Palabras clave:}}%
% \def\indexname{Índice}%
% \def\lemmaname{Lema}%
% \def\listfigurename{Figuras}%
% \def\listtablename{Tablas}%
% \def\notename{Nota}%
% \def\partname{Parte}%
% \def\prefacename{Prefacio}%
\def\problemname{Open problem}%
% \def\proofname{Demostración}%
% \def\propertyname{Propiedad}%
% \def\propositionname{Proposici\'on}%
% \def\questionname{Pregunta}%
% \def\refname{Referencias}%
% \def\remarkname{Observación}%
% \def\seename{see}%
% \def\solutionname{Solución}%
% \def\tablename{Tabla}%
% \def\theoremname{Teorema}
\def\notationname{Notation}
\def\stepsname{Algorithm}
% \def\conventionname{Convención}

% change numbers 
\let\remark\relax
\let\theorem\relax
\let\lemma\relax
\let\definition\relax
\let\proposition\relax
\let\corollary\relax
\let\exercise\relax
\let\example\relax
\let\conjecture\relax
\spnewtheorem{theorem}{\theoremname}[section]{\bfseries}{\itshape}
\renewcommand\thetheorem{\thesection.\arabic{theorem}}
\spnewtheorem{lemma}[theorem]{\lemmaname}{\bfseries}{\itshape}
\spnewtheorem{definition}[theorem]{\definitionname}{\bfseries}{\upshape}
\spnewtheorem{proposition}[theorem]{\propositionname}{\bfseries}{\itshape}
\spnewtheorem{corollary}[theorem]{\corollaryname}{\bfseries}{\itshape}
\spnewtheorem{exercise}[theorem]{\exercisename}{\bfseries}{\upshape}
\spnewtheorem{example}[theorem]{\examplename}{\bfseries}{\upshape}
\spnewtheorem{examples}[theorem]{\examplesname}{\bfseries}{\upshape}
\spnewtheorem{remark}[theorem]{\remarkname}{}{\upshape}
\spnewtheorem{conjecture}[theorem]{\conjecturename}{\bfseries}{\upshape}
\spnewtheorem{notation}[theorem]{\notationname}{\bfseries}{\upshape}
\spnewtheorem{steps}[theorem]{\stepsname}{\bfseries}{\upshape}
\spnewtheorem{convention}[theorem]{\conventionname}{\bfseries}{\upshape}

% para enumerar
\renewcommand{\labelenumi}{\textbf{\arabic{enumi})}}

\setcounter{secnumdepth}{1}

\makeindex             

\renewcommand{\I}{\operatorname{I}}
\newcommand{\II}{\operatorname{II}}

\newcommand{\GAP}{\textsf{GAP}}
\newcommand{\FK}{\mathcal{E}}
\newcommand{\ad}[1]{\operatorname{ad}\,#1}

\newcommand{\N}{\mathbb{N}}
\newcommand{\Q}{\mathbb{Q}}
\newcommand{\Z}{\mathbb{Z}}
\newcommand{\F}{\mathbb{F}}
\newcommand{\R}{\mathbb{R}}
\newcommand{\C}{\mathbb{C}}
\renewcommand{\H}{\mathbb{H}}
\newcommand{\A}{\mathbb{A}}
\newcommand{\K}{\mathbb{K}}
\newcommand{\T}{\mathbb{T}}
\renewcommand{\D}{\mathbb{D}}
\newcommand{\B}{\mathbb{B}}
\newcommand{\Fun}{\operatorname{Fun}}
\newcommand{\mpl}{\operatorname{mpl}}
\newcommand{\cL}{\mathcal{L}}
\newcommand{\cE}{\mathcal{E}}
\newcommand{\cH}{\mathcal{H}}

\newcommand{\GF}{\mathsf{GF}}
\newcommand{\MAX}{\operatorname{MAX}}
\newcommand{\MIN}{\operatorname{MIN}}
\newcommand{\cf}{\operatorname{cf}}
\newcommand{\cl}{\operatorname{cl}}
\newcommand{\cd}{\operatorname{cd}}
\newcommand{\bL}{\mathbf{L}}
\newcommand{\bP}{\mathbf{P}}

\newcommand{\Nil}{\operatorname{Nil}}
\newcommand{\rad}{\operatorname{rad}}
\newcommand{\rank}{\operatorname{rank}}

\newcommand{\Aff}{\mathrm{Aff}}
\newcommand{\Ann}{\operatorname{Ann}}
\newcommand{\Der}{\operatorname{Der}}
\newcommand{\Core}{\operatorname{Core}}
\newcommand{\Soc}{\operatorname{Soc}}
\newcommand{\Fix}{\operatorname{Fix}}
\newcommand{\Rad}{\mathrm{rad}}
\newcommand{\Inn}{\mathrm{Inn}}
\newcommand{\dist}{\mathrm{dist}}
\newcommand{\Out}{\mathrm{Out}}
\newcommand{\Ext}{\mathrm{Ext}}
\newcommand{\Img}{\mathrm{im}}
\newcommand{\Hol}{\operatorname{Hol}}
\newcommand{\Hom}{\operatorname{Hom}}
\newcommand{\Alg}{\operatorname{Alg}}
\newcommand{\Bil}{\operatorname{Bil}}
\newcommand{\op}{\operatorname{op}}
\newcommand{\gr}{\operatorname{gr}}
\newcommand{\Syl}{\mathrm{Syl}}
\newcommand{\id}{\operatorname{id}}
\newcommand{\Aut}{\operatorname{Aut}}
\newcommand{\End}{\operatorname{End}}
\newcommand{\Irr}{\operatorname{Irr}}
\newcommand{\Alt}{\mathbb{A}}
\newcommand{\Sym}{\mathbb{S}}
\newcommand{\lcm}{\mathrm{mcm}}
\newcommand{\diag}{\operatorname{diag}}
\newcommand{\spec}{\operatorname{Spec}}
\newcommand{\supp}{\operatorname{supp}}
\newcommand{\trace}{\operatorname{traza}}
\newcommand{\sgn}{\operatorname{signo}}

\newcommand{\inner}{\operatorname{inn}}
\newcommand{\ext}{\operatorname{ext}}
\newcommand{\im}{\operatorname{im}}
\newcommand{\Ret}{\operatorname{Ret}}

\newcommand{\GL}{\mathbf{GL}}
\newcommand{\SL}{\mathbf{SL}}
\newcommand{\PSL}{\mathbf{PSL}}
\newcommand{\PGL}{\mathbf{PGL}}

\newcommand{\legendre}[2]{\left(\frac{#1}{#2}\right)}

%\newcommand{\char}{\operatorname{char}}

% multiset
\def\multiset#1#2{\ensuremath{\left(\kern-.3em\left(\genfrac{}{}{0pt}{}{#1}{#2}\right)\kern-.3em\right)}}

% column vector
\newcount\colveccount
\newcommand*\colvec[1]{
\global\colveccount#1
\begin{pmatrix}
	\colvecnext
	}
	\def\colvecnext#1{
	#1
	\global\advance\colveccount-1
	\ifnum\colveccount>0
	\\
	\expandafter\colvecnext
	\else
\end{pmatrix}
\fi
}

% numero como secciones
\renewcommand{\thesection}{\arabic{chapter}}
\renewcommand{\thesubsection}{\Alph{section}}

% To remove Springer from the title page
\usepackage{etoolbox}
\makeatletter
\patchcmd{\@maketitle}{{\Large Springer\par}}{}{}{}
\makeatother

\begin{document}
 
\lstset{language=GAP,
  showstringspaces=false,
  xleftmargin=0.6cm,
  xrightmargin=0.6cm,
  basicstyle=\small\ttfamily,
  frame=single,
  framerule=0pt,
}


\author{Leandro Vendramin}
\title{Associative algebras}
\subtitle{Notes}
\maketitle

\frontmatter

%\include{dedic}
%\include{foreword}
%\preface

The notes correspond to the master  
course \emph{Associative Algebra} of the 
Vrije Universiteit Brussel, 
Faculty of Sciences, 
Department of Mathematics and Data Sciences. The course
is divided into thirteen two-hours lectures. 

The material is heavility based on \cite{MR3308118}, \cite{MR1449137} and 
\cite{MR798076}. 

%Thanks go to Arne van Antwerpen, Luca Descheemaeker, Lucas Simons
%and Geoffrey Jassens. 

This version 
was compiled on \today~at~\currenttime.

\bigskip
\begin{flushright}
Leandro Vendramin\\Brussels, Belgium\par
\end{flushright}

%\include{acknow}

\tableofcontents

%\include{acronym}

\mainmatter

\chapter{Semisimple algebras}

Let $K$ be a field. A vector space $A$ (over $K$) is an \textbf{algebra} if it has an associative multiplication
$A\times A\to A$, $(a,b)\mapsto ab$, such that 
$(\lambda a+\mu b)c=\lambda(ac)+\mu(bc)$ and  
$a(\lambda b+\mu c)=\lambda(ab)+\mu(ac)$ 
for all $a,b,c\in A$ and $\lambda,\mu\in K$. Sometimes we will consider algebras with identities, i.e. 
an element $1_A\in A$ such that $1_Aa=a1_A=a$
for all $a\in A$. The identity will be usually denoted by $1$. 

\index{Algebra!conmutative}
An algebra $A$ is \textbf{conmutative} if $ab=ba$ for all $a,b\in A$. 

\index{Algebra!dimension}
The \textbf{dimension} of an algebra $A$ is the dimension of $A$ as a vector space. This is why we want to consider algebras, as 
they are linear version of rings. Quite often our arguments will use the dimension of the underying vector space.  

\begin{example}
	Every field $K$ an algebra over $K$. 
\end{example}

\begin{example}
	If $K$ is a field, then $K[X]$ is an algebra over $K$. 
\end{example}

Similarly, the polynomial ring $K[X,Y]$ and the ring $K[[X]]$ of power series are examples of algebrs over $K$. 

\begin{example}
	If $A$ is an algebra, then  $M_n(A)$ is an algebra. 
\end{example}

\begin{example}
    The set of continuous maps $[0,1]\to\R$ is a real algebra with the usual point-wise operations $(f+g)(x)=f(x)+g(x)$ and $(fg)(x)=f(x)g(x)$. 
\end{example}

\index{Homomorphism!of algebrass}
An algebra \textbf{homomorphism} is a ring homomorphism $f\colon A\to B$ that is also a linear map. The complex conjugation map  
$\C\to \C$, $z\mapsto\overline{z}$, is a ring homomorphism that is not an algebra homomorphism over $\C$. 

\begin{definition}
	\index{Algebra!ideal}
	An \textbf{ideal} of an algebra is an ideal of the underlying ring that is also a subspace.
\end{definition}

Similarly one defines left and right ideals of an algebra.

If $A$ is an algebra, then every left ideal of the ring $A$ is a left ideal of the algebra $A$. Indeed, if $L$ is a left ideal of $A$ 
and $\lambda\in K$ and $x\in L$, then 
\[
	\lambda x=\lambda (1_Ax)=(\lambda 1_A)x.
\]
Since $\lambda 1_A\in A$, it follows that  $\lambda L=(\lambda
1_A)L\subseteq L$. Similarly, every right ideal of the ring $A$ is a right ideal of the algebra $A$.

% \begin{exercise}
% 	Demuestre que si $A$ es un álgebra, entonces todo ideal a derecha
% 	del anillo $A$ es un ideal a derecha del álgebra $A$.
% \end{exercise}

If $A$ is an algebra and $I$ is an ideal of $A$, then the quotient ring $A/I$ has a unique algebra
structure such that the canonical map  
$A\to A/I$, $a\mapsto a+I$, is an algebra homomorphism. 

\begin{example}
Let $n\in\N$. Then $K[X]/(X^n)$ is a finite-dimensional algebra. It is the \textbf{truncated polynomial algebra}.  
\end{example}

\index{Algebraic element}
\index{Algebra!algebraic}
Let $A$ be an algebra over the field $K$. An element $a\in A$ is 
\textbf{algebraic} over $K$ if there exists a non-zero polynomial $f\in K[X]$
such that $f(a)=0$. If every element of $A$ is algebraic, then $A$ is said to be \text{algebraic}. As an example, in 
the algebra $\R$ over $\Q$, the element $\sqrt{2}$ is algebraic, as $\sqrt{2}$ is a root of the polynomial $X^2-2\in\Q[X]$. A famous theorem of Lindemann proves that   
$\pi$ is not algebraic over $\Q$. Every element of the real algebra $\R$ is algebraic.

\begin{proposition}
	\label{lem:algebraic}
	Every finite-dimensional algebra is algebraic.
\end{proposition}

\begin{proof}
   Let $A$ be an algebra with $\dim A=n$ and let $a\in A$. Since  
	$\{1,a,a^2,\dots,a^n\}$ has $n+1$ elements, it is a linearly dependent set. Thus there exists 
	a non-zero polynomial $f\in K[X]$ such that $f(a)=0$.
\end{proof}

Let $A$ be a finite-dimensional algebra. If $M$ is a module over $A$, then $M$ is a vector space with  
\[
\lambda m=(\lambda 1_A)\cdot m, 
\]
where 
$\lambda\in K$ and $m\in M$. 
Moreover, $M$ is finitely generated if and only if $M$ is finite-dimensional.  

Trabajemos con $A$-módulos finitamente generados. 

\index{Representación!regular de un álgebra}
Observemos que $A$ es un $A$-módulo con la multiplicación a izquierda, es decir $a\cdot b=ab$, $a,b\in A$.
Este módulo se conoce como la \textbf{representación regular} de $A$. 

\begin{definition}
\index{Módulo!simple}
Diremos que un $A$-módulo $M$ es
\textbf{simple} si $M\ne\{0\}$ y los únicos submodulos de $M$ son $\{0\}$ y $M$. 
\end{definition}

\begin{definition}
\index{Módulo!semisimple}
Diremos que $M$ es \textbf{semisimple} si $M\ne\{0\}$ y además $M$
es suma directa de finitos submódulos simples. 
\end{definition}

La suma directa de módulos semisimples es semisimple. 

\begin{lemma}[Schur]
\index{Lema!de Schur}
Si $S$ y $T$ son $A$-módulos simples y $f\colon S\to T$ es un morfismo no nulo, entonces
$f$ es un isomorfismo. 	
\end{lemma}

\begin{proof}
Como $f\ne 0$, $\ker f$ es un submódulo propio de $S$. Como $S$ es simple, entonces $\ker f=\{0\}$. Similarmente $f(S)$ 
es un submódulo no nulo de $T$ y luego $f(S)=T$ por la simplicidad de $T$. 	
\end{proof}

\begin{proposition}
    Si $A$ es un álgebra de dimensión finita y $S$ es un módulo simple 
    entonces $S$ es de dimensión finita.
\end{proposition}

\begin{proof}
    Sea $s\in S\setminus\{0\}$. Como $S$ es simple, $\varphi\colon A\to S$, $a\mapsto a\cdot s$, es
    un epimorfismo. En particular, $A/\ker\varphi\simeq S$ y luego $\dim S=\dim (A/\ker\varphi)\leq \dim A$. 
\end{proof}

Veamos una caracterización de la semisimplicidad.
 
\begin{proposition}
\label{pro:semisimple}
	Sea $M$ un $A$-módulo de dimensión finita. Las siguientes afirmaciones son equivalentes:
	\begin{enumerate}
		\item $M$ es semisimple.
		\item $M=\sum_{i=1}^k S_i$, donde los $S_i$ son submódulos simples de $M$.
		\item Si $S$ es un submódulo de $M$, existe un submódulo $T$ de $M$ tal que $M=S\oplus T$.    
	\end{enumerate}
\end{proposition}

\begin{proof}
	Demostremos que $(2)\implies(3)$.
	Sea $N\ne\{0\}$ un submódulo de $M$. Como $N\ne\{0\}$ y $\dim M<\infty$, existe
	un submódulo no nulo $T$ de $M$ de dimensión maximal
	tal que $N\cap T=\{0\}$. Si $S_i\subseteq N\oplus T$ para todo $i\in\{1,\dots,k\}$, entonces, como $M$ es suma de los $S_i$, tenemos $M=N\oplus T$. 
	Si, en cambio, existe algún $i\in\{1,\dots,k\}$ tal que $S_i\not\subseteq N\oplus T$, entonces $S_i\cap (N\oplus T)\subseteq S_i$. Como $S_i$ es simple,
	se tiene que $S_i\cap (N\oplus T)=\{0\}$. Luego $N\cap (S_i\oplus T)=\{0\}$, una contradicción a la maximalidad de $T$.  
	
	La implicación $(1)\implies(2)$ es trivial. 
	
	Veamos ahora que $(2)\implies(1)$. Sea $J$ un subconjunto de $\{1,\dots,k\}$ maximal tal que 
	la suma de los $S_j$ con $j\in J$ es directa. Sea $N=\oplus_{j\in J}S_j$. Veamos que $M=N$. 
	Para cada $i\in\{1,\dots,k\}$, se tiene que $S_i\cap N=\{0\}$ o bien que $S_i\cap N=S_i$, pues
	$S_i$ es simple. Si $S_i\cap N=S_i$ para todo $i\in\{1,\dots,k\}$, entonces $S_i\subseteq N$ para todo $i\in\{1,\dots,k\}$.  
	Si, en cambio, existe $i\in\{1,\dots,k\}$ tal que $S_i\cap N=\{0\}$, entonces $N$ y $S_i$ estarán en suma directa, 
	una contradicción a la maximalidad del conjunto $J$.
	
	Demostremos por último que $(3)\implies(1)$. 
	Procederemos por inducción en $\dim M$. Si $\dim M=1$ el resultado es trivial. Si $\dim M\geq1$, 
	sea $S$ un submódulo no nulo de $M$ de dimensión minimal. En particular, $S$ es simple. 
	Por hipótesis sabemos que existe un submódulo $T$ de $M$ tal que $M=S\oplus T$. Veamos que $T$ verifica la hipótesis. 
	Si $X$ es un submódulo de $T$, entonces, como en particular $T$ es un submódulo de $M$, existe un submódulo $Y$ de $M$ tal que
	$M=X\oplus Y$. Luego 
	\[
	T=T\cap M=T\cap (X\oplus Y)=X\oplus (T\cap Y),
	\]
	pues $X\subseteq T$. 
	Como $\dim T<\dim M$ y además $T\cap Y$ es un submódulo de $T$, la hipótesis inductiva 
	implica que $T$ es suma directa de módulos simples. Luego $M$ también es suma
	directa de submódulos simples. 
\end{proof}

\begin{proposition}
Si $M$ es un $A$-módulo semisimple y $N$ es un submódulo, entonces $N$ y $M/N$ son semisimples.	
\end{proposition}

\begin{proof}
	Supongamos que $M=S_1+\cdots+ S_k$, donde los $S_i$ son submódulos simples. Si $\pi\colon M\to M/N$ es el morfismo canónico, el lema de Schur nos dice que  
	cada restricción $\pi|_{S_i}$ es cero o un isomorfismo. Luego
	\[
	M/N=\pi(M)=\sum_{i=1}^k(\pi|_{S_i})(S_i)
	\]
	es también una suma finita de módulos simples. Como además existe un submódulo $T$ tal que 
	$M=N\oplus T$, se tiene que $N\simeq M/T$ es también semisimple.    
\end{proof}

\begin{definition}
\index{Álgebra!semisimple}
Un álgebra $A$ se dirá \textbf{semisimple} si todo $A$-módulo finitamente generado es semisimple. 
\end{definition}

\begin{proposition}
Sea $A$ un álgebra de dimensión finita. Entonces $A$ es semisimple si y sólo si la representación
regular de $A$ es semisimple.
\end{proposition}

\begin{proof}
Demostremos la implicación no trivial. Sea $M$ un $A$-módulo finitamente generado, digamos $M=(m_1,\dots,m_k)$. 
La función 
\[
\bigoplus_{i=1}^k A\to M,\quad
(a_1,\dots,a_k)\mapsto \sum_{i=1}^k a_i\cdot m_i,
\]
es un epimorfismo de $A$-módulos. Como
$A$ es semisimple, $\oplus_{i=1}^kA$ es semisimple. 
Luego $M$ es semisimple por ser isomorfo al cociente de un semisimple. 
\end{proof}

\begin{theorem}
Sea $A$ un álgebra semisimple de dimensión finita. Si $\prescript{}{A}A=\oplus_{i=1}^k S_i$, donde los $S_i$ son submódulos simples y 
$S$ es un $A$-módulo simple, entonces $S\simeq S_i$ para algún $i\in\{1,\dots,k\}$. 
\end{theorem}

\begin{proof}
Sea $s\in S\setminus\{0\}$. La función $\varphi\colon A\to S$, $a\mapsto a\cdot s$, es un morfismo de $A$-módulos  
sobreyectivo. Como $\varphi\ne 0$, existe $i\in\{1,\dots,k\}$ tal que alguna restricción 
$\varphi|_{S_i}\colon S_i\to S$ es no nula. 
Por el lema de Schur, $\varphi|_{S_i}$ es un isomorfismo.  	
\end{proof}

Como aplicación inmediata tenemos que
un álgebra semisimple $A$ de dimensión finita admite, salvo isomorfismo, únicamente finitos módulos simples. Cuando digamos
que $S_1,\dots,S_k$ son los simples de $A$ estaremos refiriéndonos a que los $S_i$ son 
representantes de las clases de isomorfismo de todos los $A$-módulos simples, es decir 
que todo simple es isomorfo a alguno de los $S_i$ y además 
$S_i\not\simeq S_j$ si $i\ne j$. 

\medskip
Si $A$ y $B$ son álgebras, $M$ es un $A$-módulo y $N$ es un $B$-módulo, entonces
$A\times B$ actúa en $M\oplus N$ por
\[
(a,b)\cdot (m,n)=(a\cdot m,b\cdot n).
\]
Todo módulo $M$ finitamente generado sobre un anillo de división es libre, es decir
posee que una base. Tal como pasa en espacios vectoriales, vale además que
todo conjunto linealmente independiente de $M$ puede extenderse a una base.  

\medskip
Recordemos que si $V$ es un $A$-módulo, $\End_A(V)$ se define como el 
conjunto de morfismos de módulos $V\to V$. En realidad, 
$\End_A(V)$ es un álgebra con las operaciones: $(f+g)(v)=f(v)+g(v)$, 
$(af)(v)=af(v)$ y $(fg)(v)=f(g(v))$ para todo $f,g\in\End_A(V)$, $a\in A$ y $v\in V$. 

\begin{lemma}
	Sea $D$ un álgebra de división y sea $V$ un $D$-módulo finitamente generado. Entonces
	$V$ es un $\End_D(V)$-módulo simple y además existe $n\in\N$ tal que 
	$\End_D(V)\simeq nV$ es semisimple.
\end{lemma}

\begin{proof}
	Sea $\{v_1,\dots,v_n\}$ una base de $V$. La función
	\[
		\End_D(V)\to \underbrace{V\oplus\cdots\oplus V}_{\text{$n$-veces}},\quad
		f\mapsto (f(v_1),\dots,f(v_n)),
	\]
	es un isomorfismo de $\End_D(V)$-módulos. Luego 
	\[
		\End_D(V)\simeq \bigoplus_{i=1}^n V=nV.
	\]
	
	Falta ver que $V$ es simple. Para eso alcanza con demostrar que $V=(v)$ 
	para todo $v\in V\setminus\{0\}$. Sea $v\in V\setminus\{0\}$. 
	Si $w\in V\setminus\{0\}$, existen $w_2,\dots,w_n$ tal que $\{w,w_2,\dots,w_n\}$ 
	es una base de $V$. Existe $f\in\End_D(V)$ tal que
	$f\cdot v=f(v)=w$. En consecuencia, $w\in (v)$ y entonces $V=(v)$.  
\end{proof}

En lenguaje matricial, el lema anterior nos dice que si $D$ es un álgebra de división, entonces 
$D^{n}$ es un $M_n(D)$-módulo simple y que $M_n(D)\simeq n D^n$ como $M_n(D)$-módulos. 

\begin{theorem}
Sea $A$ un álgebra de dimensión finita y sean 
$S_1,\dots,S_k$ los representantes de las clases de isomorfismo de los $A$-módulos simples. 
Si \[
M\simeq n_1S_1\oplus\cdots\oplus n_kS_k,
\]
entonces
los $n_j$ quedan únivocamente determinados. 
\end{theorem}

\begin{proof}
	Como los $S_j$ son módulos simples no isomorfos, 
	el lema de Schur nos dice que si $i\ne j$ entones $\Hom_A(S_i,S_j)=\{0\}$.
	Para cada $j\in\{1,\dots,k\}$ tenemos entonces que  
	\begin{align*}
		\Hom_A(M,S_j) &\simeq \Hom_A\left(\bigoplus_{i=1}^k n_i S_i,S_j\right)
		\simeq n_j\Hom_A(S_j,S_j). 
	\end{align*} 
	Como $M$ y los $S_j$ son espacios vectoriales de dimensión finita, $\Hom_A(M,S_j)$ y $\Hom_A(S_j,S_j)$ 
	son también espacios vectoriales de dimensión finita. 
	Además $\dim\Hom_A(S_j,S_j)\geq 1$ pues $\id\in\Hom_A(S_j,S_j)$. 
	Luego los $n_j$ quedan unívocamente determinados, pues 
	\[ 
	n_j=\frac{\dim\Hom_A(M,S_j)}{\dim\Hom_A(S_j,S_j)}.\qedhere
	\]
\end{proof}

%Antes de demostrar el teorema de Artin--Wedderburn necesitamos varios resutados elementales 
%sobre matrices. 

Si $A$ es un álgebra, definimos el \textbf{álgebra opuesta} $A^{\op}$ como
el espacio vectorial $A$ con el producto $(a,b)\mapsto ba=a\cdot_{\op}b$. 

\begin{lemma}
	\label{lem:A^op}
    Si $A$ es un álgebra, $A^{\op}\simeq\End_A(A)$ como álgebras. 
\end{lemma}

\begin{proof}
	Primero observemos que $\End_A(A)=\{\rho_a:a\in A\}$, donde $\rho_a\colon
	A\to A$ está dado por $x\mapsto xa$. En efecto, si $f\in\End_A(A)$
	entonces $f(1)=a\in A$. Además $f(b)=f(b1)=bf(1)=ba$ y luego
	$f=\rho_a$.  Tenemos entonces una biyección $\End_A(A)\to A^{\op}$ que es
	morfismo de álgebras pues 
    \[
		\rho_a\rho_b(x)=\rho_a(\rho_b(x))=\rho_a(xb)=x(ba)=\rho_{ba}(x).\qedhere
    \]
\end{proof}

\begin{lemma}
	\label{lem:Mn_op}
	Si $A$ es un álgebra y $n\in\N$, entonces $M_n(A)^{\op}\simeq
	M_n(A^{\op})$ como álgebras.   
\end{lemma}

\begin{proof}
	Sea $\psi\colon M_n(A)^{\op}\to M_n(A^{\op})$ dada por $X\mapsto X^T$,
	donde $X^T$ es la traspuesta de $X$. Como $\psi$ es una transformación lineal biyectiva, basta
	ver
	que $\psi$ es morfismo. Si $i,j\in\{1,\dots,n\}$, $a=(a_{ij})$ y $b=(b_{ij})$ entonces
	\begin{align*}
		(\psi(a)\psi(b))_{ij}&=\sum_{k=1}^n \psi(a)_{ik}\psi(b)_{kj}=\sum_{k=1}^n a_{ki}\cdot_{\op}b_{jk}\\
		&=\sum_{k=1}^n b_{jk}a_{ki}=(ba)_{ji}=((ba)^T)_{ij}=\psi(a\cdot_{\op}b)_{ij}.\qedhere
	\end{align*}
\end{proof}

\begin{lemma}
	\label{lem:simple}
	Si $S$ es un módulo simple y $n\in\N$, entonces 
	\[
		\End_A(nS)\simeq M_n(\End_A(S))
	\]
	como álgebras.
\end{lemma}

\begin{proof}
	Sea $(\varphi_{ij})$ una matriz con entradas en $\End_A(S)$. Vamos a definir
	una función $nS\to nS$ de la siguiente forma:
	\[
	\begin{pmatrix}
	x_1\\
	\vdots\\
	x_n	
	\end{pmatrix}
	\mapsto 
		%\colvec[3]{x_1}{\vdots}{x_n}\mapsto 
		\begin{pmatrix}
			\varphi_{11} & \cdots & \varphi_{1n}\\
			\cdots & \ddots & \vdots\\
			\varphi{n1} & \cdots & \varphi_{nn}
		\end{pmatrix}
		%\colvec[3]{x_1}{\vdots}{x_n}
		\begin{pmatrix}
		x_1\\
		\vdots\\
		x_n	
		\end{pmatrix}
		=\begin{pmatrix}
			\varphi_{11}(x_1)+\cdots+\varphi_{1n}(x_n)\\
			\vdots\\
			\varphi_{n1}(x_1)+\cdots+\varphi_{nn}(x_n)
		\end{pmatrix}.
	\]
	Dejamos como ejercicio demostrar que esta aplicación define un morfismo inyectivo 
	de álgebras
	\[
		M_n(\End_A(S))\to\End_A(nS).
	\]
	Este morfismo es sobreyectivo pues si $\psi\in\End(nS)$ y para cada
	$i,j\in\{1,\dots,n\}$ es posible definir a los $\psi_{ij}$ mediante las ecuaciones
	\[
		\psi\begin{pmatrix}
		x\\
		0\\
		\vdots\\
		0	
		\end{pmatrix}
		=\begin{pmatrix}
		\psi_{11}(x)\\
		\psi_{21}(x)\\
		\vdots\\
		\psi_{n1}(x)
		\end{pmatrix},\dots,
		\psi\begin{pmatrix}
		0\\
		0\\
		\vdots\\
		x	
		\end{pmatrix}
		=\begin{pmatrix}
		\psi_{1n}(x)\\
		\psi_{2n}(x)\\
		\vdots\\
		\psi_{nn}(x)
		\end{pmatrix}.\qedhere
	%	\psi\colvec[4]{x}{0}{\vdots}{0}=\colvec[4]{\psi_{11}(x)}{\psi_{21}(x)}{\vdots}{\psi{n1}(x)},\quad 
	%	\psi\colvec[4]{0}{0}{\vdots}{x}=\colvec[4]{\psi_{1n}(x)}{\psi_{2n}(x)}{\vdots}{\psi{nn}(x)}.\qedhere
	\]
\end{proof}

%\begin{exercise}
%	Sea $\{M_i:i\in I\}$ una colección de módulos y sea $N$ un módulo. Demuestre que
%	\begin{align*}
%		&\Hom_R\left(\bigoplus_{i\in I}M_i,N\right)\simeq \prod_{i\in I}\Hom_R(M_i,N),\\
%		\shortintertext{y que}
%		&\Hom_R\left(N,\prod_{i\in I}M_i\right)\simeq \prod_{i\in I}\Hom_R(N,M_i).
%	\end{align*}
%\end{exercise}

\begin{theorem}[Artin--Wedderburn]
\index{Teorema!de Artin--Wedderburn}
Sea $A$ un álgebra semisimple y de dimensión finita, digamos con 
$k$ clases de isomorfismos de $A$-módulos simples. Entonces 
\[
A\simeq M_{n_1}(D_1)\times\cdots\times M_{n_k}(D_k)
\]
para ciertos $n_1,\dots,n_k\in\N$ y ciertas álgebras de división $D_1,\dots,D_k$.
\end{theorem}

\begin{proof}
	Al agrupar los finitos
	submódulos simples de la representación regular de $A$ podemos escribir 
	\[
	A=\bigoplus_{i=1}^k n_iS_i,
	\]
	donde los $S_i$ son submódulos simples tales que $S_i\not\simeq S_j$ si
	$i\ne j$. Dejamos como ejercicio verificar que, gracias al lema de Schur, tenemos 
	\begin{align*}
		\End_A(A)\simeq\End_A\left(\bigoplus_{i=1}^kn_iS_i\right)
		\simeq \prod_{i=1}^k\End_A(n_iS_i)
		\simeq\prod_{i=1}^kM_{n_i}(\End_A(S_i)), 
	\end{align*}
	donde cada $D_i=\End_A(S_i)$ es un álgebra de división. 
	%En particular, 
	%$A$ tiene $k$ submódulos simples. 
	Tenemos entonces que %álgebras de división $D_1,\dots,D_k$ tales que 
	\[
		\End_A(A)\simeq\prod_{i=1}^kM_{n_i}(D_i).
	\]
	Como $\End_A(A)\simeq
	A^{\op}$, entonces 
	\begin{align*}
		A=(A^{\op})^{\op}\simeq \prod_{i=1}^kM_{n_i}(D_i)^{\op}\simeq \prod_{i=1}^kM_{n_i}(D_i^{\op}).
	\end{align*}
	Como además 
	cada $D_i$ es un álgebra de división, cada $D_i^{\op}$ también lo es.
\end{proof}

Utilizaremos el teorema de Wedderburn en el caso de los números complejos. 
%\begin{lemma}
%Si $A$ es un álgebra compleja de dimensión finita y $S$ es un $A$-módulo simple, entonces
%$\End_A(S)\simeq\C$. 	
%\end{lemma}
%
%\begin{proof}
%Si $\varphi\in\End_A(S)$, entonces $\varphi$ tiene un autovalor $\lambda\in\C$. Como entonces 
%$\varphi-\lambda\id$ no es un isomorfismo, el lema de Schur implica que $\varphi-\lambda\id=0$, 
%es decir $\varphi=\lambda\id$. 	
%\end{proof}

\begin{corollary}[Mollien]
	Si $A$ es un álgebra compleja de dimensión finita semisimple, entonces
	\[
	A\simeq\prod_{i=1}^k M_{n_i}(\C)
	\]  
	para ciertos $n_1,\dots,n_k\in\N$. 
\end{corollary}

\begin{proof}
	Vimos en la demostración del teorema de Wedderburn que 
	\[
	A\simeq \prod_{i=1}^k M_{n_i}(\End_A(S_i)),
	\]
	donde $S_1,\dots,S_k$ son representantes de las clases de 
	isomorfismos de los $A$-módulos simples y cada $\End_A(S_i)$ es un álgebra de división. 
	Veamos que 
	\[
	\End_A(S_i)=\{\lambda\id:\lambda\in\C\}\simeq\C
	\]
	para todo $i\in\{1,\dots,k\}$. En efecto, si 
	$f\in\End_A(S_i)$, entonces $f$ tiene un autovalor $\lambda\in\C$. Como entonces 
	$f-\lambda\id$ no es un isomorfismo, el lema de Schur implica que $f-\lambda\id=0$, 
	es decir $f=\lambda\id$. Luego $\End_A(S_i)\to\C$, $\varphi\mapsto\lambda$, 
	es un isomorfismo de álgebras. En particular, 
	\[
	A\simeq \prod_{i=1}^k M_{n_i}(\C).\qedhere
	\]
\end{proof}

\begin{exercise}
Sean $A$ y $B$ álgebras. Demuestre que los ideales de $A\times B$ son 
de la forma $I\times J$, donde $I$ es un ideal de $A$ y $J$ es un ideal de $B$. 
\end{exercise}

%\index{Módulo!fiel}
%\index{Anulador!de un módulo}
%Recordemos que un $A$-módulo $M$ se dice \textbf{fiel} si el \textbf{anulador} 
%\[
%\Ann(M)=\{a\in A:a\cdot M=0\}
%\]
%de $M$ es nulo. Observemos que $\Ann(M)$ es un ideal de $A$.
\begin{definition}
\index{Álgebra!simple}  
Un álgebra $A$ se dice \textbf{simple} si sus únicos ideales son $\{0\}$ y $A$. 
\end{definition}

%Sabemos que toda álgebra $A$ posee al menos un ideal maximal.  
%
%\begin{example}
%Sea $A$ un álgebra. Sabemos que exist un ideal maximal $I$. El cociente $A/I$ es un $A$-módulo simple. 	
%\end{example}

\begin{proposition}
Sea $A$ un álgebra simple de dimensión finita. Entonces existe un ideal a izquierda no nulo $I$ de dimensión minimal. 
Este ideal es un $A$-módulo simple y todo $A$-módulo simple es isomorfo a $I$. 	
\end{proposition}

\begin{proof}
	Como $A$ es de dimensión finita y $A$ es un ideal a izquierda de $A$, existe un ideal a izquierda no nulo $I$ de dimensión minimal. 
	La minimalidad de $\dim I$ implica que $I$ es simple como $A$-módulo. 
	
	Sea $M$ un $A$-módulo simple. En particular, $M\ne\{0\}$. 
	Como 
	\[
	\Ann(M)=\{a\in A:a\cdot M=\{0\}\}
	\]
	es un ideal de $A$ y además $1\in A\setminus\Ann(M)$, la simplicidad de $A$ implica que
	$\Ann(M)=\{0\}$ y luego $I\cdot M\ne \{0\}$ (pues $I\cdot m\ne 0$ para todo $m\in M$ implica que 
	$I\subseteq\Ann(M)$ e $I$ es no nulo, una contradicción).  
	Sea $m\in M$ tal que $I\cdot m\ne\{0\}$. La función
	\[
	\varphi\colon I\to M,\quad
	x\mapsto x\cdot m,
	\]
	es un morfismo de módulos. Como $I\cdot m\ne\{0\}$, el morfismo $\varphi$ es no nulo. 
	Como $I$ y $M$ son $A$-módulos simples, el lema de Schur implica que $\varphi$ es un isomorfismo. 
\end{proof}

Si $D$ es un álgebra de división, 
el álgebra de matrices $M_n(D)$ es un álgebra simple. La proposición anterior nos dice
en particular que $M_n(D)$ tiene una única clase de isomorfismos de $M_n(D)$-módulos simples. Como sabemos, 
estos módulos son isomorfos a $D^n$. 

\begin{proposition}
Sea $A$ un álgebra de dimensión finita. Si $A$ es simple, entonces $A$ es semisimple.	
\end{proposition}

\begin{proof}
	Sea $S$ la suma de los submódulos simples de la representación regular de $A$. 
	Afirmamos que $S$ es un ideal de $A$. Sabemos
	que $S$ es un ideal a izquierda, pues los submódulos de la representación regular de $A$ son exactamente los ideales a izquierda de $A$. 
	Para ver que $Sa\subseteq S$ para todo $a\in A$, debemos
	demostrar que $Ta\subseteq S$ para todo submódulo simple $T$ de $A$. Si $T\subseteq A$ es un submódulo simple y $a\in A$, 
	sea $f\colon T\to Ta$, $t\mapsto ta$. Como $f$ es un morfismo de $A$-módulos y $T$ es simple, $\ker f=\{0\}$ o bien $\ker T=T$. Si $\ker T=T$, entonces
	$f(T)=Ta=\{0\}\subseteq S$. Si $\ker f=\{0\}$, entonces $T\simeq f(T)=Ta$ y luego $Ta$ es simple y entonces $Ta\subseteq S$. 
	
	Como $S$ es un ideal de $A$ y $A$
	es un álgebra simple, entonces $S=\{0\}$ o bien $S=A$.  Como $S\ne\{0\}$, pues 
	existe un ideal a izquierda no nulo $I$ de $A$ tal que $I\ne\{0\}$ de dimensión minimal,  
	se concluye que $S=A$, es decir la representación regular de $A$ 
	es semisimple (por ser suma de submódulos simples) y luego el álgebra 
	$A$ es semisimple. 
\end{proof}

%\begin{lemma}
%	Sea $A=B\times C$ 
%	un producto directo de álgebras. Si $K$ es un ideal de $A$ si y sólo si $K=I\times J$ 
%	para algún ideal $I$ de $A$ y un ideal $J$ de $B$. 
%\end{lemma}
%
%\begin{proof}
%	Consideramos los morfismos de anillos
%	\begin{align*}
%		& p_B\colon B\times C\to B, &&p_B(b,c)=b,\\
%		& p_C\colon B\times C\to C, && p_C(b,c)=c.  	
%	\end{align*}
%	Si $K$ es un ideal de $A$, definimos
%	$I=p_B(K)$ y $J=p_C(K)$. Veamos que $I$ es un ideal de $B$. Como $K$ es un subgrupo aditivo de $A$ y $p_B$ es un morfismo, entonces
%	$I$ es un subgrupo aditivo de $B$. Si $x\in I=p_B(K)$, entonces
%	existe $y\in C$ tal que $(x,y)\in K$. Como $(bx,0)=(b,0)(x,y)\in K$, entonces 
%	\[
%	bx=p_B(b,0)p_B(x,y)=p_B((b,0)(x,y))\in p_B(K)=I.
%	\]
%	Similarmente se demuestra que $J$ es un ideal de $C$. Afirmamos ahora que $K=I\times J$. 
%	Si $(x,y)\in K$, entonces trivialente $x\in I$ e $y\in J$. Por otro lado, 
%	si $(x,y)\in I\times J$, entonces $x\in I=p_B(K)$ y luego existe $c\in C$ tal que
%	$(x,c)\in K$. Como $K$ es un ideal, $(1,0)(x,c)=(x,0)\in K$. Similarmente vemos que 
%	$(0,y)\in K$. 
%	En consecuencia, $(x,y)=(x,0)+(0,y)\in K$.   
%%\end{proof}
%
%El resultado anterior puede extenderse por inducción. Si $A=A_1\times\cdots\times A_k$ es producto directo de álgebras, 
%todo ideal de $A$ es de la forma $I_1\times\cdots I_k$, donde $I_j$ es un ideal de $A_j$ para todo $j\in\{1,\dots,k\}$. 
\begin{theorem}[Wedderburn]
\index{Teorema!de Wedderburn}
%	Sea $A$ un álgebra de dimensión finita. 
%	Las siguientes afirmaciones son equivalentes:
%	\begin{enumerate}
%	\item $A$ tiene un módulo simple y fiel.
%	\item $A$ es semisimple y tiene una única clase de isomorfismos de módulos simples.
%	\item $A\simeq M_n(\C)$ para algún $n\in\N$.
%	\item $A$ es simple.	
%	\end{enumerate}
	Sea $A$ un álgebra de dimensión finita. 
	Si $A$ es simple, entonces $A\simeq M_n(D)$ para algún $n\in\N$ y alguna álgebra de división $D$.
\end{theorem}


\begin{proof}
%	Veamos que $(2)\implies(3)$. Como $A$ es semisimple, el teorema de Artin--Wedderburn implica que $A\simeq\prod_{i=1}^k M_{n_i}(\C)$ 
%	y que $A$ tiene $k$ clases de isomorfismos de $A$-módulos simples. Luego $k=1$ y entonces
%	$A\simeq M_n(\C)$ para algún $n\in\N$. 
	Como $A$ es simple, entonces $A$ es semisimple. El teorema de Artin--Wedderburn implica que $A\simeq\prod_{i=1}^k M_{n_i}(D_i)$ 
	para ciertos $n_1,\dots,n_k$ y ciertas álgebras de división $D_1,\dots,D_k$. Además $A$ tiene
	$k$ clases de isomorfismos de módulos simples. Como $A$ es simple,
	$A$ tiene solamente una clase de isomorfismos de módulos simples. Luego $k=1$ y entonce
	$A\simeq M_n(D)$ para algún $n\in\N$ y alguna álgebra de división $D$. 
	%	Veamos que $(1)\implies(2)$. Sea $M$ un $A$-módulo simple y fiel. Sea $K$ un ideal de 
%	$A$ de dimensión mínima tal que $K$ es el núcleo de algún morfismo $f\colon A\to nM$. Si $a\in\ker f$ es tal que $a\ne 0$, entonces 
%	existe $m\in M$ tal que $a\cdot m\ne 0$, pues $\Ann(M)=\{0\}$. Luego
%	\[
%	g\colon A\to nM\oplus M=(n+1)m,\quad
%	x\mapsto (f(x),x\cdot m),
%	\]
%	es un morfismo de $A$-módulos tal que $a\ker f\setminus\ker g$, una contradicción a la minimalidad de $K$. 
=======
\begin{definition}
	An \textbf{algebra} (over the field $K$) is a vector space (over $K$) 
	with an associative multiplication $A\times A\to A$ such that
	$a(\lambda b+\mu c)=\lambda(ab)+\mu(ac)$ and
	$(\lambda a+\mu b)c=\lambda(ac)+\mu (bc)$ for all $a,b,c\in A$, and 
	that contains an element $1_A\in A$ such that $1_Aa=a1_A=a$ for all $a\in A$.   
\end{definition}

Note that an algebra over $K$ is a ring $A$ that is a vector space
(over $K$) such that the map $K\to A$, $\lambda\mapsto \lambda1_A$, is injective. 

\begin{definition}
	An algebra $A$ is \textbf{commutative} if $ab=ba$ for all $a,b\in A$. 
\end{definition}

\begin{example}
	The field $\R$ is a real algebra and similarly 
	$\C$ is a complex algebra. Moreover, $\C$ is a real algebra. 
\end{example}

Any field $K$ is an algebra over $K$.

\begin{example}
	Let $K$ be a field. Then $K[X]$, $K[X,Y]$ and $K[[X]]$ are algebras over $K$.
\end{example}

\begin{example}
	If $A$ is an algebra, then $M_n(A)$ is an algebra. 	
\end{example}

The dimension of an algebra is by 
definition the dimension of the 
underlying vector space. 

\begin{definition}
	Let $A$ and $B$ be algebras. A map $f\colon A\to B$ is an \textbf{algebra homomorphism} 
	if $f$ is linear and it is a ring homomorphism.  	
\end{definition}

The map $\C\to\C$, $z\mapsto\overline{z}$, is a ring homomorphism that
is not $\C$-linear, so it is not an $\C$-algebra homomorphism. 

\begin{example}
	Let $G$ be a finite group. The vector space 
	$\C[G]$ with basis $\{g:g\in G\}$
	is an algebra with multiplication
	\[
	\left(\sum_{g\in G}\lambda_gg\right)\left(\sum_{h\in G}\mu_hh\right)
	=\sum_{g,h\in G}\lambda_g\mu_h(gh).
	\] 	
	Note that $\dim\C[G]=|G|$ and
	$\C[G]$ is commutative if and only $G$ is abelian. 
	This is the \textbf{complex group algebra} of $G$. 
\end{example}

Two basic exercises about group algebras.
 
\begin{exercise}
	Let $G$ be a non-trivial finite group. 
	Then $\C[G]$ has zero divisors. 
\end{exercise}

\begin{exercise}
	Let $A$ be an algebra and $G$ be a finite group. 
	If $f\colon G\to\mathcal{U}(R)$ is a group homomorphism, 
	then there exists an algebra homomorphism 
	$\varphi\colon K[G]\to A$ such that $\varphi|_G=f$.   	
\end{exercise}


\begin{definition}
	Let $A$ be an algebra. An (left) \textbf{ideal} of $A$ is an 
	(left) ideal of the ring $A$ that is also a subspace. 
\end{definition}

Let $A$ be an algebra over $K$. If $I$ is a left ideal of the ring $A$, then 
$I$ is a subspace (over $K$), as $\lambda a=\lambda(1_Aa)=(\lambda 1_A)a$ 
for all $\lambda\in K$ and $a\in A$.  
%
%If $I$ is an ideal of $A$, then $A/I$ is an algebra and 
%the canonical map $\pi\colon A\to A/I$, $a\mapsto a+I$, is a surjective
%algebra homomorphism such that $\ker\pi=I$. 

\begin{definition}
	Let $A$ be an algebra. A \textbf{module} over $A$ is a module $M$ of the ring $A$. 
\end{definition}

Note that if $M$ is a module over $A$, 
then $M$ is a vector space 
with $\lambda m=(\lambda 1_A)m$ for all $\lambda\in K$ and $m\in M$. 
 	
\begin{exercise}
	Let $A$ be an algebra and $M$ be a module over $A$. 
	Then $M$ is finitely generated if and only if $M$ is finite-dimensional.
\end{exercise}

An important example of a module is given by the left representation. The 
algebra $A$ is a module over $A$ with the left multiplication. 

\begin{definition}
	Let $A$ be an algebra and $M$ be a module over $A$. Then 
	$M$ is \textbf{simple} if $M\ne\{0\}$ and $\{0\}$ and $M$ 
	are the only submodules of $M$.	
\end{definition}

\begin{definition}
	Let $A$ be a finite-dimensional 
	algebra and $M$ be a finite-dimensional module over $A$. Then 
	$M$ is \textbf{semisimple} if $M$ is a direct sum of 
	finitely many simple submodules.  
\end{definition}

Clearly, a finite direct sum of semisimples is semisimple. 

\begin{lemma}[Schur]
	Let $A$ be an algebra. If $S$ and $T$ are
	simple modules and $f\colon S\to T$ is a non-zero module homomorphism, 
	then $f$ is an isomorphism. 
\end{lemma}





\backmatter
%\include{glossary}
%\include{soluciones}

\bibliographystyle{abbrv}
\bibliography{refs}

\printindex



\end{document}






