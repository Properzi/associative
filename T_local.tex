\section{Project: When a group algebra is local?}
\label{section:local}

\begin{proposition}
    \label{pro:augmentation}
    Let $R$ be a commutative ring with one. 
    Let $f\colon G\to H$ be a group homomorphism with kernel $K$. Then
    \[
    \varphi\colon R[G]\to R[H],
    \quad 
    \sum\lambda_ig_i\mapsto \sum\lambda_if(g),
    \]
    is a ring homomorphism with kernel the ideal 
    of $R[G]$ generated by $\{k-1:k\in K\}$. 
\end{proposition}

\begin{proof}
    A direct calculation shows that the map $\varphi$ is a well-defined ring homomorphism. Let 
    $S=\{k-1:k\in K\}$. Then $(S)\subseteq \ker\varphi$. 
    
    Let us show that 
    $\ker\varphi\subseteq (S)$. Let $\alpha=\sum r_ig_i\in\ker\varphi$. Then 
    \[
    \varphi(\alpha)=\sum r_if(g_i)=0.
    \]
    Let 
    $\{Kg_{i_1},\dots,Kg_{i_k}\}$ be the subset of pairwise distinct cosets 
    of $Kg_1,\dots,Kg_n$. Write  
    \[
    \alpha=\sum\sum s_{ij}k_{ij}g_{i_j}
    \]
    for some $s_{ij}\in R$ and $k_{ij}\in K$. Then 
    \begin{equation}
    \label{eq:0=s_ij}
    0=\varphi(\alpha)=\sum\sum s_{ij}\varphi(k_{ij}g_{i_j})
    =\sum\sum s_{ij}f(g_{i_j}),
    \end{equation}
    as $K=\ker f$. Note that 
    \[
    f(g_{i_j})=f(g_{i_k})\implies 
    g_{i_j}g_{i_k}^{-1}\in K\implies 
    g_{i_j}K=g_{i_k}K.
    \]
    Thus $f(g_{i_j})\ne f(g_{i_k})$ for $j\ne k$. Since $R[H]$ is a free $R$-module 
    with basis $\{h:h\in H\}$, Equality
    \eqref{eq:0=s_ij}
    implies that $\sum_i s_{ij}=0$ for all $j$. Thus
    \[
    \alpha=\sum\sum s_{ij}k_{ij}g_{i_j}=\sum\sum s_{ij}(k_{ij}-1)g_{i_j}\in (S).\qedhere
    \]
\end{proof}

\begin{corollary}
\label{cor:R[G/N]}
    Let $R$ be a commutative ring with one. If 
    $G$ is a group and $N$ is a normal subgroup of $G$, then
    \[
    R[G/N]\simeq R[G]/I,
    \]
    where $I$ is the ideal of $R[G]$ generated by $\{n-1:n\in N\}$. 
\end{corollary}

\begin{proof}
    Apply the previous proposition to the canonical map $\pi\colon G\to G/N$ to get
    a ring homomorphism $\varphi\colon R[G]\to R[G/N]$. The kernel of $\varphi$ is the ideal $I$ 
    generated by the set 
    \[
    \{g-1:g\in\ker\pi=N\}.
    \]
    Since 
    $\pi$ is surjective, $\varphi$ is surjective. 
    By the first isomorphism theorem, the claim follows. 
%    Then the claim follows from the first isomorphism theorem. 
\end{proof}

Let $K$ be a field and $G$ be a group. We write $A(K[G])$ to denote
the ideal of $K[G]$ generated by the set $\{g-1:g\in G\}$. This ideal is known as the
\emph{augmentation ideal} of $K[G]$. 

\begin{corollary}
\label{cor:local_K[N]_and_K[G/N]}
    Let $K$ be a field. 
    Let $G$ be a group and $N$ be a central subgroup of $G$. If $K[N]$ and $K[G/N]$ are local, 
    then $K[G]$ is local. 
\end{corollary}

\begin{proof}
    By Corollary \ref{cor:R[G/N]}, $K[G/N]\simeq K[G]/I$, where 
    $I$ is the ideal of $K[G]$ generated by $\{n-1:n\in N\}$. Since $N\subseteq Z(G)$, 
    $I$ is central in $K[G]$. Note that 
    \[
    I=A(K[N])K[G].
    \]
    Let $\alpha\in A(K[G])$. 
    Since $K[G/N]$ is local, $A(K[G/N])$ is nil by Theorem \ref{thm:local}. Since 
    \[
    K[G]/I\simeq K[G/N],
    \]
    this implies that 
    there exists $m$ such that $\alpha^m\in I$. Since $K[N]$ is local,  
    $A(K[N])$ is nil by Theorem \ref{thm:local}. Moreover, $K[N]$ is central in $K[G]$, because $N\subseteq Z(G)$. This implies that $I=A(K[N])K[G]$ is also nil. In particular, 
    $\alpha$ is nil. Hence 
    $K[G]$ is nil and therefore $K[G]$ is local by Theorem \ref{thm:local}. 
\end{proof}


\begin{exercise}
\label{xca:augmentation}
    Let $R$ be a unitary commutative ring and $G$ be a group. 
    Prove that 
    the map $R[G]\to R$, $\sum_{g\in G}r_gg\mapsto\sum_{g\in G}r_g$, is a surjective
    ring homomorphism with kernel $A(R[G])$.  
\end{exercise}

\begin{lemma}
    Let $K$ be a field and $G$ be a finite group. 
    The following statements are equivalent: 
    \begin{enumerate}
        \item $K[G]$ is local. 
        \item $A(K[G])\subseteq J(K[G])$. 
        \item $A(K[G])$ is nil.
        \item $A(K[G])=J(K[G])$. 
    \end{enumerate}
\end{lemma}

\begin{proof}
    Let us prove that $1)\implies 2)$. Since $K[G]$ is local, 
    $R\setminus\mathcal{U}(K[G])=J(K[G])$ by Theorem \ref{thm:local}. Since $K[G]\setminus\mathcal{U}(K[G])$ contains 
    every proper ideal of $K[G]$, 
    \[
    A(K[G])\subseteq J(K[G]).
    \]

    We now prove that $2)\implies 3)$. Since $G$ is finite, $K[G]$ is artinian. By Lemma \ref{lem:J(R)_nil}, 
    $J(K[G])$ is nil. Hence $A(K[G])$ is nil. 

    We now prove that $3)\implies 4)$. Since $J(K[G])$ contains every nil ideal (see Proposition~\ref{pro:nilJ}), 
    $A(K[G])\subseteq J(K[G])$. On the other hand, $K[G]/A(K[G])\simeq K$. Since $K$ is a field, the correspondence theorem
    implies that $A(K[G])=J(K[G])$. 

    Finally, we prove that $4)\implies 1)$. Since $A(K[G])=J(K[G])$, Exercise \ref{xca:augmentation} implies that 
    $K[G]/J(K[G])\simeq K$. Since $K$ is a field, 
    it is, in particular, a division ring. Thus $K[G]$ is local by Theorem \ref{thm:local}.     
\end{proof}


\begin{exercise}
\label{xca:C_p:local}
    Let $p$ be a prime number, $K$ be a field of characteristic $p$ and 
    $G$ be a cyclic group of order $p$. Prove that $K[G]$ is local. 
\end{exercise}


\begin{exercise}
\label{xca:K[G]_domain_easy}
    Let $K$ be a field and $G$ be a finite group. Then 
    $K[G]$ is a domain if and only if $|G|=1$. 
\end{exercise}


\begin{theorem}
    Let $K$ be a field and $G$ be a non-trivial finite group. 
    Then $K[G]$ is local if and only if $K$ is of characteristic $p>0$ and $G$ is a $p$-group. 
\end{theorem}

\begin{proof}
    Let us first prove $\implies$. Assume first that $K$ is a field of characteristic zero. 
    By Maschke's theorem, $J(K[G])=\{0\}$. By Theorem \ref{thm:local}, 
    $K[G]$ is a division ring. In particular, $K[G]$ is a domain, 
    a contradiction (see Exercise \ref{xca:K[G]_domain_easy}).

    Assume now that $K$ is of characteristic $p>0$. 
    Let $q$ be a prime divisor of $|G|$ and $g\in G$ an element of order $q$. 
    Since 
    \[
    (1-g)(1+\cdots+g^{q-1})=1-g^q=0,
    \]
    $1-g\not\in\mathcal{U}(K[G])$ and $1+\cdots+g^{q-1}\not\in\mathcal{U}(K[G])$. It follows
    that $1-g^m\not\in\mathcal{U}(K[G])$ for all $m\geq0$. By Theorem \ref{thm:local}, 
    $K[G]\setminus J(K[G])$ is an ideal. Thus  
    \[
    q1_G=1+\cdots+g^{q-1}+\sum_{m=1}^{q-1}(1-g^m)\not\in\mathcal{U}(K[G])
    \]
    If $q\ne 0$ in $K$, then $q1_G\in\mathcal{U}(K[G])$. Hence $q=0$ in $K$ and
    therefore $p$ divides $q$. We conclude that $G$ is a $p$-group. 

    We now prove $\impliedby$. Let $G$ be a $p$-group and $K$ be a field of characteristic $p>0$. We proceed
    by induction on $|G|$. 
    If $|G|=p$, $K[G]$ is a local ring (see Exercise \ref{xca:C_p:local}).
    % \[
    % K[G]\simeq K[X]/(X^p-1)\simeq K[X]/((X-1)^p), 
    % \]
    % as $X^p-1=(X-1)^p$. But $K[X]/((X-1)^p)$ is a commutative local ring  
    If $|G|>p$, let $Z=Z(G)$. Since $G$ is a $p$-group, $|Z|\geq p$. Let $N$ be a subgroup of $Z$ of order $p$. 
    Then $|N|<|G|$ and $|G/N|<|G|$. By the inductive hypothesis, both 
    $K[N]$ and $K[G/N]$ are local. By Corollary \ref{cor:local_K[N]_and_K[G/N]}, $K[G]$ is local too. 
\end{proof}
