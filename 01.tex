\lecture{}

\topic{Semisimple algebras}

We will devote two lectures to the study of 
finite-dimensional semisimple algebras. The main goal is to
prove Artin--Wedderburn's theorem. 

\begin{definition}
	\index{Algebra}
	An \textbf{algebra} (over the field $K$) is a vector space (over $K$) 
	with an associative multiplication $A\times A\to A$ such that
	$a(\lambda b+\mu c)=\lambda(ab)+\mu(ac)$ and
	$(\lambda a+\mu b)c=\lambda(ac)+\mu (bc)$ for all $a,b,c\in A$, and 
	that contains an element $1_A\in A$ such that $1_Aa=a1_A=a$ for all $a\in A$.   
\end{definition}

Note that an algebra over $K$ is a ring $A$ that is a vector space
(over $K$) such that the map $K\to A$, $\lambda\mapsto \lambda1_A$, is injective. 

\begin{definition}
	\index{Algebra!commutative}
	An algebra $A$ is \textbf{commutative} if $ab=ba$ for all $a,b\in A$. 
\end{definition}

\index{Algebra!dimension}
The \textbf{dimension} of an algebra $A$ is the dimension of $A$ as a vector space. This is why we want to consider algebras, as 
they are linear version of rings. Quite often our arguments will use the dimension of the underlying vector space.  

\begin{example}
	The field $\R$ is a real algebra and similarly 
	$\C$ is a complex algebra. Moreover, $\C$ is a real algebra. 
\end{example}

Any field $K$ is an algebra over $K$.

\begin{example}
	If $K$ is a field, then $K[X]$ is an algebra over $K$. 
\end{example}

Similarly, the polynomial ring $K[X,Y]$ and the ring $K[[X]]$ of power series
are examples of algebra over $K$. 

\begin{example}
	If $A$ is an algebra, then  $M_n(A)$ is an algebra. 
\end{example}

\begin{example}
	The set of continuous maps $[0,1]\to\R$ is a real algebra with the usual
	point-wise operations $(f+g)(x)=f(x)+g(x)$ and $(fg)(x)=f(x)g(x)$. 
\end{example}

\begin{example}
	Let $n\in\Z_{>0}$. Then $K[X]/(X^n)$ is a finite-dimensional algebra. 
    It is the \textbf{truncated polynomial algebra}.  
\end{example}

\begin{example}
	Let $G$ be a finite group. The vector space 
	$\C[G]$ with basis $\{g:g\in G\}$
	is an algebra with multiplication
	\[
	\left(\sum_{g\in G}\lambda_gg\right)\left(\sum_{h\in G}\mu_hh\right)
	=\sum_{g,h\in G}\lambda_g\mu_h(gh).
	\] 	
	Note that $\dim\C[G]=|G|$ and
	$\C[G]$ is commutative if and only $G$ is abelian. 
	This is the \textbf{complex group algebra} of $G$. 
\end{example}

\begin{definition}
    \index{Homomorphism!of algebras}
    An algebra \textbf{homomorphism} is a ring homomorphism $f\colon A\to B$ that is also a linear map. 
\end{definition}

The complex conjugation map  
$\C\to \C$, $z\mapsto\overline{z}$, is a ring homomorphism that is not an algebra homomorphism over $\C$. 

%Two basic exercises about group algebras.
 
\begin{exercise}
	Let $G$ be a non-trivial finite group. 
	Then $\C[G]$ has zero divisors. 
\end{exercise}

\begin{exercise}
	Let $A$ be an algebra and $G$ be a finite group. 
	If $f\colon G\to\mathcal{U}(A)$ is a group homomorphism, 
	then there exists an algebra homomorphism 
	$\varphi\colon K[G]\to A$ such that $\varphi|_G=f$.   	
\end{exercise}

\begin{definition}
 	\index{Algebra!ideal}
 	An \textbf{ideal} of an algebra is an ideal of the underlying ring.
\end{definition}

Similarly one defines left and right ideals of an algebra.

If $A$ is an algebra, then every left ideal of the ring $A$ is a vector space.  
Indeed, if $I$ is a left ideal of $A$ 
and $\lambda\in K$ and $x\in I$, then 
\[
	\lambda x=\lambda (1_Ax)=(\lambda 1_A)x.
\]
Since $\lambda 1_A\in A$, it follows that  $\lambda I=(\lambda
1_A)L\subseteq I$. 
Similarly, every right ideal of the ring $A$ is a vector space. 

If $A$ is an algebra and $I$ is an ideal of $A$, then the quotient ring $A/I$ has a unique algebra
structure such that the canonical map  
$A\to A/I$, $a\mapsto a+I$, is a surjective algebra homomorphism with kernel $I$. 

\begin{definition}
    \index{Algebraic element}
    \index{Algebra!algebraic}
    Let $A$ be an algebra over the field $K$. An element $a\in A$ is 
    \textbf{algebraic} over $K$ if there exists a non-zero polynomial $f\in K[X]$
    such that $f(a)=0$. 
\end{definition}

If every element of $A$ is algebraic, then $A$ is said to be \text{algebraic} 

In the algebra $\R$ over $\Q$, the element $\sqrt{2}$ is algebraic, as $\sqrt{2}$ is a root of the polynomial $X^2-2\in\Q[X]$. A famous theorem of Lindemann proves that $\pi$ is not algebraic over $\Q$. Every element of the real algebra $\R$ is algebraic.

\begin{proposition}
	\label{lem:algebraic}
	Every finite-dimensional algebra is algebraic.
\end{proposition}

\begin{proof}
   Let $A$ be an algebra with $\dim A=n$ and let $a\in A$. Since  
	$\{1,a,a^2,\dots,a^n\}$ has $n+1$ elements, it is a linearly dependent set. Thus there exists 
	a non-zero polynomial $f\in K[X]$ such that $f(a)=0$.
\end{proof}

\begin{definition}
    A \textbf{module} $M$ over an algebra $A$ is a module 
    over the ring $A$.
\end{definition}

Similarly one defines submodules and module homomorphisms. 

\begin{example}
If $V$ is a module over an algebra $A$, one defines $\End_A(V)$ as the set
of module homomorphisms $V\to V$. The set  
$\End_A(V)$ is indeed an algebra with 
\[
(f+g)(v)=f(v)+g(v),\quad 
(af)(v)=af(v)
\quad\text{and}
\quad 
(fg)(v)=f(g(v))
\]
for all $f,g\in\End_A(V)$, $a\in A$ and $v\in V$. 
\end{example}

Let $A$ be a finite-dimensional algebra. 
If $M$ is a module over the ring $A$, then $M$ is a vector space with  
\[
\lambda m=(\lambda 1_A)\cdot m, 
\]
where $\lambda\in K$ and $m\in M$. Moreover, $M$ is finitely generated if and only if $M$ is finite-dimensional.  

% In this chapter we will work with finitely generated modules. 

\begin{example}
An algebra  $A$ is a module over $A$ with left multiplication, that is $a\cdot b=ab$, $a,b\in A$.
This module is the (left) \textbf{regular representation} of $A$ and it will be denoted by $\prescript{}{A}{A}$. 
\end{example}

\begin{definition}
	Let $A$ be an algebra and $M$ be a module over $A$. Then 
	$M$ is \textbf{simple} if $M\ne\{0\}$ and $\{0\}$ and $M$ 
	are the only submodules of $M$.	
\end{definition}

\begin{definition}
	Let $A$ be a finite-dimensional 
	algebra and $M$ be a finite-dimensional module over $A$. Then 
	$M$ is \textbf{semisimple} if $M$ is a direct sum of 
	finitely many simple submodules.  
\end{definition}

Clearly, a finite direct sum of semisimples is semisimple. 

\begin{lemma}[Schur]
	Let $A$ be an algebra. If $S$ and $T$ are
	simple modules and $f\colon S\to T$ is a non-zero module homomorphism, 
	then $f$ is an isomorphism. 
\end{lemma}

\begin{proof}
Since $f\ne 0$, $\ker f$ is a proper submodule of $S$. Since $S$ is simple, it follows 
that $\ker f=\{0\}$. Similarly, $f(S)$ 
is a non-zero submodule of $T$ and hence $f(S)=T$, as $T$ is simple. 	
\end{proof}

\begin{proposition}
    If $A$ is a finite-dimensional algebra and $S$ is a simple module, then $S$ is finite-dimensional. 
\end{proposition}

\begin{proof}
    Let $s\in S\setminus\{0\}$. Since $S$ is simple, $\varphi\colon A\to S$, $a\mapsto a\cdot s$, is a surjective 
    module homomorphism. 
    In particular, by the first isomorphism theorem, $A/\ker\varphi\simeq S$ and hence $\dim S=\dim (A/\ker\varphi)\leq \dim A$. 
\end{proof}

\begin{proposition}
\label{pro:semisimple}
	Let $M$ be a finite-dimensional module. The following statements are equivalent.
	\begin{enumerate}
		\item $M$ is semisimple.
		\item $M=\sum_{i=1}^k S_i$, where each $S_i$ is a simple submodule of $M$. 
		\item If $S$ is a submodule of $M$, then there is a submodule $T$ of $M$ such that $M=S\oplus T$.    
	\end{enumerate}
\end{proposition}

\begin{proof}
	We first prove that $2)\implies3)$.
	Let $N\ne\{0\}$ be a submodule of $M$. Since $N\ne\{0\}$ and $\dim M<\infty$, there exists a submodule 
	$T$ of $M$ of maximal dimension such that 
	$N\cap T=\{0\}$. If $S_i\subseteq N\oplus T$ for all $i\in\{1,\dots,k\}$, then, as $M$ is the sum of the $S_i$, it follows that
	$M=N\oplus T$. 
	If, however, there exists $i\in\{1,\dots,k\}$ such that $S_i\not\subseteq N\oplus T$, then $S_i\cap (N\oplus T)\subseteq S_i$. 
	Since the module $S_i$ is simple,
	it follows that $S_i\cap (N\oplus T)=\{0\}$. Thus $N\cap (S_i\oplus T)=\{0\}$, a contradiction to the maximality of
	$\dim T$.  
	
	The implication  $1)\implies2)$ is trivial. 
	
% 	We now prove that $2)\implies1)$. Let $J$ be a subset of $\{1,\dots,k\}$ maximal with the property that 
% 	the sum $\sum_{j\in J}S_j$ is direct. Let $N=\oplus_{j\in J}S_j$. We claim that $M=N$. 
% 	For each $i\in\{1,\dots,k\}$, either $S_i\cap N=\{0\}$ or $S_i\cap N=S_i$, as
% 	$S_i$ is simple. If $S_i\cap N=S_i$ for all $i\in\{1,\dots,k\}$, then $S_i\subseteq N$ for all $i\in\{1,\dots,k\}$.  
% 	If there exists $i\in\{1,\dots,k\}$ such that $S_i\cap N=\{0\}$, then $N$ and $S_i$ are in direct sum, a contradiction to the maximality of the set $J$. 
	Finally, we prove that $3)\implies1)$. 
	We proceed by induction on $\dim M$. The result is clear if $\dim M=1$. Assume that $\dim M\geq2$ and   
	let $S$ be a non-zero submodule of $M$ of minimal dimension. In particular, $S$ is simple. 
    By assumption, there exists a submodule $T$ of $M$ such that $M=S\oplus T$. We claim that $T$ satisfies the assumptions. 
	If $X$ is a submodule of $T$, then, since $T$ is also a submodule of $M$, there exists a submodule $Y$ of $M$ such that 
	$M=X\oplus Y$. Thus  
	\[
	T=T\cap M=T\cap (X\oplus Y)=X\oplus (T\cap Y),
	\]
	as $X\subseteq T$. 
	Since $\dim T<\dim M$ and $T\cap Y$ is a submodule of $T$, the inductive hypothesis implies 
	that $T$ is a direct sum of simple submodules. Hence $M$ is a direct sum of simple submodules. 
\end{proof}

\begin{proposition}
    If $M$ is a semisimple module and $N$ is a submodule, then $N$ and $M/N$ are semisimple.	
\end{proposition}

\begin{proof}
	Assume that $M=S_1+\cdots+ S_k$, where each $S_i$ is a simple submodule. If $\pi\colon M\to M/N$ 
	is the canonical map, Schur's lemma implies that each restriction $\pi|_{S_i}$ 
	is either zero or an isomorphism with the image. Since  
	\[
	M/N=\pi(M)=\sum_{i=1}^k(\pi|_{S_i})(S_i),
	\]
	it follows that $M/N$ is a direct sum of finitely many simples. 
	
	We now prove that $N$ is semisimple. By assumption, 
	there exists a submodule $T$ such that 
	$M=N\oplus T$. The quotient
	$M/T$ is semisimple by the previous paragraph, so it follows that 
	\[
	N\simeq N/\{0\}=N/(N\cap T)\simeq (N\oplus T)/T=M/T
	\]
	is also semisimple.     
\end{proof}

