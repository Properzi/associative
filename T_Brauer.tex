\section{Project: The Brauer group}

Fix a field $K$. Recall that a $K$-algebra $A$ is \emph{simple} 
if $\{0\}$ and $A$ are the only ideals of $A$. 
For example, if $D$ is a division algebra, then $D$ and $M_n(D)$ are simple algebras. 

\begin{example}
    If $a,b\in K\setminus\{0\}$, let $H_K(a,b)$ be the $K$-algebra 
    with basis $\{1,i,j,k\}$ and multiplication given by
    \[
    i^2=a,\quad
    j^2=b,\quad 
    ij=-ji=k.
    \]
    The quaternion algebra $H_K(a,b)$ is a simple $K$-algebra, as either $H_K(a,b)$ is a division algebra
    or $H_K(a,b)\simeq M_2(K)$. 
\end{example}

A well-known particular case: $\mathbb{H}=H_{\R}(-1,-1)$. 

\begin{definition}
    \index{Algebra!central simple}
    A \emph{central simple algebra} is a finite-dimensional algebra $K$-algebra
    such that $A$ is simple and $Z(A)=K$.
\end{definition}

For example, $\C$ is a complex central simple algebra and
it is not a real central simple algebra, as $\Z(\C)=\C$. Moreover, 
$\mathbb{H}$ and $\R$ are central simple algebras over $\R$.  

\begin{exercise}
    Prove that $\mathbb{H}\otimes_{\R}\mathbb{H}\simeq M_4(\R)$.  
\end{exercise}

The previous exercise shows that the tensor product of central simple algebras is not 
necessarily a central simple algebra. 

Wedderburn's theorem states that every finite-dimensional 
simple algebra is isomorphic to $M_n(D)$ 
for some $n$ and some division algebra $D$. 

\begin{exercise}
    Prove that the $n$ in Wedderburn's theorem is unique and
    the division algebra $D$ is unique up to isomorphism. 
\end{exercise}

Let $A$ and $B$ be central simple $K$-algebras. By Wedderburn's theorem, 
$A\simeq M_n(D)$ and $B\simeq M_m(E)$ for some $m,n>0$ and 
division algebras $D$ and $E$. We define 
\[
    A\sim B\Longleftrightarrow D\simeq E.
\]

\begin{exercise}
    Prove that $\sim$ is an equivalence relation. 
\end{exercise}

%\begin{example}
If $D$ is a central division $K$-algebra, then $D=M_1(D)\sim M_n(D)$ for all $n$.
%\end{example}

\begin{exercise}
    Let $D$ be a $K$-algebra. 
    Prove that $D\otimes_K M_n(K)\simeq M_n(D)$ as $K$-algebras
\end{exercise}

\begin{exercise}
    Prove that $M_n(K)\otimes_K M_m(K)\simeq M_{nm}(K)$. 
\end{exercise}

If $A$ is a central simple algebra, $[A]$ will denote
the equivalence class of $A$ under the relation $\sim$, that is 
$[A]=\{B:B\sim A\}$. 

\begin{exercise}
    Prove that the collection of equivalence classes of central simple 
    algebras is a set. 
\end{exercise}

One way to solve the previous exercise is to recall that, by definition, central simple algebras
are finite-dimensional. Then that the underlying 
vector space of a central simple algebra over $K$ is $K^n$ for some $n$. Algebra
structures over $K^n$ form a set, as they are indeed a subset of $\Hom(K^n\otimes K^n,K^n)$. 

\begin{theorem}
    Let $\operatorname{Br}(K)$ be the set of equivalence classes
    of central simple $K$-algebras. Then $\operatorname{Br}(K)$ with 
    the operation 
    \begin{equation}
        \label{eq:Brauer}
        [A][B]=[A\otimes_KB]
    \end{equation}
    is an abelian group. 
\end{theorem}

\begin{proof}[Sketch of the proof]
    We need to show that the product of $\operatorname{Br}(K)$ is well-defined. There are several
    things to prove:
    \begin{enumerate}
        \item $A\otimes_KB$ is a finite-dimensional central simple $K$-algebra.  
        \item The multiplication $[A][B]=[A\otimes_KB]$ is well-defined, that is $A\sim A_1$ and $B\sim B_1$ imply 
        that $A\otimes_KB\sim A_1\otimes_KB_1$. 
    \end{enumerate}
    To prove 1) we note that $A\otimes_KB$ is a finite-dimensional $K$-algebra, as 
    \[
    \dim_K(A\otimes_KB)=(\dim_KA)(\dim_KB).
    \]
    It is central, as 
    $Z(A\otimes_KB)\simeq Z(A)\otimes_KZ(B)$. Finally, it is simple, as 
    there exists a bijective correspondence between ideals of $A$ and 
    ideals of $A\otimes_KB$. 
    
    Let us prove 2). Write $A\simeq M_n(D)$, $A_1\simeq M_{n_1}(D)$, $B\simeq M_{m}(E)$ and
    $B_1\simeq M_{m_1}(E)$ for some division $K$-algebras $D$ and $E$. Since the
    tensor product is associative and commutative, 
    \begin{align*}
        A\otimes_KB &\simeq M_n(D)\otimes_KM_m(E)\\
        &\simeq D\otimes_KM_n(K)\otimes_KE\otimes_KM_m(K)\\
        &\simeq D\otimes_KE\otimes_KM_{nm}(K)\\
        &\simeq M_{nm}(D\otimes_KE).
    \end{align*}
    Note that $D\otimes_KE$ is maybe not a division algebra, but it is indeed
    a finite-dimensional central simple algebra. By Wedderburn's theorem, 
    $D\otimes_KE\simeq M_p(F)$ for some division $K$-algebra $F$ and some $p$. This implies that
    \[
    A\otimes_KB\simeq M_{nmp}(F).
    \]
    Similarly, $A_1\otimes_KB_1\simeq M_{n_1m_1p}(F)$ and thus $A\otimes_KB\sim A_1\otimes_KB_1$. 
    
    Now we need to prove that $\operatorname{Br}(K)$ is a group. The multiplication \eqref{eq:Brauer}
    is associative and commutative since the tensor product $\otimes_K$ is associative 
    and multiplicative. The identity of $\operatorname{Br}(K)$ is
    $[K]$, as $[A][K]=[A\otimes_KK]=[A]$. Finally, the inverse of $[A]$ is $[A^{\op}]$, as 
    \[
    [A][A^{\op}]=[A\otimes_KA^{\op}]=[M_n(K)].\qedhere
    \]
\end{proof}

\begin{exercise}
    Let $D$ be a division algebra. Compute the center of $M_n(D)$. 
\end{exercise}

Let us compute some examples:

\begin{proposition}
    $\operatorname{Br}(\C)=\{0\}$.
\end{proposition}

\begin{proof}
    Let $A$ be a complex central simple algebra. Then $A\simeq M_n(D)$ for 
    some complex division algebra $D$. We claim that $D\simeq\C$. Let $m=\dim D$ and $\alpha\in D$. 
    Since $\{1,\alpha,\dots,\alpha^m\}$ has $m+1$ elements, it is a linearly dependent set. This means
    that there exists $\lambda_0,\dots,\lambda_m\in\C$ not all zero such that 
    $0=\sum_{i=0}^m\lambda_i\alpha^i$. Thus the non-zero polynomial $f=\sum_{i=0}^m\lambda_iX^i\in\C[X]$
    is such that $f(\alpha)=0$. Since $\C$ is algebraically closed, 
    there exist $\alpha_0,\dots,\alpha_N\in\C$ and $a\in\C\setminus\{0\}$ such that 
    \[
    f=a\prod_{i=0}^N(X-\alpha_i).
    \]
    Since $D$ is a division algebra, there exists $i\in\{0,\dots,m\}$ such that
    $\alpha=\alpha_i$. In particular, $\alpha\in\C$. Therefore $[A]=[\C]$
    and hence $\operatorname{Br}(A)=\{0\}$. 
\end{proof}

An application of Wedderburn's little theorem:

\begin{proposition}
    Let $F$ be a finite field. Then 
    $\operatorname{Br}(F)=\{0\}$.
\end{proposition}

\begin{proof}
    Let $A$ be a central simple algebra over $F$. Then 
    $A\simeq M_n(D)$ for some division $F$-algebra $D$. Since $\dim_FD<\infty$
    and $F$ is finite, $F=Z(A)\simeq Z(M_m(D))\simeq Z(D)=D$ 
    by Wedderburn's little theorem 
    and hence $[A]=[F]$. 
\end{proof}

An application of Frobenius' theorem:

\begin{proposition}
    $\operatorname{Br}(\R)$ is the cyclic group of order two. 
\end{proposition}

\begin{proof}
    Let $A$ be a central simple real algebra. Then $A\simeq M_n(D)$ 
    where either $D\simeq\R$ or $D\simeq\mathbb{H}$ by Frobenius' theorem, as 
    \[
    \R\simeq Z(A)\simeq Z(M_n(D))\simeq Z(D)
    \]
    and $\Z(\C)=\C$. Thus $\operatorname{Br}(\R)$ has only two elements, that is
    $\operatorname{Br}(\R)=\{[\R],[\mathbb{H}]\}$. 
\end{proof}


% \subsection{Brauer's group and cohomology}

Let $L/K$ be a Galois extension of degree $n$.  
Extending scalars we obtain a group homomorphism
\[
    \operatorname{res}\colon\operatorname{Br}(K)\to\operatorname{Br}(L),\quad
    [A]\mapsto [A\otimes_KL],
\]
known as the \emph{restriction homomorphism}. 

\begin{exercise}
    Prove that $\operatorname{res}$ is well-defined.
\end{exercise}

\begin{definition}
    Let $L/K$ be a Galois extension of degree $n$. 
    The \emph{restricted Brauer group} is $\operatorname{Br}(L/K)$ is defined
    as the kernel of the restriction homomorphism. 
\end{definition}

% Note that $\operatorname{Br}(L/K)$ is a subgroup of 
% $\operatorname{Br}(K)$. 

Recall that the Galois group $G$ of $L/K$ is a finite group. Let 
$Z^2(G,L^{\times})$ be the set of maps $\alpha\colon G\times G\to L^{\times}$
such that 
\[
\alpha(g,h)\alpha(gh,k)=g(\alpha(h,k))\alpha(g,hk)
\]
for all $g,h,k\in G$. 

We say that 
$\alpha\in Z^2(G,L^{\times})$ and $\beta\in Z^2(G,L^{\times})$ 
are equivalent if and only if
there exists $\{\delta_g:g\in G\}\subseteq L$ such that 
\[
\beta(g,h)=\delta_gg(\delta_h)\alpha(g,h)\delta^{-1}_{gh}
\]
for all $g,h\in G$. 

The second cohomology group $H^2(G,L^{\times})$ is defined as the
set of equivalence classes of $Z^2(G,L^{\times})$. One proves that
$H^2(G,L^{\times})$ is indeed an abelian group. 

\begin{exercise}
Let $G$ be a finite group. 
For $\alpha\in Z^2(G,L^{\times})$ let us consider the 
crossed product $L_t^{\alpha}G$ of $G$ by $K$ 
given by 
\[
L_t^{\alpha}G=\left\{\sum_{g\in G}\lambda_ge_g:\lambda_g\in L\right\}.
\]
\begin{enumerate}
    \item Prove that the product 
\[
    (\lambda_ge_g)(\lambda_he_h)=\lambda_gg(\lambda_y)\alpha(g,h)e_{gh}.
\]
is associative. 
\item Prove that $e=\alpha(1,1)^{-1}e_1$ is such that 
$ee_g=e_ge=e_g$ for all $g\in G$. 
\item Prove that 
each $e_g$ is invertible with inverse 
\[
e_g^{-1}=\alpha(g^{-1},g)^{-1}\alpha(1,1)^{-1}e_{g^{-1}}.
\]
\end{enumerate}
\end{exercise}

% \begin{exercise}
%     Prove that 
% \end{exercise}

\begin{theorem}
    Let $L/K$ be a Galois extension of degree $n$ and group $G$. Then
    \[
    \operatorname{Br}(L/K)\simeq H^2(G,L^{\times}).
    \]
\end{theorem}

The isomorphism of the theorem is given by 
    \[
    H^2(G,L^{\times})\to\operatorname{Br}(L/K)\subseteq\operatorname{Br}(K),
    \quad
    [\alpha]\mapsto [L_t^{\alpha}G],
    \]
We do not have time to prove the theorem in detail, as it requires
some tools that are outside the scope of our course. 

\begin{corollary}
    $\operatorname{Br}(K)$ is a torsion group.
\end{corollary}

\begin{proof}[Sketch of the proof]
    The theorem implies that for every finite Galois extension $L/K$
    one has $\operatorname{Br}(L/K)\simeq H^2(G,L^{\times})$ is a torsion group, 
    as $|G|H^2(G,L^{\times})=\{0\}$. To finish the proof note that
    $\operatorname{Br}(K)=\bigcup\operatorname{Br}(L/K)$, where the union 
    is taken over all finite Galois extensions $L/K$. 
\end{proof}

The theorem can be used to compute Brauer groups. Let us give an example. We know
that $\C/\R$ is a Galois extension with Galois group isomorphic to $\Z/2$. Thus
\[
\operatorname{Br}(\R)=\operatorname{Br}(\C/\R)\simeq H^2(\Z/2,\C^{\times})\simeq\Z/2.
\]


% \begin{proof}[Sketch of the proof]
%     We need to show that the map
%     \[
%     H^(G,L^{\times})\to\operatorname{Br}(L/K)\subseteq\operatorname{Br}(K),
%     \quad
%     [\alpha]\mapsto [L_t^{\alpha}G],
%     \]
%     is a group isomorphism. We need to show that the map is well-defined: if $[\alpha]=[\beta]$, then
%     $\alpha$ and $\beta$ are related by $\{\delta_g:g\in G\}$. Thus 
%     the map $e_g\mapsto\delta_ge_g$ induces an isomorphism $L_t^{\alpha}G\to L_t^{\beta}G$ and, in particular, 
%     $L_t^{\alpha}G\sim L_t^{\beta}G$. 
% \end{proof}
