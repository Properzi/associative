\chapter{}

\topic{Frobenius's theorem}


\begin{theorem}[Frobenius]
	\label{thm:Frobenius}
	\index{Frobenius'!theorem}
	Every finite-dimensioanal real division algebra is isomorphic to $\R$, $\C$
	or $\H$.
\end{theorem}

We present an elementaly proof. We shall need some lemmas. 

\begin{lemma}
	\label{lem:trick_frobenius1}
	Let $D$ be a real division algebra such that $\dim D=n$. If $x\in D$, then
	there exists $\lambda\in\R$ such that $x^2+\lambda x\in\R$.
\end{lemma}

\begin{proof}
	Since $\dim D=n$, the set $\{1,x,x^2,\dots,x^n\}$ is linearly dependent. So
	there exists a non-zero polynomial $f(X)\in\R[X]$ of degree $\leq n$ such
	that $f(x)=0$. Without loss of generality we may assume that the leading
	coefficient of $f(X)$ is one. Then we can write $f(X)$ as a product of
	polynomials of degree $\leq2$, say 
	\[
		f(X)=(X-\alpha_1)\cdots (X-\alpha_r)(X^2+\lambda_1 X+\mu_1)\cdots (X^2+\lambda_s X+\mu_s).
	\]
	Since $D$ is a division algebra and $f(x)=0$, some factor of $f(X)$ is
	zero. If $x-\lambda_j\ne 0$ for all $j$, then $x$ is a root of some
	$X^2+\lambda_k X+\mu_k$. In any case, there exists $\lambda\in\R$ such that
	$x^2+\lambda x\in\R$. 
\end{proof}

\begin{lemma}
	\label{lem:trick_frobenius2}
	Let $D$ be a real division algebra of dimension $n$. Then
	\[
		V=\{x\in D:x^2\in\R,x^2\leq 0\}
	\]
	is a subspace of $D$ such that $D=\R\oplus V$.
\end{lemma}

\begin{proof}
	Let $x\in D\setminus V$ be such that $x^2\in\R$. Since $x^2>0$, it follows
	that $x^2=\alpha^2$ for some $\alpha\in\R$. Thus $x=\pm\alpha\in\R$, as $D$
	is a division algebra and $(x-\alpha)(x+\alpha)=x^2-\alpha^2=0$.

	We claim that $V$ is a subspace of $D$.  Note that $0\in V$ and 
	that if $x\in V$, then $\lambda x\in V$ for all $\lambda\in\R$.  Let 
	$x,y\in V$. If $\{x,y\}$ is linearly dependent, then $x+y\in V$.
	If not, we claim that 
	$\{1,x,y\}$ is linearly independent. If there exist 
	$\alpha,\beta,\gamma\in\R$ such that $\alpha x+\beta y+\gamma=0$, then 
	\[
	\alpha^2x^2=\beta^2y^2+2\beta\gamma y+\gamma^2=(-\beta y-\gamma)^2.
	\]
	This implies that $2\beta\gamma y\in\R$ and thus $\beta\gamma=0$. Hence 
	$\alpha=\beta=\gamma=0$. The previous lemma implies that there exist 
	$\lambda,\mu\in\R$ such that 
	\[
		(x+y)^2+\lambda(x+y)\in\R,\quad
		(x-y)^2+\mu(x-y)\in\R.
	\]
	Since 
	\[
		(x+y)^2+(x-y)^2=2x^2+2y^2\in\R,
	\]
	it follows that $(\lambda+\mu)x+(\lambda-\mu)y\in\R$. Since  $\{1,x,y\}$ is linearly 
	independent, 
	$\lambda=\mu=0$. Thus $(x+y)^2\in\R$. If 
	$x+y\not\in V$, then, the first paragraph of the proof implies that 
	$x+y\in\R$, a constradiction. 

	Clearly, $\R\cap V=0$. If $x\in D\setminus\R$, then the previous lemma 
	implies that $x^2+\lambda x\in\R$ for some 
	$\lambda\in\R$. We claim that $x+\lambda/2\in V$. If not, since 
	\[
	(x+\lambda/2)^2=x^2+\lambda x+(\lambda/2)^2\in\R,
	\]
	it follows that $x+\lambda/2\in\R$ and thus $x\in\R$. Hence 
	$x=-\lambda/2+(x+\lambda/2)\in\R\oplus V$.
\end{proof}

\begin{lemma}
	\label{lem:trick_frobenius3}
	Let $D$ be a real algebra of (real) dimension $n$. If $n>2$, then
	there exist $i,j,k\in D$ such that $\{1,i,j,k\}$ is linearly independent and 
	\begin{align}
	\label{eq:H}
	&i^2=j^2=k^2=-1, && ij=-ji=k, && ki=-ik=j, && jk=-kj=i.
	\end{align}
\end{lemma}

\begin{proof}
	Let $V=\{x\in D:x^2\in\R,x^2\leq 0\}$ be the subspace of Lemma~\ref{lem:trick_frobenius2}. 
	For $x,y\in V$ let $x\circ
	y=xy+yx=(x+y)^2-x^2-y^2\in\R$. If $x\ne0$, then $x\circ
	x=2x^2\ne0$. Since $\dim V=n-1$, there exist $y,z\in V$ such that $\{y,z\}$ is 
	linearly independent. Let 
	\[
		x=z-\frac{z\circ y}{y\circ y}y.
	\]
	Since $\{y,z\}$ is linearly independent, $x\ne0$. Moreover, since 
	\[
		x\circ y
		=\left(z-\frac{z\circ y}{y\circ y}\right)\circ y
		=zy-\frac{z\circ y}{y\circ y}y^2+yz-\frac{z\circ y}{y\circ y}y^2
		=z\circ y-\frac{z\circ y}{y\circ y}y\circ y=0,
	\]
	it follows that $xy=-yx$. 
	Let  
	\[
		i=\frac{1}{\sqrt{-x^2}}x,
		\quad
		j=\frac{1}{\sqrt{-y^2}}y,
		\quad
		k=ij. 
	\]
	A direct calculation shows that the formulas of ~\eqref{eq:H} hold. For example, 
	\[
		ji=\frac{1}{\sqrt{-y^2}}\frac{1}{\sqrt{-x^2}}yx=\frac{1}{\sqrt{-x^2}}\frac{1}{\sqrt{-y^2}}(-xy)=-k.\qedhere
	\]
\end{proof}

Now we are finally 
ready to prove the theorem: 

\begin{proof}[Proof of \ref{thm:Frobenius}]
	Let $D$ be a real division algebra and let $n=\dim D$. If $n=1$, then 
	$D\simeq\R$. If $n=2$, the subspace $V$ of Lemma~\ref{lem:trick_frobenius2} 
	is non-zero and thus there exists $i\in D$ such that 
	$i^2=-1$. Hence $D\simeq\C$. Lemma~\ref{lem:trick_frobenius3}
	implies that $n\ne3$. If $n=4$, then $D\simeq\H$. Suppose that 
	$n>4$.  By Lemma~\ref{lem:trick_frobenius3} there exist
	$i,j,k\in D$ such that $\{1,i,j,k\}$ is linearly independent 
	and that the formulas of~\eqref{eq:H} hold. Let 
	\[
		V=\{x\in D:x^2\in\R,x^2\leq 0\}.
	\]
	By Lemma~\ref{lem:trick_frobenius2}, $\dim V=n-1$. Thus there exists 
	$x\in V\setminus\langle i,j,k\rangle$. Let 
	\[
		e=x+\frac{i\circ x}{2}i+\frac{j\circ x}{2}j+\frac{k\circ x}{2}k\in V\setminus\{0\}.
	\]
	A direct calculation shows that $i\circ e=j\circ e=k\circ e=0$. Then 
	\[
		ek=e(ij)=(ei)j=-(ie)j=-i(ej)=i(je)=(ij)e=ke,
	\]
	a contradiction. 
\end{proof}

%\section{El pequeño teorema de Wedderburn}
\topic{Wedderburn's little theorem}

Vamos a dar una demostración completamente elemental de un famoso teorema de
Wedderburn.  Antes necesitamos repasar algunos conceptos básicos sobre
polinomios ciclotómicos.

\begin{definition}
	\index{Polinomio ciclotómico}
	El $n$-polinomio ciclotómico se define como
	\begin{equation}
		\label{eq:ciclotomico}
		\Phi_n(X)=\prod(X-\zeta),
	\end{equation}
	donde el producto se hace sobre todas las $n$-raíces primitivas de la
	unidad. 
\end{definition}

\begin{example}
	Veamos algunos ejemplos:
	\begin{align*}
		&\Phi_2=X-1,\\
		&\Phi_3=X^2+X+1,\\
		&\Phi_4=X^2+1,\\
		&\Phi_5=X^4+X^3+X^2+X+1,\\
		&\Phi_6=X^2-X+1,\\
		&\Phi_7=X^6+X^5+\cdots+X+1.
	\end{align*}
\end{example}

\begin{lemma}
	Sea $n\in\Z_{>0}$. Entonces 
	\[
		X^n-1=\prod_{d\mid n}\Phi_d(X).
	\]
\end{lemma}

\begin{proof}
	Escribimos
	\[
		X^n-1=\prod_{j=1}^n (X-e^{2\pi ij/n})
		=\prod_{d\mid n}\prod_{\substack{1\leq j\leq n\\\gcd(j,n)=d}}(X-e^{2\pi ij/n})
		=\prod_{d\mid n}\Phi_d(X).
	\]
\end{proof}

\begin{lemma}
	Sea $n\in\Z_{>0}$. Entonces $\Phi_n(X)\in\Z[X]$.
\end{lemma}

\begin{proof}
	Procederemos por inducción en $n$. El caso $n=1$ es trivial pues
	$\Phi_1(X)=X-1$. Supongamos entonces $\Phi_d(X)\in\Z[X]$ para todo $d<n$.
	Entonces 
	\[
		\prod_{d\mid n,d\ne n}\Phi_d(X)\in\Z[X]
	\]
	y es un polinomio mónico. Luego $\Phi_n(X)/\prod_{d\mid
	n,d<n}\Phi_d(X)\in\Z[X]$.
\end{proof}

\begin{theorem}[Wedderburn]
	\index{Teorema!de Wedderburn}
	Todo anillo de división finito es un cuerpo. 
\end{theorem}

\begin{proof}
	Sea $K=Z(D)$. Entonces $K$ es un cuerpo finito, digamos $|K|=q$. Sea
	$n=\dim_KD$.  Vamos a demostrar que $n=1$. Supongamos que $n>1$. 
	La ecuación de clases para el grupo $D^\times=D\setminus\{0\}$ implica que
	\begin{equation}
		\label{eq:clases}
		q^n-1=q-1+\sum_{j=1}^m \frac{q^n-1}{q^{d_j}-1},
	\end{equation}
	donde $1<\frac{q^n-1}{q^{d_j}-1}\in\Z$ para todo $j\in\{1,\dots,m\}$. 
	Como $d^{d_j}-1$ divide a $q^n-1$, cada $d_j$ divide a $n$. En particular,
	la fórmula~\eqref{eq:ciclotomico} implica que podemos escribir
	\begin{equation}
		\label{eq:trick_ciclotomico}
		X^n-1=\Phi_n(X)(X^{d_j}-1)h(X)
	\end{equation}
	para algún polinomio $h(X)\in\Z[X]$. 
	Al evaluar~\eqref{eq:trick_ciclotomico} en $X=q$  
	obtenemos que $\Phi_n(q)$ divide a $q^n-1$ y que $\Phi_n(q)$
	divide a $\frac{q^n-1}{q^{d_j}-1}$. Entonces, por~\eqref{eq:clases}, 
	$\Phi_n(q)$ divide a $q-1$. Luego 
	\[
		q-1\geq |\Phi_n(q)|=\prod |q-\zeta|>q-1
	\]
	pues cada $|q-\zeta|>q-1$ (basta dibujar $q$ y $\zeta$ en el plano
	complejo), una contradicción.
\end{proof}

Veamos como corolario una aplicación al último teorema de Fermat en anillos
finitos. Demostraremos el siguiente resultado:

\begin{theorem}
	Sea $R$ un anillo unitario finito. Entonces para todo $n\geq1$ existen $x,y,z\in
	R\setminus\{0\}$ tales que $x^n+y^n=z^n$ si y sólo si $R$ no es un anillo
	de división.
\end{theorem}

\begin{proof}
	Supongamos primero que $R$ es de división. Por el teorema de Wedderburn,
	$R$ es entonces un cuerpo finito, digamos $|R|=q$. Como entonces
	$x^{q-1}=1$ para todo $x\in R\setminus\{0\}$, se concluye que la ecuación
	$x^{q-1}+y^{q-1}=z^{q-1}$ no tiene solución.

	Supongamos ahora que $R$ no es de división. Como entonces, en particular,
	$R$ no es un cuerpo, $|R|>2$ y luego $x+y=z$ tiene solución en
	$R\setminus\{0\}$ (tomar por ejemplo $x=1$, $y=z-1$ y $z\not\in\{0,1\}$).
	Como $R$ es finito, $R$ es artiniano a izquierda y entonces el radical de
	Jacobson $J(R)$ es nilpotente. Si $J(R)\ne 0$, existe entonces $a\in
	R\setminus\{0\}$ tal que $a^2=0$ y luego $a^n=0$ para todo $n\geq2$. En
	este caso, la ecuación $x^n+y^n=z^n$ tiene solución en $R\setminus\{0\}$ si
	$n\geq 2$ (tomar por ejemplo $x=a$, $y=z=1$). Si $J(R)=0$, entonces, $R$ es
	semisimple y luego, por el teorema de Wedderburn,
	\[
		R\simeq \prod_{i=1}^k M_{n_i}(D_i)
	\]
	donde los $D_i$ son cuerpos finitos (por ser anillos de división finitos).
	Como $R$ no es un cuerpo, hay dos posibilidades: o bien $n_i>1$ para algún
	$i\in\{1,\dots,k\}$, o bien $k\geq 2$ y $n_i=1$ para todo
	$i\in\{1,\dots,k\}$. En el primer caso, como $M_{n_i}(D_i)$ tiene elementos
	no nulos cuyo cuadrado es cero, $R$ también los tiene, y luego, tal como se
	hizo antes, vemos que $x^n+y^n=z^n$ tiene solución. En el segundo caso,
	$x=(1,0,0,\dots,0)$, $y=(0,1,0,\dots,0)$ y $z=(1,1,0,\dots,0)$ es una
	solución de $x^n+y^n=z^n$.
\end{proof}
