\chapter{}

\topic{Frobenius's theorem}

\begin{theorem}[Frobenius]
	\label{thm:Frobenius}
	\index{Frobenius'!theorem}
	Every finite-dimensional real division algebra is isomorphic to $\R$, $\C$
	or $\H$.
\end{theorem}

We present an elementary proof. We shall need some lemmas. 

\begin{lemma}
	\label{lem:trick_frobenius1}
	Let $D$ be a real division algebra such that $\dim D=n$. If $x\in D$, then
	there exists $\lambda\in\R$ such that $x^2+\lambda x\in\R$.
\end{lemma}

\begin{proof}
	Since $\dim D=n$, the set $\{1,x,x^2,\dots,x^n\}$ is linearly dependent. So
	there exists a non-zero polynomial $f(X)\in\R[X]$ of degree $\leq n$ such
	that $f(x)=0$. Without loss of generality, we may assume that the leading
	coefficient of $f(X)$ is one. Then we can write $f(X)$ as a product of
	polynomials of degree $\leq2$, say 
	\[
		f(X)=(X-\alpha_1)\cdots (X-\alpha_r)(X^2+\lambda_1 X+\mu_1)\cdots (X^2+\lambda_s X+\mu_s).
	\]
	Since $D$ is a division algebra and $f(x)=0$, some factor of $f(X)$ is
	zero. If $x-\lambda_j\ne 0$ for all $j$, then $x$ is a root of some
	$X^2+\lambda_k X+\mu_k$. In any case, there exists $\lambda\in\R$ such that
	$x^2+\lambda x\in\R$. 
\end{proof}

\begin{lemma}
	\label{lem:trick_frobenius2}
	Let $D$ be a real division algebra of dimension $n$. Then
	\[
		V=\{x\in D:x^2\in\R,x^2\leq 0\}
	\]
	is a subspace of $D$ such that $D=\R\oplus V$.
\end{lemma}

\begin{proof}
	Let $x\in D\setminus V$ be such that $x^2\in\R$. Since $x^2>0$, it follows
	that $x^2=\alpha^2$ for some $\alpha\in\R$. Thus $x=\pm\alpha\in\R$, as $D$
	is a division algebra and $(x-\alpha)(x+\alpha)=x^2-\alpha^2=0$.

	We claim that $V$ is a subspace of $D$.  Note that $0\in V$ and 
	that if $x\in V$, then $\lambda x\in V$ for all $\lambda\in\R$.  Let 
	$x,y\in V$. If $\{x,y\}$ is linearly dependent, then $x+y\in V$.
	If not, we claim that 
	$\{1,x,y\}$ is linearly independent. If there exist 
	$\alpha,\beta,\gamma\in\R$ such that $\alpha x+\beta y+\gamma=0$, then 
	\[
	\alpha^2x^2=\beta^2y^2+2\beta\gamma y+\gamma^2=(-\beta y-\gamma)^2.
	\]
	This implies that $2\beta\gamma y\in\R$ and thus $\beta\gamma=0$. Hence 
	$\alpha=\beta=\gamma=0$. The previous lemma implies that there exist 
	$\lambda,\mu\in\R$ such that 
	\[
		(x+y)^2+\lambda(x+y)\in\R,\quad
		(x-y)^2+\mu(x-y)\in\R.
	\]
	Since 
	\[
		(x+y)^2+(x-y)^2=2x^2+2y^2\in\R,
	\]
	it follows that $(\lambda+\mu)x+(\lambda-\mu)y\in\R$. Since  $\{1,x,y\}$ is linearly 
	independent, 
	$\lambda=\mu=0$. Thus $(x+y)^2\in\R$. If 
	$x+y\not\in V$, then, the first paragraph of the proof implies that 
	$x+y\in\R$, a contradiction. 

	Clearly, $\R\cap V=0$. If $x\in D\setminus\R$, then the previous lemma 
	implies that $x^2+\lambda x\in\R$ for some 
	$\lambda\in\R$. We claim that $x+\lambda/2\in V$. If not, since 
	\[
	(x+\lambda/2)^2=x^2+\lambda x+(\lambda/2)^2\in\R,
	\]
	it follows that $x+\lambda/2\in\R$ and thus $x\in\R$. Hence 
	$x=-\lambda/2+(x+\lambda/2)\in\R\oplus V$.
\end{proof}

\begin{lemma}
	\label{lem:trick_frobenius3}
	Let $D$ be a real algebra of (real) dimension $n$. If $n>2$, then
	there exist $i,j,k\in D$ such that $\{1,i,j,k\}$ is linearly independent and 
	\begin{align}
	\label{eq:H}
	&i^2=j^2=k^2=-1, && ij=-ji=k, && ki=-ik=j, && jk=-kj=i.
	\end{align}
\end{lemma}

\begin{proof}
	Let $V=\{x\in D:x^2\in\R,x^2\leq 0\}$ be the subspace of Lemma~\ref{lem:trick_frobenius2}. 
	For $x,y\in V$ let $x\circ
	y=xy+yx=(x+y)^2-x^2-y^2\in\R$. If $x\ne0$, then $x\circ
	x=2x^2\ne0$. Since $\dim V=n-1$, there exist $y,z\in V$ such that $\{y,z\}$ is 
	linearly independent. Let 
	\[
		x=z-\frac{z\circ y}{y\circ y}y.
	\]
	Since $\{y,z\}$ is linearly independent, $x\ne0$. Moreover, since 
	\[
		x\circ y
		=\left(z-\frac{z\circ y}{y\circ y}\right)\circ y
		=zy-\frac{z\circ y}{y\circ y}y^2+yz-\frac{z\circ y}{y\circ y}y^2
		=z\circ y-\frac{z\circ y}{y\circ y}y\circ y=0,
	\]
	it follows that $xy=-yx$. 
	Let  
	\[
		i=\frac{1}{\sqrt{-x^2}}x,
		\quad
		j=\frac{1}{\sqrt{-y^2}}y,
		\quad
		k=ij. 
	\]
	A direct calculation shows that the formulas of ~\eqref{eq:H} hold. For example, 
	\[
		ji=\frac{1}{\sqrt{-y^2}}\frac{1}{\sqrt{-x^2}}yx=\frac{1}{\sqrt{-x^2}}\frac{1}{\sqrt{-y^2}}(-xy)=-k.\qedhere
	\]
\end{proof}

Now we are finally 
ready to prove the theorem: 

\begin{proof}[Proof of \ref{thm:Frobenius}]
	Let $D$ be a real division algebra and let $n=\dim D$. If $n=1$, then 
	$D\simeq\R$. If $n=2$, the subspace $V$ of Lemma~\ref{lem:trick_frobenius2} 
	is non-zero and thus there exists $i\in D$ such that 
	$i^2=-1$. Hence $D\simeq\C$. Lemma~\ref{lem:trick_frobenius3}
	implies that $n\ne3$. If $n=4$, then $D\simeq\H$. Suppose that 
	$n>4$.  By Lemma~\ref{lem:trick_frobenius3} there exist
	$i,j,k\in D$ such that $\{1,i,j,k\}$ is linearly independent 
	and that the formulas of~\eqref{eq:H} hold. Let 
	\[
		V=\{x\in D:x^2\in\R,x^2\leq 0\}.
	\]
	By Lemma~\ref{lem:trick_frobenius2}, $\dim V=n-1$. Thus there exists 
	$x\in V\setminus\langle i,j,k\rangle$. Let 
	\[
		e=x+\frac{i\circ x}{2}i+\frac{j\circ x}{2}j+\frac{k\circ x}{2}k\in V\setminus\{0\}.
	\]
	A direct calculation shows that $i\circ e=j\circ e=k\circ e=0$. Then 
	\[
		ek=e(ij)=(ei)j=-(ie)j=-i(ej)=i(je)=(ij)e=ke,
	\]
	a contradiction. 
\end{proof}

%\section{El pequeño teorema de Wedderburn}


\topic{Jacobson's commutativity theorem}

\begin{exercise}
    \index{Ring!boolean}
    A ring $R$ is \textbf{boolean} if $x^2=x$ for all $x\in R$. 
    Prove that boolean rings are commutative. 
\end{exercise}

To prove this fact, note that $1=(-1)^2=-1$. This means that $R$ has
characteristic two. Let $x,y\in R$. Since $x+y=(x+y)^2=x^2+xy+yx+y^2$. 
it follows that $0=xy+yx$ and hence $xy=yx$.

\begin{definition}
	\index{Ring!reduced}
	A ring $R$ is \textbf{reduced} if $x^2=0$ implies $x=0$. 
\end{definition}

For example, boolean rings and domains are reduced. Moreover, the ring 
$\Z^n$ with point-wise multiplication is reduced (and has zero divisors).

\begin{exercise}
    Prove that idempotents of reduced rings are central. 
\end{exercise}
% 	Todo idempotente de un anillo reducido es central.

The previous exercise is used to solve the following problem.

\begin{exercise}
    Let $R$ be a ring such that $x^3=x$ for all $x\in R$. Prove that
    $R$ is commutative. 
\end{exercise}

This exercise is harder. Even harder is the following exercise:

\begin{exercise}
    Let $R$ be a ring such that $x^4=x$ for all $x\in R$. Prove
    that $R$ is commutative. 
\end{exercise}

Other exercises about reduced rings. 

\begin{exercise}
\label{xca:reduced}
    Prove that a ring is reduced if 
    and only it has no non-zero nilpotent elements. 
\end{exercise}

\begin{exercise}
\label{xca:domain<=>prime+reduced}
    A ring is a domain if and only if
    it is both prime and reduced. 
\end{exercise}

\begin{exercise}
\label{xca:reduced=>semiprime}
    Reduced rings are semiprime.
\end{exercise}
    
In this lecture, we will use
structure theorems to prove the following amazing (and quite useless) 
beautiful result:

\begin{theorem}[Jacobson]
\label{thm:commutativity}
\index{Jacobson's commutativity theorem}
	Let $R$ be a ring such that 
	for each $x\in R$ there exists $n(x)\geq2$ such that 
	$x^{n(x)}=x$. Then $R$ is commutative. 
\end{theorem}

We shall need the following lemma.

% The structure 
% of reduced rings is described by the following result:

% \begin{theorem}[Andrnakievich--Ryabukhin]
% \index{Andrnakievich--Ryabukhin's theorem}
% 	Let $R$ be a non-zero ring. Then $R$ is reduced if and only if
% 	$R$ is a subdirect product of domains. 
% \end{theorem}
%https://ysharifi.wordpress.com/2010/06/04/about-reduced-rings-2/
% \begin{proof}
% 	Supongamos que $R$ es reducido.\framebox{?}

% 	Supongamos ahora que $R$ es producto subdirecto de la familia $\{R_i:i\in I\}$ de dominios. Sea 
% 	$f\colon R\to \prod_{i\in I}R_i$, $f(x)=(x_i)_{i\in I}$, el morfismo inyectivo. 
% 	Si $x\in R$ es tal que $x^2=0$ entonces 
% 	\[
% 		(0)_{i\in I}=f(0)=f(x^2)=f(x)^2=(x_i^2)_{i\in I}
% 	\]
% 	y luego, como cada $R_i$ es un dominio, se concluye que $x_i=0$ para todo
% 	$i\in I$.
% \end{proof}

% Let x,y \in R. Suppose that R is a domain. Then R is reduced and if xRy=(0), then xy=0 and thus, since R is a domain, either x=0 or y=0. So R is a prime ring. Now suppose that R is both prime and reduced and xy=0. Then, by Remark 1, yRx=(0) and so, since R is prime, either y=0 or x=0. To prove that every reduced ring is semiprime, suppose that xRx=(0) for some x \in R. Then x^2 \in xRx =(0) and so x=0. \ \Box    

% \begin{lemma}
% 	\label{lem:reducido}
% 	Todo idempotente de un anillo reducido es central.
% \end{lemma}

% \begin{proof}
% 	Sea $e\in R$ tal que $e^2=e$ y sea $x\in R$. Como
% 	\[
% 	(ex-exe)^2=exex-exexe-exex+exexe=0
% 	\]
% 	y el anillo $R$ es reducido, $ex=exe$. 
% 	Similarmente la igualdad 
% 	$(xe-exe)^2=0$ implica que $xe=exe$. Luego $ex=exe=xe$ para todo $x\in R$.
% \end{proof}


% \begin{proposition}
% 	Sea $R$ un anillo tal que $x^3=x$ para todo $x\in R$. Entonces $R$ es
% 	conmutativo.
% \end{proposition}

% \begin{proof}
% 	El anillo $R$ es reducido pues si $x^2=0$ entonces $x=x^3=0$.  Como en $R$
% 	todo cuadrado es idempotente (pues $x^2=x^4=(x^2)^2$ para todo $x\in R$),
% 	el lema~\ref{lem:reducido} implica que todo cuadrado es central. 
% 	Si $x\in R$, entonces 
% 	$2x=(x^2+x)-2x^2$ 
% 	es central. Como además 
% 	\[
% 		1+x=(1+x)^3=1+3x+3x^2+x,
% 	\]
% 	se tiene que $3x=-3x^2$ es un elemento central. Luego $x=3x-2x$ es central.
% \end{proof}



% \begin{proposition}
% 	Sea $R$ un anillo tal que $x^4=x$ para todo $x\in R$. Entonces $R$ es conmutativo.
% \end{proposition}

% \begin{proof}
% 	El anillo $R$ es reducido pues si $x^2=0$ entonces $x=x^4=0$.  Como
% 	$x=x^4=(-x)^4=-x$ para todo $x\in R$, el anillo $R$ tiene característica
% 	dos. Todo elemento de la forma $z^2+z$ es idempotente pues
% 	$(z^2+z)^2=z^4+2z^3+z^2=z^2+z$. Luego, por el 
% 	lema~\ref{lem:reducido}, todo elemento de la forma $z^2+z$ es central. Si $x,y\in R$, entonces 
% 	\[
% 		x^2y+yx^2=(x^2+y)^2-(x^2+y)-(x^2+x)-(y^2+y)
% 	\]
% 	es central. Como entonces $x^2(x^2y+yx^2)=(x^2y+yx^2)x^2$, se concluye que $xy=yx$.
% \end{proof}

% Parece difícil poder extender los resultados anteriores a otros exponentes. Antes 
% de enunciar y demostrar un teorema de Jacobson que generaliza los resultados mencionados, necesitamos
% dos lemas:

\begin{lemma}
	\label{lem:k_finito}
	Let $K$ be a finite field of characteristic $p>0$. There exists 
	$n\in\Z_{>0}$ such that $|K|=p^n$ and $x^{p^n}=x$ for all $x\in K$. Moreover, 
	if $K\setminus\{0\}=\{x_1,\dots,x_{p^n-1}\}$, then 
	$X^{p^n}-X=(X-x_1)\cdots(X-x_{p^n-1})X$. 
\end{lemma}

\begin{proof}
	The field $K$ is a $(\Z/p)$-vector space. If $\dim_{\Z/p} K=n$, then 
	$|K|=p^n$. In particular, $K\setminus\{0\}$ is an abelian
	group of order $p^{n}-1$ and hence, by Lagrange's theorem, 
	$x^{p^n-1}=1$ for all $x\in K\setminus\{0\}$. Thus $x^{p^n}=x$ for all $x\in K$
	and hence every $x\in K$ is a root of the polynomial 
	$X^{p^n}-X$ of degree $p^n$. 
\end{proof}

Let $R$ be a ring. For each $r\in R$ the map $\ad{r}\colon
R\to R$, $x\mapsto rx-xr$, is a derivation. This means that
$\ad(xy)=(\ad x)y+x(\ad y)$ for all $x,y\in R$. 
By induction one proves that 
\begin{equation}
	\label{eq:Leibniz}
	(\ad{r})^n(x)=\sum_{k=0}^n(-1)^k\binom{n}{k}r^{n-k}xr^k
\end{equation}
for all $x\in R$ and $n\in\Z_{>0}$. If $p$ 
is a prime number, 
$p$ divides $\binom{p}{k}$ for all $k\in\{1,\dots,p-1\}$. This fact
is needed to solve the following exercise:

\begin{exercise}
    Let $p$ be a prime number and $R$ be a ring of characteristic $p$. 
    Prove that $(\ad{r})^{p^n}=\ad{r^{p^n}}$. 
\end{exercise}

% \begin{lemma}
% 	\label{lem:Jacobson}
% 	Sean $p$ un primo, 
% 	$R$ un anillo de característica $p$ y $r\in R$. 
% 	Entonces $(\ad{r})^{p^n}=\ad{r^{p^n}}$. 
% \end{lemma}

% \begin{proof}
% 	Procederemos por inducción en $n$. Supongamos que $n=1$. Si $k\in\{1,\dots,p-1\}$, entonces 
% 	$p$ no divide a $k!(p-k)!$ y luego $p$ divide a $\binom{p}{k}$. La igualdad~\eqref{eq:Leibniz} se
% 	transforma entonces en\dots
% \end{proof}

Now we are ready to prove Jacobson's commutativity theorem. 

\begin{proof}[Proof of Theorem \ref{thm:commutativity}]
    We divide the proof in several steps and claims. We may assume that 
    $R$ is non-zero. 
    
    \begin{claim}
        $J(R)=\{0\}$. 
    \end{claim}
    
    Let $x\in J(R)$ and $n=n(x)$. Since $-x^{n-1}\in J(R)$, 
    there exists $y\in R$ such that $-x^{n-1}\circ y=-x^{n-1}+y-x^{n-1}y=0$. Thus
    \[
    -x^{n-1}+y=x^{n-1}y\implies
    -x+xy=x(-x^{n-1}+y)=x^ny=xy.
    \]
    This implies that $x=0$.
    
    \begin{claim}
        Without loss of generality we may assume that $R$ is primitive. 
    \end{claim}
	
	Let $\{P_i:i\in I\}$ be the collection of primitive ideals of $R$. 
	The map 
	$R\to \prod_{i\in I}R/P_i$, $r\mapsto (r+P_i)_{i\in I}$, 
	is an injective homomorphism, since
	its kernel is 
	\[
	\bigcap_{i\in I} P_i=J(R)=\{0\}.
	\]
	Note that
	$R$ is commutative if and only if each $R/P_i$ is commutative. Moreover, 
	each $R/P_i$ 
	satisfies the assumption, that is 
	$(x+P_i)^{n(x)}=x^{n(x)}+P_i=x+P_i$, and 
	and is a primitive ring. 
	
    \begin{claim}
        $R$ is a division ring. 
    \end{claim}

    By Jacobson's density theorem,
    there exists a division ring $D$ and 
    a $D$-vector space $V$ such that 
    $R$ is dense in $V$. We claim that $\dim_DV=1$. If $\dim_DV\geq 2$, 
    let $\{v_1,v_2\}\subseteq V$ be a linearly independent set. Then
    there exists $f\in R$ such that $f(v_1)=v_2$ and $f(v_2)=0$. This implies
    that $f^{k}(v_1)=0$ for all $k\geq 2$ and $f(v_1)\ne 0$. This contradicts
    the fact that $f^{n}=f$ for $n=n(f)$. Thus $R\simeq D^{\op}$, 
    a division ring. 
    
    \begin{claim}
        $R$ has positive characteristic. 
    \end{claim}

    Since $R$ is a division ring, $2=1+1\in R$. There exists $n\geq2$ 
    such that $2^n=2$. In particular, 
    $2(2^{n-1}-1)=0$. This implies the claim. 
    
    \begin{claim}
        Every non-zero subring of $R$ is a division ring. 
    \end{claim}
    
    Let $S\subseteq R$ is a non-zero subring of $R$. If $x\in S\setminus\{0\}$, 
    then 
    $x^{n(x)}=x$. In particular, $x^{-1}=x^{n(x)-2}\in S$.
    
    \begin{claim}
        $R$ is commutative.
    \end{claim}
    
    Let us assume that $R$ is not commutative. Let $x\in R\setminus Z(R)$. 
    Since $R$ has positive characteristic, there exists $m>0$ such that 
    $mx=0$. Moreover, since $R$ is a division ring and 
    $x^{n(x)}=x$, it follows that $x^{n(x)-1}=1$. These facts imply that
    the subring $K$ of $R$ generated by
    $x$ is finite. By Wedderburn's theorem, $K$ is 
    a finite field. 
    Thus 
    $|K|=p^k$ for some prime number $p$ and some $k>0$ and 
    \[
    x^{p^k}=x.
    \]
    Note that $R$ is a $K$-vector space
    and $\delta=\ad x\colon R\to R$, $y\mapsto xy-yx$, is a $K$-linear map. Moreover, 
    by the lemma, 
    \[
    \delta^{p^k}=(\ad x)^{p^k}=\ad \left(x^{p^k}\right)=\ad x=\delta
    \]
    and 
    \begin{equation}
        \label{eq:delta}
        \delta(\delta-x_1\id)\cdots (\delta-x_{p^{k-1}\id})=0
    \end{equation}
    if $K=\{0,x_1,\dots,x_{p^k-1}\}$. Since $x$ is not central, 
    $\delta$ is non-zero. So there exists $y\in R$ such that $\delta(y)\ne 0$. 
    Evaluating \eqref{eq:delta} in $y$ and using that $R$ is a division ring 
    we obtain that 
    \[
    x_iy=\delta(y)=xy-yx
    \]
    for some $i$. Let $R_0$ be the subring of $R$ generated by $x$ and $y$. 
    Since $xy-yx=\delta(y)\ne 0$, the ring $R_0$ is a 
    non-commutative division ring. Note that 
    $yx=(x-x_i)y\in Ky$, as $x\in K$ and $x_i\in K$. By induction one proves
    that $yx^j\subseteq Ky$ for all $j\geq 1$ and hence
    $y^iK\subseteq Ky^i$ for
    all $i\geq1$. This implies that
    \[
    K+Ky+\cdots+Ky^{n(y)-2}\subseteq R
    \]
    is a subring. It follows that $K+Ky+\cdots+Ky^{n(y)-2}=R_0$, 
    as it is a subring of $R$ included in $R_0$ 
    that contains $x$ and $y$. Since 
    $R_0$ is a finite division ring, it is a field 
    by Wedderburn's theorem, a contradiction since
    it is non-commutative.
\end{proof}

% \topic{Andrunakevic--Rjabuhin's theorem}
%https://ysharifi.wordpress.com/2010/06/04/about-reduced-rings-1/
% \begin{exercise}
% \label{xca:reduced}
%     Let $R$ be a ring and $I$ be an ideal of $R$.
%     Prove that $I$ is prime if and only if $xRy\subseteq I$ implies
%     either $x\in I$ or $y\in I$. 
% \end{exercise}

% \begin{sol}{xca:reduced}
%     Let $A$ and $B$ be ideals such that $AB\subseteq I$. If 
%     $A\not\subseteq I$ and $B\not\subseteq J$, let 
%     $x\in A\setminus P$ and $y\in B\setminus P$. Then
%     $xRy\subseteq AB\subseteq I$, a contradiction. 
%     Conversely, if $xRy\subseteq I$ and $x\not I$ and $y\not\in I$, 
%     then $A=(x)\not\subseteq I$ and $B=(y)\not\subseteq P$.
% \end{sol}

% \begin{theorem}[Andrunakevic--Rjabuhin]
% \index{Andrunakevic--Rjabuhin's theorem}
% 	Let $R$ be a non-zero ring. Then $R$ is reduced if and only 
% 	if $R$ is a subdirect product of domains.
% \end{theorem}

% We shall need some lemmas. 

% \begin{lemma}
%     Let $P$ be a minimal prime ideal of $R$. Then 
%     $S=R\setminus P$ is multiplicatively closed if and only if
%     $s_1\cdots s_k\ne 0$ for all $k\geq 1$ and $s_1,\dots,s_k\in S$. 
% \end{lemma}

% \begin{proof}
%     Let $T=\{s_1\cdots s_k:k\geq 1,s_1,\dots,s_k\in S\}$. Clearly $T$ 
%     is multiplicatively closed and $S\subseteq T$. We claim that 
%     $T\subseteq S$. Let $X=\{I:\text{$I$ is an ideal of $R$ such that $I\cap T=\emptyset$}\}$. 
%     Then $X\ne\emptyset$, as $\{0\}\in X$. If $C$ is a chain in $X$, then
%     $\cup_{I\in C}I$ is an upper bound, so there exists a maximal element $Q\in X$ 
%     by Zorn's lemma. It is an exercise to show that $Q$ is prime. Since 
%     $Q\cap T=\emptyset$, it follows that $Q\cap S=\emptyset$, that is $Q\subseteq P$. 
%     Since $P$ is minimal, $P=Q\in X$. In particular, $T\subseteq S$. 
% \end{proof}

% \begin{lemma}
%     Let $P$ be a minimal prime ideal of $R$. 
%     Then $R/P$ is a domain.
% \end{lemma}

% \begin{proof}
    
% \end{proof}


% \begin{proof}
% 	If $R$ is reduced, then every prime ideal contains a minimal prime ideal. 
% 	Thus $\cap_{i\in I}P'_i=\{0\}$, where $\{P'_i:\in I\}$ is the collection
% 	of minimal prime ideals. Then each $R/P'_i$ is a domain and 
% 	there is an injective map $R\to\prod_{i\in I}R/P'_i$. 
	
% 	Supongamos ahora que $R$ es producto subdirecto de la familia $\{R_i:i\in I\}$ de dominios. Sea 
% 	$f\colon R\to \prod_{i\in I}R_i$, $f(x)=(x_i)_{i\in I}$, el morfismo inyectivo. 
% 	Si $x\in R$ es tal que $x^2=0$ entonces 
% 	\[
% 		(0)_{i\in I}=f(0)=f(x^2)=f(x)^2=(x_i^2)_{i\in I}
% 	\]
% 	y luego, como cada $R_i$ es un dominio, se concluye que $x_i=0$ para todo
% 	$i\in I$.
% \end{proof}

%\chapter{Álgebras centrales simples}

\section{Central algebras}

\begin{definition}
	Un álgebra se dice \textbf{central} si $Z(A)=k$. 
\end{definition}

\begin{proposition}
	Sean $A$ un álgebra unitaria y $n\in\N$. Entonces $A$ es central si y sólo si
	$M_n(A)$ es central.
\end{proposition}

\begin{proof}
	Si $M_n(A)$ es central y $z\in Z(A)$, entonces $zI\in Z(M_n(A))=kI$. Luego
	$z\in k$. Recíprocamente, si $X\in Z(M_n(A))$ entonces, como
	$XE_{kl}=E_{kl}X$ para todo $k\ne l$, $X=aI$ para algún $a\in A$. Además
	$XaE_{11}=aE_{11}X$ y luego $a\in Z(A)=k1$. 
\end{proof}

\begin{example}
	$M_n(k)$ es un álgebra central.
\end{example}

\begin{example}
	$\H$ es central como álgebra real.
\end{example}

\begin{example}
	$\C$ es central como álgebra compleja pero no lo es como álgebra real.
\end{example}

El teorema de Frobenius se traduce a este contexto de la siguiente forma:

\begin{theorem}
	Toda álgebra real de división, de dimensión finita y central 
	es isomorfa a $\R$ o $\H$.
\end{theorem}

\begin{proof}
	Es consecuencia inmediata del teorema de Frobenius~\ref{thm:Frobenius}.
\end{proof}

\begin{proposition}
	Todo anillo simple unitario es un álgebra sobre su centro.
\end{proposition}

\begin{proof}
	Sea $R$ un anillo simple unitario. Basta demostrar que $Z(R)$ es un cuerpo. Si $z\in
	Z(R)\setminus\{0\}$ entonces $zR$ es un ideal no nulo de $R$. Por la 
	simplicidad de $R$ se tiene $zR=R$. Luego $z$ es inversible. 
\end{proof}


\section{Skolem--Noether theorem}

Sea $A$ un álgebra. Consideramos las funciones $L\colon A\to\End_k(A)$,
$a\mapsto L_a$, y $R\colon A\to\End_k(A)$, $a\mapsto R_a$, dadas por
$L_a(x)=ax$ y $R_a(x)=xa$. Es fácil verificar que $L$ y $R$ son lineales y que
valen las siguientes fórmulas:
\begin{align*}
	L_{ab}=L_aL_b, && R_{ab}=R_bR_a, &&	L_aR_b=R_bL_a
\end{align*}
para todo $a,b\in A$.

\begin{definition}
	\index{Álgebra!de multiplicadores}
	Sea $A$ un álgebra. El \textbf{álgebra de multiplicadores} de $A$ se define como
	\[
		M(A)=\left\{\sum_{j=1}^n L_{a_i}R_{b_i}:n\in\N,\,a_1,\dots,a_n,b_1,\dots,b_n\in A\right\}.
	\]
\end{definition}

Obviamente $M(A)$ es una subálgebra de $\End_k(A)$. Más aún, si $A$ es
unitaria, entonces $M(A)$ está generada por los $L_a$ y los $R_b$, $a,b\in A$.

\begin{remark}
	\label{rem:SkolemNoether}
	Si $f\in M(A)$ existen $a_1,\dots,a_n,b_1,\dots,b_n\in A$ tales que 
	\[
		f=\sum_{i=1}^n L_{a_i}R_{b_i}
	\]
	y el conjunto $\{b_1,\dots,b_n\}$ es linealmente independiente. En efecto,
	si tomamos $n$ minimal entonces los $b_j$ son linealmente independientes:
	si $b_n=\sum_{j=1}^{n-1}\lambda_jb_j$ entonces
	$f=\sum_{i=1}^{n-1}L_{a_i+\lambda_ia_n}R_{b_i}$, que contradice la
	minimalidad de $n$.
\end{remark}

\begin{lemma}
	\label{lem:SkolemNoether1}
	Sea $A$ un álgebra central simple. Si $\sum_{i=1}^n L_{a_i}R_{b_i}=0$ y el
	conjunto $\{b_1,\dots,b_n\}$ (resp. $\{a_1,\dots,a_n\}$) es linealmente
	independiente, entonces $a_i=0$ (resp. $b_i=0$) para todo
	$i\in\{1,\dots,n\}$.
\end{lemma}

\begin{proof}
	Primero observemos que el resultado es válido para $n=1$. Queremos
	demostrar que si $a_1xb_1=0$ para todo $x\in A$ y $b_1\ne0$ entonces
	$a_1=0$. Supongamos que $a_1\ne 0$. Entonces el ideal de $A$ generado por
	$a_1$ es no nulo y luego es igual a $A$. Esto implica que existen
	$u_1,\dots,u_m,v_1,\dots,v_m\in A$ tales que $1=\sum_{j=1}^m u_ja_1v_j$.
	Podemos escribir entonces 
	\[
		0=\sum_{j=1}^m L_{u_j}(L_{a_1}R_{b_1})L_{v_j}=\sum_{j=1}^m L_{u_ja_1v_j}R_{b_1}=R_{b_1}
	\]
	y luego $b_1=0$. 

	Supongamos que el lema no es cierto y sea $n>1$ el menor entero positivo
	donde el lema es falso. Supongamos que $a_n\ne 0$. Como $A$ es simple, el
	ideal generado por $a_n$ es $A$ y luego existen
	$u_1,\dots,u_m,v_1,\dots,v_m\in A$ tales que $1=\sum_{j=1}^m u_ja_1v_j$.
	Entonces
	\[
		0=\sum_{j=1}^m L_{u_j}\left(\sum_{i=1}^n L_{a_i}R_{b_i}\right)L_{v_j}=\sum_{i=1}^n\sum_{j=1}^m L_{u_ja_iv_j}R_{b_i}=\sum_{i=1}^n L_{c_i}R_{b_i},
	\]
	donde $c_i=\sum_{j=1}^m u_ja_iv_j$ y obviamente $c_n=1$. Como 
	\[
		0=L_x\left(\sum_{i=1}^n L_{c_i}R_{b_i}\right)-\left(\sum_{i=1}^n L_{c_i}R_{b_i}\right)L_x=\sum_{i=1}^{n-1}L_{xc_i-c_ix}R_{b_i}
	\]
	para todo $x\in A$, la minimalidad de $n$ implica que $xc_i-c_ix=0$ para
	todo $x\in A$. Luego, como $A$ es central, $c_i\in k$ para todo
	$i\in\{1,\dots,n-1\}$. Al evaluar $0=\sum_{i=1}^n L_{c_i}R_{b_i}$ en $1_A$
	se obtiene que $0=c_1b_1+\cdots+c_nb_n$, una contradicción a la
	independencia lineal de $\{b_1,\dots,b_n\}$. 
\end{proof}

\begin{lemma}
	\label{lem:SkolemNoether2}
	Si $A$ es un álgebra central simple de dimensión finita, entonces $M(A)=\End_k(A)$. 
\end{lemma}

\begin{proof}
	Sea $\{a_1,\dots,a_n\}$ una base de $A$. El conjunto
	$\{L_{a_i}R_{a_j}:1\leq i,j\leq n\}$ es linealmente independiente: si
	$\sum_{i,j=1}^n\lambda_{ij}L_{a_i}R_{a_j}=0$ entonces
	$\sum_{i=1}^nL_{a_i}R_{c_i}=0$, donde
	$c_i=\sum_{j=1}^n\lambda_{ij}R_{a_j}$. Como los $a_i$ son linealmente
	independientes, el lema~\ref{lem:SkolemNoether1} implica que $c_i=0$ para
	todo $i\in\{1,\dots,n\}$, una contradicción a la independencia lineal de los $a_j$.  
	Luego $\dim_kM(A)\geq n^2=\dim\End_k(A)$.
\end{proof}

\begin{definition}
	\index{Automorfismo!interior}
	Sea $R$ un anillo unitario. Un automorfismo $f\in\Aut(R)$ se dice
	\textbf{interior} si existe un elemento inversible $r\in R$ tal que
	$f(x)=rxr^{-1}$ para todo $x\in R$.
\end{definition}

\begin{example}
	El automorfismo $\C\to\C$ dado por $z\mapsto\overline{z}$ no es interior.
\end{example}

\begin{example}
	Sea $\lambda\in k\setminus\{0\}$ y sea $R=k[X]$. El automorfismo $k[X]\to
	k[X]$, $f(X)\mapsto f(X+\lambda)$, no es interior.
\end{example}

\begin{example}
	Sea $R$ un anillo. El automorfismo $R\times R\to R\times R$, $(x,y)\mapsto
	(y,x)$, no es interior.
\end{example}

\begin{theorem}[Skolem--Noether]
	\index{Teorema!de Skolem--Noether}
	\label{thm:SkolemNoether}
	Si $A$ es un álgebra central simple de dimensión finita, todo automorfismo
	de $A$ es interior.
\end{theorem}

\begin{proof}	
	Sea $f\in\Aut(A)$. Gracias al lema~\ref{lem:SkolemNoether2}, 
	$f=\sum_{i=1}^n	L_{a_i}R_{b_i}$. 
	Sin perder generalidad podemos suponer que $a_1\ne 0$ y
	que $\{b_1,\dots,b_n\}$ es linealmente independiente
	(observación~\ref{rem:SkolemNoether}). Como $f$ es morfismo,
	$L_{f(x)}f=fL_x$ para todo $x\in A$. Entonces
	\[
		0=\sum_{i=1}^n L_{f(x)a_i-a_ix}R_{b_i}
	\]
	y luego, por el lema~\ref{lem:SkolemNoether1}, $f(x)a_1-a_1x=0$ para todo
	$x\in A$. Para terminar la demostración basta ver que $a_1$ es inversible:
	Como $a_1\ne 0$ y $A$ es simple, el ideal de $A$ generado por $a_1$ es $A$; esto nos permite escribir 
	$1=\sum_{i=1}^m u_ja_1v_j$ y luego $a_1$ es inversible pues 
	\[
		\left(\sum_{j=1}^m u_jf(v_j)\right)a_1=a_1\left(\sum_{j=1}^m f^{-1}(u_j)v_j\right)=1.
	\]
\end{proof}


%\begin{corollary}
%	\label{cor:SkolemNoether1}
%	Sea $A$ es un álgebra de dimensión finita unitaria. Si $a\in A$, entonces
%	$a$ es inversible o es un divisor de cero. 
%\end{corollary}
%
%\begin{proof}
%	Como $A$ es de dimensión finita, $A$ es algebraica. Existe entonces un
%	polinomio $f=\sum_{j=1}^n \lambda_jX^j\in k[X]$ (que podemos suponer de grado
%	mínimo) tal que $f(a)=0$. Al escribir
%	\[
%		0=f(a)=a(\lambda_na^{n-1}+\cdots+\lambda_2a+\lambda_1)+a_0
%	\]
%	vemos que existe un polinomio $g=\lambda_nX^{n-1}+\cdots+\lambda_1\in k[X]$
%	tal que $g(a)\ne 0$ (por la minimalidad de $n$) y $ag(a)=-\lambda_0$. Si
%	$a$ no es un divisor de cero, entonces $\lambda_0\ne 0$ y luego
%	$a^{-1}=-\lambda_0^{-1}g(a)$. 
%\end{proof}
%
%\begin{corollary}
%	Sea $A$ un álgebra de dimensión finita unitaria y sean $a,b\in A$. Si
%	$ab=1$, entonces $ba=1$.
%\end{corollary}
%
%\begin{proof}
%	Es consecuencia inmediata del corolario~\ref{cor:SkolemNoether1}.
%\end{proof}
%
%\begin{corollary}
%	Sea $D$ un álgebra de división de dimensión finita y sea $A$ una subálgebra
%	de $D$. Entonces $A$ es un álgebra de división.
%\end{corollary}
%
%\begin{proof}
%	Sea $a\in A\setminus\{0\}$. Como existe $d\in D$ tal que $ad=1$. Como $a$ es algebraico, 
%	existe $f\in k[X]$ de grado mínimo tal que $f(a)=0$. Luego $a$ es inversible con 
%	$a^{-1}=-\lambda_0^{-1}g(a)$ para algún $g\in k[X]$ tal que $g(a)\ne 0$. En
%	particular, $a^{-1}\in A$ y además $A$ es unitaria. 
%\end{proof}
