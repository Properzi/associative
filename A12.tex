
\section{Left-ordered groups}

\begin{definition}
	\index{Group!left-ordered}
	A group $G$ is \emph{left-ordered} if there is a total order 
	$<$ in $G$ such that $x<y$ implies $zx<zy$ for all $x,y,z\in G$.
\end{definition}

If $G$ is left-ordered, the \emph{positive cone} of $G$ is defined as $P(G)=\{x\in G:1<x\}$. 

\begin{exercise}
	Let $G$ be left-ordered with positive cone $P$. Prove that 
	$P$ is closed under multiplication and that 
	$G=P\cup P^{-1}\cup \{1\}$ (disjoint union).
\end{exercise}

\begin{exercise}
\label{xca:LO_cone}
	Let $G$ be a group and $P$ be a subset closed under multiplication. Assume that 
	$G=P\cup P^{-1}\cup \{1\}$ (disjoint union). Prove that $x<y$ if and only if
	$x^{-1}y\in P$ turns $G$ into a left-ordered group with positive cone $P$.
\end{exercise}

Left-ordered groups behave nicely with respect to extensions. Let $G$ be a group
and $N$ be a left-ordered normal subgroup of $G$. If $\pi\colon G\to G/N$ is the 
canonical map and $G/N$ is left-ordered, then
$G$ is left-ordered with
$x<y$ if and only if either $\pi(x)<\pi(y)$ or $\pi(x)=\pi(y)$ and $1<x^{-1}y$. 


\begin{proposition}
	Let $G$ be a group and $N$ be a normal subgroup of $G$. 
	If $N$ and $G/N$ are left-ordered, then so is $G$.
\end{proposition}

\begin{proof}
	Since $N$ and $G/N$ are both left-ordered, there exist positive cones 
	$P(N)$ and $P(G/N)$. Let $\pi\colon G\to G/N$ be the canonical map and 
	\[
		P(G)=\{x\in G:\pi(x)\in P(G/N)\text{ or }x\in P(N)\}.
	\]	
	A routine calculation shows that $P(G)$ is closed under multiplication 
	and that $G$ decomposes as $G=P(G)\cup P(G)^{-1}\cup \{1\}$ (disjoint union). It follows
	from Exercise \ref{xca:LO_cone} that 
	$G$ is left-ordered. 
\end{proof}


%
%\begin{theorem}
%	\label{theorem:}
%	Si $G$ tiene una serie finita subnormal $1=G_0\triangleleft
%	G_1\triangleleft\cdots\triangleleft G_n=G$ y cada cociente $G_{i+1}/G_i$ es
%	abeliano libre de torsión, entonces $G$ es ordenable a derecha. Si además
%	$G$ es libre de torsión y nilpotente, entonces $G$ es biordenable.
%\end{theorem}

We now present a criterion for detecting left-ordered groups. We shall need 
a lemma. 

\begin{lemma}
	\label{lem:fg}
	Let $G$ be a finitely generated group. If $H$ is a finite-index subgroup, 
	then $H$ is finitely generated. 
\end{lemma}

\begin{proof}
	Assume that $G$ is generated by $\{g_1,\dots,g_m\}$. Assume that
	for each $i$ there exists $k$ such that $g_i^{-1}=g_k$. Let $\{t_1,\dots,t_n\}$ be
	a transversal of $H$ in $G$. For $i\in\{1,\dots,n\}$ and 
	$j\in\{1,\dots,m\}$ write 
	\[
		t_ig_j=h(i,j)t_{k(i,j)}.
	\]
	We claim that $H$ is generated by the $h(i,j)$. For $x\in H$, write 
	\begin{align*}
	x &=g_{i_1}\cdots g_{i_s}\\
	&= (t_1g_{i_1})g_{i_2}\cdots g_{i_s}\\
	&= h(1,i_1)t_{k_1}g_{i_2}\cdots g_{i_s}\\
	&= h(1,i_1)h(k_1,i_2)t_{k_2}g_{i_3}\cdots g_{i_s}\\
	&= h(1,i_1)h(k_1,i_2)\cdots h(k_{s-1},g_{i_s})t_{k_s},
	\end{align*}
	where $k_1,\dots,k_{s-1}\in\{1,\dots,n\}$. Since $t_{k_s}\in H$, it follows that 
	$t_{k_s}=t_1\in H$.
\end{proof}

Now the theorem.

\begin{theorem}
	Let $G$ be a finitely generated torsion-free group. If $A$ is an abelian normal
	subgroup such that $G/A$ is finite and cyclic, then $G$ is left-ordered. 
\end{theorem}

\begin{proof}
	Note that if $A$ is trivial, then so is $G$. Let us assume that $A\ne\{1\}$. 
    Since $(G:A)$ is finite, $A$ is finitely generated by the previous lemma. 
    We proceed by induction on the number of generators of $A$. Since 
    $G/A$ is cyclic, there exists $x\in G$ such that $G=\langle A,x\rangle$. Then
    $[x,A]=\langle [x,a]:a\in A\rangle$ is a normal subgroup of $G$ such that 
    $A/C_A(x)\simeq [x,A]$ (because $a\mapsto [x,a]$ is a group homomorphism $A\to A$
    with image $[x,A]$ and kernel $C_A(x)$). If $\pi\colon G\to G/[x,A]$ is the canonical map, then
    $G/[x,A]=\langle \pi(A),\pi(x)\rangle$ and thus $G/[x,A]$ is abelian, as 
    $[\pi(x),\pi(A)]=\pi[x,A]=1$. Moreover, $G/[x,A]$ is finitely generated, as $G$
    is finitely generated. Since $(G:A)=n$ and $G$ is torsion-free, it follows that 
    $1\ne x^n\in A$. Hence $x^n\in C_A(x)$ and therefore $1\leq \rank C_A(x)<\rank A$ (if $\rank
    C_A(x)=\rank A$, then $[x,A]$ would be a torsion subgroup of $A$, a contradiction
    since $x\not\in A$). So 
    \[
    \rank[x,A]=\rank (A/C_A(x))\leq\rank A-1
    \]
    and hence $\rank (A/[x,A])\geq 1$. We proved that $A/[x,A]$ is infinite and hence 
    $G/[x,A]$ is infinite. 

    Since $G/[x,A]$ is infinite, abelian and finitely generated, there exists a normal subgroup
    $H$ of $G$ such that $[x,A]\subseteq H$ and $G/H\simeq\Z$. The subgroup 
    $B=A\cap H$ is abelian, normal in $H$ and such that $H/B$ is cyclic
    (because it is isomorphic to a subgroup of $G/A$). Since $\rank B<\rank A$, the inductive hypothesis implies that $H$ is left-ordered. Hence $G$ is left-ordered. 
\end{proof}

\index{Lagrange--Rhemtulla's theorem} 
Lagrange and Rhemtulla proved that the integral isomorphism problem 
has an affirmative solution for left-ordered groups. More precisely,
if $G$ is left-ordered and $H$ is a group such that $\Z[G]\simeq\Z[H]$, then
$G\simeq H$, see \cite{MR240183}.

\begin{theorem}[Malcev--Neumann]
	\index{Malcev--Neumann theorem}
	Let $G$ be left-ordered group. Then $K[G]$ has no zero divisors 
	and no non-trivial units. 
\end{theorem}

\begin{proof}
	If $\alpha=\sum_{i=1}^na_ig_i\in K[G]$ and
	$\beta=\sum_{j=1}^mb_jh_j\in K[G]$, then 
	\begin{equation}
		\label{eq:producto}
		\alpha\beta=\sum_{i=1}^n\sum_{j=1}^ma_ib_j(g_ih_j).
	\end{equation}
	Without loss of generality we may assume that $a_i\ne 0$ for
	all $i$ and $b_j\ne 0$ for all $j$. Moreover, we may assume that 
	$g_1<g_2<\cdots<g_n$. Let $i,j$ be such that 
	\[
		g_ih_j=\min\{g_ih_j:1\leq i\leq n,1\leq j\leq m\}.
	\]
	Then $i=1$, as $i>1$ implies
	$g_1h_j<g_ih_j$, a contradiction. Since $g_1h_j\ne g_1h_k$ whenever 
	$k\ne j$, there exists a unique minimal element in the left hand side of Equality~\eqref{eq:producto}. The same argument shows that there is a unique
	maximal element in~\eqref{eq:producto}. Thus 
	$\alpha\beta\ne 0$, as $a_1b_j\ne 0$, and therefore $K[G]$ has no zero divisors. 
	If, moreover, $n>1$ or $m>1$, then~\eqref{eq:producto} contains at least two
	terms than cancel out and thus  
	$\alpha\beta\ne1$. It follows that units of $K[G]$ are trivial. 
\end{proof}

\index{Formanek's theorem}
\index{Farkas--Snider theorem}
\index{Brown's theorem}
Formanek proved that the zero divisors conjecture is true 
in the case of torsion-free super solvable. Brown and, independently, 
Farkas and Snider proved that the conjecture is true 
in the case of groups algebras (over fields of characteristic zero) of 
polycyclic-by-finite torsion-free groups. These results
can be found in Chapter 13 of
Passman's book \cite{MR798076}. 

