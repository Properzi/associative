\section{Lecture: 12/12/2024}
\label{11}

\subsection{Tensor products}

The \emph{tensor product} of the vector spaces (over $K$) $U$ and $V$ 
is the quotient vector space $K[U\times V]/T$, where $K[U\times V]$ 
is the vector space with basis 
\[
\{(u,v):u\in U,v\in V\}
\]
and $T$ is the subspace 
generated by elements of the form 
\[
		(\lambda u+\mu u',v)-\lambda(u,v)-\mu(u',v),\quad
		(u,\lambda v+\mu v')-\lambda(u,v)-\mu(u,v')
	\]
for $\lambda,\mu\in K$, $u,u'\in U$ and $v,v'\in V$.
The tensor product of $U$ and $V$ will be denoted by $U\otimes_KV$ or 
$U\otimes V$ when the base field is clear from the context. For $u\in U$ and 
$v\in V$ we write $u\otimes v$ to denote the coset $(u,v)+T$.

\begin{theorem}
	Let $U$ and $V$ be vector spaces. Then there exists a bilinear map 
	\[
    U\times V\to U\otimes V,\quad (u,v)\mapsto u\otimes v,
    \]
    such that 
	each element of $U\otimes V$ is a finite sum of the form 
	\[
		\sum_{i=1}^N u_i\otimes v_i
	\]
	for some $u_1,\dots,u_N\in U$ and $v_1,\dots,v_N\in V$. 
	Moreover, if $W$ is a vector space and 
 \[
 \beta\colon U\times V\to W
 \]
 is a bilinear map, 
	there exists a linear map 
	$\overline{\beta}\colon U\otimes V\to W$ such that $\overline{\beta}(u\otimes
	v)=\beta(u,v)$ for all $u\in U$ and $v\in V$.
\end{theorem}

\begin{proof}
    By definition, the map
    \[
	U\times V\to U\otimes V,\quad
	(u,v)\mapsto u\otimes v,
	\]
	is bilinear. From the definitions, it follows that
	$U\otimes V$ is a finite linear combination of elements of the form 
	$u\otimes v$, where $u\in U$ and $v\in V$. Since $\lambda(u\otimes
	v)=(\lambda u)\otimes v$ for all $\lambda\in K$, the first claim follows.

	Since the elements of $U\times V$ form a basis of $K[U\times V]$, there exists
	a linear map 
	\[
		\gamma\colon K[U\times V]\to W,\quad
	\gamma(u,v)=\beta(u,v). 
	\]
	Since $\beta$ is bilinear by assumption, $T\subseteq\ker\gamma$. It follows that there exists 
	a linear map $\overline{\beta}\colon U\otimes V\to
	W$ such that  
	\[
	\begin{tikzcd}
		K[U\times V] \arrow[r]\arrow[d] & W \\
		U\otimes V\arrow[ur, dashrightarrow]
	\end{tikzcd}
	\]
	commutes. In particular, $\overline{\beta}(u\otimes v)=\beta(u,v)$. 
\end{proof}

\begin{exercise}
	\label{xca:tensorial_unicidad}
	Prove that the properties of the previous theorem characterize tensor products up to isomorphism. 
\end{exercise}

Some properties:
%Observemos
%que todo elemento de $U\otimes V$ es una suma finita
%de la forma 
%\[
%	\sum_{i=1}^N u_i\otimes v_i
%\]
%para $N\in\N$, $u_i\in U$ y $v_i\in V$. Esta expresión no es única. Vale además
%que $u\otimes 0=0=0\otimes v$ para todo $u\in U$ y $v\in V$.

\begin{proposition}
	Let $\varphi\colon U\to U_1$ and $\psi\colon V\to V_1$ be linear maps. There
	exists a unique linear map 
	$\varphi\otimes\psi\colon U\otimes V\to U_1\otimes V_1$ such that
	\[
		(\varphi\otimes\psi)(u\otimes v)=\varphi(u)\otimes\psi(v)
	\]
	for all $u\in U$ and $v\in V$.
\end{proposition}

\begin{proof}
	Since $U\times V\to U_1\otimes V_1$,
	$(u,v)\mapsto\varphi(u)\otimes\psi(v)$, is bilinear, there exists a linear map
	$U\otimes V\to U_1\otimes V_1$, $u\otimes
	v\to\varphi(u)\otimes\psi(v)$. Thus 
	\[
		\sum u_i\otimes v_i\mapsto\sum\varphi(u_i)\otimes\psi(v_i)
	\]
	is well-defined. 
\end{proof}

\begin{exercise}
    Prove the following statements:
	\begin{enumerate}
		\item $(\varphi\otimes\psi)(\varphi'\otimes\psi')=(\varphi\varphi')\otimes(\psi\psi')$.
		\item If $\varphi$ and $\psi$ are isomorphisms, then 
			$\varphi\otimes\psi$ is an isomorphism. 
		\item $(\lambda\varphi+\lambda'\varphi')\otimes\psi=\lambda\varphi\otimes\psi+\lambda'\varphi'\otimes\psi$.
		\item $\varphi\otimes(\lambda\psi+\lambda'\psi')=\lambda\varphi\otimes\psi+\lambda'\varphi\otimes\psi'$.
		\item If $U\simeq U_1$ and $V\simeq V_1$, then $U\otimes V\simeq U_1\otimes V_1$.
	\end{enumerate}
\end{exercise}

The following proposition is extremely useful:

\begin{proposition}
	If $U$ and $V$ are vector spaces, then  
	$U\otimes V\simeq V\otimes U$.
\end{proposition}

\begin{proof}
	Since $U\times V\to V\otimes U$, $(u,v)\mapsto v\otimes u$, is bilinear, there exists 
	a linear map 
    \[
    U\otimes V\to V\otimes U,\quad u\otimes
	v\mapsto v\otimes u.
    \]
    Similarly, there exists a linear map 
	\[
    V\otimes U\to U\otimes V,\quad  v\otimes u\mapsto
	u\otimes v.
    \]
    Thus $U\otimes V\simeq V\otimes U$.
\end{proof}

\begin{exercise}
	\label{xca:UxVxW}
    Prove that $(U\otimes V)\otimes W\simeq U\otimes(V\otimes W)$.
\end{exercise}

\begin{exercise}
	\label{xca:UxK}
	Prove that $U\otimes K\simeq U\simeq K\otimes U$.
\end{exercise}

\begin{proposition}
	\label{pro:U_LI}
	Let $U$ and $V$ be vector spaces. 
	If $\{u_1,\dots,u_n\}$ is a linearly independent subset of $U$ and 
	$v_1,\dots,v_n\in V$ is such that $\sum_{i=1}^n u_i\otimes v_i=0$, then 
    $v_i=0$ for all $i\in\{1,\dots,n\}$.
\end{proposition}

\begin{proof}
	Let $i\in\{1,\dots,n\}$ and 
	\[
	f_i\colon U\to K,
	\quad
	f_i(u_j)=\delta_{ij}=\begin{cases}
	1 & \text{if $i=j$},\\
	0 & \text{otherwise}.
	\end{cases}
	\]
	Since the map 
 \[
 U\times V\to V,\quad (u,v)\mapsto f_i(u)v,
 \]
 is bilinear, there exists 
	a linear map 
	$\alpha_i\colon U\otimes V\to V$ such that $\alpha_i(u\otimes
	v)=f_i(u)v$. Thus
	\[
		v_i=\sum_{j=1}^n\alpha_i(u_j\otimes v_j)=\alpha_i\left(\sum_{j=1}^nu_j\otimes v_j\right)=0.\qedhere
	\]
\end{proof}

\begin{exercise}
	\label{xca:uxv=0}
	Prove that $u\otimes v=0$ and $v\ne 0$ imply $u=0$.
\end{exercise}

\begin{theorem}
    Let $U$ and $V$ be vector spaces. 
	If $\{u_i:i\in I\}$ is a basis of $U$ and $\{v_j:j\in J\}$ is a basis of $V$, then 
	$\{u_i\otimes v_j:i\in I,j\in J\}$ is a basis of $U\otimes
	V$.
\end{theorem}

\begin{proof}
	The $u_i\otimes v_j$ are generators of $U\otimes V$, as  
    $u=\sum_i\lambda_iu_i$ and $v=\sum_j\mu_jv_j$ imply 
	\[
    u\otimes v=\sum_{i,j}\lambda_i\mu_ju_i\otimes v_j.
    \]
	We now prove that the $u_i\otimes v_j$ are linearly independent. We need to show that
	each finite subset of the $u_i\otimes v_j$
	is linearly independent. If $\sum_k\sum_l\lambda_{kl}u_{i_k}\otimes
	v_{j_l}=0$, then 
	\[
0=\sum_{k}u_{i_k}\otimes\left(\sum_{l}\lambda_{kl}v_{j_l}\right).\]
    Since  
	the $u_{i_k}$ are linearly independent, Proposition~\ref{pro:U_LI}
	implies that $\sum_{l}\lambda_{kl}v_{j_l}=0$. Thus $\lambda_{kl}=0$ for all 
	$k,l$, as the $v_{j_l}$ are linearly independent. 
\end{proof}

If $U$ and $V$ are finite-dimensional vector spaces, then 
\[
	\dim(U\otimes V)=(\dim U)(\dim V).
\]

\begin{corollary}
	If $\{u_i:i\in I\}$ is a basis of $U$, then every element of $U\otimes V$
	can be written uniquely as a finite sum $\sum_{i}u_i\otimes v_i$.
\end{corollary}

\begin{proof}
	Every element of the tensor product $U\otimes V$ is a finite sum 
	of the form $\sum_i x_i\otimes y_i$, where $x_i\in U$ and $y_i\in V$. If  
	$x_i=\sum_j\lambda_{ij}u_j$, then 
	\[
		\sum_i x_i\otimes y_i=\sum_i\left(\sum_j\lambda_{ij}u_j\right)\otimes y_i		
		=\sum_j u_j\otimes\left(\sum_i\lambda_{ij}y_i\right).\qedhere 
	\]
\end{proof}

%\begin{corollary}
%	Todo elemento no nulo de $U\otimes V$ puede escribirse como una suma finita
%	$\sum_{i=1}^N u_i\otimes v_i$ para un conjuntos $\{u_i:1\leq i\leq
%	N\}\subseteq U$ y $\{v_i:1\leq i\leq N\}\subseteq V$ linealmente
%	independientes.
%\end{corollary}
%
%\begin{proof}
%	tomar $N$ minimal	
%\end{proof}

\begin{exercise}
\label{xca:tensor_algebras}
    Let $A$ and $B$ be algebras. Prove that $A\otimes B$ 
    is an algebra with 
	\[
		(a\otimes b)(x\otimes y)=ax\otimes by.
	\]
\end{exercise}

% \begin{proof}
% 	Para $x\in A$, $y\in B$ consideramos $R_x\otimes R_y\in\End_K(A\otimes B)$.
% 	Como la función $A\times B\to\End_K(A\otimes B)$, $(x,y)\mapsto R_x\otimes
% 	R_y$, es bilineal, existe una función lineal $\varphi\colon A\otimes
% 	B\to\End_K(A\otimes B)$, $\varphi(x\otimes y)=R_x\otimes R_y$. Para $u,v\in A\otimes B$ definimos
% 	\[
% 		uv=\varphi(v)(u).
% 	\]
% 	Esta operación es bilineal pues por ejemplo
% 	\[
% 		u(v+w)=\varphi(v+w)(u)=(\varphi(v)+\varphi(w))(u)=\varphi(v)(u)+\varphi(w)(u)=uv+uw.
% 	\]
% 	Además
% 	$(a\otimes b)(x\otimes y)=\varphi(x\otimes y)(a\otimes b)=(R_x\otimes R_y)(a\otimes b)=ax\otimes by$.
% 	Un cálculo sencillo muestra que este producto es asociativo.
% \end{proof}

\begin{exercise}
    Let $K$ be a field and $A,B,C$ be $K$-algebras. 
    Prove the following statements:
	\begin{enumerate}
		\item $A\otimes B\simeq B\otimes A$.
		\item $(A\otimes B)\otimes C\simeq A\otimes(B\otimes C)$.
		\item $A\otimes K\simeq A\simeq K\otimes A$.
		\item If $A\simeq A_1$ and $B\simeq B_1$, then $A\otimes B\simeq A_1\otimes B_1$.
	\end{enumerate}
\end{exercise}

Some examples:

\begin{proposition}
	If $G$ and $H$ are groups, then $K[G]\otimes K[H]\simeq K[G\times H]$.
\end{proposition}

\begin{proof}
	The set $\{g\otimes h:g\in G,h\in H\}$ is a basis of $K[G]\otimes K[H]$ and 
	the elements of $G\times H$ form a basis of $K[G\times H]$. There exists a linear isomorphism 
	\[
	K[G]\otimes K[H]\to K[G\times H], 
	\quad 
	g\otimes h\mapsto (g,h),
	\]
	that is multiplicative. Thus $K[G]\otimes K[H]\simeq K[G\times H]$
	as algebras. 
\end{proof}

\begin{proposition}
\label{pro:AKX=AX}
	If $A$ is an algebra, then $A\otimes K[X]\simeq A[X]$.	
\end{proposition}

\begin{proof}
	Each element of the tensor product $A\otimes K[X]$ can be written uniquely as a finite sum of
	the form $\sum a_i\otimes X^i$. Routine calculations show that 
	$A\otimes K[X]\mapsto A[X]$, $\sum a_i\otimes X^i\mapsto \sum a_iX^i$, is a 
	linear algebra isomorphism. 
\end{proof}

\begin{exercise}
\label{xca:AM=MA}
	Prove that if $A$ is an algebra, then $A\otimes M_n(K)\simeq M_n(A)$. In
	particular, $M_n(K)\otimes M_m(K)\simeq M_{nm}(K)$.
\end{exercise}

Proposition \ref{pro:AKX=AX} and Exercise \ref{xca:AM=MA} 
are examples of a procedure known as \emph{scalar extensions}. 

\begin{theorem}
\label{thm:extensions_scalars}
	Let $A$ be an algebra over $K$ and $E$ be an extension of $K$ (this just simply means that
	$K$ is a subfield of $E$). Then 
	$A^E=E\otimes_KA$ is an algebra over $E$ with respect to
	the scalar multiplication 
	\[
		\lambda(\mu\otimes a)=(\lambda\mu)\otimes a,
	\]
	for all $\lambda,\mu\in E$ and $a\in A$.
\end{theorem}

\begin{proof}
	Let $\lambda\in E$. Since $E\times A\to E\otimes_KA$,
	$(\mu,a)\mapsto (\lambda\mu)\otimes a$, is $K$-bilinear, there exists 
	a linear map $E\otimes_KA\to E\otimes_KA$, $\mu\otimes a\mapsto
	(\lambda\mu)\otimes a$. The scalar multiplication is then well-defined and 
	\[
	\lambda(u+v)=\lambda u+\lambda v
	\]
	for all $\lambda\in E$ and $u,v\in E\otimes_KA$. Moreover, 
	\[
	(\lambda+\mu)u=\lambda u+\mu u,
	\quad
	(\lambda\mu)u=\lambda(\mu u),
	\quad
	\lambda(uv)=(\lambda u)v=u(\lambda v)
	\]
	for all $u,v\in E\otimes_KA$ and $\lambda,\mu\in E$.
\end{proof}

\begin{exercise}
    Prove the following statements:
    \begin{enumerate}
		\item $\{1\}\otimes A$ is a subalgebra of $A^E$ isomorphic to $A$.
		\item If $\{a_i:i\in I\}$ is a basis of $A$, then $\{1\otimes a_i:i\in
			I\}$ is a basis of $A^E$.
	\end{enumerate}
\end{exercise}

\begin{exercise}
	Prove that if $G$ is a group and $K$ is a subfield of $E$, then
	\[
	E\otimes_K K[G]\simeq E[G].
	\]
\end{exercise}

\subsection{Formanek's theorem, II}

The combination of technique known as extensions of scalars we have seen in the previous section
and Formanek's theorem for rational group algebras 
yield the following general result. 

\begin{theorem}[Formanek]
	\index{Formanek's theorem}
	Let $K$ be a field of characteristic zero and let $G$ be a group. 
	If every element of $K[G]$ is invertible or a zero divisor, 
	then $G$ is locally finite. 
\end{theorem}

\begin{proof}
	Since $K$ is of characteristic zero, $\Q\subseteq K$. Then $K[G]\simeq
	K\otimes_{\Q}\Q[G]$. Each $\beta\in K\otimes_{\Q}\Q[Q]$ can be written
	uniquely as 
	\[
		\beta=1\otimes\beta_0+\sum k_i\otimes\beta_i,
	\]
	where $\{1,k_1,k_2,\dots,\}$ is a basis of $K$ as a $\Q$-vector space. 
	Let $\alpha\in\Q[G]$ and let $\beta\in K[G]$ be such that $\alpha\beta=1$. Since
	\[
	1\otimes 1=(1\otimes\alpha)\beta=1\otimes \alpha\beta_0+\sum k_i\otimes \alpha\beta_i,
	\]
	it follows that $\alpha\beta_0=1$. Similarly, if
	$\alpha\beta=0$, then $\alpha\beta_j=0$ for all $j$. Since 
	each $\alpha\in\Q[G]$ is invertible or a zero divisor, Formanek's theorems 
	for $\Q$ applies. 
\end{proof}

% \section*{Rickart's theorem}

% En esta sección vamos a demostrar que para cualquier grupo $G$ el radical de
% Jacobson de $\C[G]$ es cero. Demostraremos también que el radical de Jacobson
% de $\R[G]$ es cero.

% \begin{definition}
% 	\index{Anillo!con involución}
% 	\index{Involución!de un anillo}
% 	Sea $R$ un anillo. Una \emph{involución} del anillo $R$ es un morfismo
% 	aditivo $R\to R$, $x\mapsto x^*$, tal que $x^{**}=x$ y $(xy)^*=y^*x^*$ para
% 	todo $x,y\in R$.
% \end{definition}

% De la definición se deduce inmediatamente que si $R$ es unitario, entonces
% $1^*=1$.

% \begin{example}
% 	La conjugación $z\mapsto\overline{z}$ es una involución de $\C$.
% \end{example}

% \begin{example}
% 	La trasposición $X\mapsto X^T$ es una involución del
% 	anillo $M_n(K)$.
% \end{example}

% \begin{example}
% 	Sea $G$ un grupo. Entonces
% 	$\left(\sum_{g\in G}\alpha_gg\right)^*=\sum_{g\in G}\overline{\alpha_g}g^{-1}$ 
% 	es una involución de $\C[G]$.
% \end{example}

% Dado un grupo $G$, se define la \emph{traza} de un elemento
% $\alpha=\sum_{g\in G}\alpha_gg\in K[G]$ como $\trace(\alpha)=\alpha_1$. Es
% fácil ver que $\trace\colon K[G]\to K$, $\alpha\mapsto\trace(\alpha)$ es una
% función $K$-lineal tal que $\trace(\alpha\beta)=\trace(\beta\alpha)$.

% \begin{exercise}
% 	Sea $G$ un grupo finito y $K$ un cuerpo tal que su característica no divide al orden de $G$.
% 	Demuestre las siguientes afirmaciones:
% 	\begin{enumerate}
% 		\item Si $\alpha\in K[G]$ es nilpotente, entonces $\trace(\alpha)=0$.
% 		\item Si $\alpha\in K[G]$ es idempotente, entonces $\trace(\alpha)=\dim
% 			K[G]\alpha/|G|$.
% 	\end{enumerate}
% \end{exercise}

% \begin{exercise}
% 	Demuestre que 
% 	$\langle\alpha,\beta\rangle=\trace(\alpha\beta^*)$, $\alpha,\beta\in\C[G]$, 
% 	define un producto interno en $\C[G]$.
% \end{exercise}

% \begin{lemma}
% 	\label{lem:algebraico}
% 	Sea $G$ un grupo. Si $J(\C[G])\ne 0$, entonces existe $\alpha\in J(\C[G])$ tal que 
% 	$\trace(\alpha^{2^m})\in\R_{\geq1}$ 
% 	para todo $m\geq1$.
% \end{lemma}

% \begin{proof}
% 	Sea $\alpha=\sum_{g\in G}\alpha_gg\in\C[G]$. Entonces	
% 	\[
% 		\trace(\alpha^*\alpha)
% 		=\sum_{g\in G}\overline{\alpha_g}\alpha_g
% 		=\sum_{g\in G}|\alpha_g|^2\geq|\alpha_1|^2
% 		=|\trace(\alpha)|^2.
% 	\]
% 	Al usar esta fórmula para algún $\alpha$ tal que $\alpha^*=\alpha$ y usar
% 	inducción se obtiene que $\trace(\alpha^{2^m})\geq|\trace(\alpha)|^{2^m}$
% 	para todo $m\geq1$. 

% 	Sea $\beta=\sum_{g\in G}\beta_gg\in J(\C[G])$ tal que $\beta\ne0$. Como
% 	$\trace(\beta^*\beta)=\sum_{g\in G}|\beta_g|^2\ne0$ y $J(\C[G])$ es un ideal, 
% 	\[
% 		\alpha=\frac{\beta^*\beta}{\trace(\beta^*\beta)}\in J(\C[G]).
% 	\]
% 	Este elemento $\alpha$ cumple que $\alpha^*=\alpha$ y $\trace(\alpha)=1$.
% 	Luego $\trace(\alpha^{2^m})\geq 1$ para todo $m\geq1$.
% \end{proof}

% El ejercicio~\ref{exa:norma} implica que $\C[G]$ con
% $\dist(\alpha,\beta)=|\alpha-\beta|$ es un espacio métrico. En este espacio
% métrico, la función $\C[G]\to\C$, $\alpha\mapsto \trace(\alpha)$, es una
% función continua.

% \begin{lemma}
% 	\label{lem:phi_diferenciable}
% 	Sea $\alpha\in J(\C[G])$. La función
% 	\[
% 		\varphi\colon\C\to\C[G],\quad
% 		\varphi(z)=(1-z\alpha)^{-1},
% 	\]
% 	es continua, diferenciable y $\varphi(z)=\sum_{n\geq0}\alpha^nz^n\in\C[G]$ si $|z|$
% 	es suficientemente pequeño.
% \end{lemma}

% \begin{proof}	
% 	Sean $y,z\in\C$. Como $\varphi(y)$ y $\varphi(z)$ conmutan, 
% 	\begin{equation}
% 		\label{eq:Rickart}
% 		\begin{aligned}
% 			\varphi(y)-\varphi(z)&=\left( (1-z\alpha)-(1-y\alpha)\right)(1-y\alpha)^{-1}(1-z\alpha)^{-1}\\
% 			&=(y-z)\alpha\varphi(y)\varphi(z).
% 		\end{aligned}
% 	\end{equation}
% 	Entonces $|\varphi(y)|\leq|\varphi(z)|+|y-z||\alpha\varphi(y)||\varphi(z)|$ y luego
% 	\[
% 		|\varphi(y)|\left( 1-|y-z||\alpha\varphi(z)|\right)\leq|\varphi(z)|.
% 	\]
% 	Fijado $z$ podemos elegir $y$ suficientemente cerca de $z$ de forma tal que
% 	se cumpla que  $1-|y-z||\alpha\varphi(z)|\geq1/2$. Luego
% 	$|\varphi(y)|\leq2|\varphi(z)|$. De la igualdad~\eqref{eq:Rickart} se
% 	obtiene entonces $|\varphi(y)-\varphi(z)|\leq2|y-z||\alpha||\varphi(z)|^2$
% 	y luego $\varphi$ es una función continua. Por la
% 	expresión~\eqref{eq:Rickart}, 
% 	\[
% 	\varphi'(z)
% 	=\lim_{y\to z}\frac{\varphi(y)-\varphi(z)}{y-z}
% 	=\lim_{y\to z}\alpha\varphi(y)\varphi(z)
% 	=\alpha\varphi(z)^2
% 	\]
% 	para todo $z\in\C$.

% 	Si $z$ es tal que $|z||\alpha|=|z\alpha|<1$, entonces 
% 	\[
% 		\varphi(z)-\sum_{n=0}^Nz^n\alpha^n
% 		=\varphi(z)\left(1-(1-z\alpha)\sum_{n=0}^Nz^n\alpha^n\right)
% 		=\varphi(z)(z\alpha)^{N+1}
% 	\]
% 	y luego
% 	\[
% 		\left|\varphi(z)-\sum_{n=0}^Nz^n\alpha^n\right|\leq|\varphi(z)||z\alpha|^{N+1}.
% 	\]
% 	Como $\varphi(z)$ está acotada cerca de $z=0$, se concluye que
% 	$\left|\varphi(z)-\sum_{n=0}^Nz^n\alpha^n\right|\to0$ si $N\to\infty$.
% \end{proof}

% Estamos en condiciones de demostrar el teorema de Rickart:

% \begin{theorem}[Rickart]
% 	\index{Teorema!de Rickart}
% 	Si $G$ es un grupo, entonces $J(\C[G])=0$.
% \end{theorem}

% \begin{proof}
% 	Sea $\alpha\in J(\C[G])$ y sea $\varphi(z)=(1-\alpha z)^{-1}$. Sea 
% 	$f\colon\C\to \C$ dada por
% 	$f(z)=\trace\varphi(z)=\trace\left((1-z\alpha)^{-1}\right)$. Por el lema~\ref{lem:phi_diferenciable},
% 	$f(z)$ es una función entera tal que $f'(z)=\trace(\alpha\varphi(z)^2)$ y
% 	\begin{equation}
% 		\label{eq:Taylor}
% 		f(z)=\sum_{n=0}^\infty z^n\trace(\alpha^n)
% 	\end{equation}
% 	si $|z|$ es suficientemente pequeño. En particular, la
% 	igualdad~\eqref{eq:Taylor} es la expansión en serie de Taylor para $f(z)$
% 	en el origen. Esto implica que esta serie tiene radio de convergencia
% 	infinito y converge a $f(z)$ para todo $z\in\C$. En particular,
% 	\begin{equation}
% 		\label{eq:limite}
% 		\lim_{n\to\infty}\trace(\alpha^n)=0.
% 	\end{equation}
% 	Por otro lado, si $\alpha\ne0$ el lema~\ref{lem:algebraico} implica que
% 	$\trace(\alpha^{2^m})\geq1$ para todo $m\geq0$, lo que contradice el límite
% 	calculado en~\eqref{eq:limite}. Luego $\alpha=0$.
% \end{proof}

% Para demostrar un corolario necesitamos dos lemas:

% \begin{lemma}[Nakayama]
% 	\label{lem:Nakayama}
% 	\index{Lema!de Nakayama}
% 	Sea $R$ un anillo unitario y sea $M$ un $R$-módulo finitamente generado. Si
% 	$J(R)M=M$, entonces $M=0$.
% \end{lemma}

% \begin{proof}
% 	Supongamos que $M$ está generado por los elementos $x_1,\dots,x_n$. Como $x_n\in M=J(R)M$, 
% 	existen $r_1,\dots,r_n\in J(R)$ tales que $x_n=r_1x_1+\cdots+r_nx_n$, es decir
% 	$(1-r_n)x_n=\sum_{j=1}^{n-1}r_jx_j$. 
% 	Como $1-r_n$ es inversible, existe $s\in R$ tal que $s(1-r_n)=1$. Luego
% 	$x_n=\sum_{j=1}^{n-1}sr_jx_j$ 
% 	y entonces $M$ está generado por $x_1,\dots,x_{n-1}$. Al repetir este
% 	procedimiento una cierta cantidad finita de veces, se obtiene que $M=0$.
% \end{proof}

% \begin{lemma}
% 	\label{lem:Rickart}
% 	Sea $\iota\colon R\to S$ un morfismo de anillos unitarios. Si 
% 	\[
% 	S=\iota(R)x_1+\cdots+\iota(R)x_n,
% 	\]
% 	donde cada $x_j$ cumple que $x_jy=yx_j$ para todo $y\in\iota(R)$, entonces
% 	$\iota(J(R))\subseteq J(S)$.
% \end{lemma}

% \begin{proof}
% 	Veamos que $J=\iota(J(R))$ actúa trivialmente en cada $S$-módulo simple $M$.
% 	Si $M$ es un $S$-módulo simple, escribimos $M=Sm$ para algún $m\ne0$. Es
% 	claro que $M$ es un $R$-módulo con $r\cdot m=\iota(r)m$. Como
% 	\[
% 		M=Sm=(\iota(R)x_1+\cdots+\iota(R)x_n)m=\iota(R)(x_1m)+\cdots+\iota(R)(x_nm),
% 	\]
% 	$M$ es finitamente generado como $\iota(R)$-módulo. Además $J(R)\cdot
% 	M=JM=\iota(J)M$ es un $S$-submódulo de $M$ pues
% 	\[
% 		x_j(JM)=(x_jJ)M=(Jx_j)M=J(x_jM)\subseteq JM.
% 	\]
% 	Como $M\ne0$, el lema de Nakayama implica que $J(R)\cdot M\subsetneq M$. Luego,
% 	como $M$ es un $S$-módulo simple, se concluye que $J(R)M=0$.
% \end{proof}

% \begin{corollary}
% 	Si $G$ es un grupo, entonces $J(\R[G])=0$. 
% \end{corollary}

% \begin{proof}
% 	Sea $\iota\colon \R[G]\to\C[G]$ la inclusión canónica. Como 
% 	\[
% 	\C[G]=\R[G]+i\R[G],
% 	\]
% 	el lema~\ref{lem:Rickart} y el teorema de Rickart implican que
% 	$\iota(J(\R[G]))\subseteq J(\C[G])=0$. Luego $J(\R[G])=0$ pues $\iota$ es
% 	inyectiva. 
% \end{proof}




\subsection{Wedderburn's little theorem}

\begin{definition}
	The $n$-th cyclotomic polynomial 
	is defined as the polynomial
	\begin{equation}
		\label{eq:ciclotomico}
		\Phi_n(X)=\prod(X-\zeta),
	\end{equation}
	where the product is taken over all 
	$n$-th primitive roots of one. 
\end{definition}

Some examples:
	\begin{align*}
		&\Phi_2=X-1,\\
		&\Phi_3=X^2+X+1,\\
		&\Phi_4=X^2+1,\\
		&\Phi_5=X^4+X^3+X^2+X+1,\\
		&\Phi_6=X^2-X+1,\\
		&\Phi_7=X^6+X^5+\cdots+X+1.
	\end{align*}

\begin{lemma}
	If $n\in\Z_{>0}$, then
	\[
		X^n-1=\prod_{d\mid n}\Phi_d(X).
	\]
\end{lemma}

\begin{proof}
	Write 
	\[
		X^n-1=\prod_{j=1}^n (X-e^{2\pi ij/n})
		=\prod_{d\mid n}\prod_{\substack{1\leq j\leq n\\\gcd(j,n)=d}}(X-e^{2\pi ij/n})
		=\prod_{d\mid n}\Phi_d(X).\qedhere 
	\]
\end{proof}

\begin{lemma}
	If $n\in\Z_{>0}$, then $\Phi_n(X)\in\Z[X]$.
\end{lemma}

\begin{proof}
	We proceed by induction on $n$. The case $n=1$ is trivial, as 
	$\Phi_1(X)=X-1$. Assume that $\Phi_d(X)\in\Z[X]$ for all $d<n$.
	Then 
	\[
		\prod_{d\mid n,d\ne n}\Phi_d(X)\in\Z[X]
	\]
	is a monic polynomial. Thus $\Phi_n(X)/\prod_{d\mid
	n,d<n}\Phi_d(X)\in\Z[X]$.
\end{proof}

\begin{theorem}[Wedderburn]
\label{thm:Wedderburn} 
	\index{Wedderburn's little theorem}
	Every finite division ring is a field. 
\end{theorem}

\begin{proof}
    Let $D$ be a finite division ring   
	and $K=Z(D)$. Then $K$ is a finite field, say $|K|=q$. 
	We claim that $|q-\zeta|>q-1$ for all $n$-th 
	root of one $\zeta\ne 1$.  In fact, write $\zeta=\cos\theta+i\sin\theta$. Then 
	$\cos\theta<1$ and 
	\[
	|q-\zeta|^2=q^2-(2\cos\theta)q+1>(q-1)^2.
	\]
	
	Note that
	$D$ is a $K$-vector space. Let 
	$n=\dim_KD$.  We claim that $n=1$. If $n>1$, the 
	class equation for the group 
	$D^\times=D\setminus\{0\}$ 
        % is 
        % \[
        % |D^\times|=|Z(D)\setminus\{0\}|+\sum_{j=1}^m (G:C_D(x_j)),
        % \]
        % where $x_1,\dots,x_m\in D$ and $(D^{\times}:C_{D^\times}(x_j))>1$ for all $j$. 
        % ver Nicholson page 370
        implies that 
	\begin{equation}
		\label{eq:clases}
		q^n-1=q-1+\sum_{j=1}^m \frac{q^n-1}{q^{d_j}-1},
	\end{equation}
	where $1<\frac{q^n-1}{q^{d_j}-1}\in\Z$ for all $j\in\{1,\dots,m\}$. 
	Since $d^{d_j}-1$ divides $q^n-1$, each $d_j$ divides $n$. In particular,
	~\eqref{eq:ciclotomico} implies that 
	\begin{equation}
		\label{eq:trick_ciclotomico}
		X^n-1=\Phi_n(X)(X^{d_j}-1)h(X)
	\end{equation}
	for some $h(X)\in\Z[X]$. 
	By evaluating~\eqref{eq:trick_ciclotomico} in $X=q$  
	we obtain that $\Phi_n(q)$ divides $q^n-1$ and that $\Phi_n(q)$
	divides $\frac{q^n-1}{q^{d_j}-1}$. By~\eqref{eq:clases}, 
	$\Phi_n(q)$ divides $q-1$. 
	Thus  
	\[
		q-1\geq |\Phi_n(q)|=\prod |q-\zeta|>q-1,
	\]
	as each $|q-\zeta|>q-1$, 
	a contradiction. 
\end{proof}

There are several proofs of Wedderburn's theorem. 
For example, \cite{MR360687} contains a proof that uses only 
elementary linear algebra. In \cite[Chapter 14]{MR896269}
the theorem is proved using group theory. 

\begin{theorem}
\label{thm:division}
    Let $D$ be a division ring of characteristic $p>0$. 
    If $G$ is a subgroup of $D\setminus\{0\}$, then 
    $G$ is cyclic. 
\end{theorem}

We shall need a lemma. The lemma 
uses a well-known result from elementary number theory:
If $\varphi$ is the Euler function that 
counts the positive integers up to a given integer $n$ 
that are relatively prime to $n$, then 
\[
\sum_{d\mid n}\varphi(d)=n.
\]
Let us present a quick group-theoretical proof. 
Let $G=\langle g\rangle$ be the cyclic group of order $n$.
Then 
\[
n=|G|=\sum_{1\leq d\leq n}|\{g\in G:|g|=d\}|
=\sum_{d\mid n}|\{g\in G:|g|=d\}|
\]
by Lagrange's Theorem. Since $G$ is cyclic, 
for each $d\mid n$, 
$\langle g^{n/d}\rangle$ is the 
unique subgroup of $G$ of order $d$. Now the claim follows, 
as each subgroup
of the form $\langle g^{n/d}\rangle$ has $\varphi(d)$ 
generators. 

\begin{lemma}
    Let $K$ be a field. 
    Any finite subgroup of $K\setminus\{0\}$ is cyclic. 
\end{lemma}

\begin{proof}
    Let $G$ be a finite subgroup of $K\setminus\{0\}$ and
    $n=|G|$. For a divisor $d$ of $n$, let
    $f(d)$ be the number of elements of $G$ of order $d$. 
    Then 
    \begin{equation}
    \label{eq:sum_f}
    \sum_{d\mid n}f(d)=n.
    \end{equation}
    
    We claim that if $d\mid n$ is such that
    $f(d)\ne0$, then $f(d)=\varphi(d)$, where $\varphi$ is
    the Euler function. In fact, if $f(d)\ne 0$, 
    then there exists $g\in G$ such that $|g|=d$. Let 
    $H=\langle g\rangle$ be the subgroup of $G$ generated by $g$.
    Every element of $H$ is a root of the polynomial
    $p(X)=X^d-1\in K[X]$. Since $p(X)$ has at most $d$ roots, 
    $H$ is the set of roots of $p(X)$. In particular, 
    $g^m\in H$ and $|g^m|=d$ if and only if $\gcd(m,d)=1$. Hence
    $f(d)=\varphi(d)$.

    Since $\sum_{d\mid n}\varphi(d)$ and \eqref{eq:sum_f}, 
    it follows that $f(n)=\varphi(n)\ne0$.
    Hence there exists $g\in G$ such that 
    $|g|=n=|G|$ and $G$ is cyclic.
\end{proof}

\begin{proof}[Proof of Theorem \ref{thm:division}]
    Let $F=\sum_{g\in G}(\Z/p)g$. Then $F$ is a finite 
    subring of $D$. Since $D$ is a domain, $F$ is a domain. 
    Let $\alpha\in F\setminus\{0\}$. 
    Then 
    $\{\lambda\alpha:\lambda\in F\}=F$. Since $\lambda\alpha=1$
    for some $\lambda\in F$, $F$ is a division ring. By Wedderburn's 
    theorem, $F$ is a field. Note that $G\subseteq F$. 
    Therefore $G$ is cyclic by the previous lemma. 
\end{proof}

% todo:  (draw a picture of $q$ and $\zeta$ in the complex plane)
% explicar mejor!

% Veamos como corolario una aplicación al último teorema de Fermat en anillos
% finitos. Demostraremos el siguiente resultado:

% \begin{theorem}
% 	Sea $R$ un anillo unitario finito. Entonces para todo $n\geq1$ existen $x,y,z\in
% 	R\setminus\{0\}$ tales que $x^n+y^n=z^n$ si y sólo si $R$ no es un anillo
% 	de división.
% \end{theorem}

% \begin{proof}
% 	Supongamos primero que $R$ es de división. Por el teorema de Wedderburn,
% 	$R$ es entonces un cuerpo finito, digamos $|R|=q$. Como entonces
% 	$x^{q-1}=1$ para todo $x\in R\setminus\{0\}$, se concluye que la ecuación
% 	$x^{q-1}+y^{q-1}=z^{q-1}$ no tiene solución.

% 	Supongamos ahora que $R$ no es de división. Como entonces, en particular,
% 	$R$ no es un cuerpo, $|R|>2$ y luego $x+y=z$ tiene solución en
% 	$R\setminus\{0\}$ (tomar por ejemplo $x=1$, $y=z-1$ y $z\not\in\{0,1\}$).
% 	Como $R$ es finito, $R$ es artiniano a izquierda y entonces el radical de
% 	Jacobson $J(R)$ es nilpotente. Si $J(R)\ne 0$, existe entonces $a\in
% 	R\setminus\{0\}$ tal que $a^2=0$ y luego $a^n=0$ para todo $n\geq2$. En
% 	este caso, la ecuación $x^n+y^n=z^n$ tiene solución en $R\setminus\{0\}$ si
% 	$n\geq 2$ (tomar por ejemplo $x=a$, $y=z=1$). Si $J(R)=0$, entonces, $R$ es
% 	semisimple y luego, por el teorema de Wedderburn,
% 	\[
% 		R\simeq \prod_{i=1}^k M_{n_i}(D_i)
% 	\]
% 	donde los $D_i$ son cuerpos finitos (por ser anillos de división finitos).
% 	Como $R$ no es un cuerpo, hay dos posibilidades: o bien $n_i>1$ para algún
% 	$i\in\{1,\dots,k\}$, o bien $k\geq 2$ y $n_i=1$ para todo
% 	$i\in\{1,\dots,k\}$. En el primer caso, como $M_{n_i}(D_i)$ tiene elementos
% 	no nulos cuyo cuadrado es cero, $R$ también los tiene, y luego, tal como se
% 	hizo antes, vemos que $x^n+y^n=z^n$ tiene solución. En el segundo caso,
% 	$x=(1,0,0,\dots,0)$, $y=(0,1,0,\dots,0)$ y $z=(1,1,0,\dots,0)$ es una
% 	solución de $x^n+y^n=z^n$.
% \end{proof}

\subsection{Zsigmondy's theorem}

One of Wedderburn's original proof of Theorem \ref{thm:Wedderburn} 
uses a 
result proved
by Zsigmondy \cite{MR1546236}. Zsigmondy's theorem is 
quite popular in mathematical contests. 

%Let $a>b\geq1$ be such that $\gcd(a,b)=1$. Let $n\geq2$. 
%We say that
%a prime divisor $p$ is a primitive divisor 
%of $a^n-b^n$ if $p\mid a^n-b^n$ and 
%$p\nmid a^k-b^k$ for all $k\in\{1,\dots,n-1\}$. 

\begin{theorem}[Zsigmondy]
\index{Zsigmondy's theorem}
    Let $a>b\geq1$ be such that $\gcd(a,b)=1$ and $n\geq2$. 
    Then there exists a prime divisor of $a^n-b^n$ that does not
    divide $a^k-b^k$ for all $k\in\{1,\dots,n-1\}$ except 
    when $n=2$ and $a+b$ is a power of two or $(a,b,n)=(2,1,6)$. 
\end{theorem}

% https://angyansheng.github.io/blog/an-elementary-proof-of-zsigmondys-theorem

\begin{proof}
    See for example \cite{MR3172590}. 
\end{proof}

We now quickly sketch a proof of Wedderburn's theorem \ref{thm:Wedderburn} 
based on Zsigmondy's theorem.  

Let $D$ be a division ring of dimension $n$ over $\Z/p$ for a prime
number $p$. Assume 
first that there exists a prime number $q$ such that 
$q\nmid p$ and the order of $p$ modulo $q$ is $n$. Let $x\in D\setminus\{0\}$ be an element of order $q$ and $F$ be the subring
of $D$ generated by $g$. Note that $F$ is a finite-dimensional
$(\Z/p)$-vector space. Let $m=\dim F$. 
Since $g^{p^m-1}=1$, $q$ divides $p^m-1$. Thus $m=n$ and
hence $D=F$ is commutative. 

Assume now that there is no prime number $q$ such that 
$q\nmid p$ and the order of $p$ modulo $q$ is $n$. By Zsigmondy's 
theorem, $n=2$ or $n=6$ and $p=2$. If $n=2$, then 
$D$ is commutative, as it is the subring generated by
any element of $D\setminus \Z/p$. If $n=6$ and $p=2$, then 
the order of 2 modulo 9 is 6. Since $D\setminus\{0\}$ contains
a subgroup of order 9 and all groups of order 9 are abelian, 
we can use the previous argument to complete the proof. 


\subsection{Fermat's last theorem in finite rings}

\begin{theorem}
\index{Fermat's last theorem for finite rings}
    Let $K$ be a finite field and $A$ be a finite-dimensional $K$-algebra.
    For $n\geq1$, there exist $x,y,z\in A\setminus\{0\}$ 
    such that $x^n+y^n=z^n$ if and only if 
    $A$ is not a division algebra.
\end{theorem}

\begin{proof}
    Assume first that $A$ is a division algebra. By Wedderburn's theorem, 
    $A$ is a finite field, say $|A|=q$. Then $x^{q-1}=1$ for all $x\in A\setminus\{0\}$.
    Hence $x^n+y^n=z^n$ does not have a solution. 
    
    Conversely, assume that $A$ is not a division algebra. In particular, 
    $A$ is not a field and $|A|>2$. The equation $x+y=z$ has a solution in $A\setminus\{0\}$ (for example, $x=1$, $y=z-1$ and $z\not\in\{0,1\}$ is a solution). Since
    $\dim A<\infty$, the Jacobson radical $J(A)$ is nilpotent. There are two 
    cases to consider. 
    
    If $J(A)\ne\{0\}$, 
    then there exists $a\in A\setminus\{0\}$ such that $a^2=0$. Thus $a^n=0$ 
    for all $n\geq2$. Hence $x^n+y^n=z^n$ has a non-trivial
    solution in $A\setminus\{0\}$ for all $n\geq2$ (for example, take 
    $x=a$ and $y=z=1$).
    
    If $J(A)=\{0\}$, then $A$ is semisimple and 
    $A\simeq\prod_{i=1}^k M_{n_i}(D_i)$ for (finite) division rings $D_1,\dots,D_k$
    and integers $n_1,\dots,n_k$. By Wedderburn's theorem, each $D_i$ is a finite
    field. We consider two possible cases. 
    
    If there exists $i\in\{1,\dots,k\}$ such that $n_i>1$, then
    $M_{n_i}(D_i)$ has non-zero elements such that their squares are zero. Thus 
    there exists $x\in A\setminus\{0\}$ such that $x^2=0$. In particular, 
    $x^n+y^n=z^n$ has a solution. 
    
    If $k\geq 2$, then $x=(1,0,0,\dots,0)$, $y=(0,1,0,\dots,0)$ 
    and $z=(1,1,0,\dots,0)$ is a solution of $x^n+y^n=z^n$.
\end{proof}
