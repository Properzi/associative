\section{Lecture: 17/10/2024}
\label{03}

\subsection{Group algebras}

Let $K$ be a field, and $G$ be a group. The \emph{group algebra} $K[G]$ 
is the vector space (over $K$) with basis $\{g:g\in G\}$ 
and the algebra structure is given by the multiplication
\[
	\left(\sum_{g\in G}\lambda_gg\right)\left(\sum_{h\in G}\mu_hh\right)
	=\sum_{g,h\in G}\lambda_g\mu_h(gh).
\]
Every element of $K[G]$ is a finite sum of the form $\sum_{g\in G}\lambda_gg$.

\begin{exercise}
\label{xc:K[G]notsimple}
    If $G$ is non-trivial, then $K[G]$ is not simple. 
\end{exercise}

\begin{exercise}
\label{xca:K_cyclic}
	Let $G=C_n$ be the (multiplicative) cyclic group of order $n$. Prove that 
	$K[G]\simeq K[X]/(X^n-1)$. 
\end{exercise}

\begin{exercise}
\label{xca:abelian=>domain}
	Let $G$ be a finitely-generated torsion-free abelian group. Prove that 
	$K[G]$ is a domain. 
\end{exercise}



\begin{exercise}
	Let $G$ be a group and $\alpha=\sum_{g\in G}\lambda_gg\in K[G]$.  
	The \emph{support} of $\alpha$ is the set 
	\[
		\supp\alpha=\{g\in G:\lambda_g\ne 0\}.
	\]
	Prove that if $g\in G$, then 
	$\supp(g\alpha)=g(\supp\alpha)$ and $\supp(\alpha g)=(\supp\alpha)g$.
\end{exercise}

\begin{exercise}
\label{xca:invertible_subgroups}
	Let $G$ be a group and $H$ be a subgroup of $G$. Let $\alpha\in K[H]$. Prove that 
    $\alpha$ is invertible (resp. a left zero divisor) in $K[H]$ if and only if 
	$\alpha$ is invertible (resp. a left zero divisor) in
	$K[G]$.
\end{exercise}

% El objetivo de esta sección es calcular el radical de Jacobson del álgebra de
% grupo de un grupo finito. Comenzamos con un ejemplo:

\begin{exercise}
	Let $G=C_2=\langle g\rangle\simeq\Z/2$ the (multiplicative) 
	group with two elements. Note that every element of $K[G]$ is of the form
	$a+bg$ for some $a,b\in K$. Prove the following statements:
	\begin{enumerate}
	    \item If the characteristic of $K$ is different from two, then 
	    \[
		K[G]\to K\times K,
		\quad
		a1+bg\mapsto (a+b,a-b),
	\]
	is an algebra isomorphism. 
	\item If the characteristic of $K$ is two, then 
	\[
	K[G]\to \begin{pmatrix}
			K & K\\
			0 & K
		\end{pmatrix},
		\quad
		a1+bg\mapsto\begin{pmatrix}
			a+b & b\\
			0 & a+b
		\end{pmatrix},
	\]
	is an algebra isomorphism. 
	\end{enumerate}
\end{exercise}

If $A$ is an algebra over $K$ and $\rho\colon G\to \mathcal{U}(A)$
is a group homomorphism, where $\mathcal{U}(A)$ is the group of units of $A$, then 
the map \[
	K[G]\to A,\quad 
\sum_{g\in G}\lambda_gg\mapsto\sum_{g\in G}\lambda_g\rho(g),
\]
is an algebra homomorphism. 

\begin{exercise}
	Let $G=C_3$ be the (multiplicative) group of three elements. Prove that
	$\R[G]\simeq\R\times\C$.
% 	Escribamos $G=\langle g:g^3=1\rangle$ y sea 
% 	\[
% 		\varphi\colon\R[G]\to\R\times\C,
% 		\quad
% 		g\mapsto (1,\omega),
% 	\]
% 	donde $\omega$ es una raíz cúbica primitiva de la unidad. Entonces
% 	$\varphi$ es inyectivo pues
% 	$0=\varphi(a1+bg+cg^2)=(a+b+c,a+b\omega+c\omega^2)$ implica que $a=b=c=0$.
% 	Luego $\varphi$ es un isomorfismo pues
% 	$\dim_\R\R[G]=\dim_\R(\R\times\C)=3$. 
\end{exercise}

\begin{exercise}
\label{xca:isos_dihedral}
	Let $G=\langle r,s:r^3=s^2=1,\,srs=r^{-1}\rangle$ be the dihedral group of six elements. 
	Prove the following statements:
	\begin{enumerate}
	    \item $\C[G]\simeq\C\times\C\times M_2(\C)$.
	    \item $\Q[G]\simeq\Q\times\Q\times M_2(\Q)$.
	\end{enumerate}  
% 	Sea $\omega$ una raíz cúbica de la unidad y sean  
% 	\[
% 		R=\begin{pmatrix}
% 			\omega & 0\\
% 			0 & \omega^2
% 		\end{pmatrix},
% 		\quad
% 		S=\begin{pmatrix}
% 			0 & 1\\
% 			1 & 0
% 		\end{pmatrix}.
% 	\]
% 	Un cálculo sencillo muestra que $R^2=S^2=I$ y que $SRS=R^{-1}$. Sea
% 	\[
% 		\varphi\colon\C[G]\to\C\times\C\times M_2(\C),\quad
% 		r\mapsto (1,1,R),\quad
% 		s\mapsto (1,-1,S).
% 	\]
% 	Es fácil ver que $\varphi$ es un morfismo de álgebras. Veamos que es
% 	biyectivo. Como $\dim_{\C}\C[G]=\dim_{\C}(\C\times\C\times M_2(\C))=6$,
% 	basta ver que $\varphi$ es inyectivo. Si 
% 	\[
% 		\alpha=a_0+a_1r+a_2r^2+(b_0+b_1r+b_2r^2)s\in\ker\varphi,
% 	\]
% 	entonces 
% 	\[
% 		0=\varphi(\alpha)=\left(u,v,\begin{pmatrix} \alpha_{11} & \alpha_{12}\\\alpha_{21}&\alpha_{22}\end{pmatrix}\right), 
% 	\]
% 	donde
% 	\begin{align*}
% 		&u = a_0+a_1+a_2+b_0+b_1+b_2, && v = a_0+a_1+a_2-b_0-b_1-b_2,\\
% 		&\alpha_{11}=a_0+a_1\omega+a_2\omega^2, && \alpha_{12}=b_0+b_1\omega+b_2\omega^2,\\
% 		&\alpha_{21}=b_0+b_2\omega+b_1\omega^2, && \alpha_{22}=a_0+a_2\omega+a_1\omega^2.
% 	\end{align*}
% 	Un cálculo sencillo muestra que estas ecuaciones implican que
% 	$\alpha=0$ y luego $\varphi$ es inyectiva.  
\end{exercise}



Maschke's theorem states that, if $G$ is a finite group, 
then the group algebra $\C[G]$ is semisimple. By Mollien's theorem, 
\[
\C[G]\simeq \prod_{i=1}^k M_{n_i}(\C),
\]
where $k$ is the number of (isomorphism classes of) 
simple $\C[G]$-modules. Moreover, 
\[
|G|=\dim\C[G]=\sum_{i=1}^k n_i^2.
\]

\begin{theorem}
    Let $G$ be a finite group. The number of simple 
    modules of $\C[G]$ coincides with the number of conjugacy classes of $G$. 
\end{theorem}

\begin{proof}
    By Mollien's theorem, $\C[G]\simeq\prod_{i=1}^kM_{n_i}(\C)$. Thus 
    \[
		Z(\C[G])\simeq\prod_{i=1}^kZ(M_{n_i}(\C))\simeq\C^k.
	\]
	In particular, $\dim Z(\C[G])=k$. If $\alpha=\sum_{g\in
	G}\lambda_gg\in Z(\C[G])$, then $h^{-1}\alpha h=\alpha$ for all $h\in
	G$. Thus 
	\[
		\sum_{g\in G}\lambda_{hgh^{-1}}g=
		\sum_{g\in g}\lambda_g h^{-1}gh=\sum_{g\in G}\lambda_gg
	\]
	and hence $\lambda_{g}=\lambda_{hgh^{-1}}$ for all $g,h\in G$. A basis for 
	$Z(\C[G])$ is given by elements of the form 
	\[
		\sum_{g\in K}g,
	\]
	where $K$ is a conjugacy class of $G$. Therefore $\dim Z(\C[G])$ is equal to 
	the number of conjugacy classes of $G$.
\end{proof}

\subsection{Which algebras are group algebras?}

\begin{example}
\label{exa:C4}
    Let $G=C_4$ be the cyclic group of order four. Then
    $G$ has four simple modules and 
    $\C[G]\simeq\C^4$. 
\end{example}

\begin{example}
\label{exa:S3}
    Let $G=\Sym_3$. Then $G$ has three simple modules and
    \[
    \C[G]\simeq\C\times\C\times M_2(\C).
    \]
\end{example}

\begin{problem}[Brauer]
\index{Brauer's problem}
    Which algebras are group algebras? 
\end{problem}

This question might be impossible to answer, but it is extremely interesting. 
Examples \ref{exa:C4} and \ref{exa:S3} show
that $\C^4$ and $\C^2\times M_2(\C)$ are complex group algebras. 

\begin{exercise}
    Is $\C^2\times M_2(\C)\times M_3(\C)$ a complex group algebra?  
\end{exercise}

% No. Let $G$ be a group of order 15. Since groups of order 15 are abelian,
% $G$ has 15 conjugacy classes. 

\subsection{The isomorphism problem for group algebras}


Recall that if $R$ is a unitary commutative ring 
and $G$ is a group, then one defines the group ring $R[G]$ (see Appendix \ref{section:Hurewicz}). 
% More precisely,
% $R[G]$ is the set of finite linear combinations
% \[
%     \sum_{g\in G}\lambda_gg
% \]
% where $\lambda_g\in R$ and $\lambda_g=0$ for all but finitely many $g\in G$.
% One easily proves that $R[G]$ is a ring with
% addition
% \[
% \left(\sum_{g\in G}\lambda_gg\right)+\left(\sum_{g\in G}\mu_gg\right)
% =\sum_{g\in G}(\lambda_g+\mu_g)(g)
% \]
% and multiplication
% \[
% \left(\sum_{g\in G}\lambda_gg\right)\left(\sum_{h\in G}\mu_hh\right)
% =\sum_{g,h\in G}\lambda_g\mu_h(gh).
% \]
Note that $R[G]$ is a left $R$-module with
\[
\lambda\left(\sum_{g\in G}\lambda_gg\right)=\sum_{g\in G}(\lambda\lambda_g)g.
\]

In this section, we will briefly discuss the
following natural problem: 

\begin{question}[The isomorphism problem]
\label{question:iso}
 Let $R$ be a ring and $G$ and $H$ be groups. Assume
 that $R[G]\simeq R[H]$ (as $R$-algebras). Does $G\simeq H$?
\end{question}

For general information on Question \ref{question:iso} we refer to the survey paper \cite{MR4472590}.

\begin{exercise}
    Prove that if $G$ and $H$ are isomorphic groups, then $K[G]\simeq K[H]$.
\end{exercise}

\begin{exercise}
    Let $G$ and $H$ be groups. Prove that if
    $\Z[G]\simeq\Z[H]$, then $R[G]\simeq R[H]$ for any commutative ring $R$.
\end{exercise}

The previous exercise suggest the importance of the following 
instance of Question \ref{question:iso}:

\begin{question}
\label{question:IP}
    Let $G$ and $H$ be groups. Does $\Z[G]\simeq\Z[H]$ imply $G\simeq H$?
\end{question}

Although there are several cases where
the isomorphism problem has an affirmative answer (e.g. abelian groups,
metabelian groups, nilpotent groups, nilpotent-by-abelian groups, simple groups,
abelian-by-nilpotent groups), it is false in general. In 2001
Hertweck found a counterexample of order $2^{21}97^{28}$, see \cite{MR1847590}.

\begin{question}[The modular isomorphism problem]
\label{question:MIP}
    Let $p$ be a prime number. Let
    $G$ and $H$ be finite $p$-groups and let $K$ be a field of characteristic $p$.
    Does $K[G]\simeq K[H]$ imply $G\simeq H$?
\end{question}

Question \ref{question:MIP} has an affirmative answer in several cases. However,
this is not true in general. This question was recently answered by Garc\'ia, Margolis and
del R\'io \cite{MR4373245}. They found two non-isomorphic groups $G$ and $H$ both of order $512$
such that $K[G]\simeq K[H]$ for all field $K$
of characteristic two.

\subsection{Primitive rings}
\label{Primitive rings}

We will consider (possibly non-unitary) rings. Thus  
a \emph{ring} is an abelian group $R$ with an associative multiplication 
$(x,y)\mapsto xy$ such that $(x+y)z=xz+yz$ and $x(y+z)=xy+xz$ for all $x,y,z\in
R$. If there is an element $1\in R$ such that $x1=1x=x$ for all $x\in R$, we 
say that $R$ is a \emph{unitary ring}.  A \emph{subring} $S$ of $R$ is an additive
subgroup of $R$ closed under multiplication. 

\begin{example}
    $\Z$ is a (unitary) ring and 
	$2\Z=\{2m:m\in\Z\}$ is a (non-unitary) ring.  
\end{example}

A \emph{left ideal} (resp. \emph{right ideal}) is a subring $I$ of $R$ such that 
$rI\subseteq I$ (resp. $Ir\subseteq I$) for all $r\in R$. An \emph{ideal}
(also two-sided ideal) of $R$ is a subring $I$ of $R$ that is both a left and a right ideal of $R$.

\begin{example}
	If $I$ and $J$ are both ideals of a ring $R$, then the sum 
 \[
 I+J=\{x+y:x\in I,y\in J\}
 \]
 and
	the intersection $I\cap J$ are both ideals of $R$. The product $IJ$, defined as the additive
	subgroup of $R$ generated by $\{xy:x\in I,y\in J\}$, is also an ideal of $R$. 
\end{example}

\begin{example}
	If $R$ is a ring, the set $Ra =\{xa: x\in R\}$ is a left ideal
	of $R$. Similarly, the set $aR =\{ax: x\in R\}$ is a right ideal of $R$. The set $RaR$, which is
	defined as the additive subgroup of $R$ generated by $\{xay: x, y\in R\}$, is a
	ideal of $R$.
\end{example}

\begin{example}
	If $R$ is a unitary ring, then $Ra$ is the left ideal generated by $a$, $aR$ is
	the right ideal generated by $a$ and $RaR$ is the ideal generated by $a$. 
	If $R$ is not unitary, the left ideal generated by $a$ is $Ra+\Z a$,
	the right ideal generated by $a$ is $aR+\Z a$ and the ideal generated by 
	$a$ is $RaR+Ra+aR+\Z a$.
\end{example}

The following exercise asks to prove the \emph{Chinese Remainder Theorem}  
for arbitrary rings.

\begin{bonus}
    \label{xca:chinese}
    \index{Chinese Remainder Theorem}
    Let $R$ be a ring and $I_1,\dots,I_n$ be ideals such that 
    $I_j+I_k=R$ whenever $j\ne k$ and $R=I_j+R^2$ for all $j$. Prove that 
    \[
    R/(I_1\cap\cdots\cap I_n)\simeq R/I_1\times\cdots\times R/I_n.
    \]
\end{bonus}

In the previous exercise, the condition $R=I_j+R^2$ trivially holds in the case of rings with one. 

\begin{definition}
\index{Ring!simple}
A ring $R$ is said to be \emph{simple} if $R^2\ne\{0\}$ and the only ideals of 
$R$ are $\{0\}$ and~$R$.  
\end{definition}

The condition $R^2\ne\{0\}$ is trivially satisfied in the case of rings
with identity, as 
\[
1\in R^2=\{r_1r_2:r_1,r_2\in R\}.
\]

\begin{example}
	Division rings are simple.
\end{example}

Let $S$ be a unitary ring. Recall that $M_n(S)$ is the ring of $n\times n$ square matrices 
with entries in $S$.  If $A=(a_{ij})\in M_n(S)$ and $E_{ij}$ is the matrix
such that $(E_{ij})_{kl}=\delta_{ik}\delta_{jl}$, then
\begin{equation}
	\label{eq:trick}
E_{ij}AE_{kl}=a_{jk}E_{il}
\end{equation}
for all $i,j,k,l\in\{1,\dots,n\}$. 

\begin{example}
	If $D$ is a division ring, then $M_n(D)$ is simple. 
\end{example}

Let $R$ be a ring. A left $R$-module (or module, for short)  
is an abelian group $M$ together with a map $R\times M\to M$, $(r,m)\mapsto r\cdot m$, such that
\begin{align*}
	&(r+s)\cdot m=r\cdot m+s\cdot m, &&
	r\cdot (m+n)=r\cdot m+r\cdot s, && r\cdot (s\cdot m)=(rs)\cdot m    
\end{align*}
for all $r,s\in R$, $m,n\in M$.  If $R$ has an identity 
$1$ and $1\cdot m=m$ holds for all $m\in M$, the module $M$ is said to be 
\emph{unitary}.  If $M$ is a unitary module, then $M=R\cdot M$. %\ne\{0\}$.

\begin{exercise}
\label{xca:center_simple}
Let $R$ be a simple unitary ring. 
\begin{enumerate}
    \item Prove that the center $Z(R)$ of $R$ is a field.
    \item Prove that $R$ is an algebra over $Z(R)$. 
\end{enumerate}
\end{exercise}

% Let $0\ne x\in Z(R)$. Then $Rx$ is a non-zero ideal of $R$.
% Since $R$ is simple, $Rx=R$. Thus $rx=1$ for some $r\in R$. 
% It follows that $x\in\mathcal{U}(R)$.

\begin{definition}
\label{Module!simple}
    A module $M$ is said to be 
    \emph{simple} if $R\cdot M\ne\{0\}$ and 
    the only submodules of $M$ are $\{0\}$ and $M$.
    If $M$ is a simple module, then $M\ne\{0\}$.
\end{definition}

If $R$ is a unitary ring and $M$ is a simple 
module, then $M$ is unitary. 

%\begin{remark}
%	Si $R$ es unitario y $M$ es un módulo simple, entonces $M$ es unitario.
%\end{remark}

\begin{lemma}
	\label{lemma:simple}
	Let $M$ be a non-zero module. Then $M$ is simple if and only if $M=R\cdot m$
	for all $0\ne m\in M$.
\end{lemma}

\begin{proof}
	Assume that $M$ is simple.  Let $m\ne 0$. Since $R\cdot m$ is a submodule of the simple 
	module $M$, either $R\cdot m=\{0\}$ or $R\cdot m=M$.  Let $N=\{n\in M:R\cdot n=\{0\}\}$. Since $N$ is a 
	submodule of $M$ and $R\cdot M\ne\{0\}$, $N=\{0\}$. Therefore $R\cdot m=M$, as $m\ne0$.
	Now assume that $M=R\cdot m$ for all $m\ne0$. Let $L$ be a non-zero submodule of 
	$M$ and let $0\ne x\in L$. Then $M=L$, as $M=R\cdot x\subseteq L$. 
\end{proof} 

\begin{example}
	Let $D$ be a division ring and let $V$ be a non-zero vector space (over $D$). If 
	$R=\End_D(V)$, then $V$ is a simple $R$-module with $fv=f(v)$, $f\in R$.
	$v\in V$. 
% 	Para ver que $V$ es simple como $R$-módulo basta ver que $Rv=V$ para todo
% 	$v\ne0$. Sean $v,w\in V$, $v\ne0$.  Al completar $v\ne0$ a una base de $V$,
% 	vemos que existe $f\in R$ tal que $f(v)=w$. Luego $V$ es simple.
\end{example}

\begin{example}
	\label{exa:I_k}
	Let $n\geq2$.  If $D$ is a division ring and $R=M_n(D)$, then each 
	\[
	I_k=\{ (a_{ij})\in R:a_{ij}=0\text{ for $j\ne k$}\}
	\]
	is an $R$-module isomorphic to $D^n$. 
	Thus $M_{n}(D)$ is a simple ring that is not a simple $M_n(D)$-module.
\end{example}

\begin{definition}
\index{Minimal left ideal}
A left ideal $L$ of a ring $R$ is said to be \emph{minimal} if $L\ne\{0\}$ and 
$L$ does not properly contain non-zero left ideals of $R$. 
\end{definition}

Similarly one defines
right minimal ideals and minimal ideals. 

\begin{example}
	Let $D$ be a division ring and let $R=M_n(D)$. Then $L=RE_{11}$ 
	is a minimal left ideal.
\end{example}

\begin{example}
	Let $L$ be a non-zero left ideal. If $RL\ne\{0\}$, then
	$L$ is minimal if and only if $L$ is a simple $R$-module. 
\end{example}

\begin{definition}
\index{Ideal!regular}
\index{Left ideal!regular}
\label{def:regular}
A left (resp. right) ideal $L$ of $R$ is said to be \emph{regular} if
there exists $e\in R$ such that $r-re\in L$ (resp.  $r-er\in L$) for all $r\in R$.
\end{definition}

If $R$ is a ring with identity, every left (or right) ideal is regular. 

\begin{definition}
\index{Ideal!maximal}
\index{Left ideal!maximal}
A left (resp. right) ideal $I$ of $R$ is said to be \emph{maximal} if $I\ne R$ and $I$ is not properly contained 
in a proper left (resp. right) ideal of $R$. 
\end{definition}

Similarly, one defines maximal ideals. 

A standard application of Zorn's lemma proves 
that every unitary ring contains a maximal left (or right) ideal.  

\begin{proposition}
	\label{proposition:R/I}
	Let $R$ be a ring and $M$ be a module. Then $M$ is simple if and only if
	$M\simeq R/I$ for some maximal regular left ideal $I$. 	
\end{proposition}

\begin{proof}
	Assume that $M$ is simple. Then $M=R\cdot m$ for some $m\ne0$ by 
	Lemma~\ref{lemma:simple}. The map $\phi\colon R\to M$, $r\mapsto r\cdot m$, 
	is a surjective homomorphism of $R$-modules, 
	so the first isomorphism theorem implies that 
	$M\simeq R/\ker\phi$. Since $\ker\phi$ is an ideal of $R$, it is 
	in particular a left ideal of $R$. 
	
	We claim that $I=\ker\phi$ is a maximal left ideal. 
	The correspondence theorem 
	and the simplicity of $M$ imply that $I$ is a 
	maximal left ideal (because each left ideal $J$ such that 
	$I\subseteq J$ yields a submodule of $R/I$).

	We claim that $I$ is regular. Since $M=R\cdot m$, there exists $e\in R$ such that $m=e\cdot m$. If
	$r\in R$, then $r-re\in I$ since 
	$\phi(r-re)=\phi(r)-\phi(re)=r\cdot m-r\cdot (e\cdot m)=0$.

    Now assume that $I$ is a maximal left ideal that is regular. 
    The correspondence theorem implies that 
    $R/I$ has no non-zero proper submodules. 
    
    We claim that 
    $R\cdot (R/I)\ne\{0_{R/I}\}$. Assume that $R\cdot (R/I)=\{0_{R/I}\}$ and let $r\in R$. 
    The regularity of $I$ implies that 
    there exists $e\in R$ such that $r-re\in I$. Hence $r\in I$, as  
	\[
	I=r\cdot (e+I)=re+I=r+I,
	\]
	a contradiction to the maximality of $I$. 
\end{proof}


%\section{Nilpotencia}
%
%Recordemos que si $I$ es un ideal de un anillo $R$, se define $I^n$ como el
%subgrupo aditivo generado por el conjunto $\{y_1\dots y_n:y_j\in I\}$. 
%
%\begin{definition}
%	Un ideal $I$ de un anillo $R$ se dice \emph{nilpotente} si $I^n=0$ para
%	algún $n\in\N$.
%\end{definition}
%
%Recordemos que un elemento $x$ de un anillo $R$ se dice \emph{nilpotente} si
%existe $n\in\N$ tal que $x^n=0$. 
%
%\begin{definition}
%	Un ideal $I$ de un anillo $R$ se dice \emph{nil} si todo elemento de $I$
%	es nilpotente.
%\end{definition}
%
%\begin{remark}
%	Un ideal nilpotente es nil. 
%\end{remark}
%
%\begin{example}
%	Sea $R=\C[x_1,x_2,\dots]/(x_1,x_2^2,x_3^3,\dots)$. El ideal
%	$I=(x_1,x_2,x_3,\dots)$ es nil en $R$ pues está generado por elementos
%	nilpotentes pero no es nilpotente. Si lo fuera, existiría $k\in\N$ tal que
%	$I^k=0$, y luego $x_i^k=0$ para todo $i$, una contradicción pues
%	$x_{k+1}^k\ne0$. 	
%\end{example}
%
%% example 2.7 del libro de springer
%% problema de kothe 
%
%\begin{lemma}
%	Si $I$ y $J$ son ideales nilpotentes, $I+J$ es nilpotente.
%\end{lemma}
%
%\begin{proof}
%	
%\end{proof}
%
%Un ideal $N$ de un anillo $R$ se dice \emph{maximal-nilpotente} si $N$ es
%nilpotente y no está propiamente contenido en ningún ideal nilpotente de $R$.
%
%\begin{lemma}
%	Si el anillo $R$ contiene un ideal maximal-nilpotente $N$ entonces todo
%	ideal nilpotente está contenido en $N$.
%\end{lemma}
%
%\begin{proof}
%	
%\end{proof}
%
%\section{Anillos primos y semiprimos}
%
%\index{Dominio}
%Recordemos que un anillo $R$ se dice un \emph{dominio} si para todo $a,b\in
%R$ tales que $ab=0$ se tiene $a=0$ o $b=0$.
%Una generalización al caso no conmutativo es la siguiente:
%
%\begin{definition}
%	\index{Anillo!primo}
%	Un anillo $R$ se dice \emph{primo} si para todo $a,b\in R$ tales que
%	$aRb=0$ se tiene $a=0$ o $b=0$.
%\end{definition}
%
%\begin{lemma}
%	Sea $R$ un anillo. Las siguientes propiedades son equivalentes:
%	\begin{enumerate}
%		\item $R$ es primo.
%		\item Si $I,J\subseteq R$ son ideales a izquierda tales que $IJ=0$
%			entonces $I=0$ o $J=0$.
%%		\item Si $I,J\subseteq R$ son ideales a derecha tales que $IJ=0$ 
%%			entonces $I=0$ o $J=0$.
%		\item Si $I,J\subseteq R$ son ideales tales que $IJ=0$ entonces $I=0$ o
%			$J=0$.
%	\end{enumerate}
%\end{lemma}
%
%\begin{proof}
%	Vamos a demostrar que $(1)\implies(2)\implies(4)\implies(1)$. 
%	La implicación $(2)\implies(4)$ es trivial.
%
%	Veamos que $(4)\implies(1)$. Sean $a,b\in R$ tales que $aRb=0$.
%	Como entonces $(RaR)(RbR)=R(aRb)R=0$, $RaR=0$ o bien $RbR=0$. Supongamos
%	sin pérdida de generalidad que $RaR=0$. Entonces $Ra$ y $aR$ son ideales
%	biláteros tales que $(Ra)R=R(aR)=0$. Al aplicar la hipótesis, $Ra=aR=0$.
%	Como $\Z a$ es un ideal de $R$ tal que $(\Z a)R=0$, se concluye al aplicar
%	la hipótesis que $a=0$.
%
%	Veamos que $(1)\implies(2)$. Supongamos que $J\ne 0$, sea $y\in
%	J\setminus\{0\}$ y sea $x\in I$. Como 
%	$xRy\subseteq IRJ=I(RJ)\subseteq IJ=0$, 
%	se concluye, al usar la hipótesis, que $x=0$. 
%\end{proof}
%
%\begin{proposition}
%	Un anillo conmutativo es primo si y sólo si es un dominio. 
%\end{proposition}
%
%\begin{proof}
%	Supongamos que $R$ es un anillo primo. Si $a,b\in R$ son tales que $ab=0$
%	entonces $aRb=(ab)R=0$ y luego $a=0$ o bien $b=0$. Supongamos ahora que $R$
%	es un dominio. Si $a,b\in R$ son tales que $aRb=0$ entonces $(ab)R=0$ y
%	luego $a=0$ o bien $b=0$ pues $ab=0$. 
%\end{proof}
%
%\begin{definition}
%	\index{Anillo!semiprimo}
%	Un anillo $R$ se dice \emph{semiprimo} si para todo $a\in R$ tal que
%	$aRa=0$ se tiene $a=0$.
%\end{definition}
%
%\begin{lemma}
%	Sea $R$ un anillo. Las siguientes aifrmaciones son equivalentes:
%	\begin{enumerate}
%		\item $R$ es semiprimo.
%		\item Si $I$ es un ideal a izquierda tal que $I^2=0$ entonces $I=0$.
%		\item Si $I$ es un ideal tal que $I^2=0$ entonces $I=0$.
%		\item $R$ no tiene ideales nilpotentes no nulos.
%	\end{enumerate}
%\end{lemma}
%
%\begin{proof}
%	Primero vamos a demostrar que $(1)\implies(2)\implies(3)\implies(1)$ 
%
%	La implicación $(5)\implies(4)$ es trivial.	
%	Demostremos entonces que
%	$(4)\implies(5)$. Sea $I$ un ideal nilpotente no nulo y sea $n\in\N$ el
%	mínimo tal que $I^n=0$. Como $(I^{n-1})^2=0$, por hipótesis se tiene que
%	$I^{n-1}=0$, una contradicción.
%\end{proof}
%

Let $R$ be a ring and $M$ be a left $R$-module. For a 
subset $N\subseteq M$
we define the \emph{annihilator} of $N$ as the subset 
\[
\Ann_R(N)=\{r\in R:r\cdot n=0\text{ for all }n\in N\}.
\]

\begin{example}
	$\Ann_{\Z}(\Z/n)=n\Z$.
\end{example}

\begin{exercise}
    Let $R$ be a ring and $M$ be a module. If $N\subseteq M$ is a subset, then 
	$\Ann_R(N)$ is a left ideal of $R$. If $N\subseteq M$ is a submodule of $R$, then 
	$\Ann_R(N)$ is an ideal of $R$. 
\end{exercise}

% \begin{lemma}
% 	\label{lemma:Ann}
% 	Let $R$ be a ring and $M$ be a module. If $N\subseteq M$ is a subset, then 
% 	$\Ann_R(N)$ is a left ideal of $R$. If $N\subseteq M$ is a submodule of $R$, then 
% 	$\Ann_R(N)$ is an ideal of $R$. 
% \end{lemma}

% \begin{proof}
% 	We left as an exercise to prove that $\Ann_R(N)$ is an additive subgroup of $R$. Then $\Ann_R(N)$
% 	is a left ideal, as $R\Ann_R(N)\subseteq\Ann_R(N)$. Indeed, if $r\in R$,
% 	$s\in\Ann_R(N)$ and $n\in N$, then $(rs)n=r(sn)=r0=0$. 
	
% 	If $N$ is a submodule, $\Ann_R(N)R\subseteq\Ann_R(N)$ since if 
% 	$s\in\Ann_R(N)$, $r\in R$ and $n\in N$, $rn\in\Ann_R(N)$, then
% 	$(sr)n=s(rn)=0$.
% \end{proof}

\begin{definition}
\index{Module!faithful}
A module $M$ is said to be \emph{faithful} if $\Ann_R(M)=\{0\}$. 
\end{definition}

\begin{example}
	If $K$ is a field, then $K^n$ is a faithful unitary $M_n(K)$-module.
\end{example}

\begin{example}
	If $V$ is vector space over a field $K$, then $V$ is faithful unitary $\End_K(V)$-module.
\end{example}

\begin{definition}
\index{Ring!primitive}
A ring $R$ is said to be \emph{primitive} if there exists a faithful simple $R$-module.  
\end{definition}

Since 
we are considering left modules, our definition of primitive rings is that of left primitive rings.
By convention, a primitive ring
will always mean a left primitive ring. 
The use 
of right modules yields to the notion of right primitive rings.  

\begin{exercise}
	\label{xca:simple=>prim}
	If $R$ is a simple unitary ring, then $R$ is primitive. 
\end{exercise}

\begin{exercise}
	\label{xca:prim+conm=cuerpo}
	If $R$ is a commutative ring (maybe without identity), then $R$ is primitive if and only if $R$ is a field. 
\end{exercise}

\begin{example}
	The ring $\Z$ is not primitive. 
\end{example}

\begin{definition}
\index{Ideal!primitive}
An ideal $P$ of a ring $R$ is said to be \emph{primitive} if $P=\Ann_R(M)$
for some simple $R$-module $M$. 
\end{definition}

As we work with left modules, our definition of primitive rings refers to \emph{left primitive rings}. Of course, one can also define right primitive rings. As Bergman showed, there exist rings that are right primitive but not left primitive; see \cite{MR175940,MR167497}. 

\begin{lemma}
	\label{lemma:primitivo}
	Let $R$ be a ring and $P$ be an ideal of $R$. Then $P$ is primitive if and only if 
	$R/P$ is a primitive ring.
\end{lemma}

\begin{proof}
	Assume that $P=\Ann_R(M)$ for some $R$-module $M$. Then $M$ is a simple 
	$(R/P)$-module with 
	\[
	(r+P)\cdot m=r\cdot m,\quad r\in R,\;m\in M. 
	\]
	This operation 
	is well-defined, as 
	$P=\Ann_R(M)$. Since $M$ is a simple $R$-module, it follows that $M$ is 
	a simple $(R/P)$-module. Moreover, $\Ann_{R/P}M=\{0\}$. Indeed, if 
	$(r+P)\cdot M=\{0\}$, then $r\in\Ann_RM=P$ and hence $r+P=P$.

	Assume now that $R/P$ is primitive. Let $M$ be a faithful simple $(R/P)$-module. 
	Then 
    \[
    r\cdot m=(r+P)\cdot m, \quad r\in R,\; m\in M,
    \]
    turns $M$ into an $R$-module. It follows that $M$ is simple and that $P=\Ann_R(M)$. 
\end{proof}

%\begin{example}
%	Si $I$ es un ideal maximal de un anillo unitario $R$, entonces $I$ es
%	primitivo. Como $I$ es ideal maximal y regular (pues $1\in R$), el cociente
%	$R/I$ es un anillo unitario simple y luego $R/I$ es primitivo por la
%	proposición~\ref{proposition:simple=>prim}. 
%\end{example}

%\begin{example}
%	Si $I$ es un ideal primitivo de un anillo conmutativo $R$, entonces $I$ es
%	maximal pues $R/I$ es un cuerpo (por ser primitivo y conmutativo), ver
%	proposición~\ref{proposition:prim+conm=cuerpo}.
%\end{example}

\begin{example}
	Let $R_1,\dots,R_n$ be primitive rings and $R=R_1\times\cdots\times
	R_n$. Then each 
    \[
    P_i=R_1\times\cdots\times R_{i-1}\times\{0\}\times
	R_{i+1}\times\cdots\times R_n
    \]
    is a primitive ideal of $R$ since 
	$R/P_i\simeq R_i$.
\end{example}

%Recordemos que un ideal a izquierda $L$ de $R$ se dice \emph{minimal} si
%$L\ne0$ y $L$ no contiene propiamenete a otros ideales a izquierda no nulos de
%$R$.
%
%\begin{example}
%	Sea $L$ un ideal a izquierda de $R$ tal que $RL\ne0$. Entonoces $L$ es
%	simple si y sólo si $L$ es minimal.
%\end{example}

\begin{lemma}
	\label{lemma:maxprim}
	Let $R$ be a ring. If $P$ is a primitive ideal, there exists a regular 
        maximal left ideal $I$ such that $P=\{x\in R:xR\subseteq I\}$.
	Conversely, if $I$ is a regular maximal left ideal, then 
	$\{x\in R:xR\subseteq I\}$ is a primitive ideal. 
\end{lemma}

\begin{proof}
	Assume that $P=\Ann_R(M)$ for some simple $R$-module $M$. By
	Proposition~\ref{proposition:R/I}, there exists a regular maximal 
	left ideal 
	$I$ such that $M\simeq R/I$. Then 
    \[
    P=\Ann_R(R/I)=\{x\in
	R:xR\subseteq I\}.
    \]

	Conversely, let $I$ be a regular maximal left ideal. By
	Proposition~\ref{proposition:R/I}, $R/I$ is a simple $R$-module. Then
	\[
	\Ann_R(R/I)=\{x\in R:xR\subseteq I\}
	\]
	is a primitive ideal.
\end{proof}

%\begin{remark}
%	Una consecuencia trivial del lema~\ref{lemma:maxprim} es la siguiente: en
%	un anillo unitario, todo ideal a izquierda maximal contiene un ideal
%	primitivo.
%\end{remark}

\begin{exercise}
\label{xca:maximal=>primitive}
    Maximal ideals of unitary rings are primitive.  
\end{exercise}

\begin{exercise}
\label{xca:primitive=>maximal}
	Prove that every primitive ideal of a commutative ring is maximal.
\end{exercise}

\begin{bonus}
\label{xca:M_n(R)primitive}
    Prove that $M_n(R)$ is primitive if and only if $R$ is primitive.
\end{bonus}

% Si $P$ es primitivo, entonces $R/P$ es un cuerpo(por ser primitivo y conmutativo) y luego $P$ es maximal
