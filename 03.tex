\chapter{}
\label{03}

\begin{definition}
\index{Algebra!simple}  
    An algebra $A$ is \textbf{simple} if $A\ne\{0\}$ and $\{0\}$ and $A$ are the only ideals of $A$. 
\end{definition}

\begin{proposition}
	Let $A$ be a finite-dimensional simple algebra. There exists a non-zero left ideal 
	$I$ of minimal dimension. This ideal is a simple 
	$A$-module, and every simple $A$-module is isomorphic to $I$.
\end{proposition}

\begin{proof}
	Since $A$ is finite-dimensional and $A$ is a left ideal of $A$, there exists a non-zero left ideal of minimal dimension. The minimality 
	of $\dim I$ implies that $I$ is a simple $A$-module. 
	
	Let $M$ be a simple $A$-module. In particular, $M\ne\{0\}$. 
	Since  
	\[
	\Ann(M)=\{a\in A:a\cdot M=\{0\}\}
	\]
	is an ideal of $A$ and $1\in A\setminus\Ann(M)$, the simplicity of $A$ implies that 
	$\Ann(M)=\{0\}$ and hence $I\cdot M\ne \{0\}$ (because $I\cdot m\ne 0$ for all $m\in M$ yields  
	$I\subseteq\Ann(M)$ and $I$ is non-zero, a contradiction). Let $m\in M$ be such 
	that $I\cdot m\ne\{0\}$. The map 
	\[
	\varphi\colon I\to M,\quad
	x\mapsto x\cdot m,
	\]
	is a module homomorphism. Since $I\cdot m\ne\{0\}$, the map $\varphi$ is non-zero. 
	Since both $I$ and $M$ are simple, Schur's lemma implies that $\varphi$ is an isomorphism. 
\end{proof}

If $D$ is a division algebra, then $M_n(D)$ is a simple algebra. The previous proposition
implies that the algebra $M_n(D)$ has a unique isomorphism class of simple modules. Each simple module
is isomorphic to $D^n$. 

\begin{proposition}
Let $A$ be a finite-dimensional algebra. If $A$ is simple, then $A$ is semisimple. 
\end{proposition}

\begin{proof}
	Let $S$ be the sum of the simple submodules appearing in the regular representation of $A$. 
	We claim that $S$ is an ideal of $A$. We knot that $S$ is a left ideal, as the submodules of the regular representation
	are exactly the left ideals of $A$. To show that $Sa\subseteq S$ for all $a\in A$ we need to prove that 
	$Ta\subseteq S$ for all simple submodule $T$ of $A$. If $T\subseteq A$ is a simple submodule and $a\in A$, 
	let $f\colon T\to Ta$, $t\mapsto ta$. Since $f$ is a module homomorphism and $T$ is simple, it follows that
	either $\ker f=\{0\}$ or $\ker T=T$. If $\ker T=T$, then 
	$f(T)=Ta=\{0\}\subseteq S$. If $\ker f=\{0\}$, then $T\simeq f(T)=Ta$ and hence $Ta$ is simple. Hence $Ta\subseteq S$. 
	
	Since $S$ is an ideal of $A$ and $A$
	is a simple algebra, it follows either $S=\{0\}$ or $S=A$.  Since $S\ne\{0\}$, because  
	there exists a non-zero left ideal $I$ of $A$ such that $I\ne\{0\}$ is of minimal dimension, 
	it follows that $S=A$, that is, the regular representation of $A$ is semisimple (because it is a sum of simple submodules). Therefore 
	$A$ is semisimple. 
\end{proof}

\begin{theorem}[Wedderburn]
\index{Wedderburn's theorem}
	Let $A$ be a finite-dimensional algebra. If $A$ is simple, then 
	$A\simeq M_n(D)$ for some $n\in\Z_{>0}$ and some division algebra $D$. 
\end{theorem}

\begin{proof}
	Since $A$ is simple, it follows that $A$ is semisimple. Artin--Wedderburn theorem implies that $A\simeq\prod_{i=1}^k M_{n_i}(D_i)$ 
	for some $n_1,\dots,n_k$ and some division algebras $D_1,\dots,D_k$. Moreover, $A$ has 
	$k$ isomorphism classes of simple modules. Since $A$ is simple,
	$A$ has only one isomorphism class of simple modules. Thus $k=1$ and hence 
	$A\simeq M_n(D)$ for some $n\in\Z_{>0}$ and some division algebra $D$. 
\end{proof}





\topic{Primitive rings}
\label{Primitive rings}

We will consider rings possibly without identity. Thus  
a \textbf{ring} is an abelian group $R$ with an associative multiplication 
$(x,y)\mapsto xy$ such that $(x+y)z=xz+yz$ and $x(y+z)=xy+xz$ for all $x,y,z\in
R$. If there is an element $1\in R$ such that $x1=1x=x$ for all $x\in R$, we 
say that $R$ is a ring (or a unitary ring).  A \textbf{subring} $S$ of $R$ is an additive
subgroup of $R$ closed under multiplication. 

\begin{example}
	$2\Z=\{2m:m\in\Z\}$ is a ring.  
\end{example}

A \textbf{left ideal} (resp. \textbf{right ideal}) is a subring $I$ of $R$ such that 
$rI\subseteq I$ (resp. $Ir\subseteq I$) for all $r\in R$. An \textbf{ideal}
(also two-sided ideal) of $R$ is a subring $I$ of $R$ that is both a left and a right ideal of $R$.

\begin{example}
	If $I$ and $J$ are both ideals of $R$, then the sum $I+J=\{x+y:x\in I,y\in J\}$ and
	the intersection $I\cap J$ are both ideals of $R$. The product $IJ$, defined as the additive
	subgroup of $R$ generated by $\{xy:x\in I,y\in J\}$, is also an ideal of $R$. 
\end{example}

\begin{example}
	If $R$ is a ring, the set $Ra =\{xa: x\in R\}$ is a left ideal
	of $R$. Similarly, the set $aR =\{ax: x\in R\}$ is a right ideal of $R$. The set $RaR$, which is
	defined as the additive subgroup of $R$ generated by $\{xay: x, y\in R\}$, is a
	ideal of $R$.
\end{example}

\begin{example}
	Ir $R$ is a unitary ring, then $Ra$ is the left ideal generated by $a$, $aR$ is
	the right ideal generated by $a$ and $RaR$ is the ideal generated by $a$. 
	If $R$ is not unitary, the left ideal generated by $a$ is $Ra+\Z a$,
	the right ideal generated by $a$ is $aR+\Z a$ and the ideal generated by 
	$a$ is $RaR+Ra+aR+\Z a$.
\end{example}

\begin{definition}
A ring $R$ is said to be \textbf{simple} if $R^2\ne\{0\}$ and the only ideals of 
$R$ are $\{0\}$ and $R$.  
\end{definition}

The condition $R^2\ne\{0\}$ is trivially satisfied in the case of rings
with identity, as $1\in R^2=\{r_1r_2:r_1,r_2\in R\}$.

\begin{example}
	Division rings are simple.
\end{example}

Let $S$ be a unitary ring. Recall that $M_n(S)$ is the ring of $n\times n$ square matrices 
with entries in $S$.  If $A=(a_{ij})\in M_n(S)$ y $E_{ij}$ is the matrix
such that $(E_{ij})_{kl}=\delta_{ik}\delta_{jl}$, then
\begin{equation}
	\label{eq:trick}
E_{ij}AE_{kl}=a_{jk}E_{il}
\end{equation}
for all $i,j,k,l\in\{1,\dots,n\}$. 

\begin{example}
	If $D$ is a division ring, then $M_n(D)$ is simple. 
\end{example}

Let $R$ be a ring. A left $R$-module (or module, for short)  
is an abelian group $M$ together with a map $R\times M\to M$, $(r,m)\mapsto r\cdot m$, such that
\begin{align*}
	&(r+s)\cdot m=r\cdot m+s\cdot m, &&
	r\cdot (m+n)=r\cdot m+r\cdot s, && r\cdot (s\cdot m)=(rs)\cdot m    
\end{align*}
for all $r,s\in R$, $m,n\in M$.  If $R$ has an identity 
$1$ and $1\cdot m=m$ holds for all $m\in M$, the module $M$ is said to be 
\textbf{unitary}.  If $M$ is a unitary module, then $M=R\cdot M$. %\ne\{0\}$.

\begin{definition}
A module $M$ is said to be 
\textbf{simple} if $R\cdot M\ne\{0\}$ and the only submodules of $M$ are $\{0\}$ and $M$.
If $M$ is a simple module, then $M\ne\{0\}$.
\end{definition}

%\begin{remark}
%	Si $R$ es unitario y $M$ es un módulo simple, entonces $M$ es unitario.
%\end{remark}

\begin{lemma}
	\label{lemma:simple}
	Let $M$ be a non-zero module. Then $M$ is simple if and only if $M=R\cdot m$
	for all $0\ne m\in M$.
\end{lemma}

\begin{proof}
	Assume that $M$ is simple.  Let $m\ne 0$. Since $R\cdot m$ is a submodule of the simple 
	module $M$, either $R\cdot m=\{0\}$ or $R\cdot m=M$.  Let $N=\{n\in M:R\cdot n=\{0\}\}$. Since $N$ is a 
	submodule of $M$ and $R\cdot M\ne\{0\}$, $N=\{0\}$. Therefore $R\cdot m=M$, as $m\ne0$.
	Now assume that $M=R\cdot m$ for all $m\ne0$. Let $L$ be a non-zero submodule of 
	$M$ and let $0\ne x\in L$. Then $M=L$, as $M=R\cdot x\subseteq L$. 
\end{proof} 

\begin{example}
	Let $D$ be a division ring and let $V$ be a non-zero vector space (over $D$). If 
	$R=\End_D(V)$, then $V$ is a simple $R$-module with $fv=f(v)$, $f\in R$.
	$v\in V$. 
% 	Para ver que $V$ es simple como $R$-módulo basta ver que $Rv=V$ para todo
% 	$v\ne0$. Sean $v,w\in V$, $v\ne0$.  Al completar $v\ne0$ a una base de $V$,
% 	vemos que existe $f\in R$ tal que $f(v)=w$. Luego $V$ es simple.
\end{example}

\begin{example}
	\label{exa:I_k}
	Let $n\geq2$.  If $D$ is a division ring and $R=M_n(D)$, then each 
	\[
	I_k=\{ (a_{ij})\in R:a_{ij}=0\text{ for $j\ne k$}\}
	\]
	is an $R$-module isomorphic to $D^n$. 
	Thus $M_{n}(D)$ is a simple ring that is not a simple $M_n(D)$-module.
\end{example}

\begin{definition}
A left ideal $L$ of a ring $R$ is said to be \textbf{minimal} if $L\ne\{0\}$ and 
$L$ does not strictly contain other left ideals of $R$. 
\end{definition}

Similarly one defines
right minimal ideals and minimal ideals. 

\begin{example}
	Let $D$ be a division ring and let $R=M_n(D)$. Then $L=RE_{11}$ 
	is a minimal left ideal.
\end{example}

\begin{example}
	Let $L$ be a non-zero left ideal. If $RL\ne\{0\}$, then
	$L$ is minimal if and only if $L$ is a simple $R$-module. 
\end{example}

\begin{definition}
A left (resp. right) ideal $L$ of $R$ is said to be \textbf{regular} if
there exists $e\in R$ such that $r-re\in L$ (resp.  $r-er\in L$) for all $r\in R$.
\end{definition}

If $R$ is a ring with identity, every left (or right) ideal is regular. 

\begin{definition}
A left (resp. right) ideal $I$ of $R$ is said to be \textbf{maximal} if $I\ne M$ and $I$ is not properly contained 
in any other left (resp. right) ideal of $R$. 
\end{definition}

Similarly, one defines maximal ideals. 

A standard application of Zorn's lemma proves that every unitary ring contains a maximal left (or right) ideal.  

% \begin{exercise}
% Prove that every ring with identity contains a maximal ideal.
% \end{exercise}

\begin{proposition}
	\label{proposition:R/I}
	Let $R$ be a ring and $M$ be a module. Then $M$ is simple if and only if
	$M\simeq R/I$ for some maximal regular left ideal $I$. 	
\end{proposition}

\begin{proof}
	Assume that $M$ is simple. Then $M=R\cdot m$ for some $m\ne0$ by 
	Lemma~\ref{lemma:simple}. The map $\phi\colon R\to M$, $r\mapsto r\cdot m$, 
	is a surjective homomorphism of $R$-modules, so the first isomorphism theorem implies that 
	$M\simeq R/\ker\phi$. 
	
	We claim that $I=\ker\phi$ is a maximal ideal. The correspondence theorem 
	and the simplicity of $M$ imply that $I$ is a maximal ideal (because each left ideal $J$ such that 
	$I\subseteq J$ yields a submodule of $R/I$).

	We claim that $I$ is regular. Since $M=Rm$, there exists $e\in R$ such that $m=e\cdot m$. If
	$r\in R$, then $r-re\in I$ since 
	$\phi(r-re)=\phi(r)-\phi(re)=r\cdot m-r\cdot (e\cdot m)=0$.

    Now assume that $I$ is a maximal left ideal that is regular. The correspondence theorem implies that 
    $R/I$ has no non-zero proper submodules. 
    
    We claim that 
    $R\cdot (R/I)\ne0$. If $R\cdot (R/I)=\{0\}$ and $r\in R$, then 
    the regularity of $I$ implies that 
    there exists $e\in R$ such that $r-re\in I$. Hence $r\in I$, as  
	\[
	0=r\cdot (e+I)=re+I=r+I,
	\]
	a contradiction to the maximality of $I$. 
\end{proof}


%\section{Nilpotencia}
%
%Recordemos que si $I$ es un ideal de un anillo $R$, se define $I^n$ como el
%subgrupo aditivo generado por el conjunto $\{y_1\dots y_n:y_j\in I\}$. 
%
%\begin{definition}
%	Un ideal $I$ de un anillo $R$ se dice \textbf{nilpotente} si $I^n=0$ para
%	algún $n\in\N$.
%\end{definition}
%
%Recordemos que un elemento $x$ de un anillo $R$ se dice \textbf{nilpotente} si
%existe $n\in\N$ tal que $x^n=0$. 
%
%\begin{definition}
%	Un ideal $I$ de un anillo $R$ se dice \textbf{nil} si todo elemento de $I$
%	es nilpotente.
%\end{definition}
%
%\begin{remark}
%	Un ideal nilpotente es nil. 
%\end{remark}
%
%\begin{example}
%	Sea $R=\C[x_1,x_2,\dots]/(x_1,x_2^2,x_3^3,\dots)$. El ideal
%	$I=(x_1,x_2,x_3,\dots)$ es nil en $R$ pues está generado por elementos
%	nilpotentes pero no es nilpotente. Si lo fuera, existiría $k\in\N$ tal que
%	$I^k=0$, y luego $x_i^k=0$ para todo $i$, una contradicción pues
%	$x_{k+1}^k\ne0$. 	
%\end{example}
%
%% example 2.7 del libro de springer
%% problema de kothe 
%
%\begin{lemma}
%	Si $I$ y $J$ son ideales nilpotentes, $I+J$ es nilpotente.
%\end{lemma}
%
%\begin{proof}
%	
%\end{proof}
%
%Un ideal $N$ de un anillo $R$ se dice \textbf{maximal-nilpotente} si $N$ es
%nilpotente y no está propiamente contenido en ningún ideal nilpotente de $R$.
%
%\begin{lemma}
%	Si el anillo $R$ contiene un ideal maximal-nilpotente $N$ entonces todo
%	ideal nilpotente está contenido en $N$.
%\end{lemma}
%
%\begin{proof}
%	
%\end{proof}
%
%\section{Anillos primos y semiprimos}
%
%\index{Dominio}
%Recordemos que un anillo $R$ se dice un \textbf{dominio} si para todo $a,b\in
%R$ tales que $ab=0$ se tiene $a=0$ o $b=0$.
%Una generalización al caso no conmutativo es la siguiente:
%
%\begin{definition}
%	\index{Anillo!primo}
%	Un anillo $R$ se dice \textbf{primo} si para todo $a,b\in R$ tales que
%	$aRb=0$ se tiene $a=0$ o $b=0$.
%\end{definition}
%
%\begin{lemma}
%	Sea $R$ un anillo. Las siguientes propiedades son equivalentes:
%	\begin{enumerate}
%		\item $R$ es primo.
%		\item Si $I,J\subseteq R$ son ideales a izquierda tales que $IJ=0$
%			entonces $I=0$ o $J=0$.
%%		\item Si $I,J\subseteq R$ son ideales a derecha tales que $IJ=0$ 
%%			entonces $I=0$ o $J=0$.
%		\item Si $I,J\subseteq R$ son ideales tales que $IJ=0$ entonces $I=0$ o
%			$J=0$.
%	\end{enumerate}
%\end{lemma}
%
%\begin{proof}
%	Vamos a demostrar que $(1)\implies(2)\implies(4)\implies(1)$. 
%	La implicación $(2)\implies(4)$ es trivial.
%
%	Veamos que $(4)\implies(1)$. Sean $a,b\in R$ tales que $aRb=0$.
%	Como entonces $(RaR)(RbR)=R(aRb)R=0$, $RaR=0$ o bien $RbR=0$. Supongamos
%	sin pérdida de generalidad que $RaR=0$. Entonces $Ra$ y $aR$ son ideales
%	biláteros tales que $(Ra)R=R(aR)=0$. Al aplicar la hipótesis, $Ra=aR=0$.
%	Como $\Z a$ es un ideal de $R$ tal que $(\Z a)R=0$, se concluye al aplicar
%	la hipótesis que $a=0$.
%
%	Veamos que $(1)\implies(2)$. Supongamos que $J\ne 0$, sea $y\in
%	J\setminus\{0\}$ y sea $x\in I$. Como 
%	$xRy\subseteq IRJ=I(RJ)\subseteq IJ=0$, 
%	se concluye, al usar la hipótesis, que $x=0$. 
%\end{proof}
%
%\begin{proposition}
%	Un anillo conmutativo es primo si y sólo si es un dominio. 
%\end{proposition}
%
%\begin{proof}
%	Supongamos que $R$ es un anillo primo. Si $a,b\in R$ son tales que $ab=0$
%	entonces $aRb=(ab)R=0$ y luego $a=0$ o bien $b=0$. Supongamos ahora que $R$
%	es un dominio. Si $a,b\in R$ son tales que $aRb=0$ entonces $(ab)R=0$ y
%	luego $a=0$ o bien $b=0$ pues $ab=0$. 
%\end{proof}
%
%\begin{definition}
%	\index{Anillo!semiprimo}
%	Un anillo $R$ se dice \textbf{semiprimo} si para todo $a\in R$ tal que
%	$aRa=0$ se tiene $a=0$.
%\end{definition}
%
%\begin{lemma}
%	Sea $R$ un anillo. Las siguientes aifrmaciones son equivalentes:
%	\begin{enumerate}
%		\item $R$ es semiprimo.
%		\item Si $I$ es un ideal a izquierda tal que $I^2=0$ entonces $I=0$.
%		\item Si $I$ es un ideal tal que $I^2=0$ entonces $I=0$.
%		\item $R$ no tiene ideales nilpotentes no nulos.
%	\end{enumerate}
%\end{lemma}
%
%\begin{proof}
%	Primero vamos a demostrar que $(1)\implies(2)\implies(3)\implies(1)$ 
%
%	La implicación $(5)\implies(4)$ es trivial.	
%	Demostremos entonces que
%	$(4)\implies(5)$. Sea $I$ un ideal nilpotente no nulo y sea $n\in\N$ el
%	mínimo tal que $I^n=0$. Como $(I^{n-1})^2=0$, por hipótesis se tiene que
%	$I^{n-1}=0$, una contradicción.
%\end{proof}
%

Let $R$ be a ring and $M$ be a left $R$-module. For a 
subset $N\subseteq M$
we define the \textbf{annihilator} of $N$ as the subset 
\[
\Ann_R(N)=\{r\in R:r\cdot n=0\text{ for all }n\in N\}.
\]

\begin{example}
	$\Ann_{\Z}(\Z/n)=n\Z$.
\end{example}

\begin{exercise}
    Let $R$ be a ring and $M$ be a module. If $N\subseteq M$ is a subset, then 
	$\Ann_R(N)$ is a left ideal of $R$. If $N\subseteq M$ is a submodule of $R$, then 
	$\Ann_R(N)$ is an ideal of $R$. 
\end{exercise}

% \begin{lemma}
% 	\label{lemma:Ann}
% 	Let $R$ be a ring and $M$ be a module. If $N\subseteq M$ is a subset, then 
% 	$\Ann_R(N)$ is a left ideal of $R$. If $N\subseteq M$ is a submodule of $R$, then 
% 	$\Ann_R(N)$ is an ideal of $R$. 
% \end{lemma}

% \begin{proof}
% 	We left as an exercise to prove that $\Ann_R(N)$ is an additive subgroup of $R$. Then $\Ann_R(N)$
% 	is a left ideal, as $R\Ann_R(N)\subseteq\Ann_R(N)$. Indeed, if $r\in R$,
% 	$s\in\Ann_R(N)$ and $n\in N$, then $(rs)n=r(sn)=r0=0$. 
	
% 	If $N$ is a submodule, $\Ann_R(N)R\subseteq\Ann_R(N)$ since if 
% 	$s\in\Ann_R(N)$, $r\in R$ and $n\in N$, $rn\in\Ann_R(N)$, then
% 	$(sr)n=s(rn)=0$.
% \end{proof}

\begin{definition}
A module $M$ is said to be \textbf{faithful} if $\Ann_R(M)=\{0\}$. 
\end{definition}

\begin{example}
	If $K$ is a field, then $K^n$ is a faithful unitary $M_n(K)$-module.
\end{example}

\begin{example}
	If $V$ is vector space over a field $K$, then $V$ is faithful unitary $\End_K(V)$-module.
\end{example}

\begin{definition}
\index{Ring!primitive}
A ring $R$ is said to be \textbf{primitive} if there exists a faithful simple $R$-module.  
\end{definition}

Since 
we are considering left modules, our definition of primitive rings is that of left primitive rings.
By convention, a primitive ring
will always mean a left primitive ring. 
The use 
of right modules yields to the notion of right primitive rings.  

\begin{exercise}
	\label{xca:simple=>prim}
	If $R$ is a simple unitary ring, then $R$ is primitive. 
\end{exercise}

\begin{exercise}
	\label{xca:prim+conm=cuerpo}
	If $R$ is a commutative ring (maybe without identity), then $R$ is primitive if and only if $R$ is a field. 
\end{exercise}

\begin{example}
	The ring $\Z$ is not primitive. 
\end{example}

