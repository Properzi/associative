\RequirePackage{amsmath} 

\documentclass[graybox,envcountsect]{svmono}

\usepackage[T1]{fontenc}
\usepackage[utf8]{inputenc}

\usepackage{amsmath}
\usepackage[notref,notcite]{showkeys}
\usepackage{float}
\usepackage{amssymb}
\usepackage{amstext}
\usepackage{mathtools}
\usepackage{anyfontsize}
\usepackage{xcolor} 
\usepackage{centernot}
\usepackage{listings}
\usepackage{multicol}
\usepackage{mathptmx}
\usepackage{newtxtext,newtxmath}
%\usepackage{txfonts}
\usepackage{datetime}
\usepackage{stmaryrd}
\usepackage{tikz-cd}

\usepackage{helvet}
\usepackage{courier}
\usepackage{type1cm}         
\usepackage{makeidx}        
\usepackage{graphicx}        
\usepackage{multicol}        
\usepackage[all]{xy}
\usepackage{hyperref} 
%\usepackage{tikz-cd}
\usepackage{colortbl}
\usepackage{chngcntr}





% Table of contents for lectures and topics
\makeatletter
\newcommand\listtopicsname{List of topics}
\newcommand\listoftopics{
    \chapter*{\listtopicsname}\@starttoc{top}}
\makeatother

\makeatletter
\newcommand\listlecturesname{Contents}
\newcommand\listoflectures{
    \chapter*{\listlecturesname}\@starttoc{lec}}
\makeatother

\newcommand{\lecture}[1]{
    \chapter{#1}
    \addcontentsline{lec}{chapter}{Lecture \thechapter}
}

\newcommand{\topic}[1]{
    \section{#1}
    \addcontentsline{top}{chapter}{\S\thesection\quad #1}
}


%\usepackage[small,bf]{caption}

\usepackage{tikz}
\usetikzlibrary{braids}
	
\usepackage[bottom]{footmisc}

% for QED
\let\proof\relax\let\endproof\relax
\let\openbox\relax
\usepackage{amsthm}

\overfullrule=1mm

%%% for Spanish
% \def\abstractname{Resumen}%
% \def\ackname{Agradecimientos}%
% \def\andname{y}%
% \def\bibname{Referencias}%
% \def\lastandname{, y}%
% \def\appendixname{Apéndice}%
\def\chaptername{Lecture}%
% \def\claimname{Afirmación}%
% \def\conjecturename{Conjetura}%
% \def\contentsname{Contenidos}%
% \def\corollaryname{Corolario}%
% \def\definitionname{Definici\'on}%
% \def\emailname{e-mail}%
% \def\examplename{Ejemplo}%
\def\examplesname{Examples}%
% \def\exercisename{Ejercicio}%
\def\figurename{Figure}%
% \def\forewordname{Foreword}%
% \def\keywordname{{\bf Palabras clave:}}%
% \def\indexname{Índice}%
% \def\lemmaname{Lema}%
% \def\listfigurename{Figuras}%
% \def\listtablename{Tablas}%
% \def\notename{Nota}%
% \def\partname{Parte}%
% \def\prefacename{Prefacio}%
\def\problemname{Open problem}%
% \def\proofname{Demostración}%
% \def\propertyname{Propiedad}%
% \def\propositionname{Proposici\'on}%
% \def\questionname{Pregunta}%
% \def\refname{Referencias}%
% \def\remarkname{Observación}%
% \def\seename{see}%
% \def\solutionname{Solución}%
% \def\tablename{Tabla}%
% \def\theoremname{Teorema}
\def\notationname{Notation}
\def\stepsname{Algorithm}
% \def\conventionname{Convención}

% change numbers 
\let\remark\relax
\let\theorem\relax
\let\lemma\relax
\let\definition\relax
\let\proposition\relax
\let\corollary\relax
\let\exercise\relax
\let\example\relax
\let\conjecture\relax

% Numerar con sección y no resetear al cambiar de capítulo
\counterwithout{section}{chapter}
\counterwithout{theorem}{chapter}
\spnewtheorem{theorem}{\theoremname}[section]{\bfseries}{\itshape}

\renewcommand\thetheorem{\thesection.\arabic{theorem}}
\spnewtheorem{lemma}[theorem]{\lemmaname}{\bfseries}{\itshape}
\spnewtheorem{definition}[theorem]{\definitionname}{\bfseries}{\upshape}
\spnewtheorem{proposition}[theorem]{\propositionname}{\bfseries}{\itshape}
\spnewtheorem{corollary}[theorem]{\corollaryname}{\bfseries}{\itshape}
\spnewtheorem{exercise}[theorem]{\exercisename}{\bfseries}{\upshape}
\spnewtheorem{example}[theorem]{\examplename}{\bfseries}{\upshape}
\spnewtheorem{examples}[theorem]{\examplesname}{\bfseries}{\upshape}
\spnewtheorem{remark}[theorem]{\remarkname}{}{\upshape}
\spnewtheorem{conjecture}[theorem]{\conjecturename}{\bfseries}{\upshape}
\spnewtheorem{notation}[theorem]{\notationname}{\bfseries}{\upshape}
\spnewtheorem{steps}[theorem]{\stepsname}{\bfseries}{\upshape}
\spnewtheorem{convention}[theorem]{\conventionname}{\bfseries}{\upshape}
\spnewtheorem{openproblem}[theorem]{\problemname}{\bfseries}{\upshape}


% Numerar con sección y no resetear al cambiar de capítulo
\counterwithout{section}{chapter}

% No sections in TOC
\setcounter{secnumdepth}{1}
\setcounter{tocdepth}{0}

 \usepackage{titlesec}
 \titleformat{\section}
   {\secsize\secstyle}{\S\thesection.}{1em}{}

% para enumerar
\renewcommand{\labelenumi}{\textbf{\arabic{enumi})}}

\makeindex             

\renewcommand{\I}{\operatorname{I}}
\newcommand{\II}{\operatorname{II}}

\newcommand{\GAP}{\textsf{GAP}}
\newcommand{\FK}{\mathcal{E}}
\newcommand{\ad}[1]{\operatorname{ad}\,#1}

%\newcommand{\N}{\mathbb{N}}
\newcommand{\Q}{\mathbb{Q}}
\newcommand{\Z}{\mathbb{Z}}
\newcommand{\F}{\mathbb{F}}
\newcommand{\R}{\mathbb{R}}
\newcommand{\C}{\mathbb{C}}
\renewcommand{\H}{\mathbb{H}}
\newcommand{\A}{\mathbb{A}}
\newcommand{\K}{\mathbb{K}}
\newcommand{\T}{\mathbb{T}}
\renewcommand{\D}{\mathbb{D}}
\newcommand{\B}{\mathbb{B}}
\newcommand{\Fun}{\operatorname{Fun}}
\newcommand{\mpl}{\operatorname{mpl}}
\newcommand{\cL}{\mathcal{L}}
\newcommand{\cE}{\mathcal{E}}
\newcommand{\cH}{\mathcal{H}}

\newcommand{\GF}{\mathsf{GF}}
\newcommand{\MAX}{\operatorname{MAX}}
\newcommand{\MIN}{\operatorname{MIN}}
\newcommand{\cf}{\operatorname{cf}}
\newcommand{\cl}{\operatorname{cl}}
\newcommand{\cd}{\operatorname{cd}}
\newcommand{\bL}{\mathbf{L}}
\newcommand{\bP}{\mathbf{P}}

\newcommand{\Nil}{\operatorname{Nil}}
\newcommand{\rad}{\operatorname{rad}}
\newcommand{\rank}{\operatorname{rank}}

\newcommand{\Aff}{\mathrm{Aff}}
\newcommand{\Ann}{\operatorname{Ann}}
\newcommand{\Der}{\operatorname{Der}}
\newcommand{\Core}{\operatorname{Core}}
\newcommand{\Soc}{\operatorname{Soc}}
\newcommand{\Fix}{\operatorname{Fix}}
\newcommand{\Rad}{\mathrm{rad}}
\newcommand{\Inn}{\mathrm{Inn}}
\newcommand{\dist}{\mathrm{dist}}
\newcommand{\Out}{\mathrm{Out}}
\newcommand{\Ext}{\mathrm{Ext}}
\newcommand{\Img}{\mathrm{im}}
\newcommand{\Hol}{\operatorname{Hol}}
\newcommand{\Hom}{\operatorname{Hom}}
\newcommand{\Alg}{\operatorname{Alg}}
\newcommand{\Bil}{\operatorname{Bil}}
\newcommand{\op}{\operatorname{op}}
\newcommand{\gr}{\operatorname{gr}}
\newcommand{\Syl}{\mathrm{Syl}}
\newcommand{\id}{\operatorname{id}}
\newcommand{\Aut}{\operatorname{Aut}}
\newcommand{\End}{\operatorname{End}}
\newcommand{\Irr}{\operatorname{Irr}}
\newcommand{\Alt}{\mathbb{A}}
\newcommand{\Sym}{\mathbb{S}}
\newcommand{\lcm}{\mathrm{mcm}}
\newcommand{\diag}{\operatorname{diag}}
\newcommand{\spec}{\operatorname{Spec}}
\newcommand{\supp}{\operatorname{supp}}
\newcommand{\trace}{\operatorname{trace}}
\newcommand{\sgn}{\operatorname{sign}}
\newcommand{\ch}{\operatorname{char}}

\newcommand{\inner}{\operatorname{inn}}
\newcommand{\ext}{\operatorname{ext}}
\newcommand{\im}{\operatorname{im}}
\newcommand{\Ret}{\operatorname{Ret}}

\newcommand{\GL}{\mathbf{GL}}
\newcommand{\SL}{\mathbf{SL}}
\newcommand{\PSL}{\mathbf{PSL}}
\newcommand{\PGL}{\mathbf{PGL}}

\newcommand{\legendre}[2]{\left(\frac{#1}{#2}\right)}

%\newcommand{\char}{\operatorname{char}}

% multiset
\def\multiset#1#2{\ensuremath{\left(\kern-.3em\left(\genfrac{}{}{0pt}{}{#1}{#2}\right)\kern-.3em\right)}}

% column vector
\newcount\colveccount
\newcommand*\colvec[1]{
\global\colveccount#1
\begin{pmatrix}
	\colvecnext
	}
	\def\colvecnext#1{
	#1
	\global\advance\colveccount-1
	\ifnum\colveccount>0
	\\
	\expandafter\colvecnext
	\else
\end{pmatrix}
\fi
}


% numero como secciones
\renewcommand{\thesection}{\arabic{section}}
%\renewcommand{\thesubsection}{\Alph{section}}

% To remove Springer from the title page
\usepackage{etoolbox}
\makeatletter
\patchcmd{\@maketitle}{{\Large Springer\par}}{}{}{}
\def\ps@bchap{%
  \let\@oddhead\@empty\let\@evenhead\@empty
  \def\@oddfoot{\reset@font\small\hfil\thepage\hfil}%
  \let\@evenfoot\@oddfoot
}

% Heading 
\def\ps@headings{%
  \let\@mkboth\markboth
  \def\@oddfoot{\reset@font\small\hfil\thepage\hfil}%
  \let\@evenfoot\@oddfoot
  \def\@evenhead{\runheadsize\runheadstyle\hfil\leftmark}%
  \def\@oddhead{\runheadsize\runheadstyle\rightmark\hfil}%
  \def\chaptermark##1{%
    \markboth{%
      {\if@chapnum Lecture \thechapter\thechapterend\fi ##1}%
    }{%
      {\if@chapnum Lecture \thechapter\thechapterend\fi ##1}}%
    }%
    \def\sectionmark##1{\markright{{\ifnum\c@secnumdepth>\z@
     \S\thesection\seccounterend\hskip\betweenumberspace\fi ##1}}}
}
\makeatother
\pagestyle{headings}

\begin{document}
 
\lstset{language=GAP,
  showstringspaces=false,
  xleftmargin=0.6cm,
  xrightmargin=0.6cm,
  basicstyle=\small\ttfamily,
  frame=single,
  framerule=0pt,
}

\author{Leandro Vendramin}
\title{Associative algebras}
\subtitle{Notes}
\maketitle

\frontmatter

%\include{dedic}
\preface

The notes correspond to the master  
course \emph{Associative Algebra} of the 
Vrije Universiteit Brussel, 
Faculty of Sciences, 
Department of Mathematics and Data Sciences. The course
is divided into thirteen two-hours lectures. 

The material is heavility based on \cite{MR3308118}, \cite{MR1449137} and 
\cite{MR798076}. 

%Thanks go to Arne van Antwerpen, Luca Descheemaeker, Lucas Simons
%and Geoffrey Jassens. 

This version 
was compiled on \today~at~\currenttime.

\bigskip
\begin{flushright}
Leandro Vendramin\\Brussels, Belgium\par
\end{flushright}


\tableofcontents
\listoftopics

\mainmatter

\chapter{}

\topic{Semisimple algebras}

We will devote two lectures to the study of 
finite-dimensional semisimple algebras. The main goal is to
prove Artin--Wedderburn's theorem. 

\begin{definition}
	\index{Algebra}
	An \textbf{algebra} (over the field $K$) is a vector space (over $K$) 
	with an associative multiplication $A\times A\to A$ such that
	$a(\lambda b+\mu c)=\lambda(ab)+\mu(ac)$ and
	$(\lambda a+\mu b)c=\lambda(ac)+\mu (bc)$ for all $a,b,c\in A$, and 
	that contains an element $1_A\in A$ such that $1_Aa=a1_A=a$ for all $a\in A$.   
\end{definition}

Note that an algebra over $K$ is a ring $A$ that is a vector space
(over $K$) such that the map $K\to A$, $\lambda\mapsto \lambda1_A$, is injective. 

\begin{definition}
	\index{Algebra!commutative}
	An algebra $A$ is \textbf{commutative} if $ab=ba$ for all $a,b\in A$. 
\end{definition}

\index{Algebra!dimension}
The \textbf{dimension} of an algebra $A$ is the dimension of $A$ as a vector space. This is why we want to consider algebras, as 
they are linear version of rings. Quite often our arguments will use the dimension of the underlying vector space.  

\begin{example}
	The field $\R$ is a real algebra and similarly 
	$\C$ is a complex algebra. Moreover, $\C$ is a real algebra. 
\end{example}

Any field $K$ is an algebra over $K$.

\begin{example}
	If $K$ is a field, then $K[X]$ is an algebra over $K$. 
\end{example}

Similarly, the polynomial ring $K[X,Y]$ and the ring $K[[X]]$ of power series
are examples of algebra over $K$. 

\begin{example}
	If $A$ is an algebra, then  $M_n(A)$ is an algebra. 
\end{example}

\begin{example}
	The set of continuous maps $[0,1]\to\R$ is a real algebra with the usual
	point-wise operations $(f+g)(x)=f(x)+g(x)$ and $(fg)(x)=f(x)g(x)$. 
\end{example}

\begin{example}
	Let $n\in\Z_{>0}$. Then $K[X]/(X^n)$ is a finite-dimensional algebra. 
    It is the \textbf{truncated polynomial algebra}.  
\end{example}

\begin{example}
	Let $G$ be a finite group. The vector space 
	$\C[G]$ with basis $\{g:g\in G\}$
	is an algebra with multiplication
	\[
	\left(\sum_{g\in G}\lambda_gg\right)\left(\sum_{h\in G}\mu_hh\right)
	=\sum_{g,h\in G}\lambda_g\mu_h(gh).
	\] 	
	Note that $\dim\C[G]=|G|$ and
	$\C[G]$ is commutative if and only $G$ is abelian. 
	This is the \textbf{complex group algebra} of $G$. 
\end{example}

\begin{definition}
    \index{Homomorphism!of algebras}
    An algebra \textbf{homomorphism} is a ring homomorphism $f\colon A\to B$ that is also a linear map. 
\end{definition}

The complex conjugation map  
$\C\to \C$, $z\mapsto\overline{z}$, is a ring homomorphism that is not an algebra homomorphism over $\C$. 

%Two basic exercises about group algebras.
 
\begin{exercise}
	Let $G$ be a non-trivial finite group. 
	Then $\C[G]$ has zero divisors. 
\end{exercise}

\begin{exercise}
	Let $A$ be an algebra and $G$ be a finite group. 
	If $f\colon G\to\mathcal{U}(A)$ is a group homomorphism, 
	then there exists an algebra homomorphism 
	$\varphi\colon K[G]\to A$ such that $\varphi|_G=f$.   	
\end{exercise}

\begin{definition}
 	\index{Algebra!ideal}
 	An \textbf{ideal} of an algebra is an ideal of the underlying ring.
\end{definition}

Similarly one defines left and right ideals of an algebra.

If $A$ is an algebra, then every left ideal of the ring $A$ is a vector space.  
Indeed, if $I$ is a left ideal of $A$ 
and $\lambda\in K$ and $x\in I$, then 
\[
	\lambda x=\lambda (1_Ax)=(\lambda 1_A)x.
\]
Since $\lambda 1_A\in A$, it follows that  $\lambda I=(\lambda
1_A)L\subseteq I$. 
Similarly, every right ideal of the ring $A$ is a vector space. 

If $A$ is an algebra and $I$ is an ideal of $A$, then the quotient ring $A/I$ has a unique algebra
structure such that the canonical map  
$A\to A/I$, $a\mapsto a+I$, is a surjective algebra homomorphism with kernel $I$. 

\begin{definition}
    \index{Algebraic element}
    \index{Algebra!algebraic}
    Let $A$ be an algebra over the field $K$. An element $a\in A$ is 
    \textbf{algebraic} over $K$ if there exists a non-zero polynomial $f\in K[X]$
    such that $f(a)=0$. 
\end{definition}

If every element of $A$ is algebraic, then $A$ is said to be \text{algebraic} 

In the algebra $\R$ over $\Q$, the element $\sqrt{2}$ is algebraic, as $\sqrt{2}$ is a root of the polynomial $X^2-2\in\Q[X]$. A famous theorem of Lindemann proves that $\pi$ is not algebraic over $\Q$. Every element of the real algebra $\R$ is algebraic.

\begin{proposition}
	\label{lem:algebraic}
	Every finite-dimensional algebra is algebraic.
\end{proposition}

\begin{proof}
   Let $A$ be an algebra with $\dim A=n$ and let $a\in A$. Since  
	$\{1,a,a^2,\dots,a^n\}$ has $n+1$ elements, it is a linearly dependent set. Thus there exists 
	a non-zero polynomial $f\in K[X]$ such that $f(a)=0$.
\end{proof}

\begin{definition}
    A \textbf{module} $M$ over an algebra $A$ is a module 
    over the ring $A$.
\end{definition}

Similarly one defines submodules and module homomorphisms. 

\begin{example}
If $V$ is a module over an algebra $A$, one defines $\End_A(V)$ as the set
of module homomorphisms $V\to V$. The set  
$\End_A(V)$ is indeed an algebra with 
\[
(f+g)(v)=f(v)+g(v),\quad 
(af)(v)=af(v)
\quad\text{and}
\quad 
(fg)(v)=f(g(v))
\]
for all $f,g\in\End_A(V)$, $a\in A$ and $v\in V$. 
\end{example}

Let $A$ be a finite-dimensional algebra. 
If $M$ is a module over the ring $A$, then $M$ is a vector space with  
\[
\lambda m=(\lambda 1_A)\cdot m, 
\]
where $\lambda\in K$ and $m\in M$. Moreover, $M$ is finitely generated if and only if $M$ is finite-dimensional.  

% In this chapter we will work with finitely generated modules. 

\begin{example}
An algebra  $A$ is a module over $A$ with left multiplication, that is $a\cdot b=ab$, $a,b\in A$.
This module is the (left) \textbf{regular representation} of $A$ and it will be denoted by $\prescript{}{A}{A}$. 
\end{example}

\begin{definition}
	Let $A$ be an algebra and $M$ be a module over $A$. Then 
	$M$ is \textbf{simple} if $M\ne\{0\}$ and $\{0\}$ and $M$ 
	are the only submodules of $M$.	
\end{definition}

\begin{definition}
	Let $A$ be a finite-dimensional 
	algebra and $M$ be a finite-dimensional module over $A$. Then 
	$M$ is \textbf{semisimple} if $M$ is a direct sum of 
	finitely many simple submodules.  
\end{definition}

Clearly, a finite direct sum of semisimples is semisimple. 

\begin{lemma}[Schur]
	Let $A$ be an algebra. If $S$ and $T$ are
	simple modules and $f\colon S\to T$ is a non-zero module homomorphism, 
	then $f$ is an isomorphism. 
\end{lemma}

\begin{proof}
Since $f\ne 0$, $\ker f$ is a proper submodule of $S$. Since $S$ is simple, it follows 
that $\ker f=\{0\}$. Similarly, $f(S)$ 
is a non-zero submodule of $T$ and hence $f(S)=T$, as $T$ is simple. 	
\end{proof}

\begin{proposition}
    If $A$ is a finite-dimensional algebra and $S$ is a simple module, then $S$ is finite-dimensional. 
\end{proposition}

\begin{proof}
    Let $s\in S\setminus\{0\}$. Since $S$ is simple, $\varphi\colon A\to S$, $a\mapsto a\cdot s$, is a surjective 
    module homomorphism. 
    In particular, by the first isomorphism theorem, $A/\ker\varphi\simeq S$ and hence $\dim S=\dim (A/\ker\varphi)\leq \dim A$. 
\end{proof}

\begin{proposition}
\label{pro:semisimple}
	Let $M$ be a finite-dimensional module. The following statements are equivalent.
	\begin{enumerate}
		\item $M$ is semisimple.
		\item $M=\sum_{i=1}^k S_i$, where each $S_i$ is a simple submodule of $M$. 
		\item If $S$ is a submodule of $M$, then there is a submodule $T$ of $M$ such that $M=S\oplus T$.    
	\end{enumerate}
\end{proposition}

\begin{proof}
	We first prove that $2)\implies3)$.
	Let $N\ne\{0\}$ be a submodule of $M$. Since $N\ne\{0\}$ and $\dim M<\infty$, there exists a submodule 
	$T$ of $M$ of maximal dimension such that 
	$N\cap T=\{0\}$. If $S_i\subseteq N\oplus T$ for all $i\in\{1,\dots,k\}$, then, as $M$ is the sum of the $S_i$, it follows that
	$M=N\oplus T$. 
	If, however, there exists $i\in\{1,\dots,k\}$ such that $S_i\not\subseteq N\oplus T$, then $S_i\cap (N\oplus T)\subseteq S_i$. 
	Since the module $S_i$ is simple,
	it follows that $S_i\cap (N\oplus T)=\{0\}$. Thus $N\cap (S_i\oplus T)=\{0\}$, a contradiction to the maximality of
	$\dim T$.  
	
	The implication  $1)\implies2)$ is trivial. 
	
% 	We now prove that $2)\implies1)$. Let $J$ be a subset of $\{1,\dots,k\}$ maximal with the property that 
% 	the sum $\sum_{j\in J}S_j$ is direct. Let $N=\oplus_{j\in J}S_j$. We claim that $M=N$. 
% 	For each $i\in\{1,\dots,k\}$, either $S_i\cap N=\{0\}$ or $S_i\cap N=S_i$, as
% 	$S_i$ is simple. If $S_i\cap N=S_i$ for all $i\in\{1,\dots,k\}$, then $S_i\subseteq N$ for all $i\in\{1,\dots,k\}$.  
% 	If there exists $i\in\{1,\dots,k\}$ such that $S_i\cap N=\{0\}$, then $N$ and $S_i$ are in direct sum, a contradiction to the maximality of the set $J$. 
	Finally, we prove that $3)\implies1)$. 
	We proceed by induction on $\dim M$. The result is clear if $\dim M=1$. Assume that $\dim M\geq2$ and   
	let $S$ be a non-zero submodule of $M$ of minimal dimension. In particular, $S$ is simple. 
    By assumption, there exists a submodule $T$ of $M$ such that $M=S\oplus T$. We claim that $T$ satisfies the assumptions. 
	If $X$ is a submodule of $T$, then, since $T$ is also a submodule of $M$, there exists a submodule $Y$ of $M$ such that 
	$M=X\oplus Y$. Thus  
	\[
	T=T\cap M=T\cap (X\oplus Y)=X\oplus (T\cap Y),
	\]
	as $X\subseteq T$. 
	Since $\dim T<\dim M$ and $T\cap Y$ is a submodule of $T$, the inductive hypothesis implies 
	that $T$ is a direct sum of simple submodules. Hence $M$ is a direct sum of simple submodules. 
\end{proof}

\begin{proposition}
    If $M$ is a semisimple module and $N$ is a submodule, then $N$ and $M/N$ are semisimple.	
\end{proposition}

\begin{proof}
	Assume that $M=S_1+\cdots+ S_k$, where each $S_i$ is a simple submodule. If $\pi\colon M\to M/N$ 
	is the canonical map, Schur's lemma implies that each restriction $\pi|_{S_i}$ 
	is either zero or an isomorphism with the image. Since  
	\[
	M/N=\pi(M)=\sum_{i=1}^k(\pi|_{S_i})(S_i),
	\]
	it follows that $M/N$ is a direct sum of finitely many simples. 
	
	We now prove that $N$ is semisimple. By assumption, 
	there exists a submodule $T$ such that 
	$M=N\oplus T$. The quotient
	$M/T$ is semisimple by the previous paragraph, so it follows that 
	\[
	N\simeq N/\{0\}=N/(N\cap T)\simeq (N\oplus T)/T=M/T
	\]
	is also semisimple.     
\end{proof}


\chapter{}


\begin{exercise}
If $A$ and $B$ are algebras, $M$ is a module over $A$ and $N$ is a module over $B$, then 
    $M\oplus N$ is a module over $A\times B$ with 
    \[
    (a,b)\cdot (m,n)=(a\cdot m,b\cdot n).
    \]
\end{exercise}


\index{Division algebra}
A \textbf{division algebra} $D$ is an algebra such that every non-zero element 
is invertible, that is for all $x\in D\setminus\{0\}$ there exists $y\in D$ such that $xy=yx=1$.  
Modules over division algebras are very much like vector spaces.  For example, 
every finitely generated module $M$ over a division algebra has a basis. 
Moreover, every linearly independent subset of
$M$ can be extended into a basis of $M$. 

\begin{proposition}
	Let $D$ be a division algebra, and $V$ be a finite-dimensional module over $D$. Then 
	$V$ is a simple module over $\End_D(V)$ and there exits $n\in\Z_{>0}$ such that  
	$\End_D(V)\simeq nV$ is semisimple.
\end{proposition}

\begin{proof}[Sketch of the proof]
	Let $\{v_1,\dots,v_n\}$ be a basis of $V$. A direct calculation shows that the map 
	\[
		\End_D(V)\to\bigoplus_{i=1}^nV=nV,\quad
		f\mapsto (f(v_1),\dots,f(v_n)),
	\]
	is an injective homomorphism of $\End_D(V)$-modules.
	Since
	\[
	\dim_D\End_D(V)=n^2=\dim_D(nV),
	\]
	it follows that the map is an isomorphism. 
	Thus 
	\[
		\End_D(V)\simeq \bigoplus_{i=1}^nV.
	\]
	
	It remains to show that $V$ is simple. It is enough to prove that $V=\End_D(V)\cdot v=(v)$ 
	for all $v\in V\setminus\{0\}$. Let $v\in V\setminus\{0\}$. If $w\in V$, then 
	there exists $f\in\End_D(V)$ such that $f\cdot v=f(v)=w$. 
	Thus $w\in (v)$ and therefore $V=(v)$.  
\end{proof}

The proposition states that if $D$ is a division algebra, then  
$D^{n}$ is a simple $M_n(D)$-module and that $M_n(D)\simeq n D^n$ as $M_n(D)$-modules. 

\begin{exercise}
    Let $M$, $N$, and $X$ be modules. Prove that 
    \begin{align}
        \Hom_A(M\oplus N,X)\simeq\Hom_A(M,X)\times\Hom_A(N,X).
    \end{align}
\end{exercise}

\begin{theorem}
Let $A$ be a finite-dimensional algebra and let 
$S_1,\dots,S_k$ be the simple modules over $A$. 
If 
\[
M\simeq n_1S_1\oplus\cdots\oplus n_kS_k,
\]
then each $n_j$ is uniquely determined.  
\end{theorem}

\begin{proof}
	Since each $S_j$ is simple and $S_i\not\simeq S_j$ if $i\ne j$, 
    Schur's lemma implies that 
	$\Hom_A(S_i,S_j)=\{0\}$ whenever $i\ne j$. 
	For each $j\in\{1,\dots,k\}$, routine calculations show that 
	\begin{align*}
		\Hom_A(M,S_j) &\simeq \Hom_A\left(\bigoplus_{i=1}^k n_i S_i,S_j\right)
		\simeq n_j\Hom_A(S_j,S_j). 
	\end{align*} 
	Since $M$ and $S_j$ are finite-dimensional vector spaces, it follows that
	$\Hom_A(M,S_j)$ and $\Hom_A(S_j,S_j)$ 
	are both finite-dimensional vector spaces.  
	Moreover, the identity $\id\colon S_j\to S_j$ 
	is a module homomorphism and hence  
%	$\id\in\Hom_A(S_j,S_j)$ and hence 
	$\dim\Hom_A(S_j,S_j)\geq 1$. 
	Thus each $n_j$ is uniquely determined, as  
	\[ 
	n_j=\frac{\dim\Hom_A(M,S_j)}{\dim\Hom_A(S_j,S_j)}.\qedhere
	\]
\end{proof}

If $A$ is an algebra, the \textbf{opposite algebra} $A^{\op}$ is the vector space 
$A$ with multiplication $A\times A\to A$, $(a,b)\mapsto ba=a\cdot_{\op}b$. Clearly,
$A$ is commutative if and only if $A=A^{\op}$. 

\begin{lemma}
	\label{lem:A^op}
    If $A$ is an algebra, then $A^{\op}\simeq\End_A(A)$ as algebras.  
\end{lemma}

\begin{proof}
	Note that $\End_A(A)=\{\rho_a:a\in A\}$, where $\rho_a\colon
	A\to A$, $x\mapsto xa$. Indeed, if $f\in\End_A(A)$, then 
	$f(1)=a\in A$. Moreover, $f(b)=f(b1)=bf(1)=ba$ and hence 
	$f=\rho_a$. The map $A^{\op}\to \End_A(A)$, $a\mapsto\rho_a$, 
	is bijective and it is an algebra homomorphism, as 
    \[
		\rho_a\rho_b(x)=\rho_a(\rho_b(x))=\rho_a(xb)=x(ba)=\rho_{ba}(x).\qedhere
    \]
\end{proof}

\begin{lemma}
	\label{lem:Mn_op}
	If $A$ is an algebra and $n\in\Z_{>0}$, then $M_n(A)^{\op}\simeq
	M_n(A^{\op})$ as algebras.   
\end{lemma}

\begin{proof}
	Let $\psi\colon M_n(A)^{\op}\to M_n(A^{\op})$, $X\mapsto X^T$,
	where $X^T$ is the transpose matrix of $X$. Since $\psi$ is a bijective linear map, it is enough
	to see that $\psi$ is a homomorphism. If $i,j\in\{1,\dots,n\}$, $a=(a_{ij})$ and $b=(b_{ij})$, then 
	\begin{align*}
		(\psi(a)\psi(b))_{ij}&=\sum_{k=1}^n \psi(a)_{ik}\psi(b)_{kj}=\sum_{k=1}^n a_{ki}\cdot_{\op}b_{jk}\\
		&=\sum_{k=1}^n b_{jk}a_{ki}=(ba)_{ji}=((ba)^T)_{ij}=\psi(a\cdot_{\op}b)_{ij}.\qedhere
	\end{align*}
\end{proof}

\begin{lemma}
	\label{lem:simple}
	If $S$ is a simple module and $n\in\Z_{>0}$, then  
	\[
		\End_A(nS)\simeq M_n(\End_A(S))
	\]
	as algebras.
\end{lemma}

\begin{proof}[Sketch of the proof]
	Let $(\varphi_{ij})$ be a matrix with entries in $\End_A(S)$. We define a map
	$nS\to nS$ as follows:
	\[
	\begin{pmatrix}
	x_1\\
	\vdots\\
	x_n	
	\end{pmatrix}
	\mapsto 
		\begin{pmatrix}
			\varphi_{11} & \cdots & \varphi_{1n}\\
			\cdots & \ddots & \vdots\\
			\varphi{n1} & \cdots & \varphi_{nn}
		\end{pmatrix}
		\begin{pmatrix}
		x_1\\
		\vdots\\
		x_n	
		\end{pmatrix}
		=\begin{pmatrix}
			\varphi_{11}(x_1)+\cdots+\varphi_{1n}(x_n)\\
			\vdots\\
			\varphi_{n1}(x_1)+\cdots+\varphi_{nn}(x_n)
		\end{pmatrix}.
	\]
	The reader should prove that the map  
	\[
		M_n(\End_A(S))\to\End_A(nS)
	\]
	is an injective algebra homomorphism. 
	It is surjective. Indeed, if $\psi\in\End_A(nS)$ and 
	$i,j\in\{1,\dots,n\}$ one defines $\psi_{ij}$ by 
	\[
		\psi\begin{pmatrix}
		x\\
		0\\
		\vdots\\
		0	
		\end{pmatrix}
		=\begin{pmatrix}
		\psi_{11}(x)\\
		\psi_{21}(x)\\
		\vdots\\
		\psi_{n1}(x)
		\end{pmatrix},\dots,
		\psi\begin{pmatrix}
		0\\
		0\\
		\vdots\\
		x	
		\end{pmatrix}
		=\begin{pmatrix}
		\psi_{1n}(x)\\
		\psi_{2n}(x)\\
		\vdots\\
		\psi_{nn}(x)
		\end{pmatrix}.\qedhere
	\]
\end{proof}

\begin{exercise}
    Prove Lemma \ref{lem:simple}.
\end{exercise}

\begin{exercise}
    Let $M$, $N$, and $X$ be modules. Prove that 
    \begin{align}
        \Hom_A(X,M\oplus N)\simeq\Hom_A(X,M)\times\Hom_A(X,N).
    \end{align}
\end{exercise}

\begin{theorem}[Artin--Wedderburn]
\index{Artin--Wedderburn theorem}
Let $A$ be a finite-dimensional semisimple algebra with  
$k$ isomorphism classes of simple modules. Then 
\[
A\simeq M_{n_1}(D_1)\times\cdots\times M_{n_k}(D_k)
\]
for some $n_1,\dots,n_k\in\Z_{>0}$ and some division algebras $D_1,\dots,D_k$.
\end{theorem}

\begin{proof}
    Decompose the regular representation as a sum of simple modules and
    gather the simples by isomorphism classes to get 	
    \[
	A=\bigoplus_{i=1}^k n_iS_i,
	\]
	where each $S_i$ is simple and $S_i\not\simeq S_j$ whenever 
	$i\ne j$. Schur's lemma implies that  
	\begin{align*}
		\End_A(A)\simeq\End_A\left(\bigoplus_{i=1}^kn_iS_i\right)
		\simeq \prod_{i=1}^k\End_A(n_iS_i)
		\simeq\prod_{i=1}^kM_{n_i}(\End_A(S_i)), 
	\end{align*}
	where each $D_i=\End_A(S_i)$ is a division algebra by Schur's lemma. 
    Thus
    \[
		\End_A(A)\simeq\prod_{i=1}^kM_{n_i}(D_i).
	\]
	Since $\End_A(A)\simeq
	A^{\op}$, it follows that  
	\begin{align*}
		A=(A^{\op})^{\op}\simeq \prod_{i=1}^kM_{n_i}(D_i)^{\op}\simeq \prod_{i=1}^kM_{n_i}(D_i^{\op}).
	\end{align*}
	Since each $D_i$ is a division algebra, each $D_i^{\op}$ is also a division algebra.
\end{proof}

\begin{corollary}[Mollien]
\index{Mollien's theorem}
	If $A$ is a finite-dimensional complex semisimple algebra
	with $k$ isomorphism classes of simple modules, 
	then 
	\[
	A\simeq\prod_{i=1}^k M_{n_i}(\C)
	\]  
	for some $n_1,\dots,n_k\in\Z_{>0}$. 
\end{corollary}

\begin{proof}
	By Wedderburn's theorem,  
	\[
	A\simeq \prod_{i=1}^k M_{n_i}(\End_A(S_i)^{\op}),
	\]
	where $S_1,\dots,S_k$ are representatives of the isomorphism classes of simple modules
	and each $\End_A(S_i)$ is a division algebra. We claim that 
	\[
	\End_A(S_i)=\{\lambda\id:\lambda\in\C\}\simeq\C
	\]
	for all $i\in\{1,\dots,k\}$. If  
	$f\in\End_A(S_i)$, then $f$ has an eigenvalue $\lambda\in\C$. Since  
	$f-\lambda\id$ is not an isomorphism, Schur's lemma implies that $f-\lambda\id=0$, 
	that is $f=\lambda\id$. Thus $\End_A(S_i)\to\C$, $f\mapsto\lambda$, 
	is an algebra isomorphism. In particular,  
	\[
	A\simeq \prod_{i=1}^k M_{n_i}(\C).\qedhere
	\]
\end{proof}

% \begin{exercise}
%     Let $A$ and $B$ be algebras. Prove that the ideals of $A\times B$ are of the form 
%     $I\times J$, where $I$ is an ideal of $A$ and $J$ is an ideal of $B$.
% \end{exercise}

\topic{Group algebras}

Let $K$ be a field, and $G$ be a group. The \textbf{group algebra} $K[G]$ 
is the vector space (over $K$) with basis $\{g:g\in G\}$ 
and the algebra structure is given by the multiplication
\[
	\left(\sum_{g\in G}\lambda_gg\right)\left(\sum_{h\in G}\mu_hh\right)
	=\sum_{g,h\in G}\lambda_g\mu_h(gh).
\]
Every element of $K[G]$ is a finite sum of the form $\sum_{g\in G}\lambda_gg$.

\begin{exercise}
\label{xc:K[G]notsimple}
    If $G$ is non-trivial, then $K[G]$ is not simple. 
\end{exercise}

\begin{exercise}
\label{xca:K_cyclic}
	Let $G=C_n$ be the (multiplicative) cyclic group of order $n$. Prove that 
	$K[G]\simeq K[X]/(X^n-1)$. 
\end{exercise}

\begin{exercise}
\label{xca:abelian=>domain}
	Let $G$ be a finitely-generated torsion-free abelian group. Prove that 
	$K[G]$ is a domain. 
\end{exercise}



\begin{exercise}
	Let $G$ be a group and $\alpha=\sum_{g\in G}\lambda_gg\in K[G]$.  
	The \textbf{support} of $\alpha$ is the set 
	\[
		\supp\alpha=\{g\in G:\lambda_g\ne 0\}.
	\]
	Prove that if $g\in G$, then 
	$\supp(g\alpha)=g(\supp\alpha)$ and $\supp(\alpha g)=(\supp\alpha)g$.
\end{exercise}

\begin{exercise}
\label{xca:invertible_subgroups}
	Let $G$ be a group and $H$ be a subgroup of $G$. Let $\alpha\in K[H]$. Prove that 
    $\alpha$ is invertible (resp. a left zero divisor) in $K[H]$ if and only if 
	$\alpha$ is invertible (resp. a left zero divisor) in
	$K[G]$.
\end{exercise}

% El objetivo de esta sección es calcular el radical de Jacobson del álgebra de
% grupo de un grupo finito. Comenzamos con un ejemplo:

\begin{exercise}
	Let $G=C_2=\langle g\rangle\simeq\Z/2$ the (multiplicative) 
	group with two elements. Note that every element of $K[G]$ is of the form
	$a+bg$ for some $a,b\in K$. Prove the following statements:
	\begin{enumerate}
	    \item If the characteristic of $K$ is different from two, then 
	    \[
		K[G]\to K\times K,
		\quad
		a1+bg\mapsto (a+b,a-b),
	\]
	is an algebra isomorphism. 
	\item If the characteristic of $K$ is two, then 
	\[
	K[G]\to \begin{pmatrix}
			K & K\\
			0 & K
		\end{pmatrix},
		\quad
		a1+bg\mapsto\begin{pmatrix}
			a+b & b\\
			0 & a+b
		\end{pmatrix},
	\]
	is an algebra isomorphism. 
	\end{enumerate}
\end{exercise}

If $A$ is an algebra over $K$ and $\rho\colon G\to \mathcal{U}(A)$
is a group homomorphism, where $\mathcal{U}(A)$ is the group of units of $A$, then 
the map \[
	K[G]\to A,\quad 
\sum_{g\in G}\lambda_gg\mapsto\sum_{g\in G}\lambda_g\rho(g),
\]
is an algebra homomorphism. 

\begin{exercise}
	Let $G=C_3$ be the (multiplicative) group of three elements. Prove that
	$\R[G]\simeq\R\times\C$.
% 	Escribamos $G=\langle g:g^3=1\rangle$ y sea 
% 	\[
% 		\varphi\colon\R[G]\to\R\times\C,
% 		\quad
% 		g\mapsto (1,\omega),
% 	\]
% 	donde $\omega$ es una raíz cúbica primitiva de la unidad. Entonces
% 	$\varphi$ es inyectivo pues
% 	$0=\varphi(a1+bg+cg^2)=(a+b+c,a+b\omega+c\omega^2)$ implica que $a=b=c=0$.
% 	Luego $\varphi$ es un isomorfismo pues
% 	$\dim_\R\R[G]=\dim_\R(\R\times\C)=3$. 
\end{exercise}

\begin{exercise}
	Let $G=\langle r,s:r^3=s^2=1,\,srs=r^{-1}\rangle$ be the dihedral group of six elements. 
	Prove the following statements:
	\begin{enumerate}
	    \item $\C[G]\simeq\C\times\C\times M_2(\C)$.
	    \item $\Q[G]\simeq\Q\times\Q\times M_2(\Q)$.
	\end{enumerate}  
% 	Sea $\omega$ una raíz cúbica de la unidad y sean  
% 	\[
% 		R=\begin{pmatrix}
% 			\omega & 0\\
% 			0 & \omega^2
% 		\end{pmatrix},
% 		\quad
% 		S=\begin{pmatrix}
% 			0 & 1\\
% 			1 & 0
% 		\end{pmatrix}.
% 	\]
% 	Un cálculo sencillo muestra que $R^2=S^2=I$ y que $SRS=R^{-1}$. Sea
% 	\[
% 		\varphi\colon\C[G]\to\C\times\C\times M_2(\C),\quad
% 		r\mapsto (1,1,R),\quad
% 		s\mapsto (1,-1,S).
% 	\]
% 	Es fácil ver que $\varphi$ es un morfismo de álgebras. Veamos que es
% 	biyectivo. Como $\dim_{\C}\C[G]=\dim_{\C}(\C\times\C\times M_2(\C))=6$,
% 	basta ver que $\varphi$ es inyectivo. Si 
% 	\[
% 		\alpha=a_0+a_1r+a_2r^2+(b_0+b_1r+b_2r^2)s\in\ker\varphi,
% 	\]
% 	entonces 
% 	\[
% 		0=\varphi(\alpha)=\left(u,v,\begin{pmatrix} \alpha_{11} & \alpha_{12}\\\alpha_{21}&\alpha_{22}\end{pmatrix}\right), 
% 	\]
% 	donde
% 	\begin{align*}
% 		&u = a_0+a_1+a_2+b_0+b_1+b_2, && v = a_0+a_1+a_2-b_0-b_1-b_2,\\
% 		&\alpha_{11}=a_0+a_1\omega+a_2\omega^2, && \alpha_{12}=b_0+b_1\omega+b_2\omega^2,\\
% 		&\alpha_{21}=b_0+b_2\omega+b_1\omega^2, && \alpha_{22}=a_0+a_2\omega+a_1\omega^2.
% 	\end{align*}
% 	Un cálculo sencillo muestra que estas ecuaciones implican que
% 	$\alpha=0$ y luego $\varphi$ es inyectiva.  
\end{exercise}



Maschke's theorem states that, if $G$ is a finite group, 
then the group algebra $\C[G]$ is semisimple. By Mollien's theorem, 
\[
\C[G]\simeq \prod_{i=1}^k M_{n_i}(\C),
\]
where $k$ is the number of (isomorphism classes of) 
simple $\C[G]$-modules. Moreover, 
\[
|G|=\dim\C[G]=\sum_{i=1}^k n_i^2.
\]

\begin{theorem}
    Let $G$ be a finite group. The number of simple 
    modules of $\C[G]$ coincides with the number of conjugacy classes of $G$. 
\end{theorem}

\begin{proof}
    By Mollien's theorem, $\C[G]\simeq\prod_{i=1}^kM_{n_i}(\C)$. Thus 
    \[
		Z(\C[G])\simeq\prod_{i=1}^kZ(M_{n_i}(\C))\simeq\C^k.
	\]
	In particular, $\dim Z(\C[G])=k$. If $\alpha=\sum_{g\in
	G}\lambda_gg\in Z(\C[G])$, then $h^{-1}\alpha h=\alpha$ for all $h\in
	G$. Thus 
	\[
		\sum_{g\in G}\lambda_{hgh^{-1}}g=
		\sum_{g\in g}\lambda_g h^{-1}gh=\sum_{g\in G}\lambda_gg
	\]
	and hence $\lambda_{g}=\lambda_{hgh^{-1}}$ for all $g,h\in G$. A basis for 
	$Z(\C[G])$ is given by elements of the form 
	\[
		\sum_{g\in K}g,
	\]
	where $K$ is a conjugacy class of $G$. Therefore $\dim Z(\C[G])$ is equal to 
	the number of conjugacy classes of $G$.
\end{proof}

\begin{example}
\label{exa:C4}
    Let $G=C_4$ be the cyclic group of order four. Then
    $G$ has four simple modules and 
    $\C[G]\simeq\C^4$. 
\end{example}

\begin{example}
\label{exa:S3}
    Let $G=\Sym_3$. Then $G$ has three simple modules and
    \[
    \C[G]\simeq\C\times\C\times M_2(\C).
    \]
\end{example}

\begin{problem}[Brauer]
\index{Brauer's problem}
    Which algebras are group algebras? 
\end{problem}

This question might be impossible to answer, but it is extremely interesting. 
Examples \ref{exa:C4} and \ref{exa:S3} show
that $\C^4$ and $\C^2\times M_2(\C)$ are complex group algebras. 

\begin{exercise}
    Is $\C^2\times M_2(\C)\times M_3(\C)$ a complex group algebra?  
\end{exercise}

% No. Let $G$ be a group of order 15. Since groups of order 15 are abelian,
% $G$ has 15 conjugacy classes. 

\section{Lecture: 10/10/2024}
\label{03}

\subsection{Group algebras}

Let $K$ be a field, and $G$ be a group. The \emph{group algebra} $K[G]$ 
is the vector space (over $K$) with basis $\{g:g\in G\}$ 
and the algebra structure is given by the multiplication
\[
	\left(\sum_{g\in G}\lambda_gg\right)\left(\sum_{h\in G}\mu_hh\right)
	=\sum_{g,h\in G}\lambda_g\mu_h(gh).
\]
Every element of $K[G]$ is a finite sum of the form $\sum_{g\in G}\lambda_gg$.

\begin{exercise}
\label{xc:K[G]notsimple}
    If $G$ is non-trivial, then $K[G]$ is not simple. 
\end{exercise}

\begin{exercise}
\label{xca:K_cyclic}
	Let $G=C_n$ be the (multiplicative) cyclic group of order $n$. Prove that 
	$K[G]\simeq K[X]/(X^n-1)$. 
\end{exercise}

\begin{exercise}
\label{xca:abelian=>domain}
	Let $G$ be a finitely-generated torsion-free abelian group. Prove that 
	$K[G]$ is a domain. 
\end{exercise}



\begin{exercise}
	Let $G$ be a group and $\alpha=\sum_{g\in G}\lambda_gg\in K[G]$.  
	The \emph{support} of $\alpha$ is the set 
	\[
		\supp\alpha=\{g\in G:\lambda_g\ne 0\}.
	\]
	Prove that if $g\in G$, then 
	$\supp(g\alpha)=g(\supp\alpha)$ and $\supp(\alpha g)=(\supp\alpha)g$.
\end{exercise}

\begin{exercise}
\label{xca:invertible_subgroups}
	Let $G$ be a group and $H$ be a subgroup of $G$. Let $\alpha\in K[H]$. Prove that 
    $\alpha$ is invertible (resp. a left zero divisor) in $K[H]$ if and only if 
	$\alpha$ is invertible (resp. a left zero divisor) in
	$K[G]$.
\end{exercise}

% El objetivo de esta sección es calcular el radical de Jacobson del álgebra de
% grupo de un grupo finito. Comenzamos con un ejemplo:

\begin{exercise}
	Let $G=C_2=\langle g\rangle\simeq\Z/2$ the (multiplicative) 
	group with two elements. Note that every element of $K[G]$ is of the form
	$a+bg$ for some $a,b\in K$. Prove the following statements:
	\begin{enumerate}
	    \item If the characteristic of $K$ is different from two, then 
	    \[
		K[G]\to K\times K,
		\quad
		a1+bg\mapsto (a+b,a-b),
	\]
	is an algebra isomorphism. 
	\item If the characteristic of $K$ is two, then 
	\[
	K[G]\to \begin{pmatrix}
			K & K\\
			0 & K
		\end{pmatrix},
		\quad
		a1+bg\mapsto\begin{pmatrix}
			a+b & b\\
			0 & a+b
		\end{pmatrix},
	\]
	is an algebra isomorphism. 
	\end{enumerate}
\end{exercise}

If $A$ is an algebra over $K$ and $\rho\colon G\to \mathcal{U}(A)$
is a group homomorphism, where $\mathcal{U}(A)$ is the group of units of $A$, then 
the map \[
	K[G]\to A,\quad 
\sum_{g\in G}\lambda_gg\mapsto\sum_{g\in G}\lambda_g\rho(g),
\]
is an algebra homomorphism. 

\begin{exercise}
	Let $G=C_3$ be the (multiplicative) group of three elements. Prove that
	$\R[G]\simeq\R\times\C$.
% 	Escribamos $G=\langle g:g^3=1\rangle$ y sea 
% 	\[
% 		\varphi\colon\R[G]\to\R\times\C,
% 		\quad
% 		g\mapsto (1,\omega),
% 	\]
% 	donde $\omega$ es una raíz cúbica primitiva de la unidad. Entonces
% 	$\varphi$ es inyectivo pues
% 	$0=\varphi(a1+bg+cg^2)=(a+b+c,a+b\omega+c\omega^2)$ implica que $a=b=c=0$.
% 	Luego $\varphi$ es un isomorfismo pues
% 	$\dim_\R\R[G]=\dim_\R(\R\times\C)=3$. 
\end{exercise}

\begin{exercise}
\label{xca:isos_dihedral}
	Let $G=\langle r,s:r^3=s^2=1,\,srs=r^{-1}\rangle$ be the dihedral group of six elements. 
	Prove the following statements:
	\begin{enumerate}
	    \item $\C[G]\simeq\C\times\C\times M_2(\C)$.
	    \item $\Q[G]\simeq\Q\times\Q\times M_2(\Q)$.
	\end{enumerate}  
% 	Sea $\omega$ una raíz cúbica de la unidad y sean  
% 	\[
% 		R=\begin{pmatrix}
% 			\omega & 0\\
% 			0 & \omega^2
% 		\end{pmatrix},
% 		\quad
% 		S=\begin{pmatrix}
% 			0 & 1\\
% 			1 & 0
% 		\end{pmatrix}.
% 	\]
% 	Un cálculo sencillo muestra que $R^2=S^2=I$ y que $SRS=R^{-1}$. Sea
% 	\[
% 		\varphi\colon\C[G]\to\C\times\C\times M_2(\C),\quad
% 		r\mapsto (1,1,R),\quad
% 		s\mapsto (1,-1,S).
% 	\]
% 	Es fácil ver que $\varphi$ es un morfismo de álgebras. Veamos que es
% 	biyectivo. Como $\dim_{\C}\C[G]=\dim_{\C}(\C\times\C\times M_2(\C))=6$,
% 	basta ver que $\varphi$ es inyectivo. Si 
% 	\[
% 		\alpha=a_0+a_1r+a_2r^2+(b_0+b_1r+b_2r^2)s\in\ker\varphi,
% 	\]
% 	entonces 
% 	\[
% 		0=\varphi(\alpha)=\left(u,v,\begin{pmatrix} \alpha_{11} & \alpha_{12}\\\alpha_{21}&\alpha_{22}\end{pmatrix}\right), 
% 	\]
% 	donde
% 	\begin{align*}
% 		&u = a_0+a_1+a_2+b_0+b_1+b_2, && v = a_0+a_1+a_2-b_0-b_1-b_2,\\
% 		&\alpha_{11}=a_0+a_1\omega+a_2\omega^2, && \alpha_{12}=b_0+b_1\omega+b_2\omega^2,\\
% 		&\alpha_{21}=b_0+b_2\omega+b_1\omega^2, && \alpha_{22}=a_0+a_2\omega+a_1\omega^2.
% 	\end{align*}
% 	Un cálculo sencillo muestra que estas ecuaciones implican que
% 	$\alpha=0$ y luego $\varphi$ es inyectiva.  
\end{exercise}



Maschke's theorem states that, if $G$ is a finite group, 
then the group algebra $\C[G]$ is semisimple. By Mollien's theorem, 
\[
\C[G]\simeq \prod_{i=1}^k M_{n_i}(\C),
\]
where $k$ is the number of (isomorphism classes of) 
simple $\C[G]$-modules. Moreover, 
\[
|G|=\dim\C[G]=\sum_{i=1}^k n_i^2.
\]

\begin{theorem}
    Let $G$ be a finite group. The number of simple 
    modules of $\C[G]$ coincides with the number of conjugacy classes of $G$. 
\end{theorem}

\begin{proof}
    By Mollien's theorem, $\C[G]\simeq\prod_{i=1}^kM_{n_i}(\C)$. Thus 
    \[
		Z(\C[G])\simeq\prod_{i=1}^kZ(M_{n_i}(\C))\simeq\C^k.
	\]
	In particular, $\dim Z(\C[G])=k$. If $\alpha=\sum_{g\in
	G}\lambda_gg\in Z(\C[G])$, then $h^{-1}\alpha h=\alpha$ for all $h\in
	G$. Thus 
	\[
		\sum_{g\in G}\lambda_{hgh^{-1}}g=
		\sum_{g\in g}\lambda_g h^{-1}gh=\sum_{g\in G}\lambda_gg
	\]
	and hence $\lambda_{g}=\lambda_{hgh^{-1}}$ for all $g,h\in G$. A basis for 
	$Z(\C[G])$ is given by elements of the form 
	\[
		\sum_{g\in K}g,
	\]
	where $K$ is a conjugacy class of $G$. Therefore $\dim Z(\C[G])$ is equal to 
	the number of conjugacy classes of $G$.
\end{proof}

\subsection{Which algebras are group algebras?}

\begin{example}
\label{exa:C4}
    Let $G=C_4$ be the cyclic group of order four. Then
    $G$ has four simple modules and 
    $\C[G]\simeq\C^4$. 
\end{example}

\begin{example}
\label{exa:S3}
    Let $G=\Sym_3$. Then $G$ has three simple modules and
    \[
    \C[G]\simeq\C\times\C\times M_2(\C).
    \]
\end{example}

\begin{problem}[Brauer]
\index{Brauer's problem}
    Which algebras are group algebras? 
\end{problem}

This question might be impossible to answer, but it is extremely interesting. 
Examples \ref{exa:C4} and \ref{exa:S3} show
that $\C^4$ and $\C^2\times M_2(\C)$ are complex group algebras. 

\begin{exercise}
    Is $\C^2\times M_2(\C)\times M_3(\C)$ a complex group algebra?  
\end{exercise}

% No. Let $G$ be a group of order 15. Since groups of order 15 are abelian,
% $G$ has 15 conjugacy classes. 

\subsection{The isomorphism problem for group algebras}


Recall that if $R$ is a unitary commutative ring 
and $G$ is a group, then one defines the group ring $R[G]$ (see Appendix \ref{section:Hurewicz}). 
% More precisely,
% $R[G]$ is the set of finite linear combinations
% \[
%     \sum_{g\in G}\lambda_gg
% \]
% where $\lambda_g\in R$ and $\lambda_g=0$ for all but finitely many $g\in G$.
% One easily proves that $R[G]$ is a ring with
% addition
% \[
% \left(\sum_{g\in G}\lambda_gg\right)+\left(\sum_{g\in G}\mu_gg\right)
% =\sum_{g\in G}(\lambda_g+\mu_g)(g)
% \]
% and multiplication
% \[
% \left(\sum_{g\in G}\lambda_gg\right)\left(\sum_{h\in G}\mu_hh\right)
% =\sum_{g,h\in G}\lambda_g\mu_h(gh).
% \]
Note that $R[G]$ is a left $R$-module with
\[
\lambda\left(\sum_{g\in G}\lambda_gg\right)=\sum_{g\in G}(\lambda\lambda_g)g.
\]

In this section, we will briefly discuss the
following natural problem: 

\begin{question}[The isomorphism problem]
\label{question:iso}
 Let $R$ be a ring and $G$ and $H$ be groups. Assume
 that $R[G]\simeq R[H]$ (as $R$-algebras). Does $G\simeq H$?
\end{question}

For general information on Question \ref{question:iso} we refer to the survey paper \cite{MR4472590}.

\begin{exercise}
    Prove that if $G$ and $H$ are isomorphic groups, then $K[G]\simeq K[H]$.
\end{exercise}

\begin{exercise}
    Let $G$ and $H$ be groups. Prove that if
    $\Z[G]\simeq\Z[H]$, then $R[G]\simeq R[H]$ for any commutative ring $R$.
\end{exercise}

The previous exercise suggest the importance of the following 
instance of Question \ref{question:iso}:

\begin{question}
\label{question:IP}
    Let $G$ and $H$ be groups. Does $\Z[G]\simeq\Z[H]$ imply $G\simeq H$?
\end{question}

Although there are several cases where
the isomorphism problem has an affirmative answer (e.g. abelian groups,
metabelian groups, nilpotent groups, nilpotent-by-abelian groups, simple groups,
abelian-by-nilpotent groups), it is false in general. In 2001
Hertweck found a counterexample of order $2^{21}97^{28}$, see \cite{MR1847590}.

\begin{question}[The modular isomorphism problem]
\label{question:MIP}
    Let $p$ be a prime number. Let
    $G$ and $H$ be finite $p$-groups and let $K$ be a field of characteristic $p$.
    Does $K[G]\simeq K[H]$ imply $G\simeq H$?
\end{question}

Question \ref{question:MIP} has an affirmative answer in several cases. However,
this is not true in general. This question was recently answered by Garc\'ia, Margolis and
del R\'io \cite{MR4373245}. They found two non-isomorphic groups $G$ and $H$ both of order $512$
such that $K[G]\simeq K[H]$ for all field $K$
of characteristic two.

\subsection{Primitive rings}
\label{Primitive rings}

We will consider (possibly non-unitary) rings. Thus  
a \emph{ring} is an abelian group $R$ with an associative multiplication 
$(x,y)\mapsto xy$ such that $(x+y)z=xz+yz$ and $x(y+z)=xy+xz$ for all $x,y,z\in
R$. If there is an element $1\in R$ such that $x1=1x=x$ for all $x\in R$, we 
say that $R$ is a \emph{unitary ring}.  A \emph{subring} $S$ of $R$ is an additive
subgroup of $R$ closed under multiplication. 

\begin{example}
    $\Z$ is a (unitary) ring and 
	$2\Z=\{2m:m\in\Z\}$ is a (non-unitary) ring.  
\end{example}

A \emph{left ideal} (resp. \emph{right ideal}) is a subring $I$ of $R$ such that 
$rI\subseteq I$ (resp. $Ir\subseteq I$) for all $r\in R$. An \emph{ideal}
(also two-sided ideal) of $R$ is a subring $I$ of $R$ that is both a left and a right ideal of $R$.

\begin{example}
	If $I$ and $J$ are both ideals of a ring $R$, then the sum 
 \[
 I+J=\{x+y:x\in I,y\in J\}
 \]
 and
	the intersection $I\cap J$ are both ideals of $R$. The product $IJ$, defined as the additive
	subgroup of $R$ generated by $\{xy:x\in I,y\in J\}$, is also an ideal of $R$. 
\end{example}

\begin{example}
	If $R$ is a ring, the set $Ra =\{xa: x\in R\}$ is a left ideal
	of $R$. Similarly, the set $aR =\{ax: x\in R\}$ is a right ideal of $R$. The set $RaR$, which is
	defined as the additive subgroup of $R$ generated by $\{xay: x, y\in R\}$, is a
	ideal of $R$.
\end{example}

\begin{example}
	If $R$ is a unitary ring, then $Ra$ is the left ideal generated by $a$, $aR$ is
	the right ideal generated by $a$ and $RaR$ is the ideal generated by $a$. 
	If $R$ is not unitary, the left ideal generated by $a$ is $Ra+\Z a$,
	the right ideal generated by $a$ is $aR+\Z a$ and the ideal generated by 
	$a$ is $RaR+Ra+aR+\Z a$.
\end{example}

The following exercise asks to prove the \emph{Chinese Remainder Theorem}  
for arbitrary rings.

\begin{bonus}
    \label{xca:chinese}
    \index{Chinese Remainder Theorem}
    Let $R$ be a ring and $I_1,\dots,I_n$ be ideals such that 
    $I_j+I_k=R$ whenever $j\ne k$ and $R=I_j+R^2$ for all $j$. Prove that 
    \[
    R/(I_1\cap\cdots\cap I_n)\simeq R/I_1\times\cdots\times R/I_n.
    \]
\end{bonus}

In the previous exercise, the condition $R=I_j+R^2$ trivially holds in the case of rings with one. 

\begin{definition}
\index{Ring!simple}
A ring $R$ is said to be \emph{simple} if $R^2\ne\{0\}$ and the only ideals of 
$R$ are $\{0\}$ and~$R$.  
\end{definition}

The condition $R^2\ne\{0\}$ is trivially satisfied in the case of rings
with identity, as 
\[
1\in R^2=\{r_1r_2:r_1,r_2\in R\}.
\]

\begin{example}
	Division rings are simple.
\end{example}

Let $S$ be a unitary ring. Recall that $M_n(S)$ is the ring of $n\times n$ square matrices 
with entries in $S$.  If $A=(a_{ij})\in M_n(S)$ and $E_{ij}$ is the matrix
such that $(E_{ij})_{kl}=\delta_{ik}\delta_{jl}$, then
\begin{equation}
	\label{eq:trick}
E_{ij}AE_{kl}=a_{jk}E_{il}
\end{equation}
for all $i,j,k,l\in\{1,\dots,n\}$. 

\begin{example}
	If $D$ is a division ring, then $M_n(D)$ is simple. 
\end{example}

Let $R$ be a ring. A left $R$-module (or module, for short)  
is an abelian group $M$ together with a map $R\times M\to M$, $(r,m)\mapsto r\cdot m$, such that
\begin{align*}
	&(r+s)\cdot m=r\cdot m+s\cdot m, &&
	r\cdot (m+n)=r\cdot m+r\cdot s, && r\cdot (s\cdot m)=(rs)\cdot m    
\end{align*}
for all $r,s\in R$, $m,n\in M$.  If $R$ has an identity 
$1$ and $1\cdot m=m$ holds for all $m\in M$, the module $M$ is said to be 
\emph{unitary}.  If $M$ is a unitary module, then $M=R\cdot M$. %\ne\{0\}$.

\begin{exercise}
\label{xca:center_simple}
Let $R$ be a simple unitary ring. 
\begin{enumerate}
    \item Prove that the center $Z(R)$ of $R$ is a field.
    \item Prove that $R$ is an algebra over $Z(R)$. 
\end{enumerate}
\end{exercise}

% Let $0\ne x\in Z(R)$. Then $Rx$ is a non-zero ideal of $R$.
% Since $R$ is simple, $Rx=R$. Thus $rx=1$ for some $r\in R$. 
% It follows that $x\in\mathcal{U}(R)$.

\begin{definition}
\label{Module!simple}
    A module $M$ is said to be 
    \emph{simple} if $R\cdot M\ne\{0\}$ and 
    the only submodules of $M$ are $\{0\}$ and $M$.
    If $M$ is a simple module, then $M\ne\{0\}$.
\end{definition}

If $R$ is a unitary ring and $M$ is a simple 
module, then $M$ is unitary. 

%\begin{remark}
%	Si $R$ es unitario y $M$ es un módulo simple, entonces $M$ es unitario.
%\end{remark}

\begin{lemma}
	\label{lemma:simple}
	Let $M$ be a non-zero module. Then $M$ is simple if and only if $M=R\cdot m$
	for all $0\ne m\in M$.
\end{lemma}

\begin{proof}
	Assume that $M$ is simple.  Let $m\ne 0$. Since $R\cdot m$ is a submodule of the simple 
	module $M$, either $R\cdot m=\{0\}$ or $R\cdot m=M$.  Let $N=\{n\in M:R\cdot n=\{0\}\}$. Since $N$ is a 
	submodule of $M$ and $R\cdot M\ne\{0\}$, $N=\{0\}$. Therefore $R\cdot m=M$, as $m\ne0$.
	Now assume that $M=R\cdot m$ for all $m\ne0$. Let $L$ be a non-zero submodule of 
	$M$ and let $0\ne x\in L$. Then $M=L$, as $M=R\cdot x\subseteq L$. 
\end{proof} 

\begin{example}
	Let $D$ be a division ring and let $V$ be a non-zero vector space (over $D$). If 
	$R=\End_D(V)$, then $V$ is a simple $R$-module with $fv=f(v)$, $f\in R$.
	$v\in V$. 
% 	Para ver que $V$ es simple como $R$-módulo basta ver que $Rv=V$ para todo
% 	$v\ne0$. Sean $v,w\in V$, $v\ne0$.  Al completar $v\ne0$ a una base de $V$,
% 	vemos que existe $f\in R$ tal que $f(v)=w$. Luego $V$ es simple.
\end{example}

\begin{example}
	\label{exa:I_k}
	Let $n\geq2$.  If $D$ is a division ring and $R=M_n(D)$, then each 
	\[
	I_k=\{ (a_{ij})\in R:a_{ij}=0\text{ for $j\ne k$}\}
	\]
	is an $R$-module isomorphic to $D^n$. 
	Thus $M_{n}(D)$ is a simple ring that is not a simple $M_n(D)$-module.
\end{example}

\begin{definition}
\index{Minimal left ideal}
A left ideal $L$ of a ring $R$ is said to be \emph{minimal} if $L\ne\{0\}$ and 
$L$ does not strictly contain other left ideals of $R$. 
\end{definition}

Similarly one defines
right minimal ideals and minimal ideals. 

\begin{example}
	Let $D$ be a division ring and let $R=M_n(D)$. Then $L=RE_{11}$ 
	is a minimal left ideal.
\end{example}

\begin{example}
	Let $L$ be a non-zero left ideal. If $RL\ne\{0\}$, then
	$L$ is minimal if and only if $L$ is a simple $R$-module. 
\end{example}

\begin{definition}
\index{Ideal!regular}
\index{Left ideal!regular}
\label{def:regular}
A left (resp. right) ideal $L$ of $R$ is said to be \emph{regular} if
there exists $e\in R$ such that $r-re\in L$ (resp.  $r-er\in L$) for all $r\in R$.
\end{definition}

If $R$ is a ring with identity, every left (or right) ideal is regular. 

\begin{definition}
\index{Ideal!maximal}
\index{Left ideal!maximal}
A left (resp. right) ideal $I$ of $R$ is said to be \emph{maximal} if $I\ne R$ and $I$ is not properly contained 
in any other left (resp. right) ideal of $R$. 
\end{definition}

Similarly, one defines maximal ideals. 

A standard application of Zorn's lemma proves 
that every unitary ring contains a maximal left (or right) ideal.  

\begin{proposition}
	\label{proposition:R/I}
	Let $R$ be a ring and $M$ be a module. Then $M$ is simple if and only if
	$M\simeq R/I$ for some maximal regular left ideal $I$. 	
\end{proposition}

\begin{proof}
	Assume that $M$ is simple. Then $M=R\cdot m$ for some $m\ne0$ by 
	Lemma~\ref{lemma:simple}. The map $\phi\colon R\to M$, $r\mapsto r\cdot m$, 
	is a surjective homomorphism of $R$-modules, 
	so the first isomorphism theorem implies that 
	$M\simeq R/\ker\phi$. Since $\ker\phi$ is an ideal of $R$, it is 
	in particular a left ideal of $R$. 
	
	We claim that $I=\ker\phi$ is a maximal left ideal. 
	The correspondence theorem 
	and the simplicity of $M$ imply that $I$ is a 
	maximal left ideal (because each left ideal $J$ such that 
	$I\subseteq J$ yields a submodule of $R/I$).

	We claim that $I$ is regular. Since $M=R\cdot m$, there exists $e\in R$ such that $m=e\cdot m$. If
	$r\in R$, then $r-re\in I$ since 
	$\phi(r-re)=\phi(r)-\phi(re)=r\cdot m-r\cdot (e\cdot m)=0$.

    Now assume that $I$ is a maximal left ideal that is regular. 
    The correspondence theorem implies that 
    $R/I$ has no non-zero proper submodules. 
    
    We claim that 
    $R\cdot (R/I)\ne0$. If $R\cdot (R/I)=\{0\}$ and $r\in R$, then 
    the regularity of $I$ implies that 
    there exists $e\in R$ such that $r-re\in I$. Hence $r\in I$, as  
	\[
	0=r\cdot (e+I)=re+I=r+I,
	\]
	a contradiction to the maximality of $I$. 
\end{proof}


%\section{Nilpotencia}
%
%Recordemos que si $I$ es un ideal de un anillo $R$, se define $I^n$ como el
%subgrupo aditivo generado por el conjunto $\{y_1\dots y_n:y_j\in I\}$. 
%
%\begin{definition}
%	Un ideal $I$ de un anillo $R$ se dice \emph{nilpotente} si $I^n=0$ para
%	algún $n\in\N$.
%\end{definition}
%
%Recordemos que un elemento $x$ de un anillo $R$ se dice \emph{nilpotente} si
%existe $n\in\N$ tal que $x^n=0$. 
%
%\begin{definition}
%	Un ideal $I$ de un anillo $R$ se dice \emph{nil} si todo elemento de $I$
%	es nilpotente.
%\end{definition}
%
%\begin{remark}
%	Un ideal nilpotente es nil. 
%\end{remark}
%
%\begin{example}
%	Sea $R=\C[x_1,x_2,\dots]/(x_1,x_2^2,x_3^3,\dots)$. El ideal
%	$I=(x_1,x_2,x_3,\dots)$ es nil en $R$ pues está generado por elementos
%	nilpotentes pero no es nilpotente. Si lo fuera, existiría $k\in\N$ tal que
%	$I^k=0$, y luego $x_i^k=0$ para todo $i$, una contradicción pues
%	$x_{k+1}^k\ne0$. 	
%\end{example}
%
%% example 2.7 del libro de springer
%% problema de kothe 
%
%\begin{lemma}
%	Si $I$ y $J$ son ideales nilpotentes, $I+J$ es nilpotente.
%\end{lemma}
%
%\begin{proof}
%	
%\end{proof}
%
%Un ideal $N$ de un anillo $R$ se dice \emph{maximal-nilpotente} si $N$ es
%nilpotente y no está propiamente contenido en ningún ideal nilpotente de $R$.
%
%\begin{lemma}
%	Si el anillo $R$ contiene un ideal maximal-nilpotente $N$ entonces todo
%	ideal nilpotente está contenido en $N$.
%\end{lemma}
%
%\begin{proof}
%	
%\end{proof}
%
%\section{Anillos primos y semiprimos}
%
%\index{Dominio}
%Recordemos que un anillo $R$ se dice un \emph{dominio} si para todo $a,b\in
%R$ tales que $ab=0$ se tiene $a=0$ o $b=0$.
%Una generalización al caso no conmutativo es la siguiente:
%
%\begin{definition}
%	\index{Anillo!primo}
%	Un anillo $R$ se dice \emph{primo} si para todo $a,b\in R$ tales que
%	$aRb=0$ se tiene $a=0$ o $b=0$.
%\end{definition}
%
%\begin{lemma}
%	Sea $R$ un anillo. Las siguientes propiedades son equivalentes:
%	\begin{enumerate}
%		\item $R$ es primo.
%		\item Si $I,J\subseteq R$ son ideales a izquierda tales que $IJ=0$
%			entonces $I=0$ o $J=0$.
%%		\item Si $I,J\subseteq R$ son ideales a derecha tales que $IJ=0$ 
%%			entonces $I=0$ o $J=0$.
%		\item Si $I,J\subseteq R$ son ideales tales que $IJ=0$ entonces $I=0$ o
%			$J=0$.
%	\end{enumerate}
%\end{lemma}
%
%\begin{proof}
%	Vamos a demostrar que $(1)\implies(2)\implies(4)\implies(1)$. 
%	La implicación $(2)\implies(4)$ es trivial.
%
%	Veamos que $(4)\implies(1)$. Sean $a,b\in R$ tales que $aRb=0$.
%	Como entonces $(RaR)(RbR)=R(aRb)R=0$, $RaR=0$ o bien $RbR=0$. Supongamos
%	sin pérdida de generalidad que $RaR=0$. Entonces $Ra$ y $aR$ son ideales
%	biláteros tales que $(Ra)R=R(aR)=0$. Al aplicar la hipótesis, $Ra=aR=0$.
%	Como $\Z a$ es un ideal de $R$ tal que $(\Z a)R=0$, se concluye al aplicar
%	la hipótesis que $a=0$.
%
%	Veamos que $(1)\implies(2)$. Supongamos que $J\ne 0$, sea $y\in
%	J\setminus\{0\}$ y sea $x\in I$. Como 
%	$xRy\subseteq IRJ=I(RJ)\subseteq IJ=0$, 
%	se concluye, al usar la hipótesis, que $x=0$. 
%\end{proof}
%
%\begin{proposition}
%	Un anillo conmutativo es primo si y sólo si es un dominio. 
%\end{proposition}
%
%\begin{proof}
%	Supongamos que $R$ es un anillo primo. Si $a,b\in R$ son tales que $ab=0$
%	entonces $aRb=(ab)R=0$ y luego $a=0$ o bien $b=0$. Supongamos ahora que $R$
%	es un dominio. Si $a,b\in R$ son tales que $aRb=0$ entonces $(ab)R=0$ y
%	luego $a=0$ o bien $b=0$ pues $ab=0$. 
%\end{proof}
%
%\begin{definition}
%	\index{Anillo!semiprimo}
%	Un anillo $R$ se dice \emph{semiprimo} si para todo $a\in R$ tal que
%	$aRa=0$ se tiene $a=0$.
%\end{definition}
%
%\begin{lemma}
%	Sea $R$ un anillo. Las siguientes aifrmaciones son equivalentes:
%	\begin{enumerate}
%		\item $R$ es semiprimo.
%		\item Si $I$ es un ideal a izquierda tal que $I^2=0$ entonces $I=0$.
%		\item Si $I$ es un ideal tal que $I^2=0$ entonces $I=0$.
%		\item $R$ no tiene ideales nilpotentes no nulos.
%	\end{enumerate}
%\end{lemma}
%
%\begin{proof}
%	Primero vamos a demostrar que $(1)\implies(2)\implies(3)\implies(1)$ 
%
%	La implicación $(5)\implies(4)$ es trivial.	
%	Demostremos entonces que
%	$(4)\implies(5)$. Sea $I$ un ideal nilpotente no nulo y sea $n\in\N$ el
%	mínimo tal que $I^n=0$. Como $(I^{n-1})^2=0$, por hipótesis se tiene que
%	$I^{n-1}=0$, una contradicción.
%\end{proof}
%

Let $R$ be a ring and $M$ be a left $R$-module. For a 
subset $N\subseteq M$
we define the \emph{annihilator} of $N$ as the subset 
\[
\Ann_R(N)=\{r\in R:r\cdot n=0\text{ for all }n\in N\}.
\]

\begin{example}
	$\Ann_{\Z}(\Z/n)=n\Z$.
\end{example}

\begin{exercise}
    Let $R$ be a ring and $M$ be a module. If $N\subseteq M$ is a subset, then 
	$\Ann_R(N)$ is a left ideal of $R$. If $N\subseteq M$ is a submodule of $R$, then 
	$\Ann_R(N)$ is an ideal of $R$. 
\end{exercise}

% \begin{lemma}
% 	\label{lemma:Ann}
% 	Let $R$ be a ring and $M$ be a module. If $N\subseteq M$ is a subset, then 
% 	$\Ann_R(N)$ is a left ideal of $R$. If $N\subseteq M$ is a submodule of $R$, then 
% 	$\Ann_R(N)$ is an ideal of $R$. 
% \end{lemma}

% \begin{proof}
% 	We left as an exercise to prove that $\Ann_R(N)$ is an additive subgroup of $R$. Then $\Ann_R(N)$
% 	is a left ideal, as $R\Ann_R(N)\subseteq\Ann_R(N)$. Indeed, if $r\in R$,
% 	$s\in\Ann_R(N)$ and $n\in N$, then $(rs)n=r(sn)=r0=0$. 
	
% 	If $N$ is a submodule, $\Ann_R(N)R\subseteq\Ann_R(N)$ since if 
% 	$s\in\Ann_R(N)$, $r\in R$ and $n\in N$, $rn\in\Ann_R(N)$, then
% 	$(sr)n=s(rn)=0$.
% \end{proof}

\begin{definition}
\index{Module!faithful}
A module $M$ is said to be \emph{faithful} if $\Ann_R(M)=\{0\}$. 
\end{definition}

\begin{example}
	If $K$ is a field, then $K^n$ is a faithful unitary $M_n(K)$-module.
\end{example}

\begin{example}
	If $V$ is vector space over a field $K$, then $V$ is faithful unitary $\End_K(V)$-module.
\end{example}

\begin{definition}
\index{Ring!primitive}
A ring $R$ is said to be \emph{primitive} if there exists a faithful simple $R$-module.  
\end{definition}

Since 
we are considering left modules, our definition of primitive rings is that of left primitive rings.
By convention, a primitive ring
will always mean a left primitive ring. 
The use 
of right modules yields to the notion of right primitive rings.  

\begin{exercise}
	\label{xca:simple=>prim}
	If $R$ is a simple unitary ring, then $R$ is primitive. 
\end{exercise}

\begin{exercise}
	\label{xca:prim+conm=cuerpo}
	If $R$ is a commutative ring (maybe without identity), then $R$ is primitive if and only if $R$ is a field. 
\end{exercise}

\begin{example}
	The ring $\Z$ is not primitive. 
\end{example}

\begin{definition}
\index{Ideal!primitive}
An ideal $P$ of a ring $R$ is said to be \emph{primitive} if $P=\Ann_R(M)$
for some simple $R$-module $M$. 
\end{definition}

\begin{lemma}
	\label{lemma:primitivo}
	Let $R$ be a ring and $P$ be an ideal of $R$. Then $P$ is primitive if and only if 
	$R/P$ is a primitive ring.
\end{lemma}

\begin{proof}
	Assume that $P=\Ann_R(M)$ for some $R$-module $M$. Then $M$ is a simple 
	$(R/P)$-module with 
	\[
	(r+P)\cdot m=r\cdot m,\quad r\in R,\;m\in M. 
	\]
	This operation 
	is well-defined, as 
	$P=\Ann_R(M)$. Since $M$ is a simple $R$-module, it follows that $M$ is 
	a simple $(R/P)$-module. Moreover, $\Ann_{R/P}M=\{0\}$. Indeed, if 
	$(r+P)\cdot M=\{0\}$, then $r\in\Ann_RM=P$ and hence $r+P=P$.

	Assume now that $R/P$ is primitive. Let $M$ be a faithful simple $(R/P)$-module. 
	Then 
    \[
    r\cdot m=(r+P)\cdot m, \quad r\in R,\; m\in M,
    \]
    turns $M$ into an $R$-module. It follows that $M$ is simple and that $P=\Ann_R(M)$. 
\end{proof}

%\begin{example}
%	Si $I$ es un ideal maximal de un anillo unitario $R$, entonces $I$ es
%	primitivo. Como $I$ es ideal maximal y regular (pues $1\in R$), el cociente
%	$R/I$ es un anillo unitario simple y luego $R/I$ es primitivo por la
%	proposición~\ref{proposition:simple=>prim}. 
%\end{example}

%\begin{example}
%	Si $I$ es un ideal primitivo de un anillo conmutativo $R$, entonces $I$ es
%	maximal pues $R/I$ es un cuerpo (por ser primitivo y conmutativo), ver
%	proposición~\ref{proposition:prim+conm=cuerpo}.
%\end{example}

\begin{example}
	Let $R_1,\dots,R_n$ be primitive rings and $R=R_1\times\cdots\times
	R_n$. Then each 
    \[
    P_i=R_1\times\cdots\times R_{i-1}\times\{0\}\times
	R_{i+1}\times\cdots\times R_n
    \]
    is a primitive ideal of $R$ since 
	$R/P_i\simeq R_i$.
\end{example}

%Recordemos que un ideal a izquierda $L$ de $R$ se dice \emph{minimal} si
%$L\ne0$ y $L$ no contiene propiamenete a otros ideales a izquierda no nulos de
%$R$.
%
%\begin{example}
%	Sea $L$ un ideal a izquierda de $R$ tal que $RL\ne0$. Entonoces $L$ es
%	simple si y sólo si $L$ es minimal.
%\end{example}

\begin{lemma}
	\label{lemma:maxprim}
	Let $R$ be a ring. If $P$ is a primitive ideal, there exists a regular 
        maximal left ideal $I$ such that $P=\{x\in R:xR\subseteq I\}$.
	Conversely, if $I$ is a regular maximal left ideal, then 
	$\{x\in R:xR\subseteq I\}$ is a primitive ideal. 
\end{lemma}

\begin{proof}
	Assume that $P=\Ann_R(M)$ for some simple $R$-module $M$. By
	Proposition~\ref{proposition:R/I}, there exists a regular maximal 
	left ideal 
	$I$ such that $M\simeq R/I$. Then 
    \[
    P=\Ann_R(R/I)=\{x\in
	R:xR\subseteq I\}.
    \]

	Conversely, let $I$ be a regular maximal left ideal. By
	Proposition~\ref{proposition:R/I}, $R/I$ is a simple $R$-module. Then
	\[
	\Ann_R(R/I)=\{x\in R:xR\subseteq I\}
	\]
	is a primitive ideal.
\end{proof}

%\begin{remark}
%	Una consecuencia trivial del lema~\ref{lemma:maxprim} es la siguiente: en
%	un anillo unitario, todo ideal a izquierda maximal contiene un ideal
%	primitivo.
%\end{remark}

\begin{exercise}
\label{xca:maximal=>primitive}
    Maximal ideals of unitary rings are primitive.  
\end{exercise}

\begin{exercise}
\label{xca:primitive=>maximal}
	Prove that every primitive ideal of a commutative ring is maximal.
\end{exercise}

\begin{bonus}
\label{xca:M_n(R)primitive}
    Prove that $M_n(R)$ is primitive if and only if $R$ is primitive.
\end{bonus}

% Si $P$ es primitivo, entonces $R/P$ es un cuerpo(por ser primitivo y conmutativo) y luego $P$ es maximal

\section{Lecture: 24/02/2024}
\label{04}


\subsection{Jacobson's radical}
%Let us discuss the Jacobson radical and radical rings. 

\begin{definition}
\index{Jacobson radical}
Let $R$ be a ring. The \emph{Jacobson radical} $J(R)$
is the intersection of all the annihilators of simple left $R$-modules. If $R$ does not
have simple left $R$-modules, then $J(R)=R$. 
\end{definition}

From the definition, it follows
that $J(R)$ is an ideal. Moreover, 
	\[
		J(R)=\bigcap\{P:\text{$P$ left primitive ideal}\}.
	\]

	If $I$ is an ideal of $R$ and $n\in\Z_{>0}$, $I^n$ is the additive subgroup of $R$ 
generated by the set $\{y_1\dots y_n:y_j\in I\}$. 

\begin{definition}
\index{Ideal!nilpotent}
\index{Left ideal!nilpotent}
An ideal $I$ of $R$ is \emph{nilpotent} 
if $I^n=\{0\}$ for some $n\in\Z_{>0}$.
\end{definition}

Similarly, one defines right or left nilpotent ideals. 
Note that an ideal $I$ is nilpotent if and only if there exists $n\in\Z_{>0}$ such that 
$x_1x_2\cdots x_n=0$ for all $x_1,\dots,x_n\in I$.  

\begin{definition}
	An element $x$ of a ring is said to be \emph{nil} (or nilpotent) if $x^n=0$ for some $n\in\Z_{>0}$. 
\end{definition}

\begin{definition}
\index{Ideal!nil}
\index{Left ideal!nil}
An ideal $I$ of a ring is said to be \emph{nil} if every element of $I$ is nil. 
\end{definition}

Similarly, one defines right or left nil ideals. 
Note that every nilpotent ideal is nil, as $I^n=\{0\}$ implies $x^n=0$ for all 
$x\in I$.

\begin{example}
	Let \[
 R=\C[X_1,X_2,\dots]/(X_1,X_2^2,X_3^3,\dots).
    \]
    The ideal 
	\[ 
    I=(X_1,X_2,X_3,\dots)
    \]
    is nil in $R$, as it is generated by nilpotent element. However, it is not nilpotent. Indeed, if $I$ is nilpotent, then there exists $k\in\Z_{>0}$ such that 
	$I^k=\{0\}$ and hence $x_i^k=0$ for all $i$, a contradiction since 
	$x_{k+1}^k\ne0$. 	
\end{example}

\begin{proposition}
	\label{pro:nilJ}
	Let $R$ be a ring. Then every nil left ideal is contained in $J(R)$.
\end{proposition}

\begin{proof}
	Assume that there is a nil left ideal $I$ such that 
	$I\not\subseteq J(R)$. There exists a simple $R$-module $M$ such that 
	$n=x\cdot m\ne 0$ for some $x\in I$ and some $m\in M$. Since $M$ is simple,
	$R\cdot n=M$ and hence there exists $r\in R$ such that 
	\[
	(rx)\cdot m=r\cdot (x\cdot m)=r\cdot n=m\
	\]
	Thus $(rx)^k\cdot m=m$ 
    for all 
	$k\geq1$, a contradiction since $rx\in I$ is a nilpotent element. 
\end{proof}

Similarly, one proves that 
every nil right ideal is contained in the Jacobson radical. 


\begin{definition}
\index{Element!left quasi-regular}
\index{Element!quasi-regular}
Let $R$ be a ring. An element $a\in R$ is said to be 
\emph{left quasi-regular} if there exists $r\in R$ such that $r+a+ra=0$. Similarly, 
$a$ is said to be \emph{right quasi-regular} if there exists $r\in R$ such that $a+r+ar=0$. 
\end{definition}

% \begin{exercise}
% 	\label{exercise:circ}
Let $R$ be a ring. A direct calculation shows that
the \emph{Jacobson circle operation}
\[
R\times R\to R,
\quad
(r,s)\mapsto r\circ s=r+s+rs,
\]
is an associative operation with neutral element $0$.

\begin{example}
Let $R=\Z/3=\{0,1,2\}$ be the ring of integers modulo 3. 
The Jacobson circle 
operation of $R$ is shown in Table~\ref{tab:radical}.
\end{example}

	\begin{table}[bht]
  		\caption{The table of a radical ring over $\Z/3$.}
		\centering
		\begin{tabular}{c|ccc}
			$\circ$ & 0 & 1 & 2\tabularnewline
			\hline
			0 & 0 & 1 & 2\tabularnewline
			1 & 1 & 0 & 2\tabularnewline
			2 & 2 & 2 & 2\tabularnewline
		\end{tabular}
    \label{tab:radical}
 	\end{table}

%\end{exercise}

% \begin{exercise}
% 	Let $R=\Z/3=\{0,1,2\}$. Compute the multiplication table with respect to the circle 
%  	operation given by the previous exercise.  
%  	%is then 
% % 	\begin{table}[ht]
% % 		\centering
% % 		\begin{tabular}{c|ccc}
% % 			$\circ$ & 0 & 1 & 2\tabularnewline
% % 			\hline
% % 			0 & 0 & 1 & 2\tabularnewline
% % 			1 & 1 & 0 & 2\tabularnewline
% % 			2 & 2 & 2 & 2\tabularnewline
% % 		\end{tabular}
% % 		\caption{The multiplication table of the radical ring $\Z/3$.}
% % 	\end{table}
% \end{exercise}

If $R$ is unitary, an element $x\in R$ is left quasi-regular (resp. right quasi-regular)
if and only if $1+x$ is left invertible (resp. right invertible). In fact, 
if $r\in R$ is such that $r+x+rx=0$, then $(1+r)(1+x)=1+r+x+rx=1$.
Conversely, if there exists $y\in R$ such that $y(1+x)=1$, then  
\[
(y-1)\circ x=y-1+x+(y-1)x=0.
\]

\begin{example}
	If $x\in R$ is a nilpotent element, 
    then $y=\sum_{n\geq1}x^n\in R$ is left quasi-regular. 
	In fact, if there exists $N$ such that $x^N=0$, 
    then the sum defining $y$ is finite 
    and $y+(-x)+y(-x)=0$.  Is right quasi-regular?
%	En efecto, si existe $N$ tal que $x^N=0$, la suma que
%	define al elemento $y$ es finita y cumple que $y+(-x)+y(-x)=0$.  
\end{example}

\begin{definition}
A left ideal $I$ of $R$ is said to be 
\emph{left quasi-regular} (resp. right quasi-regular) if every element of $I$ is
left quasi-regular (resp. right quasi-regular). A left ideal 
is said to be \emph{quasi-regular} if is left and right quasi-regular. 
\end{definition}

Similarly 
one defines right quasi-regular ideals and quasi-regular ideals. 

\begin{lemma}
	\label{lemma:casiregular}
	Let $I$ be a left ideal of $R$. If $I$ is left quasi-regular, then 
	$I$ is quasi-regular.
\end{lemma}

\begin{proof}
	Let $x\in I$. Let us prove that $x$ is right quasi-regular. Since $I$ is
	left quasi-regular, there exists $r\in R$ such that $r\circ x=r+x+rx=0$. Since 
	$r=-x-rx\in I$, there exists $s\in R$ such that $s\circ
	r=s+r+sr=0$. Then $s$ is right quasi-regular and  
	\[
	x=0\circ x=(s\circ r)\circ x=s\circ (r\circ x)=s\circ 0=s.\qedhere
	\]
\end{proof}

% \index{Lemma!Zorn}
% Let $(A,\leq)$ be a \emph{partially order set}, this means that $A$ is a set together with a 
% reflexive, transitive, and anti-symmetric binary relation
% $R$ en $A\times A$, where $a\leq b$ if and only if $(a,b)\in R$. 
% Recall that the relation is reflexive if $a\leq a$ for all $a\in A$, the relation is transitive if 
% $a\leq b$ and $b\leq c$ imply that 
% $a\leq c$ and the relation is anti-symmetric if $a\leq b$ and $b\leq a$ imply $a=b$.
% The elements $a,b\in A$ are said to be \emph{comparable} if $a\leq b$ or $b\leq
% a$. An element $a\in A$ is said to be \emph{maximal} if 
% $c\leq a$ 
% for all $c\in A$
% that is comparable with $a$. 
% An \emph{upper bound} for a non-empty subset $B\subseteq A$ is an element $d\in
% A$ such that $b\leq d$ for all $b\in B$. A \emph{chain} in $A$ is a subset 
% $B$ such that every pair of elements of $B$ are comparable. 
% \emph{Zorn's lemma} states the following property: 
% \begin{quote}
% If $A$ is a non-empty partially ordered set such that every chain in 
% $A$ contains an upper bound in $A$, then $A$ contains a maximal element. 
% \end{quote}

% Our application of Zorn's lemma:
The following result uses Zorn's lemma. 

\begin{lemma}
	\label{lemma:maxreg}
	Let $R$ be a ring, and $x\in R$ be an element that is not left quasi-regular Then there
	exists a maximal left ideal $M$ such that 
	$x\not\in M$. Moreover, $R/M$ is a simple $R$-module and 
	$x\not\in\Ann_R(R/M)$.
\end{lemma}

\begin{proof}
	Let $T=\{r+rx:r\in R\}$. A straightforward calculation shows that $T$ is a left ideal of 
	$R$ such that $x\not\in T$ (if $x\in T$, then $r+rx=-x$ for some 
	$r\in R$, a contradiction since $x$ is not left quasi-regular). 

	The only left ideal of $R$ containing 
	$T\cup\{x\}$ is $R$. Indeed, if there exists a left ideal $U$ containing $T$, then 
    $x\not\in U$, since otherwise every $r\in R$ could be written as 
	\[
    r=(r+rx)+r(-x)\in U.
    \]

	Let $\mathcal{S}$ be the set of proper left ideals of $R$ containing 
	$T$ partially ordered by inclusion. If $\{K_i:i\in I\}$ is a chain in 
	$\mathcal{S}$, then $K=\cup_{i\in I}K_i$ is an upper bound for the chain 
	($K$ is a proper, as $x\not\in K$). Zorn's lemma implies that 
	$\mathcal{S}$ admits a maximal element $M$. Thus $M$
	is a maximal left ideal such that $x\not\in M$.
 
        Moreover, $M$ is regular
	since $r-r(-x)\in T\subseteq M$ for all $r\in R$. Therefore $R/M$ is a simple 
	$R$-module by Proposition~\ref{proposition:R/I}. Since $x\cdot (x+M)\ne
	\{0\}$ (if $x^2\in M$, then  $x\in M$, as $x+x^2\in
	T\subseteq M$), it follows that $x\not\in\Ann_R(R/M)$.
\end{proof}

If $x\in R$ is not left quasi-regular, the lemma implies that there exists 
a simple $R$-module $M$ such $x\not\in\Ann_R(M)$. Thus 
$x\not\in J(R)$.

\begin{theorem}
	\label{thm:casireg_eq}
	Let $R$ be a ring and $x\in R$. The following statements are equivalent: 
	\begin{enumerate}
		\item The left ideal generated by $x$ is quasi-regular.
		\item $Rx$ is quasi-regular.
		\item $x\in J(R)$.
	\end{enumerate}
\end{theorem}

\begin{proof}
	The implication $(1)\implies(2)$ is trivial, as $Rx$ is included in the left ideal 
	generated by $x$.  
	
	We now prove $(2)\implies(3)$. If
	$x\not\in J(R)$, by definition, there exists a simple 
	$R$-module $M$ such that $x\cdot m\ne 0$ for some $m\in M$. The simplicity of $M$ implies
	that $(Rx)\cdot m=M$. Thus there exists $r\in R$ such that $(rx)\cdot m=-m$. There is an element 
	$s\in R$ such that $s+rx+s(rx)=0$ and hence 
	\[
	-m=(rx)\cdot m=(-s-srx)\cdot m=-s\cdot m+s\cdot m=0,
	\]
	a contradiction. 
	
	Finally, to prove $(3)\implies(1)$, it is enough to note that 
	$x$ is left quasi-regular. If $x\in J(R)$, 
	then $x$ is left quasi-regular by 
	the previous lemma. 
	Thus the left ideal generated by 
	$x$ is quasi-regular by Lemma~\ref{lemma:casiregular}.
\end{proof}

The theorem immediately implies the following corollary. 

\begin{corollary}
	If $R$ is a ring, then $J(R)$ is a quasi-regular ideal that contains every 
	quasi-regular left ideal. 
\end{corollary}

The following result is somewhat what we all had in mind. We first need a lemma. 

\begin{lemma}
    \label{lemma:JsupsetCR}
    Let $R$ be such that $J(R)\ne R$. If $I$ is a left quasi-regular left 
    ideal of $R$, then $I\subseteq J(R)$.
\end{lemma}

\begin{proof}
    Assume that $I\not\subseteq J(R)$. There exists a simple $R$-module 
    $N$ such that $I\cdot N\ne \{0\}$. In particular, $I\cdot n\ne \{0\}$ for some $0\ne n\in N$. Since 
    $I$ is a left ideal, $I\cdot n$ is a non-zero submodule of $N$. 
    Then $I\cdot n=N$, as $N$ is simple. There exists $x\in I$ such that $x\cdot n=-n$. 
    Since $I$ is left quasi-regular, there exists $r\in R$ such that $r+x+rx=0$.
    Thus 
    \[
	0=0\cdot n=(r+x+rx)\cdot n=r\cdot n+x\cdot n+(rx)\cdot n=r\cdot n-n-r\cdot n=-n,
    \]
    a contradiction. 
\end{proof}


% \begin{proof}
% 	Es consecuencia inmediata del teorema~\ref{thm:casireg_eq}. 
%%	Como \[
%%	J(R)=\bigcap\{\Ann_R(R/I):I\text{ ideal a izquierda maximal y
%%	regular}\}
%%	\]
%%	por el teorema~\ref{thm:J(R)}, 
%%	todo $x\in J(R)$ es casi-regular gracias al
%%	lema~\ref{lemma:K}. Si $I$ es un ideal casi-regular a izquierda, entonces
%%	$I\subseteq J(R)$ por el lema~\ref{lemma:JsupsetCR}.
%\end{proof}

%\begin{lemma}
%%	\label{lemma:maxreg}
%	Sea $R$ un anillo y sea $I\ne R$ un ideal a izquierda regular. Entonces $I$
%	está contenido en algún ideal a izquierda maximal y regular.
%\end{lemma}
%
%\begin{proof}
%	
%\end{proof}

%\begin{lemma}
%	\label{lemma:K}
%	Sean $R$ un anillo y $K=\bigcap\{I:\text{$I$ ideal a izquierda maximal y
%	regular}\}$.  Entonces $K$ es un ideal a izquierda casi-regular.
%\end{lemma}
%
%\begin{proof}
%	Gracias al lema~\ref{lemma:casiregular}, basta ver que $K$ es casi-regular
%	a izquierda.  Sea $a\in K$ y sea $T=\{r+ra:r\in R\}$.  Es claro que $T$ es
%	un ideal a izquierda regular con $e=-a$. Como $a\in K$, existe $s\in R$ tal
%	que $s+a+sa=0$. Entonces, como $-a=s+sa$, 
%	\[
%	r+re=r+r(-a)=r+r(s+sa)=r+rs+rsa\in r+T
%	\]
%	para todo $r\in R$. 
%	Si $T\ne R$, por el lema~\ref{lemma:maxreg} existe un ideal a izquierda
%	$J$, maximal y regular.  Como $a\in K\subseteq J$, $J$ es ideal a izquierda
%	y $r+ra\in T\subseteq J$ para todo $r\in R$, se concluye que $R=J$, una
%	contradicción.  Luego $T=R$ y entonces existe $r\in R$ tal que $r+ra=-a$.
%	Esto implica que $a$ es casi-regular a izquierda. 
%\end{proof}

%\begin{lemma}
%	\label{lemma:JsupsetCR}
%	Sea $R$ un anillo que admite un $R$-módulo simple. Si $I$ es un ideal a
%	izquierda casi-regular a izquierda, $I\subseteq J(R)$.
%\end{lemma}
%
%\begin{proof}
%	Supongamos que $I\not\subseteq J(R)$. Existe entonces un $R$-módulo simple
%	$N$ tal que $IN\ne 0$. Entonces $In\ne 0$ para algún $0\ne n\in N$. Como
%	$I$ es un ideal a izquierda, $In\subseteq N$ es un submódulo no nulo del
%	simple $N$. Luego $In=N$. Existe entonces $x\in I$ tal que $xn=-n$. Como
%	$I$ es casi-regular a izquierda, existe $r\in R$ tal que $r+x+rx=0$.
%	Entonces
%	\[
%		0=0n=(r+x+rx)n=rn+xn+rxn=rn-n-rn=-n,
%	\]
%	una contradicción.
%\end{proof}

The following exercise uses Zorn's lemma 
and will be used in the proof of Theorem~\ref{thm:J(R)}. 

\begin{exercise}
\label{xca:regular}
    Let $R$ be a ring. 
    Prove that every proper left ideal of $R$ that 
    is regular is contained in a maximal ideal that is regular. 
\end{exercise}

\begin{theorem}
	\label{thm:J(R)}
	Let $R$ be a ring such that $J(R)\ne R$. Then 
	\begin{align*}
		J(R)&=\bigcap\{I:\text{$I$ regular maximal left ideal of $R$}\}.
	\end{align*}
\end{theorem}

\begin{proof}
        Let 
	\[
	K=\bigcap\{I:\text{$I$ regular maximal left ideal of $R$}\}.
	\]
        
        Let us prove that $K\subseteq J(R)$. By Lemma \ref{lemma:JsupsetCR}, 
        it is enough to prove that $K$ is left quasi-regular. 
        Let $a\in K$ and $T=\{r+ra:r\in R\}$. 
        If $T=R$, then $-a=r+ra$ for some $r\in R$ and hence 
        $a$ is left quasi-regular. So we need to prove that $T=R$. Note that 
        $T$ is a regular left ideal with $e=-a$ (see Definition~\ref{def:regular}). If $T\ne R$, then $T$ is contained in a maximal left ideal $J$  
        by the previous exercise. Then 
        $a\in K\subseteq J$ and hence $ra\in J$ for all $r\in R$. Since 
        $r+ra\in T\subseteq J$ for all $r\in R$, it follows that $J=R$, a contradiction. Therefore $T=R$. 
        
        Now we prove that $J(R)\subseteq K$. 
		By
	Proposition~\ref{proposition:R/I}, 
	\[
		J(R)=\bigcap\{\Ann_R(R/I):I\text{ regular maximal left ideal of $R$}\}.
	\]
	Let $I$ be a regular maximal left ideal. If $r\in J(R)\subseteq
	\Ann_R(R/I)$, then, since $I$ is regular, there exists $e\in R$ such that
	$r-re\in I$. Since 
	\[
	re+I=r(e+I)=\{0\},
	\]
	$re\in I$ and hence $r\in I$. Thus $J(R)\subseteq K$. 
\end{proof}

\begin{example}
	Each maximal ideals of $\Z$ is of the form $p\Z=\{pm:m\in\Z\}$ for some prime number $p$. 
	Thus $J(\Z)=\cap_p p\Z=\{0\}$.
\end{example}

%\begin{example}
%	Sea $D$ un anillo de división y sea $R=D[x_1,\dots,x_n]$. Si $f\in J(R)$
%	entonces\dots Luego $J(R)=0$. 
%\end{example}

We now review some basic results useful to compute radicals. 

\begin{proposition}
	Let $\{R_i:i\in I\}$ be a family of rings. Then 
	\[
	J\left(\prod_{i\in I}R_i\right)=\prod_{i\in I}J(R_i).
	\]
\end{proposition}

\begin{proof}
	Let $R=\prod_{i\in I}R_i$ and $x=(x_i)_{i\in I}\in R$.  The left ideal 
    $Rx$ is quasi-regular if and only if each left ideal $R_ix_i$
	is quasi-regular in $R_i$, as $x$ is quasi-regular in $R$ if and only if each 
	$x_i$ is quasi-regular in $R_i$. Thus $x\in J(R)$ if and only if $x_i\in
	J(R_i)$ for all $i\in I$.
\end{proof}

For the next result, we shall need a lemma.

\begin{lemma}
	\label{lemma:trickJ1}
	Let $R$ be a ring and $x\in R$. 
	If $-x^2$ is a left quasi-regular element, then so is $x$. 
\end{lemma}

\begin{proof}
	Let $r\in R$ be such that $r+(-x^2)+r(-x^2)=0$ and $s=r-x-rx$. Then
    $x$ is left quasi-regular, as 
    \begin{align*}
		s+x+sx&=(r-x-rx)+x+(r-x-rx)x\\
		&=r-x-rx+x+rx-x^2-rx^2=r-x^2-rx^2=0.\qedhere 
\end{align*}
\end{proof}

%\begin{lemma}
%	\label{lemma:trickJ2}
%	Sea $R$ un anillo. Entonces $x\in J(R)$ si y sólo si $Rx$ es un ideal a
%	izquierda casi-regular a izquierda.
%\end{lemma}
%
%\begin{proof}
%	Si $x\in J(R)$ entonces $Rx\subseteq J(R)$ y luego todo elemento de $Rx$ es
%	casi-regular a izquierda. Recíprocamente, si $Rx$ es casi-regular a
%	izquierda, $Rx+\Z x$ es un ideal a izquierda de $R$. Si $s=rx+nx\in Rx+\Z
%	x$, entonces $-s^2$ es casi-regular a izquierda (pues $-s^2\in Rx$). Por el
%	lema~\ref{lemma:trickJ1}, $s$ es casi-regular a izquierda; en particular,
%	$x$ es casi-regular a izquierda y luego $x\in J(R)$. 
%\end{proof}

\begin{proposition}
	\label{proposition:J(I)}
	If $I$ is an ideal of $R$, then $J(I)=I\cap J(R)$. 
\end{proposition}

\begin{proof}
	Note that $I\cap J(R)$ is an ideal of $I$. Let $x\in I\cap J(R)$ and $r\in R$. 
        Since $rx$ is left quasi-regular in $R$, there exists $s\in R$ such that $s+rx+srx=0$. 
	Since $s=-rx-srx\in I$, $rx$ is left quasi-regular 
	in $I$. Thus $I\cap J(R)\subseteq J(I)$. 

	Let $x\in J(I)\subseteq I$ and $r\in R$. Since $-(rx)^2=(-rxr)x\in
	I(J(I))\subseteq J(I)$, the element $-(rx)^2$ is left quasi-regular 
	in $I$. Thus $rx$ is left quasi-regular by
	Lemma~\ref{lemma:trickJ1}.
\end{proof}


\chapter{}

\begin{definition}
\index{Ring!radical}
A ring $R$ is said to be \textbf{radical} if $J(R)=R$. 
\end{definition}

\begin{example}
	If $R$ is a ring, then $J(R)$ is a radical ring, by Proposition~\ref{proposition:J(I)}.
\end{example}

\begin{example}
	The Jacobson radical of $\Z/8$ is $\{0,2,4,6\}$. 
\end{example}

There are several characterizations of radical rings. 

\begin{theorem}
	\label{theorem:anillo_radical}
	Let $R$ be a ring. The following statements are equivalent: 
	\begin{enumerate}
		\item $R$ is radical.
		\item $R$ admits no simple $R$-modules. 
		\item $R$ does not have regular maximal left ideals.
		\item $R$ does not have primitive left ideals.
		\item Every element of $R$ is quasi-regular. 
		\item $(R,\circ)$ is a group. 
	\end{enumerate}
\end{theorem}

\begin{exercise}
    Prove Theorem \ref{theorem:anillo_radical}. 
\end{exercise}
%\begin{proof}
%	The equivalence $(1)\Longleftrightarrow(5)$ follows from 
%	Theorem~\ref{thm:casireg_eq}. 
%    
%    The equivalence $(5)\Longleftrightarrow(6)$ is left as an exercise. 
%
%	Let us prove that $(1)\implies(2)$. Assume that there exists a simple $R$-module $N$. Since 
%	$R=J(R)\subseteq\Ann_R(N)$, $R=\Ann_R(N)$. 
%	Hence $R\cdot N=\{0\}$, a contradiction to the simplicity of $N$.
%	
%	To prove $(2)\implies(3)$ we note that for each regular and maximal left ideal 
%	$I$, the quotient $R/I$ is a simple $R$-module by
%	Proposition~\ref{proposition:R/I}. 
%	
%	To prove $(1)\implies(4)$ assume that there is a primitive left ideal 
%	$I=\Ann_R(M)$, where $M$ is some simple $R$-module. Since $R=J(R)\subseteq I$, it follows that  
%    $I=R$, a contradiction to the simplicity of $M$.
%
%	Finally we prove $(4)\implies(2)$. If $M$ is a simple $R$-module, then 
%	$\Ann_R(M)$ is a primitive left ideal.
%\end{proof}

\begin{example}
	Let 
	\[
	A=\left\{\frac{2x}{2y+1}:x,y\in\Z\right\}.
	\]
	Then $A$ is a radical ring, as the inverse of the element $\frac{2x}{2y+1}$
	with respect to the circle operation 
	$\circ$ is 
	\[
	\left(\frac{2x}{2y+1}\right)'=\frac{-2x}{2(x+y)+1}.
	\]
\end{example}

\topic{Commutative rings with no maximal ideals}

There are rings with no maximal ideals. 

\begin{exercise}
\label{xca:Q_no_maximals}
    Prove that the additive group of rational numbers is an abelian group
with no maximal subgroups.
\end{exercise}

One can turn the additive group $\Q$ of rational into a non-unitary 
ring by considering the zero multiplication $xy=0$ for all $x,y\in\Q$. This 
ring has no maximal ideals. 

% Let $R$ be a commutative ring with no non-zero proper ideals. 
% Prove that if $R$ is not a field, then there exists a prime number $p$ 
% such that $R=\Z/p$ and $xy=0$ for all $x,y\in R$. 

\begin{exercise}
\label{xca:R/I_field_or_zero}
    Let $R$ be a commutative ring 
    and $I$ be an ideal of $R$. Prove that
    $I$ is maximal if and only if $R/I$ is a field
    or a ring isomorphic to 
    $\Z/p$ with zero multiplication for some prime number $p$. 
\end{exercise}

\begin{exercise}
    \label{xca:J(R)_fields}
    Let $R$ be a commutative ring. Prove that
    $J(R)$ equals the intersection of maximal ideals 
    such that $R/M$ is a field. 
\end{exercise}

\index{Characteristic of a ring}
Recall that the \textbf{characteristic of a ring} is defined as the least positive
integer $n$ such that $nx=0$ for all $x$. If no such $n$ exists, 
then we say that the ring is of characteristic zero. 

\begin{exercise}
\label{xca:characteristic}
    % Let $R$ be a ring and $n=\lcm\{ |x|:x\in R\}$, where 
    % $|x|$ denotes the additive order of $x$. If $n<\infty$, then 
    % $R$ has characteristic $n$. If $n=\infty$, then $R$ has characteristic zero. 
    % is the least common multiple of 
    % the additive orders of the elements of $R$, then  
    Let $R$ be a ring such and $p$ be a prime number. 
    If $px=0$ for all $x\in R$, then $R$ has characteristic $p$. 
\end{exercise}

% \begin{sol}{xca:characteristic}
%     Let $n$ be the characteristic of $R$. 
%     For each $x\in R$, $x\ne 0$, $px=0$. Since the additive order $|x|$ of $x$
%     divides $p$, it follows that $|x|=p$. Since $nx=0$ for all $x$, 
%     $p$ divides $n$. The minimality of the characteristic implies that $p=n$. 
% Assume that $n<\infty$. 
% Let $X=\{n:nx=0\text{ for all $x\in R$}\}$. By definition, 
% \[ 
% m\in X
% \Longleftrightarrow |x|\text{ divides }m
% \Longleftrightarrow n\leq m
% \Longleftrightarrow n\in X.
% \]
%\end{sol}

% Let $n>0$ be the characteristic of $R$. 
% If $px=0$ for all $x$, then $n=p$. Since $px=0$ for all $x$, $p\geq n$. 
% Assume that $p>n$, then $p+t=n$ for some $n$. 
% Then $0=px=(n-t)x=nx-tx=-tx$ and hence $tx=0$ for all $x$, a contradiction. 
% If $p\nmid n$, then 
% $n=pm$. Then $px=$

We now characterize commutative rings with no maximal ideals.
The result appeared in~\cite{MR0424776}. 

\begin{theorem}[Henriksen]
\label{thm:Henriksen}
\index{Henriksen's theorem}
Let $R$ be a commutative ring. Then $R$ has
no maximal ideals if and only if 
$J(R)=R$ and $R^2+pR=R$ for all prime number $p$. 
\end{theorem}

\begin{proof}
    Assume first that $R$ has no maximal ideals. Then $J(R)=R$ by 
    Exercise~\ref{xca:J(R)_fields}. Let $p$ be a prime number
    such that $I=R^2+pR\ne R$. Then $I$ is a proper ideal of $R$. 
    Let $\pi\colon R\to R/I$ be the canonical map. Since $R^2\subseteq I$, 
    $0=\pi(xy)=\pi(x)\pi(y)$ for all $x,y\in R$. Thus $R/I$ has zero multiplication. 
    Moreover, by Exercise~\ref{xca:characteristic}, 
    $R/I$ has characteristic $p$, as $pR\subseteq I$.  
    %as $0=\pi(px)=p\pi(x)$ for all $x\in R$, 
    % since $pR\subseteq I$. 
    Thus $R/I$ is a vector space 
    over the field $\Z/p$.
    Let $\{x_\alpha:\alpha\in\Lambda\}$ be a basis
    of $R/I$. Every element $x\in R/I$ can be written uniquely
    as a finite sum of the form 
    $x=\sum \lambda_{\alpha}x_{\alpha}$ for scalars $\lambda_\alpha$. 
    Let $A$ be the ring with underlying additive group $\Z/p$ and zero multiplication. 
    For a fixed
    $\beta\in\Lambda$, the map 
    \[
    \gamma\colon R/I\to A,\quad 
    x=\sum \lambda_{\alpha}x_{\alpha}\mapsto \lambda_{\beta}
    \]
    is a ring homomorphism. 
    The composition $f=\gamma\pi\colon R\to R/I\to A$ is a ring homomorphism. By Exercise~\ref{xca:R/I_field_or_zero}, $\ker f$ is a maximal ideal, a contradiction. 

    Conversely, let $M$ be a maximal ideal of $R$. If $R/M$ is a field, 
    then $J(R)\subseteq M\ne R$, a contradiction. 
    By Exercise~\ref{xca:R/I_field_or_zero},  there exists a prime
    number $p$ such that $R/M\simeq\Z/p$ as abelian groups and 
    zero multiplication (i.e. $xy\in M$ for all $x,y\in R$). 
    Let us write $A$ to denote this ring and  
    $\pi\colon R\to R/M$ be the canonical map. Note that 
    $R^2\subseteq M$. Moreover, $pR\subseteq M$, as 
    $\pi(px)=p\pi(x)=0$ for all $x\in R$. Thus $R^2+pR\subseteq M\ne R$, a contradiction. 
\end{proof}

We now present a non-trivial concrete example of a ring 
with no maximal ideals. For that purpose, we will use
the field of fractions $\R(X)$ of the real polynomial ring $\R[X]$.  

\begin{exercise}
    \label{xca:example}
    Let $R$ be the set of rational real functions of the form 
    $f(X)/g(X)$, where $f(X),g(X)\in\R(X)$ and $g(0)\ne 0$.  Prove the following statements:
    \begin{enumerate}
        \item $R$ is an integral domain with a unique maximal ideal $M=XR$.
        \item $M$ has no maximal ideals. 
    \end{enumerate}  
\end{exercise}

%The construction of the previous example does not work 
%if the field of real numbers is replaced by a field of positive characteristic. 

\begin{definition}
\index{Ring!nil}
A ring $R$ is said to be \textbf{nil} if for every $x\in R$ there
exists $n=n(x)$ such that $x^n=0$. 
\end{definition}

\begin{exercise}
    Prove that a nil ring is a radical ring. 
\end{exercise}

\begin{exercise}
    Let $\R[\![X]\!]$ be the ring of power series with real coefficients. Prove that the ideal 
    $X\R[\![X]\!]$ consisting of power series with zero constant term is a radical ring
    that is not nil. 
\end{exercise}

\begin{theorem}
	\label{thm:J(R/J)=0}
	If $R$ is a ring, then $J(R/J(R))=\{0\}$.
\end{theorem}

\begin{proof}
	If $R$ is radical, the result is trivial. Suppose then that 
	$J(R)\ne R$. Let $M$ be a simple $R$-module. Then $M$ is 
	a simple module over $R/J(R)$ with 
	\[
		(x+J(R))\cdot m=x\cdot m,\quad
		x\in R,\,m\in M.
	\]
	If $x+J(R)\in J(R/J(R))$, then  $x\cdot M=(x+J(R))\cdot M=\{0\}$. Then $x\in J(R)$, as 
	$x$ annihilates any simple module over $R$.
\end{proof}

\begin{theorem}
	Let $R$ be a ring and $n\in\Z_{>0}$. Then $J(M_n(R))=M_n(J(R))$. 
\end{theorem}

\begin{proof}
	We first prove that $J(M_n(R))\subseteq M_n(J(R))$. 
	If $J(R)=R$, the theorem is clear. Let us assume that $J(R)\ne R$ and let  
	$J=J(R)$. 
	If $M$ is a simple $R$-module, then $M^n$ is a simple $M_n(R)$-module with the usual multiplication. 
	Let $x=(x_{ij})\in J(M_n(R))$ and $m_1,\dots,m_n\in M$. Then
	\[
		x\colvec{3}{m_1}{\vdots}{m_n}=0.
	\]
	In particular, $x_{ij}\in\Ann_R(M)$ for all $i,j\in\{1,\dots,n\}$. Hence 
	$x\in M_n(J)$. 

	We now prove that $M_n(J)\subseteq J(M_n(R))$. Let 
	\[
		J_1=\begin{pmatrix}
			J & 0 & \cdots & 0\\
			J & 0 & \cdots & 0\\
			\vdots & \vdots & \ddots & \vdots\\
			J & 0 & \cdots & 0
		\end{pmatrix}
		\quad\text{and}\quad
		x=\begin{pmatrix}
			x_1 & 0 & \cdots & 0\\
			x_2 & 0 & \cdots & 0\\
			\vdots & \vdots & \ddots & \vdots\\
			x_n & 0 & \cdots & 0
		\end{pmatrix}\in J_1.
	\]
	Since $x_1$ is quasi-regular, there exists $y_1\in R$ such that $x_1+y_1+x_1y_1=0$.
	If
	\[
		y=\begin{pmatrix}
			y_1 & 0 & \cdots & 0\\
			0 & 0 & \cdots & 0\\
			\vdots & \vdots & \ddots & \vdots\\
			0 & 0 & \cdots & 0
		\end{pmatrix}, 
	\]
	then $u=x+y+xy$ is lower triangular, as  
	\[
		u=\begin{pmatrix}
			0 & 0 & \cdots & 0\\
			x_2y_1 & 0 & \cdots & 0\\
			x_3y_1 & 0 & \cdots & 0\\
			\vdots & \vdots & \ddots & \vdots\\
			x_ny_1 & 0 & \cdots & 0
		\end{pmatrix}.
	\]
	Since  
	$u^n=0$, the element
	\[
	v=-u+u^2-u^3+\cdots+(-1)^{n-1} u^{n-1}
	\]
	is such that 
	$u+v+uv=0$. Thus $x$ is right quasi-regular, as  
	\begin{align*}
		x+(y+v+yv)+x(y+v+yv)&=0,
	\end{align*}
	and therefore $J_1$ is right quasi-regular. Similarly one proves that 
	each $J_i$ is right quasi-regular and hence $J_i\subseteq J(M_n(R))$ for all 
	$i\in\{1,\dots,n\}$. In conclusion, 
	\[
	J_1+\cdots+J_n\subseteq J(M_n(R))
	\]
	and therefore $M_n(J)\subseteq J(M_n(R))$.
\end{proof}

\begin{exercise}
	Let $R$ be a unitary ring. Then  
	\[
	J(R)=\bigcap\{M:\text{$M$ is a left maximal ideal}\}.
	\]
\end{exercise}

\begin{exercise}
\label{xca:Jcon1}
	Let $R$ be a unitary ring. The
	following statements are equivalent: 
	\begin{enumerate}
		\item $x\in J(R)$.
		\item $x\cdot M=\{0\}$ for all simple $R$-module $M$.
		\item $x\in P$ for all primitive left ideal $P$.
		\item $1+rx$ is invertible for all $r\in R$.
		\item $1+\sum_{i=1}^n r_ixs_i$ is invertible 
		    for all $n$ and all $r_i,s_i\in R$.
		\item $x$ belongs to every maximal ideal maximal. 
	\end{enumerate}
\end{exercise}

The following exercise is entirely optional. 
It somewhat shows a recent application of radical rings
to solutions of the celebrated Yang--Baxter equation. 

\begin{exercise}
A pair $(X,r)$ is a \textbf{solution} to the 
Yang--Baxter equation if $X$ is a set and
$r\colon X\times X\to X\times X$ is a bijective map such that  
\[
	(r\times\id)\circ (\id\times r)\circ (r\times\id)
	=(\id\times r)\circ (r\times\id)\circ (\id\times r).
\]
The solution $(X,r)$ is said to be \textbf{involutive} 
if $r^2=\id$. By convention, we write 
\[
	r(x,y)=(\sigma_x(y),\tau_y(x)).
\]
The solution $(X,r)$ is said to be \textbf{non-degenerate}  
$\sigma_x\colon X\to X$ and 
$\tau_x\colon X\to X$ are bijective for all $x\in X$.

\begin{enumerate}
    \item Let $X$ be a set and $\sigma\colon X\to X$ be a bijective map. Prove that  
          the pair $(X,r)$, where 
          $r(x,y)=(\sigma(y),\sigma^{-1}(x))$, is an involutive non-degenerate solution. 
\end{enumerate}
Let $R$ be a radical ring. For $x,y\in R$ let 
\begin{align*}
	&\lambda_x(y)=-x+x\circ y=xy+y,\\
	&\mu_y(x)=\lambda_x(y)'\circ x\circ y=(xy+y)'x+x
\end{align*}
Prove the following statements:
\begin{enumerate}
    \setcounter{enumi}{1}
		\item $\lambda\colon (R,\circ)\to\Aut(R,+)$, $x\mapsto
			\lambda_x$, is a group homomorphism.
		\item $\mu\colon (R,\circ)\to\Aut(R,+)$, $y\mapsto\mu_y$,
    		is a group antihomomorphism.
	    \item The map 
    	\[
	        r\colon R\times R\to R\times R,\quad
	        r(x,y)=(\lambda_x(y),\mu_y(x)),
	    \]
	is an involutive non-degenerate solution to the Yang--Baxter equation. 
\end{enumerate}
\end{exercise}

%\begin{exercise}
%	Sea $A$ un anillo radical. Para $a,b\in A$ se define 
%	\[
%		\mu_b(a)=\lambda_a(b)'\circ a\circ b=(ab+b)'a+a.
%	\]
%	Demuestre que la función $\mu\colon (A,\circ)\to\Aut(A,+)$,
%	$b\mapsto\mu_b$, está bien definida y es un antimorfismo de grupos.
%\end{exercise}

\begin{exercise}
    If $D$ is a division ring and $R=D[X_1,\dots,X_n]$, then
    $J(R)=\{0\}$. 
%     Como las unidades de $R$ son los elementos no nulos de $D$,
% 	$J(R)$ es un ideal de $D$. Como $D$ es simple, $J(R)\in\{0,D\}$. Si
% 	$J(R)=D$, entonces existe  $f\in R$ tal que $-1+f+(-1)f=0$ y luego $-1=0$,
% 	una contradicción. Luego $J(R)=0$.
\end{exercise}


\begin{example}
\index{Ring!local}
    A commutative and unitary ring $R$ is \textbf{local} if it contains
    only one maximal ideal. 
	If $R$ is a local ring and $M$ is its maximal ideal, then $J(R)=M$. Some particular cases: 
	\begin{enumerate}
		\item If $K$ is a field and $R=K[\![X]\!]$, then $J(R)=(X)$. 
		\item If $p$ is a prime number and $R=\Z/p^n$, then $J(R)=(p)$. 
	\end{enumerate}
\end{example}

We finish the discussion on the Jacobson radical with 
some results in the case of unitary algebras. We first need an application of Zorn's lemma. 

\begin{exercise}
\label{xca:maximal_regular}
    Let $I$ be a proper left ideal that is left regular. Prove that $I$ is contained in a maximal left ideal 
    which is regular. 
\end{exercise}

% explain ideals of algebras without one. 

\begin{proposition}
	Let $A$ be a $K$-algebra and $I$ be a subset of $A$. Then $I$ is 
	a regular maximal left ideal of the algebra $A$ if and only if $I$ is 
	a regular maximal left ideal of the ring $A$.
\end{proposition}

\begin{proof}
	Let $I$ be a left regular maximal ideal of the ring $A$. We claim that
	$\lambda I\subseteq I$ for all $\lambda\in K$. Assume that 
	$\lambda I\not\subseteq I$ for some $\lambda$. Then $I+\lambda I$
	is an ideal of the ring $A$ that contains $I$, as 
	\[
	a(I+\lambda I)=aI+a(\lambda I)\subseteq I+\lambda (aI)\subseteq I+\lambda I.
	\]
	Since $I$ is maximal, it follows that $I+\lambda I=A$. 
	The left regularity of $I$ implies that there exists $e\in A$
	such that 
	$a-ae\in I$ for all $a\in A$. Write $e=x+\lambda y$ for $x,y\in
	I$. Then 
	\[
		e^2=e(x+\lambda y)=ex+e(\lambda y)=ex+(\lambda e)y\in I.
	\]
	Since $e-e^2\in I$ and $e^2\in I$, it follows that $e\in I$. Thus $A=I$, as
	$a-ae\in I$ for all $a\in A$, a contradiction.

	Conversely, if $I$ is a left regular maximal ideal of the algebra $A$, then 
	$I$ is a left regular ideal of the ring $A$. We claim that $I$ is a maximal left ideal of the ring of $A$. 
	There exists a regular maximal left ideal $M$ 
	of the ring $A$ that contains $I$. Since 
	$M$ is regular, it follows that $M$ is a regular maximal ideal of the algebra $A$. Thus 
	$M=I$ because $I$ is a maximal left ideal of the algebra $A$. 
% 	ejercicio~\ref{xca:Zorn:regular} sabemos que existe un ideal a izquierda
% 	maximal $L$ del anillo $A$ que contiene a $I$. Como $L$ es regular, la
% 	implicación demostrada nos dice que $L$ es un ideal a izquierda maximal y
% 	regular del anillo $A$. Luego $L=I$ por la maximalidad de $I$.
\end{proof}

\begin{exercise}
    Let $A$ be an algebra. Prove that the Jacobson
    radical of the ring $A$ coincides with the Jacobson radical of the algebra $A$. 
\end{exercise}

% \begin{proof}
% 	Es consecuencia del teorema anterior y de que el radical de Jacobson es la
% 	intersección de los ideales a izquierda maximales y regulares.
% \end{proof}

\topic{Amitsur's theorem}

We now prove an important result of Amitsur that
has several interesting applications. We first need a lemma. 

\begin{lemma}
	\label{lemma:algebraico=nil}
	Let $A$ be an algebra with one and let $x\in J(A)$. 
	Then $x$ is algebraic if and only if $x$ is nilpotent. 
\end{lemma}

\begin{proof}
    Since $x$ is algebraic, there exist $a_0,\dots,a_n\in K$ 
    not all zero such that 
    \[
		a_0+a_1x+\cdots+a_nx^n=0.
	\]
	Let $r$ be the smallest integer such that $a_r\ne 0$. Then 
	\[
		x^r(1+b_1x+\cdots+b_mx^m)=0,
	\]
	for some $b_1,\dots,b_m\in K$. Since $1+b_1x+\cdots+b_mx^m$ is a unit by 
	Exercise~\ref{xca:Jcon1}, it follows that $x^r=0$.
\end{proof}

An application:

\begin{proposition}
	\label{pro:algebraica=>Jnil}
	If $A$ is an algebraic algebra with one, then $J(A)$ is the largest nil ideal of $A$.
\end{proposition}

\begin{proof}
	The previous lemma implies that $J(A)$ is a nil ideal. 
	Proposition~\ref{pro:nilJ} now implies that $J(A)$ is the largest nil ideal of $A$. 
\end{proof}

\begin{theorem}[Amitsur]
	\label{thm:Amitsur}
	\index{Amitsur's theorem}
	Let $A$ be a $K$-algebra with one such that $\dim_KA<|K|$ (as cardinals). Then 
	$J(A)$ is the largest nil ideal of $A$. 
\end{theorem}

\begin{proof}
	If $K$ is finite, then $A$ is a finite-dimensional algebra. In particular, $A$ is algebraic and
	hence $J(A)$ is a nil ideal by Proposition~\ref{pro:algebraica=>Jnil}.

	Assume that $K$ is infinite and let $a\in J(A)$. Exercise~\ref{xca:Jcon1} implies that 
	every element of the form 
	$1-\lambda^{-1}a$, $\lambda\in K\setminus\{0\}$, is invertible. Thus  
	\[
		a-\lambda=-\lambda(1-\lambda^{-1}a)
	\]
	is invertible for all $\lambda\in K\setminus\{0\}$. Let
	$S=\{(a-\lambda)^{-1}:\lambda\in K\setminus\{0\}\}$. Since 
	\[
	(a-\lambda)^{-1}=(a-\mu)^{-1}\Longleftrightarrow\lambda=\mu,
	\]
	it follows that $|S|=|K\setminus\{0\}|=|K|>\dim_KA$. Then $S$ 
	is linearly dependent, so there are $\beta_1,\dots,\beta_n\in K$
	not all zero and distinct elements $\lambda_1,\dots,\lambda_n\in K$ such that 
	\begin{equation}
		\label{eq:Amitsur}
		\sum_{i=1}^n \beta_i(a-\lambda_i)^{-1}=0.
	\end{equation}
	Multiplying~\eqref{eq:Amitsur} by $\prod_{i=1}^n(a-\lambda_i)$ we get 
	\[
		\sum_{i=1}^n\beta_i\prod_{j\ne i}(a-\lambda_j)=0.
	\]
	We claim that $a$ is algebraic over $K$. Indeed,  
	\[
		f(X)=\sum_{i=1}^n\beta_i\prod_{j\ne i}(X-\lambda_j)
	\]
	is non-zero, as, for example, if $\beta_1\ne1$, then  
	$f(\lambda_1)=\beta_1(\lambda_1-\lambda_2)\cdots(\lambda_1-\lambda_n)\ne0$
	and $f(a)=0$. Since $a\in J(A)$ is algebraic, it follows
	$a$ is nilpotent by Lemma~\ref{lemma:algebraico=nil}.
\end{proof}

Amitsur's theorem implies the following result. 

\begin{corollary}
Let $K$ be a non-countable field. If $A$ is an algebra
over $K$ with a countable basis, then 
$J(A)$ is the largest nil ideal of $A$.
\end{corollary}

% \begin{proof}
% 	Es consecuencia del teorema de Amitsur pues $\dim_KA<|K|$. 
% \end{proof}

%We now finish the lecture with some big open problems. 


\topic{Jacobson's conjecture}

We now conclude the lecture
with two big open problems related to the Jacobson radical. The first
one is Jacobson's conjecture. 

\begin{openproblem}[Jacobson]
\label{prob:Jacobson}
\index{Jacobson conjecture}
\index{Jacobson--Herstein conjecture}
Let $R$ be a noetherian ring. Is then 
\[
\bigcap_{n\geq1}J(R)^n=\{0\}?
\]
\end{openproblem}

Open problem \ref{prob:Jacobson} was originally formulated by Jacobson in 1956 \cite{MR0222106} 
for one-sided noetherian rings. In 1965 Herstein \cite{MR188253} found a counterexample
in the case of one-sided noetherian rings 
and reformulated the conjecture as it appears here. 

\begin{exercise}[Herstein]
Let $D$ be the ring of rationals with odd denominators. Let
$R=\begin{pmatrix}
    D & \Q\\
    0 & \Q
\end{pmatrix}$. Prove that $R$ is right noetherian and 
$J(R)=\begin{pmatrix}
J(D) & \Q\\
0 & 0
\end{pmatrix}$. Prove that 
$J(R)^n\supseteq\begin{pmatrix}0&\Q\\0&0\end{pmatrix}$ and hence $\bigcap_nJ(R)^n$ is non-zero. 
\end{exercise}

\topic{K\"othe's conjecture}

The following problem is maybe the most important open 
problem in non-commutative ring theory. 

\begin{openproblem}[K\"othe]
\label{prob:Koethe}
\index{K\"othe conjecture}
Let $R$ be a ring. Is the sum 
of two arbitrary nil left ideals of $R$ is nil?
\end{openproblem}

Open problem~\ref{prob:Koethe} is the well-known K\"othe's conjecture. 
The conjecture was first formulated in 1930, see \cite{MR1545158}. It is known to be true
in several cases. In full generality, the problem is still open. In~\cite{MR306251} 
Krempa proved that
the following statements are equivalent:
\begin{enumerate}
    \item K\"othe's conjecture is true.  
    \item If $R$ is a nil ring, then $R[X]$ is a radical ring. 
    \item If $R$ is a nil ring, then $M_2(R)$ is a nil ring. 
    \item Let $n\geq2$. If $R$ is a nil ring, then $M_n(R)$ is a nil ring. 
\end{enumerate}

In 1956 Amitsur formulated the following conjecture, see for example
\cite{MR0347873}: If $R$ is a nil ring, then $R[X]$ is a nil ring. In~\cite{MR1793911} 
Smoktunowicz found a counterexample to Amitsur's conjecture. 
This counterexample suggests that K\"othe's conjecture might be false. 
A simplification of Smoktunowicz's example
appears in~\cite{MR3169522}. See \cite{MR1879880,MR2275597} for more
information on K\"othe's conjecture and related topics. 


\section{31/10/2024}



\subsection{Gilmer's theorem}

Hilbert's theorem states that 
if $R$ is a noetherian 
commutative unitary ring, then
$R[X]$ is noetherian. Following \cite{MR212007}, 
we now present the converse of Hilbert's theorem. 

\begin{theorem}[Gilmer]
\index{Gilmer's theorem}
    Let $R$ be a commutative ring. If $R[X]$ is noetherian, 
    then $R$ is unitary. 
\end{theorem}

\begin{proof}
    Let $a\in R$. For $m\geq0$, let
    \begin{align*}
    I_m&=(a,aX,aX^2,\dots,aX^m)\\
    &=R[X]a+R[X]aX+\cdots+R[X]aX^m+\Z a+\Z aX+\cdots+\Z aX^m.
    \end{align*}
    Then $I_0\subseteq I_1\subseteq\cdots I_m\subseteq I_{m+1}\subseteq\cdots$ 
    is a sequence of ideals of $R[X]$. Since $R[X]$ is noetherian,
    $I_n=I_{n+1}$ for some $n$. In particular, 
    $aX^{n+1}\in I_{n+1}=I_n$. Thus
    \[
    aX^{n+1}=\sum_{i=1}^{n+1} aX^{i-1}f_i(X)+\sum_{i=1}^{n+1} k_iaX^{i-1}
    \]
    for some $f_1(X),\dots,f_n(X)\in R[X]$ and 
    $k_1,\dots,k_n\in\Z$. Comparing the coefficient of $X^{n+1}$ 
    one gets that $a=ar$ for some $r\in R$. 
    Thus  
    \begin{equation}
        \label{eq:Gilmer}
        \text{for every $a\in R$ there exists $r\in R$ 
    such that $a=ra$.}
    \end{equation}
    
    \begin{claim}
        For every $a_1,\dots,a_n\in R$ there exists $r\in R$ 
        such that $a_i=ra_i$ for all $i$. 
    \end{claim}
    
    We proceed by induction on $n$. 
    The case $n=1$ is \eqref{eq:Gilmer}. Assume 
    that the result holds for $n-1\geq1$. By the inductive hypothesis, 
    there exists $r_1\in R$ such that $a_i=r_1a_i$ 
    for all $i\in\{1,\dots,n-1\}$. Moreover, 
    there exists $r_2\in R$ such that $a_n=ra_n$. 
    Let $r=r_1+r_2-r_1r_2$. Then
    \[
    ra_n=r_1a_n+r_2a_n-r_1r_2a_n=r_1a_n+a_n-r_1a_n=a_n.
    \]
    Moreover, for $i\in\{1,\dots,n-1\}$, 
    \[
    ra_i=r_1a_i+r_2a_i-r_1r_2a_i=a_i+r_2a_i-r_2r_1a_i=a_i+r_2a_i-r_2a_i=a_i.
    \]
    
    We now finish the proof of the theorem. 
    Let $R[X]\to R$, $f(X)\mapsto f(0)$, be an evaluation map. Since
    it is a surjective ring homomorphism, 
    $R$ is noetherian. In particular, $R$ is finitely generated, 
    say 
    \[
    R=(a_1,\dots,a_n)=Ra_1+\cdots+Ra_n+\Z a_1+\cdots+\Z a_n
    \]
    for some $a_1,\dots,a_n\in R$. 
    
    We now prove that the element $r$ from the claim we proved turns 
    $R$ into a unitary ring, that is $r=1_R$.  
    We need to show that $rb=b$ for all $b\in R$. 
    If $b\in R$, then 
    \[
    b=t_1a_1+\cdots+t_na_n+m_1a_1+\cdots+m_na_n
    \]
    for some $t_1,\dots,t_n\in R$ and $m_1,\dots,m_n\in\Z$. 
    Since $a_i=ra_i$ for all $i\in\{1,\dots,n\}$, it immediately
    follows that
    $rb=b$. 
\end{proof}

\begin{example}
    The polynomial ring $(2\Z)[X]$ is not 
    noetherian, as the ring $2\Z$ is not unitary. 
\end{example}

\subsection{Artinian modules}

\begin{definition}
\index{Module!artinian}
	Let $R$ be a ring. A module $N$ is \emph{artinian} if every decreasing sequence 
	$N_1\supseteq N_2\supseteq\cdots$ of submodules of $N$ stabilizes, that is
	there exists $n\in\Z_{>0}$ such that 
	$N_n=N_{n+k}$ for all $k\in\Z_{\geq0}$.
\end{definition}

\index{Minimal element}
Let $X$ be a set and $\mathcal{S}$ be a set of subsets of $X$. 
We say that $A\in\mathcal{S}$ is a \emph{minimal element} of $\mathcal{S}$
if there is no $Y\in\mathcal{S}$ such that $Y\subsetneq A$. 

\begin{proposition}
\label{pro:artinian_minimal}
	A module $N$ is artinian if and only if 
	every non-empty subset of submodules of $N$ 
	contains a minimal element. 
\end{proposition}

\begin{proof}
	Assume that $N$ is artinian. Let $\mathcal{S}$ be a non-empty set of submodules of $N$. 
	Suppose that $\mathcal{S}$ has no minimal element and let $N_1\in\mathcal{S}$. 
	Since $N_1$ is not minimal, there exists 
	$N_2\in\mathcal{S}$ such that $N_1\supsetneq N_2$. Now assume the 
	submodules 
	\[
	N_1\supsetneq N_2\supsetneq\cdots\supsetneq N_k
	\]
	we chosen. 
	Since $N_k$ is not minimal, there exists $N_{k+1}$ such that $N_k\supsetneq N_{k+1}$.
	This procedure produces a sequence $N_1\supsetneq
	N_2\supsetneq\cdots$ that cannot stabilize, a contradiction. 
	
	If $N_1\supseteq N_2\supseteq\cdots$ is a sequence of submodules, then 
	$\mathcal{S}=\{N_j:j\geq1\}$ has a minimal element, say $N_n$. Then
	$N_n=N_{n+k}$ for all $k$. 
\end{proof}

\index{Module!noetherian}
A module $N$ is \emph{noetherian} if for every sequence 
$N_1\subseteq N_2\subseteq\cdots$ of submodules of $N$ there exists $n\in\Z_{>0}$ such that 
$N_n=N_{n+k}$ for all $k\in\Z_{\geq0}$. 

% Let $X$ be a set and $\mathcal{S}$ be a set of subsets of $X$. We say that 
% $B\in\mathcal{S}$ is a \emph{maximal element} of $\mathcal{S}$ if
% there is no $Z\in\mathcal{S}$ such that $B\subsetneq Z$.

\begin{exercise}
    Let $M$ be a module. The following statements are equivalent:
    \begin{enumerate}
        \item $M$ is noetherian.
        \item Every submodule of $M$ is finitely generated. 
        \item Every non-empty subset $\mathcal{S}$ of submodules of $M$ contains a maximal element, that is
            an element $X\in\mathcal{S}$ such that there is no $Z\in\mathcal{S}$ such that $X\subsetneq Z$.  
    \end{enumerate}
\end{exercise}

\begin{exercise}
    Prove that a ring $R$ is left noetherian if every sequence of 
    left ideals $I_1\subseteq I_2\subseteq\cdots$ stabilizes. 
\end{exercise}

\begin{exercise}
\label{xca:AN_exact}
	Let 
	\[
	\begin{tikzcd}
		0 \arrow{r}
		& A \arrow{r}{f}
		& B \arrow{r}{g}
		& C \arrow{r}
		& 0
	\end{tikzcd}
	\]
	be an exact sequence of modules. Prove that $B$ is noetherian (resp.
	artinian) if and only if $A$ and $C$ are noetherian (resp. artinian).
\end{exercise}

% \begin{definition}
% 	Un anillo $R$ se dice \emph{noetheriano a izquierda} si el módulo 
% 	$\prescript{}{R}R$ es noetheriano.
% \end{definition}
%Similarly one defines right noetherian rings.

\begin{definition}
\index{Ring!left artinian}
	A ring $R$ is \emph{left artinian} if the module 
	$\prescript{}{R}R$ is artinian.
\end{definition}

Similarly one defines right artinian rings. 

\begin{example}
	The ring $\Z$ is noetherian. It is not artinian, as the sequence
	\[
	2\Z\supseteq
	4\Z\supseteq 8\Z\supseteq\cdots
	\]
	does not stabilize. 
\end{example}

\begin{exercise}
    Prove that a ring $R$ is left artinian if every sequence of 
    left ideals $I_1\supseteq I_2\supseteq\cdots$ stabilizes. 
\end{exercise}


\begin{definition}
\index{Module!composition series}
	\label{def:serie_de_composicion}
	A \emph{composition series} of the module $M$ is a sequence 
	\[
		\{0\}=M_0\subsetneq M_1\subsetneq M_2\subsetneq\cdots\subsetneq M_n=M
	\]
	of submodules of $M$ such that each $M_i/M_{i-1}$ is non-zero and has no non-zero 
	proper submodules. 
	In this case 
	$n$ is the length of the composition series.
\end{definition}

The previous definition makes sense also for non-unitary rings. That is why
it is required that each quotient $M_i/M_{i-1}$ has no proper submodules.

\begin{theorem}
	\label{thm:serie_de_composicion}
	A non-zero module admits a composition series if and only if it is artinian and noetherian.
\end{theorem}

\begin{proof}
	Let $M$ be a non-zero module and let $\{0\}=M_0\subsetneq
	M_1\subsetneq\cdots\subsetneq M_n=M$ be a composition series for $M$.
	We claim that each $M_i$ is artinian and noetherian. We proceed by induction on $i$. The case
	$i=0$ is trivial. Let us assume that $M_i$ is artinian and noetherian. Since 
	$M_i/M_{i+1}$ has no proper submodules and the sequence 
	\[
	\begin{tikzcd}
		0 \arrow{r}
		& M_i \arrow{r}
		& M_{i+1} \arrow{r}
		& M_{i+1}/M_i \arrow{r}
		& 0
	\end{tikzcd}
	\]
	is exact, it follows that 
	$M_{i+1}$ is artinian and noetherian, see Exercise \ref{xca:AN_exact}. 

    Conversely, let $M$ be a non-zero  artinian and noetherian module. Let $M_0=\{0\}$ and 
    $M_1$ be minimal among the non-zero  submodules of $M$ (it exists by Proposition \ref{pro:artinian_minimal}).
    If $M_1\ne M$, let 
	$M_2$ be minimal among those submodules of $M$ such that $M_1\subsetneq M_2$. This procedure
	produces a sequence 
	\[
		\{0\}=M_0\subsetneq M_1\subsetneq M_2\subsetneq\cdots
	\]
	of submodules of $M$, where each $M_{i+1}/M_i$ is non-zero and admits no
	proper submodules. Since $M$ is noetherian, the sequence stabilizes and
	hence it follows that $M_n=M$ for some $n$. 
\end{proof}

\begin{definition}
\index{Composition series!equivalence}
    Let $M$ be a module. 
	We say that the composition series
	\[
	M=V_0\supseteq V_1\supsetneq\cdots\supsetneq V_k=\{0\},
	\quad
	M=W_0\supsetneq W_1\supsetneq\cdots\supsetneq W_l=\{0\},
	\]
	are \emph{equivalent} if $k=l$ and there exists 
	$\sigma\in\Sym_k$ such that 
	$V_{i}/V_{i-1}\simeq W_{\sigma(i)}/W_{\sigma(i)-1}$
	for all $i\in\{1,\dots,k\}$.
\end{definition}

\begin{exercise}
\label{xca:Z6}
    Find all composition series
    for the $\Z$-module $\Z/6$. 
\end{exercise}

\begin{theorem}[Jordan--H\"older]
	\label{thm:JordanHolder}
	\index{Jordan--H\"older theorem}
	Any two composition series for a module are equivalent. 
\end{theorem}

\begin{proof}
    Let $M$ be a module and
    \[
		M=V_0\supsetneq V_1\supsetneq\cdots\supsetneq V_k=\{0\},
		\quad
		M=W_0\supsetneq W_1\supsetneq\cdots\supsetneq W_l=\{0\},
	\]
	be composition series of $M$. 
	We claim that these composition series are equivalent. 
	We proceed by induction on $k$. The case $k=1$ is trivial, as 
	in this case $M$ has no proper submodules and $M\supseteq\{0\}$ 
	is the only possible composition series for $M$. So
	assume the result holds for modules with composition series of length $<k$. If $V_1=W_1$, then 
	$V_1$ has composition series of lengths $k-1$ and $l-1$. The inductive hypothesis implies that 
	$k=l$ and we are done. So assume that $V_1\ne W_1$. Since $V_1$ and $W_1$ are submodules of $M$, the
	sum $V_1+W_1$ is also a submodule of $M$. Moreover, $M/V_1$ has no non-zero proper submodules
	and hence 
	$V_1+W_1=V$. Then 
	\[
		M/V_1=\frac{V_1+W_1}{V_1}\simeq\frac{V_1}{V_1\cap W_1}.
	\]
	Since $V_1$ has a composition series, $V_1$ is artinian and
	noetherian by Theorem~\ref{thm:serie_de_composicion}. The submodule $U=V_1\cap W_1$ is also 
	artinian and noetherian and hence, by Theorem \ref{thm:serie_de_composicion}, 
	admits a composition series 
	\[
		U=U_0\supsetneq U_1\supsetneq\cdots\supsetneq U_r=\{0\}.
	\]
    Thus
    $V_1\supsetneq\cdots\supsetneq V_k=\{0\}$ and  
	$V_1\supseteq U\supsetneq U_1\supsetneq\cdots\supsetneq U_r=\{0\}$ are both composition 
	series for $V_1$. The inductive hypothesis implies that 
	$k-1=r+1$ and that these composition series are equivalent. Similarly, 
	\[
		W_1\supsetneq W_2\supsetneq\cdots\supsetneq W_l=\{0\},
		\quad
		W_1\supsetneq U\supsetneq U_1\supsetneq\cdots\supsetneq U_{r}=\{0\},
	\]
    are both composition series for $W_1$ and hence $l-1=r+1$ and these composition 
    series are equivalent. Therefore $l=k$ and the proof is completed. 
\end{proof}

Jordan--H\"older theorem allows us to define the 
length of modules that admit a composition series. 

\begin{definition}
\index{Module!length}
    Let $M$ be a module with a composition series. 
    The \emph{length} $\ell(M)$ of $M$ is defined as the length of any composition series of $M$. 
\end{definition}

\index{Module!of finite length}
A module is said to be of 
finite length if it admits a composition series. 

\begin{exercise}
	If $N$ and $Q$ are modules with composition series and  
	\[
	\begin{tikzcd}
		0 \arrow[r]
		& N \arrow{r}{f}
		& M \arrow{r}{g}
		& Q \arrow[r]
		& 0
	\end{tikzcd}
	\]
	is an exact sequence of modules, then $\ell(M)=\ell(N)+\ell(Q)$.
\end{exercise}

%\begin{proof}
%	Sean $Q=Q_0\supsetneq Q_1\supsetneq\cdots\supsetneq Q_m=0$ y
%	$N=N_0\supsetneq N_1\supseteq\cdots\supsetneq N_n=0$ series de composición
%	para $Q$ y $N$ respectivamente. Entonces
%	\[
%		M=g^{-1}(Q_0)\supsetneq g^{-1}(Q_1)\supsetneq\cdots\supsetneq g^{-1}(Q_m)=f(N_0)\supsetneq f(N_1)\supsetneq\cdots\supsetneq f(N_n)=0
%	\]
%	es una serie de composición para $M$ y luego $c(M)=c(N)+c(Q)$.
%\end{proof}

\begin{exercise}
	If $A$ and $B$ are finite-length submodules of $M$, then  
	\[
	\ell(A+B)+\ell(A\cap B)=\ell(A)+\ell(B).
	\]
\end{exercise}

\begin{theorem}
	\label{thm:Jnilpotente}
	If $R$ is a left artinian ring, then $J(R)$ is nilpotent. 
\end{theorem}

\begin{proof}
	Let $J=J(R)$. Since $R$ is a left artinian ring, the sequence 
	$(J^m)_{m\in\Z_{>0}}$ of left ideals stabilizes. There exists 
	$k\in\Z_{>0}$ such that $J^k=J^l$ for all $l\geq k$. We claim that $J^k=\{0\}$. If
	$J^k\ne\{0\}$ let $\mathcal{S}$ the set of left ideals 
	$I$ such that $J^kI\ne\{0\}$. Since 
	\[
	J^kJ^k=J^{2k}=J^k\ne\{0\},
	\]
	the set $\mathcal{S}$ is non-empty. 
	Since $R$ is left artinian, $\mathcal{S}$ has a minimal element $I_0$. Since $J^kI_0\ne\{0\}$, let $x\in
	I_0\setminus\{0\}$ be such that $J^kx\ne\{0\}$. Moreover, $J^kx$ is a left ideal of $R$ 
	contained in $I_0$ and such that $J^kx\in\mathcal{S}$, as 
	$J^k(J^kx)=J^{2k}x=J^kx\ne\{0\}$. The minimality of $I_0$ implies that, $J^kx=I_0$. In particular, 
	there exists $r\in J^k\subseteq J$ such that $rx=x$. Since $-r\in
	J(R)$ is left quasi-regular, there exists $s\in R$ such that $s-r-sr=0$.
	Thus 
	\[
		x=rx=(s-sr)x=sx-s(rx)=sx-sx=0,
	\]
	a contradiction.
\end{proof}

\begin{corollary}
	Let $R$ be a left artinian ring. Each nil left ideal is nilpotent and 
	$J(R)$ is the unique maximal nilpotent ideal of $R$. 
\end{corollary}

\begin{proof}
	Let $L$ be a nil left ideal of $R$. By Proposition~\ref{pro:nilJ}, $L$
	is contained in $J(R)$. Thus $L$ is nilpotent, as $J(R)$ 
	is nilpotent by Theorem~\ref{thm:Jnilpotente}. 
\end{proof}

\subsection{Akizuki's theorem}

We now prove that 
if $R$ is a unitary commutative artinian ring, 
then $R$ is noetherian. 

\begin{exercise}
\label{xca:I_fg}
    Let $R$ be a unitary commutative ring, $I$ be an ideal of $R$
    and $M$ be an $R$-module such that $I\cdot M=\{0\}$. Prove that
    if $M$ is finitely generated, then $M$ is a finitely generated
    $(R/I)$-module with
    \[
    (r+I)\cdot m=r\cdot m,\quad r\in R,m\in M.
    \]
\end{exercise}

Recall that an ideal $I$  
of a commutative ring 
$R$ is said to be \emph{prime} if 
$xy\in I$ implies that $x\in I$ or $y\in I$. 

\begin{exercise}
    Let $R$ be an unitary commutative artinian ring. 
    \begin{enumerate}
        \item Prove that if $R$ is a domain, then $R$ is a field. 
        \item Prove that prime ideals of $R$ are maximal. 
    \end{enumerate}
\end{exercise}

% Let $x\in R\setminus\{0\}$. The sequence $(x)\supseteq (x^2)\supseteq\cdots$
% stabilizes, that is $(x^n)=(x^{n+1})$ for some $n$. There exists
% $r\in R$ such that $x^n(1-rx)=0$. Since $x^n\ne 0$, $1=rx$. 
% I prime $\implies$ $(R/I)$ domain $\implies $R/I$ field $\implies $I$ maximal 

\begin{theorem}[Akizuki]
    \index{Akizuki's theorem}
    Let $R$ be a unitary commutative ring. If $R$ is artinian, 
    then $R$ is noetherian.
\end{theorem}

\begin{proof}
    Assume that the result is not true, so there exists an ideal of $R$ 
    that is not finitely generated. 
    Let $X$ be the set of ideals of $R$ that are not finitely generated. 
    Since $X\ne\emptyset$ and $R$ is artinian, there exists a minimal 
    element $I\in X$. The minimality of $I$ implies that 
    if $J$ is an ideal of $R$ such that $J\subsetneq I$, then 
    $J$ is finitely generated. 

    \begin{claim}
        Either $RI=\{0\}$ or $RI=I$.
    \end{claim}
    
    If not, let $r\in R$ be such that $rI\ne\{0\}$ and $rI\ne I$. 
    Since $rI$ is an ideal of $R$ and
    $rI\subsetneq I$, the minimality of $I$ implies that 
    $rI$ is finitely generated. Let 
    $f\colon I\to rI$, $x\mapsto rx$. Then $f$ is a 
    surjective module homomorphism. Since $RI\ne\{0\}$, 
    $f$ is non-zero. In particular, $\ker f$ 
    is finitely generated, again by the minimality of $I$. 
    By the first isomorphism theorem, $I/\ker f\simeq rI$ as $R$-modules.
    Since $\ker f$ and $I/\ker f\simeq rI$ are finitely generated, 
    $I$ is finitely generated, a contradiction.
    
    \begin{claim}
        $M=\{r\in R:rI=\{0\}\}$ is a maximal ideal of $R$. 
    \end{claim}
    
    Routine calculations show that $M$ is an ideal. Since 
    $R$ is artinian, it is enough to show that $M$ is a prime ideal. 
    Let $rs\in M$. Then $(rs)I=\{0\}$. If $r\not\in M$, 
    then $rI\ne\{0\}$. By the previous claim, $rI=I$. Thus
    \[
    \{0\}=(rs)I=s(rI)=sI
    \]
    and hence $s\in M$.     
    
    \medskip
    Since $M$ is maximal, $K=R/M$ is a field. 
    Since $MI=\{0\}$, $I$ is an $(R/M)$-module, that is 
    $I$ is a $K$-vector space. By Exercise \ref{xca:I_fg}, 
    $\dim_KI=\infty$. Let $B$ be a basis of $I$ (as a $K$-vector space) 
    and $x_0\in B$. Let $J$ be the subspace of $I$ generated by
    $B\setminus\{x_0\}$. A direct calculation
    shows that 
    $J$ is an ideal of $R$.
    %as  
    %\[
    %J=\sum_{x\in B\setminus\{x_0\}} Kx. 
    %\]
    Since $\dim_K J=\infty$, it follows that $J$ 
    is not a finitely generated ideal of $R$ 
    (Exercise \ref{xca:I_fg}). 
    This is a contradiction, because $J$ is an ideal of $R$
    such that $J\subsetneq I$. 
\end{proof}

\chapter{}

\topic{Semiprime and semiprimitive rings}

\begin{definition}
	A ring $R$ is \textbf{semiprimitive} (or Jacobson semisimple) if  $J(R)=\{0\}$.
\end{definition}

In Lecture \ref{03} we defined primitive rings as
those rings that have a faithful simple module.  We claim that primitive rings
are semiprimitive. If $R$ is primitive, then $\{0\}$ is a primitive ideal. Since
$J(R)$ is the intersection of primitive ideals, it follows that $J(R)=\{0\}$.

\begin{example}
	If $R=\prod_{i\in I}R_i$ is a direct product of semiprimitive rings, then
	$R$ is semiprimitive, as 
	\[
		J(R)=J\left(\prod_{i\in I}R_i\right)=J\left(\prod_{i\in I}J(R_i)\right)=\{0\}.
	\]
\end{example}

\begin{example}
$\Z$ is semiprimitive, as $J(\Z)=\cap_{p}\Z/p=\{0\}$.
\end{example}

\begin{example}
	\label{exa:C[a,b]}
	Let $R=C[a,b]$ be the ring of continuous maps $f\colon [a,b]\to\R$. 
	In this case $J(R)$ is the intersection of all maximal ideals of $R$. Note that 
	each maximal ideal of $R$ is of the form 
	\[
		U_c=\{f\in C[a,b]:f(c)=0\}
	\]
	for some $c\in[a,b]$. 
	%Each $U_c$ is a maximal ideal, as 
	%$C[a,b]/U_c\simeq\R$ is a field.  
	Thus $J(R)=\cap_{a\leq c\leq
	b}U_c=\{0\}$.
\end{example}

We proved in Theorem~\ref{thm:J(R/J)=0} (Lecture \ref{04}) 
that $R/J(R)$ is semiprimive. 

%El teorema de densidad de Jacobson nos permite entonces obtener el siguiente resultado:
%
%\begin{theorem}
%	Sea $R$ un anillo no radical. Entonces $R/J(R)$ es isomorfo a un producto
%	subdirecto de anillos densos en espacios vectoriales sobre anillos de
%	división.	
%\end{theorem}
%
%\begin{proof}
%	Si $R$ no es radical, $J(R)\ne R$. Luego $R/J(R))$ es semiprimitivo por el
%	teorema~\ref{thm:semiprimitivo}. El teorema~\ref{thm:subdirecto} y el
%	teorema de densidad de Jacobson completan la demostración del teorema.
%\end{proof}


\begin{definition}
	Let $\{R_i:i\in I\}$ be a collection of rings. A subring $R$ of
	$\prod_{i\in I}R_i$ is said to be a \textbf{subdirect product} of the
	collection if each $\pi_j\colon R\to R_j$, $(r_i)_{i\in I}\mapsto r_j$, is surjective. 
\end{definition}

%El siguiente teorema justifica que indistintamente llamemos anillos
%semiprimitivos a los anillos semisimples Jacobson:

\begin{theorem}
	\label{thm:subdirecto}
	Let $R$ be a non-zero ring. Then $R$ is semiprimitive if and only if
	$R$ is isomorphic to a subdirect product of primitive rings. 
\end{theorem}

\begin{proof}
	Suppose first that $R$ is semiprimitive and let $\{P_i:i\in I\}$ be the collection of 
	primitive ideals of $R$. Each $R/P_j$ is primitive and 
	$\{0\}=J(R)=\cap_{i\in I}P_i$. For $j$ let $\lambda_j\colon R\to
	R/P_j$ and $\pi_j\colon \prod_{i\in I}R/P_i\to R/P_j$ be canonical maps
	The ring homomorphism 
	\[
		\phi\colon R\to\prod_{i\in I}R/P_i,\quad
		r\mapsto \{\lambda_i(r):i\in I\},
	\]
	is injective and satisfies $\pi_j\phi(R)=R/P_j$ for all 
	$j$.

	Assume now that $R$ is isomorphic to a subdirect product of primitive rings 
	$R_j$ and let $\varphi\colon R\to\prod_{i\in I}R_i$ be an injective homomorphism 
	such that $\pi_j(\varphi(R))=R_j$ for all $j$. For $j$ 
	let $P_j=\ker\pi_j\varphi$. Since $R/P_j\simeq R_j$, each $P_j$ is a primitive ideal. 
	If $x\in\cap_{i\in I}P_i$, then $\varphi(x)=0$ and thus $x=0$.
	Hence $J(R)\subseteq\cap_{i\in I} P_i=0$. 
\end{proof}

\begin{example}
	$\Z$ is isomorphic to a subdirect product of the fields $\Z/p$, where 
	$p$ runs over all prime numbers. 
\end{example}

\begin{example}
	The ring $C[a,b]$ 
	of Example \ref{exa:C[a,b]}
	is isomorphic to a subdirect product of the fields 
	$C[a,b]/U_c\simeq\R$.
\end{example}

\begin{definition}
	\index{Ring!semiprime}
	A ring $R$ \textbf{semiprime} if 
	$aRa=\{0\}$ implies $a=0$.
\end{definition}

\begin{proposition}
	Let $R$ be a ring. The following statements are equivalent: 
	\begin{enumerate}
		\item $R$ is semiprime.
		\item If $I$ is a left ideal such that $I^2=\{0\}$, then $I=\{0\}$.
		\item If $I$ is an ideal such that $I^2=\{0\}$, then $I=\{0\}$.
		\item $R$ does not contain non-zero nilpotent ideals.
	\end{enumerate}
\end{proposition}

\begin{proof}
	We first prove that $1)\implies2)$. If $I^2=\{0\}$ y $x\in I$, then
	$xRx\subseteq I^2=\{0\}$ and thus $x=0$. The implications $2)\implies3)$
	and $4)\implies3)$ are both trivial. Let us prove that $3)\implies4)$.  If
	$I$ is a non-zero nilpotent ideal, let $n\in\Z_{>0}$ be minimal such that
	$I^n=\{0\}$.  Since $(I^{n-1})^2=\{0\}$, it follows that $I^{n-1}=\{0\}$, a
	contradiction.  Finally, we prove that $3)\implies1)$. Let $a\in R$ be such
	that $aRa=\{0\}$. Then $I=RaR$ is an ideal of $R$ such that $I^2=\{0\}$. Thus 
	$RaR=\{0\}$. This means that $Ra$ and $aR$ are ideals such that
	$(Ra)R=R(aR)=\{0\}$ (for example, $R(aR)\subseteq RaR=\{0\}\subseteq aR$). 
	Moreover, since $(Ra)(Ra)=\{0\}$ and $(aR)(aR)=\{0\}$, it follows that
	$aR=Ra=\{0\}$. 
	This implies that $\Z a$ is an ideal of $R$, as $R(\Z a)\subseteq \Z(Ra)=\{0\}$ and 
	$(\Z a)R\subseteq aR=\{0\}$. Now $(\Z a)(\Z a)\subseteq (\Z a)R=\{0\}$ and hence
	$a=0$, as $\Z a=\{0\}$. 
\end{proof}

Two consequences:

\begin{exercise}
	A commutative ring is semiprime if and only if it does not contain non-zero
	nilpotent elements. 
\end{exercise}

\begin{corollary}
	The ring $\C[G]$ is semiprime.
\end{corollary}

\begin{proof}
	Since $J(\C[G])=\{0\}$ by Rickart's theorem and the Jacobson radical
	contains every nil ideal by Proposition~\ref{pro:nilJ}, it follows that
	$\C[G]$ does not contain non-trivial nil ideals. Thus $\C[G]$ does not
	contain non-trivial nilpotent ideals and hence $\C[G]$ is semiprime.
\end{proof}

\begin{exercise}
	Prove that $Z(\C[G])$ is semiprime.
\end{exercise}

% tomar $\alpha$ tal que $\alpha^2=0$ y sea $A=K[G]\alpha$. Como $A^2=0$, $A=0$ y entonces $\alpha=0$.

\begin{exercise}
	Let $D$ be a division ring. 
	\begin{enumerate}
		\item $D[X]$ is semiprime.
		\item $D[\![X]\!]$ is semiprime and it is not semiprimitive.
	\end{enumerate}
\end{exercise}


%\section{Anillos semiprimitivos}
%
%\begin{lemma}
%	\label{lem:Iunitario}
%	Sea $R$ un anillo y sea $I$ un ideal de $R$ unitario. Sea $e\in I$ la
%	unidad de $I$. Entonces $e$ es un idempotente central de $R$, $I=eR$ y
%	existe un ideal $J$ de $R$ tal que $R=I\oplus J$. Además $R\simeq I\times
%	J$.
%\end{lemma}
%
%\begin{proof}
%	Como $e\in I$, $eR\subseteq I$. Luego $I=eR$ pues $I=eI\subseteq eR$. Como
%	$ex\in I$ y $xe\in I$ para todo $x\in R$, $ex=(ex)e$ y $xe=e(xe)$. Luego
%	$ex=xe$ y entonces $e$ es central e idempotente. Sea $J=\{x-ex:x\in R\}$.
%	Es fácil demostrar que $J$ es un ideal tal que $R=I\oplus J$. Además
%	$R\simeq I\times J$, via $x\mapsto (ex,x-ex)$,
%\end{proof}
%
%A continuación daremos una demostración muy sencilla del teorema de Wedderburn
%en el caso de álgebras de dimensión finita.
%
%\begin{theorem}[Artin--Wedderburn]
%	Sea $R$ un anillo artiniano a izquierda y no nulo. Entonces $R$ es
%	semiprimo si y sólo si existen $n_1,\dots,n_r\in\N$ y existen anillos de
%	división $D_1,\dots,D_r$ tales que $R\simeq M_{n_1}(D_1)\times\cdots\times
%	M_{n_r}(D_r)$.
%\end{theorem}
%
%\begin{proof}
%	Procederemos por inducción en $\dim A$. Si $\dim A=1$\dots\framebox{} 
%
%	Supongamos entonces que $\dim A>1$. Si $A$ es un álgebra prima, el
%	resultado se sigue inmediatamente del teorema de Wedderburn. Supongamos
%	entonces que existe $a\in A\setminus\{0\}$ tal que $I=\{x\in A:aAx=0\}$ es
%	no nulo. Como $I$ es un ideal de $A$, $I$ es un álgebra semiprima.
%	\framebox{?} Como $a\not\in I$, $\dim I<\dim A$, y entonces, por hipótesis
%	inductiva, existen $n_1,\dots,n_s\in\N$ y álgebras de división
%	$D_1,\dots,D_s$ tales que 
%	\[
%		I\simeq M_{n_1}(D_1)\times\cdots\times M_{n_s}(D_s).
%	\]
%	En particular, $I$ es unitario. Por el lema~\ref{lem:Iunitario}, existe un
%	ideal $J$ de $A$ tal que $A\simeq I\times J$. Como $\dim J<\dim A$, la hipótesis inductiva
%	implica que existen $n_{s+1},\dots,n_r\in\N$ y álgebras de división $D_{s+1},\dots,D_r$ tales que
%	\[
%		J\simeq M_{n_{s+1}}(D_{s+1})\times\cdots\times M_{n_r}(D_r).
%	\]
%	Luego $A\simeq I\times J\simeq \prod_{j=1}^s M_{n_j}(D_j)$.
%\end{proof}
%
%\begin{corollary}
%	Sea $A$ un álgebra no nula de dimensión finita. Si $A$ es semiprima,
%	entonces $A$ es unitaria.
%\end{corollary}
%
%%Gracias al teorema de Wedderburn se puede ir un poco más lejos:
%%\begin{corollary}
%%	Sea $A$ un álgebra unitaria. Son equivalentes:
%%	\begin{enumerate}
%%		\item $A$ es semiprima.
%%		\item Todo $A$-módulo unitario es semisimple.
%%		\item $A$ es semisimple como $A$-módulo.
%%		\item Todo ideal a izquierda de $A$ es de la forma $Ae$ para algún
%%			idempotente $e\in A$. 
%%	\end{enumerate}
%%\end{corollary}
%%
%%\begin{proof}
%%	La implicación $(1)\implies(2)$ es el teorema de Wedderburn. 
%%	
%%\end{proof}
%
%\begin{example}
%	Por el teorema de Maschke sabemos que si $G$ es un grupo finito, 
%	$\C[G]$ es un álgebra semiprimitiva y luego semisimple.
%\end{example}
%




%\section{Viejo!}
%
%\begin{theorem}[Artin--Wedderburn]
%	\index{Teorema!de Artin--Wedderburn}
%	\label{thm:ArtinWedderburn}
%	Si $R$ es un anillo, las siguientes afirmaciones son equivalentes:
%	\begin{enumerate}
%		\item $R$ es un anillo no nulo semiprimitivo y artiniano a izquierda.
%		\item Existen anillos de división $D_1,\dots,D_r$ y tales que
%			\[
%				R\simeq\prod_{i=1}^r R_i,
%			\]
%			donde $R_i=\End_{D_i}(V_i)$
%		\item Existen anillos de división $D_1,\dots,D_r$ y enteros positivos
%			$n_1,\dots,n_r$ tales que 
%			\[
%			R\simeq M_{n_1}(D_1)\times\cdots\times M_{n_r}(D_r).
%		\]
%	\end{enumerate}
%\end{theorem}
%
%\begin{proof}
%	Demostremos que $(1)\implies(2)$. Como $R\ne0$ y $J(R)=0$, $R$ admite
%	ideales primitivos. Supongamos que existe un número finito de ideales
%	primitivos distintos, digamos $P_1,\dots,P_t$. Cada $R/P_j$ es un anillo
%	primitivo y es artiniano a izquierda \framebox{?}. Entonces, por el teorema
%	de Wedderburn, para cada $j\in\{1,\dots,t\}$ existen un anillo de división
%	$D_j$ y un entero positivo $n_j$ tales que $R/P_j\simeq M_{n_j}(D_j)$. En
%	particular, cada $R/P_j$ es simple y entonces $P_j$ es un ideal maximal de
%	$R$. Como $R/P_j$ es simple, $(R/P_j)^2\ne 0$ y luego $R^2\not\subseteq
%	P_j$. Por maximalidad, $R^2+P_j=R$ y además $P_i+P_j=R$ para todo $i\ne j$.
%	Por el teorema chino del resto,
%	\[
%		R=R/0=R/J(R)=R/\cap_{j=1}^t P_j\simeq R/P_1\times\cdots\times R/P_t.
%	\]
%	Sea $\iota_k\colon R/P_k\to \prod_{j=1}^t R/P_j$ la inclusión canónica.
%	Cada $\iota_k(R/P_k)$ es un ideal simple (es decir, que como anillo es
%	simple) de $\prod_{j=1}^t R/P_j\simeq R$. Luego las imágenes, digamos
%	$I_k$, de los $\iota_k(R/P_k)$ dan ideales simples de $R$ y
%	$R=I_1\times\cdots\times I_t$.
%
%	Demostremos ahora que $(3)\implies(1)$. Para cada $j$ sea
%	$R_j=M_{n_j}(D_j)$. Como cada $R_j$ es primitivo por el teorema de
%	Wedderburn, $J(R_j)=\{0\}$ para todo $j$. Luego
%	$J(R)=\prod_{i=1}^rJ(R_j)=\{0\}$ y entonces $R$ es semiprimitivo. Además
%	$R$ es artiniano a izquierda.\framebox{?}
%\end{proof}
%
%\begin{corollary}
%	Sea $R$ un anillo semiprimitivo.
%	\begin{enumerate}
%		\item Si $R$ es artiniano a izquierda, entonces $R$ es unitario.
%		\item $R$ es artiniano a izquierda si y sólo si es artiniano a derecha.
%		\item Si $R$ es artiniano a izquierda es noetheriano.
%	\end{enumerate}
%\end{corollary}
%
%\begin{proof}
%	La primera afirmación es consecuencia inmediata del teorema de
%	Artin--Wedderburn~\ref{thm:ArtinWedderburn}.
%\end{proof}
%
%\begin{corollary}
%	Sea $R$ un anillo semiprimitivo artiniano a izquierda y sea $I$ un ideal de
%	$R$. Entonces $I=Re$ para algún idempotente $e\in R$ tal que $e\in Z(R)$.
%\end{corollary}
%
%\begin{proof}
%		
%\end{proof}
%
%\begin{proposition}
%	Sea $R$ un anillo semisimple artiniano a izquierda. 
%	\begin{enumerate}
%		\item $R=I_1\times\cdots\times I_n$ donde los $I_j$ son ideales simples.
%		\item Si $J\subseteq R$ es un ideal simple, entonces existe $k\in\{1,\dots,n\}$ tal que $J=I_k$.
%		\item Si $R=J_1\times\cdots\times J_m$ donde los $J_k$ son ideales simples, entonces $n=m$ y existe
%			$\sigma\in\Sym_n$ tal que $I_k=J_{\sigma(k)}$ para todo $k\in\{1,\dots,n\}$.
%	\end{enumerate}
%\end{proposition}
%
%\begin{proof}
%\end{proof}
%
%\begin{theorem}
%	Sea $R$ un anillo unitario no nulo. Las siguientes afirmaciones son
%	equivalentes:
%	\begin{enumerate}
%		\item $R$ es semiprimitivo y artiniano a izquierda.
%		\item Todo $R$-módulo unitario es proyectivo.
%		\item Todo $R$-módulo unitario es inyectivo.
%		\item Toda sucesión exacta de $R$-módulos unitarios se parte.
%		\item Todo $R$-módulo unitario no nulo es semisimple.
%		\item $\prescript{}{R}R$ es unitario y semisimple.
%		\item Todo ideal a izquierda de $R$ es de la forma $Re$ para algún $e\in R$ indempotente.
%		\item $\prescript{}{R}R$ es suma directa de ideales a izquierda
%			minimales $L_1,\dots,L_m$ donde cada $L_j$ es de la forma $Re_j$, y
%			los $e_j$ son idempotentes ortogonales tales que
%			$e_1+\cdots+e_m=1$. 
%	\end{enumerate}
%\end{theorem}
%
%\begin{proof}
%	Veamos que $(4)\implies(5)$. Sea $M$ un módulo unitario y sea $N$ un
%	submódulo no nulo de $M$. Como la sucesión $0\to N\to M\to M/N\to 0$ es
%	exacta, se parte. Luego $M=N\oplus X$ para algún submódulo $X$ de $N$ tal
%	que $X\simeq M/N$. Como $M$ es unitario, $Rm\ne 0$ para todo $m\in
%	M\setminus\{0\}$. Luego $M$ es semisimple por el teorema~\framebox{?}.
%
%	Veamos ahora que $(5)\implies(4)$. Sea 
%	\[
%	\begin{tikzcd}
%		0 \arrow[r]
%		& N \arrow[r]
%		& M \arrow[r]
%		& X \arrow[r]
%		& 0
%	\end{tikzcd}
%	\]
%	una sucesión exacta corta de $R$-módulos. Como $f\colon N\to f(N)$ es un
%	isomorfismo y entonces $f(N)$ es un submódulo del semisimple $M$, $f(N)$ es
%	sumando directo de $M$. Sea $\pi\colon M\to f(N)$ el morfismo canónico.
%	Entonces $\pi f=f$ y $f^{-1}\pi\colon M\to A$ es un morfismo tal que
%	$(f^{-1}\pi)f=\id_N$.\framebox{?}
%
%	Demostremos que $(5)\implies(7)$. Sea $L$ un ideal a izquierda de $R$. Como
%	los ideales a izquierda de $R$ son los submódulos de $\prescript{}{R}R$,
%	existe un ideal a izquierda $N$ de $R$ tal que $R=L\oplus N$. Existen
%	entonces $e_1\in L$ y $e_2\in N$ tales que $1=e_1+e_2$. Si $x\in L$,
%	entonces $x=xe_1+xe_2$ y luego $xe_2=x-xe_1\in L\cap N=\{0\}$. Demostramos
%	entonces que $x=xe_1$ para todo $x\in L$. En particular, $e_1e_1=e_1$ y
%	$L=Re_1$. 
%
%	Demostremos que $(7)\implies(6)$. Sea $L$ un submódulo de
%	$\prescript{}{R}R$. Como entonces $L$ es un ideal a izquierda de $R$,
%	$L=Re$ para algún idempotente $e\in R$. Como el conjunto $R(1-e)$ es un
%	ideal a izquierda de $R$ tal que $R=Re\oplus R(1-e)$, se concluye que
%	$\prescript{}{R}R$ es semisimple.\framebox{?}
%
%	Veamos que $(6)\implies(1)$. Supongamos que $\prescript{}{R}R=\sum_{i\in
%	I}N_i$, donde los $N_j$ son submódulos simples de $\prescript{}{R}R$.
%	Reordenando los $N_j$ si fuera necesario, podemos suponer que existe
%	$k\in\N$ tal que $1=e_1+\cdots+e_k$ y $e_j\in N_j$ para todo
%	$j\in\{1,\dots,k\}$. Si $r\in R$, entonces $r=re_1+\cdots+re_k\in
%	\sum_{i=1}^k N_i$. Luego $R=\sum_{i=1}^k N_i$.
%	Veamos que $J(R)=0$. Si $r\in J(R)$ entonces, como $rN_i=0$ para todo
%	$i\in\{1,\dots,k\}$, se concluye que $r=r1=re_1+\cdots+re_k=0$. Probamos
%	entonces que $R$ es semiprimitivo. Falta ver que $R$ es artiniano a
%	izquierda. Para eso basta obvervar que, como
%	\[
%		\frac{N_1\oplus\cdots\oplus N_i}{N_1\oplus\cdots\oplus N_{i-1}}\simeq N_i
%	\]
%	para cada $i\in\{1,\dots,k\}$, la serie
%	\[
%	R=N_1\oplus\cdots\oplus N_k\supsetneq N_1\oplus\cdots\oplus N_{k-1}\supsetneq\cdots\supsetneq N_1\oplus N_2\supsetneq N_1\supsetneq 0
%	\]
%	es una serie de composición.\framebox{?}
%
%	Veamos que $(1)\implies(8)$. Sin perder generalidad podemos suponer que
%	\[
%	R=\prod_{i=1}^k M_{n_j}(D_j),
%	\]
%	donde los $D_j$ son anillos de división.\framebox{?}
%
%	Veamos que $(8)\implies(5)$. Sea $M$ un módulo unitario no nulo. Si $m\in
%	M$ entonces $L_im$ es un submódulo de $M$. Los $L_jm$ generan a $M$ pues 
%	\[
%	m=1m=e_1m+\cdots+e_km\in\sum L_im.
%	\]
%	Veamos que cada $L_jm$ es simple. Fijado $i$, la función $f\colon L_i\to
%	L_im$, $x\mapsto xm$, es un morfismo sobreyectivo. Como $L_i$ es un ideal a
%	izquierda minimal, $L_i$ es un submódulo simple. Luego $m\ne0$ implica que
%	$f$ es un isomorfismo. Probamos entonces que el conjunto $\{L_jm:1\leq
%	j\leq k,m\in M,L_jm\ne 0\}$ es una familia de submódulos simples cuya suma
%	es $M$.
%\end{proof}


\topic{Jacobson's density theorem}

At this point, it is convenient to recall that
modules over division rings are pretty much as vector spaces over fields. 
Modules over division rings are usually called vector spaces over division rings. 

\begin{definition}
	Let $D$ be a division ring, and $V$ be a vector space over $D$. A subring 
	$R\subseteq\End_D(V)$ is a \textbf{dense ring of linear operators} 
	of $V$ (or simple, \textbf{dense} in $V$) if for every  
	$n\in\Z_{>0}$, every linearly independent set $\{u_1,\dots,u_n\}\subseteq V$ 
	and every (not necessarily linearly independent) subset $\{v_1,\dots,v_n\}\subseteq V$ 
	there exists $f\in R$ such that $f(u_j)=v_j$ for all 
	$j\in\{1,\dots,n\}$.
\end{definition}

%\begin{lemma}
%	Sea $R$ un subanillo de $\End_D(V)$. Entonces $R$ es denso en $V$ si y sólo
%	si para todo $g\in\End_D(V)$ y todo subespacio $U$ de $V$ de dimensión
%	finita existe $f\in R$ tal que $f|_U=g|_U$.
%\end{lemma}
%
%\begin{proof}
%	Supngamos que $R$ es denso en $V$. Sean $g\in\End_D(V)$ y $U\subseteq V$ un
%	subespacio de dimensión finita. Sea $\{u_1,\dots,u_n\}$ una base de $U$.
%	Como $R$ es denso, existe $f\in R$ tal que $f(u_j)=g(u_j)$ para todo
%	$j\in\{1,\dots,n\}$ y luego $f|_U=g|_U$.
%
%	Recíprocamente, sea $n\in\N$ y sean $\{u_1,\dots,u_n\}\subseteq V$ un
%	conjunto linealmente independiente y $\{v_1,\dots,v_n\}\subseteq V$. Sean
%	$g\in\End_D(V)$ tal que $g(u_j)=v_j$ para todo $j\in\{1,\dots,n\}$ y $U$ el
%	subespacio de $V$ generado por $u_1,\dots,u_n$. Por hipótesis existe $f\in
%	R$ tal que $f|_U=g|_U$ y luego $f(u_j)=g(u_j)=v_j$ para todo
%	$j\in\{1,\dots,n\}$.
%\end{proof}

\begin{proposition}
	\label{pro:unique_dense}
	Let $D$ be a division ring and 
    $V$ be a $D$-vector space.  
	If $\dim_DV<\infty$, then $\End_D(V)$ is the only dense ring of $V$.
\end{proposition}

\begin{proof}
	Let $R$ be dense in $V$ and let $\{v_1,\dots,v_n\}$ be a basis of $V$. By definition,
	$R\subseteq\End_D(V)$. If $g\in\End_D(V)$ then, since $R$ is dense in $V$, there
	exists $f\in R$ such that $f(v_j)=g(v_j)$ for all 
	$j\in\{1,\dots,n\}$. Hence $g=f\in R$.
\end{proof}


%Ahora demostraremos el teorema de densidad de Jacobson. 
%Necesitaremos el siguiente
%lema:
%\begin{lemma}
%	\label{lem:densidad}
%	Sea $M$ un $R$-módulo simple y $D=\End_R(M)$.  Si $N$ es un subespacio de
%	$M$ tal que $\dim_DN<\infty$ y $m\in M\setminus N$, entonces existe $r\in
%	R$ tal que $rm\ne 0$ y $rN=0$.
%\end{lemma}
%
%\begin{proof}
%	Supongamos que la afirmación no es cierta y sea $N$ un contraejemplo de la
%	mínima dimensión posible. Entonces $\dim_DN\geq1$ (pues el resultado es
%	verdadero en el caso $N=0$). Sea $N_0$ un subespacio de $N$ tal que $\dim
%	N_0=\dim N-1$ y sea
%	\[
%		L=\{r\in R:rN_0=0\}.
%	\]
%	Como por la minimalidad de $N$ nuestra afirmación es cierta para $N_0$,
%	para cualquier $x\in N\setminus N_0$ se tiene que $Lx=N$ (pues existe $r\in
%	L$ tal que $rx=\ne 0$). Como $L$ es ideal a izquierda de $R$ y
%	$Lx\subseteq N$ es un submódulo, $Lx=N$ pues $N$ es simple.
%
%	Sea $w\in V\setminus U$ tal que nuestra afirmación no es cierta y sea $u\in
%	U\setminus U_0$.  La función 
%	\[
%		\delta\colon V\to V,\quad
%		v\mapsto lw,
%	\]
%	donde $v=lu\in Lu=V$ (que depende de $u$ y $w$) 
%	está bien definida: si $l_1,l_2\in L$ son tales que $v=l_1u=l_2u$ entonces $(l_1-l_2)u=0$ y luego
%	\[
%		0=\delta(0)=\delta((l_1-l_2)u)=(l_1-l_2)w=l_1w-l_2w. 
%	\]
%	Además $\delta$ es morfismo de $R$-módulos pues si $l\in L$ es tal que $v=lu$ entonces
%	\[
%		\delta(rv)=\delta(r(lu))=\delta( (rl)u)=(rl)w=r(lw)=r\delta(v)
%	\]
%	para todo $r\in R$.
%
%
%\end{proof}


\begin{theorem}[Jacobson]
	\label{thm:density}
	\index{Jacobson's density theorem}
	A ring $R$ is primitive if and only if it is isomorphic to a dense ring in 
	a vector space over a division ring.
\end{theorem}

We shall need the following lemma. 

\begin{lemma}
	\label{lem:ideal_denso}
	Let $D$ be a division ring and $V$ be a $D$-vector space. 
	If $R$ is dense in $V$ and $I$ is a non-zero ideal of $R$, then 
	$I$ is dense in $V$. 
\end{lemma}

\begin{proof}
    Fix $n\in\Z_{>0}$. Let 
    $\{u_1,\dots,u_n\}\subseteq V$ be a linearly independent set 
	and let $\{v_1,\dots,v_n\}\subseteq V$. We want to find $\gamma\in I$ such that 
	$\gamma(u_i)=v_i$ for all $i$. Since $I\ne\{0\}$, there exists 
	$h\in I\setminus\{0\}$. This means that 
	$h(u)=v\ne0$ for some $u\ne 0$. Since $R$ is 
	dense in $V$, there exist $g_1,\dots,g_n\in R$ such that 
	\[
	g_i(u_j)=\begin{cases} 
	u & \text{if $i=j$},\\
	0 & \text{otherwise}.
	\end{cases}
	\]
	Further, since $\{v\}$ is a linearly independent subset of $V$, 
	there exist $f_1,\dots,f_n\in R$ such that 
	$f_i(v)=v_i$ for all $i$. Thus $\gamma=\sum_{i=1}^n f_ihg_i\in I$ is such that 
	$\gamma(u_j)=v_j$ for all $j\in\{1,\dots,n\}$.
\end{proof}

Now we are ready to prove Jacobson's density theorem. 

\begin{proof}[Proof of Theorem \ref{thm:density}]
	If $R$ is isomorphic to a dense ring in $V$, where
	$V$ is a $D$-vector space for some division ring $D$, then $R$
	is primitive, as $V$ is a simple and faithful $R$-module. Why faithful? If 
	$f\in\Ann_R(V)$, then $f=0$ since $f(v)=0$ for all $v\in V$. Why simple? 
	If $W\subseteq V$ is a non-zero submodule, let $v\in V$ and $w\in
	W\setminus\{0\}$. There exists $f\in R$ such that $v=f(w)\in W$. 

	Now assume that $R$ is primitive. Let $V$ be a simple faithful module.
	Schur's lemma implies that $D=\End_R(V)$ is a division ring. Thus $V$ is
	a $D$-vector space with 
	\[
	D\times V\to V,\quad
	(\delta,v)\mapsto \delta v=\delta(v),
	\]
	For $r\in R$ let  
	\[
		\gamma_r\colon V\to V,\quad
		v\mapsto rv.
	\]
	A straightforward calculation shows that $\gamma_r\in\End_D(V)$ and that
    $R\to\End_D(V)$,
	$r\mapsto\gamma_r$, is a ring homomorphism. Since $V$ is faithful,
	$R\simeq\gamma(R)=\{\gamma_r:r\in R\}$. In fact, if $\gamma_r=\gamma_s$, then 
	$rv=\gamma_r(v)=\gamma_s(v)=sv$ for all $v\in V$ and hence $r=s$, as
	$(r-s)v=0$ for all $v\in V$.

	\begin{claim}
		If $U$ is a finite-dimensional submodule of $V$, 
		for each $w\in V\setminus U$ there exists $r\in R$ such that 
		$\gamma_r(U)=0$ and $\gamma_r(w)\ne0$.
	\end{claim}

	Suppose the claim is not true. Let $U$ be a counterexample of minimal  
	dimension. Then
	$\dim_DU\geq1$, as the claim holds for the zero submodule. Let 
	$U_0$ be a submodule of $U$ such that 
	$\dim U_0=\dim U-1$ and let 
	\[
		L=\{l\in R:\gamma_l(U_0)=0\}.
	\]
	The minimality of the dimension of $U$ shows that the claim is true for $U_0$, so
	any $v\in V\setminus U_0$ is such that $Lv=V$. In fact, since there exists $l\in
	L$ such that $lv=\gamma_l(v)\ne 0$ and $L$ is a left ideal of $R$, it follows
	that $Lv\subseteq V$ is a submodule and the claim follows from the simplicity of
	$V$.
	
	Let $w\in V\setminus U$ be such that the claim is not true. Let $u\in
	U\setminus U_0$. The map  
	\[
		\delta\colon V\to V,\quad
		v\mapsto lw,
	\]
	where $v=lu\in Lu=V$ (that depends both on $u$ and $w$) 
	is well-defined: if $l_1,l_2\in L$ are such that 
	$v=l_1u=l_2u$, then $(l_1-l_2)u=0$ and thus 
	\[
		0=\delta(0)=\delta((l_1-l_2)u)=(l_1-l_2)w=l_1w-l_2w. 
	\]
	Further, $\delta$ is a homomorphism of modules over $R$, as if 
	$l\in L$ is such that $v=lu$, then 
	\[
		\delta(rv)=\delta(r(lu))=\delta( (rl)u)=(rl)w=r(lw)=r\delta(v)
	\]
	for all $r\in R$.

	For every $l\in L$,  
	\[
		l(\delta(u)-w)=l\delta(u)-lw=\delta(lu)-lw=0.
	\]
	Thus $L(\delta(u)-w)=0$. This implies that $\delta(u)-w\not\in V\setminus U_0$, 
	that is $\delta(u)-w\in U_0$. Therefore  
	\[
		w=xu-(xu-w)\in Du+U_0=U,
	\]
	a contradiction.
	
	Now the theorem follows from the claim. Let 
	$u_1,\dots,u_n\in V$ be linearly independent vectors and let 
	$v_1,\dots,v_n\in V$ arbitrary vectors. Fix $i\in\{1,\dots,n\}$. 
	The previous claim with 
	\[
		U=\langle u_1,\dots,u_{i-1},u_{i+1},\dots,u_n\rangle
	\]
	and $w=u_i$ implies that there exists $r_i\in R$ such that $\gamma_{r_i}(u_j)=0$ if 
	$j\ne i$ and $\gamma_{r_i}(u_i)\ne 0$. Since there exists $s_i\in R$ such that 
	$\gamma_{s_i}\gamma_{r_i}(u_i)=v_i$, it follows that 
	$r=\sum_{j=1}^n s_jr_j\in R$ is such that $\gamma_r(u_i)=v_i$ for all 
	$i\in\{1,\dots,n\}$.
\end{proof}

%\begin{exercise}
%	Sea $R$ un anillo denso en $V$. Demuestre que $R$ es artiniano a izquierda
%	si y sólo si $V$ es de dimensión finita.
%\end{exercise}
% si V es de dimensión finita, es fácil por el lema~\ref{lem:unico_denso}
% hungerford pag 419

\begin{corollary}
	If $R$ is a primitive ring, then either there exists a division ring $D$
	such that $R\simeq\End_D(V)$ for some finite-dimensional module $V$ over $D$ or 
	for all $m\in\Z_{>0}$ there exists a subring $R_m$ of 
	$R$ and a surjective ring homomorphism $R_m\to\End_D(V_m)$ for some module 
	$V_m$ over $D$ such that $\dim_DV_m=m$.
\end{corollary}

\begin{proof}
	The ring $R$ admits a simple faithful module $V$. Furthermore, by Jacobson's density 
	theorem we may assume that there exists a division ring $D$ 
	such that $R$ is dense in a module $V$ over $D$. 
	Let $\gamma\colon R\to\End_D(V)$, $r\mapsto\gamma_r$, where 
	$\gamma_r(v)=rv$. Since $V$ is faithful, $\gamma$ is injective. Thus 
	$R\simeq\gamma(R)$. 

	If $\dim_DV<\infty$, the result follows from Proposition \ref{pro:unique_dense}. 
	Assume that $\dim_DV=\infty$ and let $\{u_1,u_2,\dots\}$ be a linearly independent set. 
	For each $m\in\Z_{>0}$ let $V_m$ be the subspace generated by $\{u_1,\dots,u_m\}$
	and $R_m=\{r\in R:rV_m\subseteq V_m\}$. Then $R_m$ is a subring of $R$. 
	Since $R$ is dense in $V$, the map 
	\[
		R_m\to \End_D(V_m),\quad
		r\mapsto\gamma_r|_{V_m}
	\]
	is a surjective ring homomorphism. 
\end{proof}


\chapter{}

\topic{Herstein's theorem}

Our aim now is to answer the following question: When
a group algebra is algebraic? 
A partial answer is given by Herstein's theorem. 

\begin{definition}
\index{Group!locally finite}
	A group $G$ is \textbf{locally finite} if every finitely generated 
	subgroup of $G$ is finite. 
\end{definition}

If $G$ is a locally finite group, then every element $g\in G$ has finite order, as
the subgroup $\langle g\rangle$ is finite because it is finitely generated.

\begin{example}
    Every finite group is locally finite
\end{example}

\begin{example}
    The group $\Z$ is not locally finite because it is torsion-free.
\end{example}

\begin{example}
\index{Pr\"ufer's group}
	Let $p$ be a prime number. 
	The \textbf{Pr\"ufer's group}  
	\[
		\Z(p^{\infty})=\{z\in\C:z^{p^n}=1\text{ para algún $n\in\Z_{>0}$}\}, 
	\]
	formed by of all $p$-roots of one, is locally finite. 
\end{example}

\begin{example}
	Let $X$ be an infinite set and $\Sym_X$ be the set of bijective maps $X\to
	X$ moving only finitely many elements of $X$. Then 
	$\Sym_X$ is locally finite.
\end{example}

\begin{proposition}
\label{pro:exact_LI}
	Let $G$ be a group and $N$ be a normal subgroup of $G$. If $N$ and $G/N$
	are locally finite, then $G$ is locally finite.
\end{proposition}

\begin{proof}
	Let $\pi\colon G\to G/N$ be the canonical map and $\{g_1,\dots,g_n\}$ be a finite subset of $G$. 
	Since $G/N$ is locally finite, the subgroup $Q$ of $G/N$ generated by 
	$\pi(g_1),\dots,\pi(g_n)$ is finite, say
	\[
		Q=\{\pi(g_1),\dots,\pi(g_n),\pi(g_{n+1}),\dots,\pi(g_m)\}.
	\]
	For each $i,j\in\{1,\dots,n\}$ there exist $u_{ij}\in N$ and 
	$k\in\{1,\dots,m\}$ uch that $g_ig_j=u_{ij}g_k$. Let $U$ be the subgroup of $G$
	generated by $\{u_{ij}:1\leq i,j\leq n\}$. Since $N$ is locally finite, $U$ is finite. Moreover, since 
	each $g_ig_jg_l$ can be written as 
	\[
		g_ig_jg_l=u_{ij}g_kg_l=u_{ij}u_{kl}g_t=ug_t
	\]
	for some $u\in U$ and $t\in\{1,\dots,m\}$, it follows that the subgroup 
	$H$ of $G$ generated by $\{g_1,\dots,g_n\}$ is finite, as 
	$|H|\leq m|U|$. 
\end{proof}

\index{Group!solvable}
Recall that a group $G$ is
\textbf{solvable} if there exists a sequence
of subgroups 
\begin{equation}
	\label{eq:resoluble}
	\{1\}=G_0\subsetneq G_1\subsetneq \cdots\subsetneq G_n=G
\end{equation}
where each $G_i$ is normal in $G_{i+1}$ and each 
quotient $G_i/G_{i-1}$ is
abelian.
\index{Group!torsion}
A group $G$ is a \textbf{torsion} group if every element of $G$
has finite order. 

\begin{proposition}
	If $G$ is a solvable torsion group, 
	then $G$ is locally finite. 
\end{proposition}

\begin{proof}
	We proceed by induction on $n$, the length of the sequence~\eqref{eq:resoluble}. 
	If $n=1$, then $G$ is finite because it is abelian and a torsion group.
	Now assume the result holds for group with resolubility length $n-1$ and let
	$G$ be a solvable group with a sequence~\eqref{eq:resoluble}. By the inductive hypothesis, 
	the normal subgroup $G_{n-1}$ of $G$ is locally finite. Since $G/G_{n-1}$ is an abelian torsion group, 
	it is locally finite, the result now follows from Proposition \ref{pro:exact_LI}.
\end{proof}

We now prove Herstein's theorem.

\begin{theorem}[Herstein]
\index{Herstein's theorem}
	If $G$ is a locally finite group, then $K[G]$ is algebraic. Conversely, if 
	$K[G]$ is algebraic and $K$ has characteristic zero, then $G$ 
	is locally finite. 
\end{theorem}

\begin{proof}
	Assume thast $G$ is locally finite. Let $\alpha\in K[G]$. The subgroup 
	$H=\langle\supp\alpha\rangle$ is finite, as it is finitely generated. Since 
	$\alpha\in K[H]$ and $\dim_KK[H]<\infty$, the set 
	$\{1,\alpha,\alpha^2,\dots\}$ is linearly dependent. Thus $\alpha$ is
	algebraic over $K$.

	Let $\{x_1,\dots,x_m\}$ be a finite subset of $G$. Adding inverses if needed,
	we may assume that $\{x_1,\dots,x_m\}$ generates the subgroup 
	$H=\langle x_1,\dots,x_m\rangle$ as a semigroup. If
	$\alpha=x_1+\dots+x_m\in K[G]$, then, since $\alpha$ is algebraic over $K$, 
	\[
		\alpha^{n+1}=a_0+a_1\alpha+\cdots+a_n\alpha^n
	\]
	for some $n\geq0$ and $a_0,\dots,x_n\in K$. Let $w=x_{i_1}\cdots
	x_{i_{n+1}}\in H$ be a word of length $n+1$. There exist positive integers 
	$c_{i_1\cdots i_m}$ such that 
	\[
		\alpha^{n+1}=(x_1+\cdots+x_m)^{n+1}
		=\sum_{\substack{{i_1+\cdots+i_m=n+1}\\{\text{$i_j$ positive integers}}}} c_{i_1\cdots i_m}x_1^{i_1}\cdots x_{m}^{i_m}.
	\]
	Since $K$
	is of characteristic zero, it follows that $w\in\supp(\alpha^{n+1})$. Since, moreover,  
	$\alpha^{n+1}=\sum_{j=0}^na_j\alpha^j$, it follows that 
	$w\in\supp(\alpha^j)$ for some $j\in\{0,\dots,n\}$. Thus each
	word in the letters $x_j$ of length $n+1$ can be written as a word in the letters $x_j$ of 
	length $\leq n$. Therefore $H$ is finite and hence $G$ is locally finite. 
\end{proof}


\topic{Formanek's theorem, I}

\begin{exercise}
\label{xca:invertible_algebraic}
	Let $A$ be an algebraic algebra and $a\in A$.
	\begin{enumerate}
		\item $a$ is a left zero divisor if and only if $a$ is a right zero divisor.
		\item $a$ is left invertible if and only if $a$ is right invertible.
		\item $a$ is invertible if and only if $a$ is not a zero divisor.
	\end{enumerate}
\end{exercise}

\begin{exercise}
	\label{exa:norma}
	For $\alpha=\sum_{g\in G}\alpha_gg\in\C[G]$ let $|\alpha|=\sum_{g\in
	G}|\alpha_g|\in\R$. Prove the following statements:
	\begin{enumerate}
		%\item $|\trace(\alpha)|\leq |\alpha|$, 
		\item $|\alpha+\beta|\leq|\alpha|+|\beta|$, and 
		\item $|\alpha\beta|\leq|\alpha||\beta|$ 
	\end{enumerate}
	for all $\alpha,\beta\in\C[G]$.
\end{exercise}

\begin{theorem}[Formanek]
	\label{thm:FormanekQ}
	\index{Formanek's theorem}
	Let $G$ be a group. If every element of $\Q[G]$ is invertible or 
	a zero divisor, then $G$ is locally finite. 
\end{theorem}

\begin{proof}
	Let $\{x_1,\dots,x_n\}$ be a finite subset of $G$. Adding inverses if needed, we may assume that 
	$\{x_1,\dots,x_n\}$ generates the subgroup
	$H=\langle x_1,\dots,x_n\rangle$ as a semigroup. Let 
	\[
		\alpha=\frac{1}{2n}(x_1+\cdots+x_n)\in\Q[G]
	\]

	We claim that $1-\alpha\in\Q[G]$ is invertible. If not, then it is a zero divisor. If there exists 
	$\delta\in\Q[G]$ such that $\delta(1-\alpha)=0$, then 
	$\delta=\delta\alpha$. Since  
	\[
		|\delta|=|\delta\alpha|\leq|\delta||\alpha|=|\delta|/2,
	\]
	it follows that $\delta=0$. Similarly, $(1-\alpha)\delta=0$ implies
	$\delta=0$. 
	
	Let $\beta=(1-\alpha)^{-1}\in\Q[G]$.  For each$k$ let  
	\[
		\gamma_k=(1+\alpha+\cdots+\alpha^k)-\beta.
	\]
	Then 
	\begin{align*}
		\gamma_k(1-\alpha)&=(1+\alpha+\cdots+\alpha^k-\beta)(1-\alpha)\\
		&=(1+\alpha+\cdots+\alpha^k)(1-\alpha)-\beta(1-\alpha)=-\alpha^{k+1}
	\end{align*}
	and thus  
	$\gamma_k=-\alpha^{k+1}\beta$. Since  
	\[
		|\gamma_k|=|-\alpha^{k+1}\beta|\leq|\beta||\alpha^{k+1}|=\frac{|\beta|}{2^{k+1}},
	\]
	it follows that $\lim_{k\to\infty}|\gamma_k|=0$. 

	We now prove that $H\subseteq\supp\beta$. If
	$H\not\subseteq\supp\beta$, let $h\in H\setminus\supp\beta$.  Assume that 
    $h=x_{i_1}\cdots x_{i_m}$ is a word in the letters $x_j$ of length $m$. Let 
    $c_j$ be the coefficient of $h$ in $\alpha^j$. Then $c_0+\cdots+c_k$ is the 
	coefficient of $h$ in $\gamma_k$, but 
	\[
		|\gamma_k|\geq c_0+c_1+\cdots+c_k\geq c_m>0
	\]
	for all $k\geq m$, as each $c_j$ is non-negative, a contradiction to 
	$|\gamma_k|\to 0$ si $k\to\infty$.
\end{proof}

\topic{Formanek's theorem, II}

The \textbf{tensor product} of the vector spaces (over $K$) $U$ and $V$ 
is the quotient vector space $K[U\times V]/T$, where $K[U\times V]$ 
is the vector space with basis 
\[
\{(u,v):u\in U,v\in V\}
\]
and $T$ is the subspace 
generated by elements of the form 
\[
		(\lambda u+\mu u',v)-\lambda(u,v)-\mu(u',v),\quad
		(u,\lambda v+\mu v')-\lambda(u,v)-\mu(u,v')
	\]
for $\lambda,\mu\in K$, $u,u'\in U$ and $v,v'\in V$.
The tensor product of $U$ and $V$ will be denoted by $U\otimes_KV$ or 
$U\otimes V$ when the base field it is clear from the context. For $u\in U$ and 
$v\in V$ we write $u\otimes v$ to denote the coset $(u,v)+T$.

\begin{theorem}
	Let $U$ and $V$ be vector spaces. Then there exists a bilinear map 
	$U\times V\to U\otimes V$, $(u,v)\mapsto u\otimes v$, such that 
	each element of $U\otimes V$ is a finite sum of the form 
	\[
		\sum_{i=1}^N u_i\otimes v_i
	\]
	for some $u_1,\dots,u_N\in U$ and $v_1,\dots,v_N\in V$. 
	Moreover, if $W$ is a vector space and $\beta\colon U\times V\to W$ is a bilinear map, 
	there exists a linear map 
	$\overline{\beta}\colon U\otimes V\to W$ such that $\overline{\beta}(u\otimes
	v)=\beta(u,v)$ for all $u\in U$ and $v\in V$.
\end{theorem}

\begin{proof}
    By definition, the map
    \[
	U\times V\to U\otimes V,\quad
	(u,v)\mapsto u\otimes v,
	\]
	is bilinear. From the definitions it follows that
	$U\otimes V$ is a finite linear combination of elements of the form 
	$u\otimes v$, where $u\in U$ and $v\in V$. Since $\lambda(u\otimes
	v)=(\lambda u)\otimes v$ for all $\lambda\in K$, the first claim follows.

	Since the elements of $U\times V$ form a basis of $K[U\times V]$, there exists
	a linear map 
	\[
		\gamma\colon K[U\times V]\to W,\quad
	\gamma(u,v)=\beta(u,v). 
	\]
	Since $\beta$ is bilinear by assumption, $T\subseteq\ker\gamma$. It follows that there exists 
	a linear map $\overline{\beta}\colon U\otimes V\to
	W$ such that  
	\[
	\begin{tikzcd}
		K[U\times V] \arrow[r]\arrow[d] & W \\
		U\otimes V\arrow[ur, dashrightarrow]
	\end{tikzcd}
	\]
	commutes. In particular, $\overline{\beta}(u\otimes v)=\beta(u,v)$. 
\end{proof}

\begin{exercise}
	\label{xca:tensorial_unicidad}
	Prove that the properties of the previous theorem characterize tensor products up to isomorphism. 
\end{exercise}

Some properties:
%Observemos
%que todo elemento de $U\otimes V$ es una suma finita
%de la forma 
%\[
%	\sum_{i=1}^N u_i\otimes v_i
%\]
%para $N\in\N$, $u_i\in U$ y $v_i\in V$. Esta expresión no es única. Vale además
%que $u\otimes 0=0=0\otimes v$ para todo $u\in U$ y $v\in V$.

\begin{proposition}
	Let $\varphi\colon U\to U_1$ and $\psi\colon V\to V_1$ be linear maps. There
	exists a unique linear map 
	$\varphi\otimes\psi\colon U\otimes V\to U_1\otimes V_1$ such that
	\[
		(\varphi\otimes\psi)(u\otimes v)=\varphi(u)\otimes\psi(v)
	\]
	for all $u\in U$ and $v\in V$.
\end{proposition}

\begin{proof}
	Since $U\times V\to U_1\otimes V_1$,
	$(u,v)\mapsto\varphi(u)\otimes\psi(v)$, is bilinear, there exists a linear map
	$U\otimes V\to U_1\otimes V_1$, $u\otimes
	v\to\varphi(u)\otimes\psi(v)$. Thus 
	\[
		\sum u_i\otimes v_i\mapsto\sum\varphi(u_i)\otimes\psi(v_i)
	\]
	is well-defined. 
\end{proof}

\begin{exercise}
    Prove the following statements:
	\begin{enumerate}
		\item $(\varphi\otimes\psi)(\varphi'\otimes\psi')=(\varphi\varphi')\otimes(\psi\psi')$.
		\item If $\varphi$ and $\psi$ are isomorphisms, then 
			$\varphi\otimes\psi$ is an isomorphism. 
		\item $(\lambda\varphi+\lambda'\varphi')\otimes\psi=\lambda\varphi\otimes\psi+\lambda'\varphi'\otimes\psi$.
		\item $\varphi\otimes(\lambda\psi+\lambda'\psi')=\lambda\varphi\otimes\psi+\lambda'\varphi\otimes\psi'$.
		\item If $U\simeq U_1$ and $V\simeq V_1$, then $U\otimes V\simeq U_1\otimes V_1$.
	\end{enumerate}
\end{exercise}

The following proposition is extremely useful:

\begin{proposition}
	If $U$ and $V$ are vector spaces, then  
	$U\otimes V\simeq V\otimes U$.
\end{proposition}

\begin{proof}
	Since $U\times V\to V\otimes U$, $(u,v)\mapsto v\otimes u$, is bilinear, there exists 
	a linear map $U\otimes V\to V\otimes U$, $u\otimes
	v\mapsto v\otimes u$. Similarly, there exists a linear map 
	$V\otimes U\to U\otimes V$, $v\otimes u\mapsto
	u\otimes v$. Thus $U\otimes V\simeq V\otimes U$.
\end{proof}

\begin{exercise}
	\label{xca:UxVxW}
    Prove that $(U\otimes V)\otimes W\simeq U\otimes(V\otimes W)$.
\end{exercise}

\begin{exercise}
	\label{xca:UxK}
	Prove that $U\otimes K\simeq K\simeq K\otimes U$.
\end{exercise}

\begin{proposition}
	\label{pro:U_LI}
	Let $U$ and $V$ be vector spaces. 
	If $\{u_1,\dots,u_n\}$ is a linearly independent subset of $U$ and 
	$v_1,\dots,v_n\in V$ is such that $\sum_{i=1}^n u_i\otimes v_i=0$, then 
    $v_i=0$ for all $i\in\{1,\dots,n\}$.
\end{proposition}

\begin{proof}
	Let $i\in\{1,\dots,n\}$ and 
	\[
	f_i\colon U\to K,
	\quad
	f_i(u_j)=\delta_{ij}=\begin{cases}
	1 & \text{if $i=j$},\\
	0 & \text{otherwise}.
	\end{cases}
	\]
	Since the map $U\times V\to V$, $(u,v)\mapsto f_i(u)v$, is bilinear, there exists 
	a linear map 
	$\alpha_i\colon U\otimes V\to V$ such that $\alpha_i(u\otimes
	v)=f_i(u)v$. Thus
	\[
		v_i=\sum_{j=1}^n\alpha_i(u_j\otimes v_j)=\alpha_i\left(\sum_{j=1}^nu_j\otimes v_j\right)=0.\qedhere
	\]
\end{proof}

\begin{exercise}
	\label{xca:uxv=0}
	Prove that $u\otimes v=0$ and $v\ne 0$ imply $u=0$.
\end{exercise}

\begin{theorem}
    Let $U$ and $V$ be vector spaces. 
	If $\{u_i:i\in I\}$ is a basis of $U$ and $\{v_j:j\in J\}$ is a basis of $V$, then 
	$\{u_i\otimes v_j:i\in I,j\in J\}$ is a basis of $U\otimes
	V$.
\end{theorem}

\begin{proof}
	The $u_i\otimes v_j$ are generators of $U\otimes V$, as  
    $u=\sum_i\lambda_iu_i$ and $v=\sum_j\mu_jv_j$ imply 
	$u\otimes v=\sum_{i,j}\lambda_i\mu_ju_i\otimes v_j$. 
	We now prove that the $u_i\otimes v_j$ are linearly independent. We need to show that
	each finite subset of the $u_i\otimes v_j$
	is linearly independent. If $\sum_k\sum_l\lambda_{kl}u_{i_k}\otimes
	v_{j_l}=0$, then 
	$0=\sum_{k}u_{i_k}\otimes\left(\sum_{l}\lambda_{kl}v_{j_l}\right)$. Since  
	the $u_{i_k}$ are linearly independent, Proposition~\ref{pro:U_LI}
	implies that $\sum_{l}\lambda_{kl}v_{j_l}=0$. Thus $\lambda_{kl}=0$ for all 
	$k,l$, as the $v_{j_l}$ are linearly independent. 
\end{proof}

If $U$ and $V$ are finite-dimensional vector spaces, then 
\[
	\dim(U\otimes V)=(\dim U)(\dim V).
\]

\begin{corollary}
	If $\{u_i:i\in I\}$ is basis of $U$, then every element of $U\otimes V$
	can be written uniquely as a finite sum $\sum_{i}u_i\otimes v_i$.
\end{corollary}

\begin{proof}
	Every element of $U\otimes V$ is a finite sum 
	$\sum_i x_i\otimes y_i$, where $x_i\in U$ and $y_i\in V$. If  
	$x_i=\sum_j\lambda_{ij}u_j$, then 
	\[
		\sum_i x_i\otimes y_i=\sum_i\left(\sum_j\lambda_{ij}u_j\right)\otimes y_i		
		=\sum_j u_j\otimes\left(\sum_i\lambda_{ij}y_i\right).\qedhere 
	\]
\end{proof}

%\begin{corollary}
%	Todo elemento no nulo de $U\otimes V$ puede escribirse como una suma finita
%	$\sum_{i=1}^N u_i\otimes v_i$ para un conjuntos $\{u_i:1\leq i\leq
%	N\}\subseteq U$ y $\{v_i:1\leq i\leq N\}\subseteq V$ linealmente
%	independientes.
%\end{corollary}
%
%\begin{proof}
%	tomar $N$ minimal	
%\end{proof}

\begin{exercise}
\label{xca:tensor_algebras}
    Let $A$ and $B$ be algebras. Prove that $A\otimes B$ 
    is an algebra with 
	\[
		(a\otimes b)(x\otimes y)=ax\otimes by.
	\]
\end{exercise}

% \begin{proof}
% 	Para $x\in A$, $y\in B$ consideramos $R_x\otimes R_y\in\End_K(A\otimes B)$.
% 	Como la función $A\times B\to\End_K(A\otimes B)$, $(x,y)\mapsto R_x\otimes
% 	R_y$, es bilineal, existe una función lineal $\varphi\colon A\otimes
% 	B\to\End_K(A\otimes B)$, $\varphi(x\otimes y)=R_x\otimes R_y$. Para $u,v\in A\otimes B$ definimos
% 	\[
% 		uv=\varphi(v)(u).
% 	\]
% 	Esta operación es bilineal pues por ejemplo
% 	\[
% 		u(v+w)=\varphi(v+w)(u)=(\varphi(v)+\varphi(w))(u)=\varphi(v)(u)+\varphi(w)(u)=uv+uw.
% 	\]
% 	Además
% 	$(a\otimes b)(x\otimes y)=\varphi(x\otimes y)(a\otimes b)=(R_x\otimes R_y)(a\otimes b)=ax\otimes by$.
% 	Un cálculo sencillo muestra que este producto es asociativo.
% \end{proof}

\begin{exercise}
    Prove the following statements:
	\begin{enumerate}
		\item $A\otimes B\simeq B\otimes A$.
		\item $(A\otimes B)\otimes C\simeq A\otimes(B\otimes C)$.
		\item $A\otimes K\simeq A\simeq K\otimes A$.
		\item If $A\otimes A_1$ and $B\otimes B_1$, then $A\otimes B\simeq A_1\otimes B_1$.
	\end{enumerate}
\end{exercise}

Some examples:

\begin{proposition}
	If $G$ and $H$ are groups, then $K[G]\otimes K[H]\simeq K[G\times H]$.
\end{proposition}

\begin{proof}
	The set $\{g\otimes h:g\in G,h\in H\}$ is a basis of $K[G]\otimes K[H]$ and 
	the elements of $G\times H$ form a basis of $K[G\times H]$. There exists a linear isomorphism 
	\[
	K[G]\otimes K[H]\to K[G\times H], 
	\quad 
	g\otimes h\mapsto (g,h),
	\]
	that is multiplicative. Thus $K[G]\otimes K[H]\simeq K[G\times H]$
	as algebras. 
\end{proof}

\begin{proposition}
\label{pro:AKX=AX}
	If $A$ is an algebra, then $A\otimes K[X]\simeq A[X]$.	
\end{proposition}

\begin{proof}
	Each element of $A\otimes K[X]$ can be written uniquely as a finite sum of
	the form $\sum a_i\otimes X^i$. Routine calculations show that 
	$A\otimes K[X]\mapsto A[X]$, $\sum a_i\otimes X^i\mapsto \sum a_iX^i$, is a 
	linear algebra isomprhism. 
\end{proof}

\begin{exercise}
\label{xca:AM=MA}
	Prove that if $A$ is an algebra, then $A\otimes M_n(K)\simeq M_n(A)$. In
	particular, $M_n(K)\otimes M_m(K)\simeq M_{nm}(K)$.
\end{exercise}

Proposition \ref{pro:AKX=AX} and Exercise \ref{xca:AM=MA} 
are examples of a procedure known as \textbf{scalar extensions}. 

\begin{theorem}
	Let $A$ be an algebra over $K$ and $E$ be an extension of $K$ (this just simply means that
	$K$ is a subfield of $E$). Then 
	$A^E=E\otimes_KA$ is an algebra over $E$ with respect to
	the scalar multiplication 
	\[
		\lambda(\mu\otimes a)=(\lambda\mu)\otimes a,
	\]
	for all $\lambda,\mu\in E$ and $a\in A$.
\end{theorem}

\begin{proof}
	Let $\lambda\in E$. Since $E\times A\to E\otimes_KA$,
	$(\mu,a)\mapsto (\lambda\mu)\otimes a$, is $K$-bilinear, there exists 
	a linear map $E\otimes_KA\to E\otimes_KA$, $\mu\otimes a\mapsto
	(\lambda\mu)\otimes a$. The scalar multiplication is then well-defined and 
	\[
	\lambda(u+v)=\lambda u+\lambda v
	\]
	for all $\lambda\in E$ and $u,v\in E\otimes_KA$. Moreover, 
	\[
	(\lambda+\mu)u=\lambda u+\mu u,
	\quad
	(\lambda\mu)u=\lambda(\mu u),
	\quad
	\lambda(uv)=(\lambda u)v=u(\lambda v)
	\]
	for all $u,v\in E\otimes_KA$ and $\lambda,\mu\in E$.
\end{proof}

\begin{exercise}
    Prove the following statements:
    \begin{enumerate}
		\item $\{1\}\otimes A$ is a subalgebra of $A^E$ isomorphic to $A$.
		\item If $\{a_i:i\in I\}$ is a basis of $A$, then $\{1\otimes a_i:i\in
			I\}$ is a basis of $A^E$.
	\end{enumerate}
\end{exercise}

\begin{exercise}
	Prove that if $G$ is a group and $K$ is a subfield of $E$, then
	$E\otimes_K K[G]\simeq E[G]$.
\end{exercise}

Now we prove Formanek's theorem:

\begin{theorem}[Formanek]
	\index{Formanek's theorem}
	Let $K$ be a field of characteristic zero and let $G$ be a group. 
	If every element of $K[G]$ is invertible or a zero divisor, 
	then $G$ is locally finite. 
\end{theorem}

\begin{proof}
	Since $K$ is of characteristic zero, $\Q\subseteq K$. Then $K[G]\simeq
	K\otimes_{\Q}\Q[G]$. Each $\beta\in K\otimes_{\Q}\Q[Q]$ can be written
	uniquely as 
	\[
		\beta=1\otimes\beta_0+\sum k_i\otimes\beta_i,
	\]
	where $\{1,k_1,k_2,\dots,\}$ is a basis of $K$ as a $\Q$-vector space. 
	Let $\alpha\in\Q[G]$ and let $\beta\in K[G]$ be such that $\alpha\beta=1$. Since
	\[
	1\otimes 1=(1\otimes\alpha)\beta=1\otimes \alpha\beta_0+\sum k_i\otimes \alpha\beta_i,
	\]
	it follows that $\alpha\beta_0=1$. Similarly, if
	$\alpha\beta=0$, then $\alpha\beta_j=0$ for all $j$. Since 
	each $\alpha\in\Q[G]$ is invertible or a zero divisor, Formanek's theorems 
	for $\Q$ applies. 
\end{proof}

% \section*{Rickart's theorem}

% En esta sección vamos a demostrar que para cualquier grupo $G$ el radical de
% Jacobson de $\C[G]$ es cero. Demostraremos también que el radical de Jacobson
% de $\R[G]$ es cero.

% \begin{definition}
% 	\index{Anillo!con involución}
% 	\index{Involución!de un anillo}
% 	Sea $R$ un anillo. Una \textbf{involución} del anillo $R$ es un morfismo
% 	aditivo $R\to R$, $x\mapsto x^*$, tal que $x^{**}=x$ y $(xy)^*=y^*x^*$ para
% 	todo $x,y\in R$.
% \end{definition}

% De la definición se deduce inmediatamente que si $R$ es unitario, entonces
% $1^*=1$.

% \begin{example}
% 	La conjugación $z\mapsto\overline{z}$ es una involución de $\C$.
% \end{example}

% \begin{example}
% 	La trasposición $X\mapsto X^T$ es una involución del
% 	anillo $M_n(K)$.
% \end{example}

% \begin{example}
% 	Sea $G$ un grupo. Entonces
% 	$\left(\sum_{g\in G}\alpha_gg\right)^*=\sum_{g\in G}\overline{\alpha_g}g^{-1}$ 
% 	es una involución de $\C[G]$.
% \end{example}

% Dado un grupo $G$, se define la \textbf{traza} de un elemento
% $\alpha=\sum_{g\in G}\alpha_gg\in K[G]$ como $\trace(\alpha)=\alpha_1$. Es
% fácil ver que $\trace\colon K[G]\to K$, $\alpha\mapsto\trace(\alpha)$ es una
% función $K$-lineal tal que $\trace(\alpha\beta)=\trace(\beta\alpha)$.

% \begin{exercise}
% 	Sea $G$ un grupo finito y $K$ un cuerpo tal que su característica no divide al orden de $G$.
% 	Demuestre las siguientes afirmaciones:
% 	\begin{enumerate}
% 		\item Si $\alpha\in K[G]$ es nilpotente, entonces $\trace(\alpha)=0$.
% 		\item Si $\alpha\in K[G]$ es idempotente, entonces $\trace(\alpha)=\dim
% 			K[G]\alpha/|G|$.
% 	\end{enumerate}
% \end{exercise}

% \begin{exercise}
% 	Demuestre que 
% 	$\langle\alpha,\beta\rangle=\trace(\alpha\beta^*)$, $\alpha,\beta\in\C[G]$, 
% 	define un producto interno en $\C[G]$.
% \end{exercise}

% \begin{lemma}
% 	\label{lem:algebraico}
% 	Sea $G$ un grupo. Si $J(\C[G])\ne 0$, entonces existe $\alpha\in J(\C[G])$ tal que 
% 	$\trace(\alpha^{2^m})\in\R_{\geq1}$ 
% 	para todo $m\geq1$.
% \end{lemma}

% \begin{proof}
% 	Sea $\alpha=\sum_{g\in G}\alpha_gg\in\C[G]$. Entonces	
% 	\[
% 		\trace(\alpha^*\alpha)
% 		=\sum_{g\in G}\overline{\alpha_g}\alpha_g
% 		=\sum_{g\in G}|\alpha_g|^2\geq|\alpha_1|^2
% 		=|\trace(\alpha)|^2.
% 	\]
% 	Al usar esta fórmula para algún $\alpha$ tal que $\alpha^*=\alpha$ y usar
% 	inducción se obtiene que $\trace(\alpha^{2^m})\geq|\trace(\alpha)|^{2^m}$
% 	para todo $m\geq1$. 

% 	Sea $\beta=\sum_{g\in G}\beta_gg\in J(\C[G])$ tal que $\beta\ne0$. Como
% 	$\trace(\beta^*\beta)=\sum_{g\in G}|\beta_g|^2\ne0$ y $J(\C[G])$ es un ideal, 
% 	\[
% 		\alpha=\frac{\beta^*\beta}{\trace(\beta^*\beta)}\in J(\C[G]).
% 	\]
% 	Este elemento $\alpha$ cumple que $\alpha^*=\alpha$ y $\trace(\alpha)=1$.
% 	Luego $\trace(\alpha^{2^m})\geq 1$ para todo $m\geq1$.
% \end{proof}

% El ejercicio~\ref{exa:norma} implica que $\C[G]$ con
% $\dist(\alpha,\beta)=|\alpha-\beta|$ es un espacio métrico. En este espacio
% métrico, la función $\C[G]\to\C$, $\alpha\mapsto \trace(\alpha)$, es una
% función continua.

% \begin{lemma}
% 	\label{lem:phi_diferenciable}
% 	Sea $\alpha\in J(\C[G])$. La función
% 	\[
% 		\varphi\colon\C\to\C[G],\quad
% 		\varphi(z)=(1-z\alpha)^{-1},
% 	\]
% 	es continua, diferenciable y $\varphi(z)=\sum_{n\geq0}\alpha^nz^n\in\C[G]$ si $|z|$
% 	es suficientemente pequeño.
% \end{lemma}

% \begin{proof}	
% 	Sean $y,z\in\C$. Como $\varphi(y)$ y $\varphi(z)$ conmutan, 
% 	\begin{equation}
% 		\label{eq:Rickart}
% 		\begin{aligned}
% 			\varphi(y)-\varphi(z)&=\left( (1-z\alpha)-(1-y\alpha)\right)(1-y\alpha)^{-1}(1-z\alpha)^{-1}\\
% 			&=(y-z)\alpha\varphi(y)\varphi(z).
% 		\end{aligned}
% 	\end{equation}
% 	Entonces $|\varphi(y)|\leq|\varphi(z)|+|y-z||\alpha\varphi(y)||\varphi(z)|$ y luego
% 	\[
% 		|\varphi(y)|\left( 1-|y-z||\alpha\varphi(z)|\right)\leq|\varphi(z)|.
% 	\]
% 	Fijado $z$ podemos elegir $y$ suficientemente cerca de $z$ de forma tal que
% 	se cumpla que  $1-|y-z||\alpha\varphi(z)|\geq1/2$. Luego
% 	$|\varphi(y)|\leq2|\varphi(z)|$. De la igualdad~\eqref{eq:Rickart} se
% 	obtiene entonces $|\varphi(y)-\varphi(z)|\leq2|y-z||\alpha||\varphi(z)|^2$
% 	y luego $\varphi$ es una función continua. Por la
% 	expresión~\eqref{eq:Rickart}, 
% 	\[
% 	\varphi'(z)
% 	=\lim_{y\to z}\frac{\varphi(y)-\varphi(z)}{y-z}
% 	=\lim_{y\to z}\alpha\varphi(y)\varphi(z)
% 	=\alpha\varphi(z)^2
% 	\]
% 	para todo $z\in\C$.

% 	Si $z$ es tal que $|z||\alpha|=|z\alpha|<1$, entonces 
% 	\[
% 		\varphi(z)-\sum_{n=0}^Nz^n\alpha^n
% 		=\varphi(z)\left(1-(1-z\alpha)\sum_{n=0}^Nz^n\alpha^n\right)
% 		=\varphi(z)(z\alpha)^{N+1}
% 	\]
% 	y luego
% 	\[
% 		\left|\varphi(z)-\sum_{n=0}^Nz^n\alpha^n\right|\leq|\varphi(z)||z\alpha|^{N+1}.
% 	\]
% 	Como $\varphi(z)$ está acotada cerca de $z=0$, se concluye que
% 	$\left|\varphi(z)-\sum_{n=0}^Nz^n\alpha^n\right|\to0$ si $N\to\infty$.
% \end{proof}

% Estamos en condiciones de demostrar el teorema de Rickart:

% \begin{theorem}[Rickart]
% 	\index{Teorema!de Rickart}
% 	Si $G$ es un grupo, entonces $J(\C[G])=0$.
% \end{theorem}

% \begin{proof}
% 	Sea $\alpha\in J(\C[G])$ y sea $\varphi(z)=(1-\alpha z)^{-1}$. Sea 
% 	$f\colon\C\to \C$ dada por
% 	$f(z)=\trace\varphi(z)=\trace\left((1-z\alpha)^{-1}\right)$. Por el lema~\ref{lem:phi_diferenciable},
% 	$f(z)$ es una función entera tal que $f'(z)=\trace(\alpha\varphi(z)^2)$ y
% 	\begin{equation}
% 		\label{eq:Taylor}
% 		f(z)=\sum_{n=0}^\infty z^n\trace(\alpha^n)
% 	\end{equation}
% 	si $|z|$ es suficientemente pequeño. En particular, la
% 	igualdad~\eqref{eq:Taylor} es la expansión en serie de Taylor para $f(z)$
% 	en el origen. Esto implica que esta serie tiene radio de convergencia
% 	infinito y converge a $f(z)$ para todo $z\in\C$. En particular,
% 	\begin{equation}
% 		\label{eq:limite}
% 		\lim_{n\to\infty}\trace(\alpha^n)=0.
% 	\end{equation}
% 	Por otro lado, si $\alpha\ne0$ el lema~\ref{lem:algebraico} implica que
% 	$\trace(\alpha^{2^m})\geq1$ para todo $m\geq0$, lo que contradice el límite
% 	calculado en~\eqref{eq:limite}. Luego $\alpha=0$.
% \end{proof}

% Para demostrar un corolario necesitamos dos lemas:

% \begin{lemma}[Nakayama]
% 	\label{lem:Nakayama}
% 	\index{Lema!de Nakayama}
% 	Sea $R$ un anillo unitario y sea $M$ un $R$-módulo finitamente generado. Si
% 	$J(R)M=M$, entonces $M=0$.
% \end{lemma}

% \begin{proof}
% 	Supongamos que $M$ está generado por los elementos $x_1,\dots,x_n$. Como $x_n\in M=J(R)M$, 
% 	existen $r_1,\dots,r_n\in J(R)$ tales que $x_n=r_1x_1+\cdots+r_nx_n$, es decir
% 	$(1-r_n)x_n=\sum_{j=1}^{n-1}r_jx_j$. 
% 	Como $1-r_n$ es inversible, existe $s\in R$ tal que $s(1-r_n)=1$. Luego
% 	$x_n=\sum_{j=1}^{n-1}sr_jx_j$ 
% 	y entonces $M$ está generado por $x_1,\dots,x_{n-1}$. Al repetir este
% 	procedimiento una cierta cantidad finita de veces, se obtiene que $M=0$.
% \end{proof}

% \begin{lemma}
% 	\label{lem:Rickart}
% 	Sea $\iota\colon R\to S$ un morfismo de anillos unitarios. Si 
% 	\[
% 	S=\iota(R)x_1+\cdots+\iota(R)x_n,
% 	\]
% 	donde cada $x_j$ cumple que $x_jy=yx_j$ para todo $y\in\iota(R)$, entonces
% 	$\iota(J(R))\subseteq J(S)$.
% \end{lemma}

% \begin{proof}
% 	Veamos que $J=\iota(J(R))$ actúa trivialmente en cada $S$-módulo simple $M$.
% 	Si $M$ es un $S$-módulo simple, escribimos $M=Sm$ para algún $m\ne0$. Es
% 	claro que $M$ es un $R$-módulo con $r\cdot m=\iota(r)m$. Como
% 	\[
% 		M=Sm=(\iota(R)x_1+\cdots+\iota(R)x_n)m=\iota(R)(x_1m)+\cdots+\iota(R)(x_nm),
% 	\]
% 	$M$ es finitamente generado como $\iota(R)$-módulo. Además $J(R)\cdot
% 	M=JM=\iota(J)M$ es un $S$-submódulo de $M$ pues
% 	\[
% 		x_j(JM)=(x_jJ)M=(Jx_j)M=J(x_jM)\subseteq JM.
% 	\]
% 	Como $M\ne0$, el lema de Nakayama implica que $J(R)\cdot M\subsetneq M$. Luego,
% 	como $M$ es un $S$-módulo simple, se concluye que $J(R)M=0$.
% \end{proof}

% \begin{corollary}
% 	Si $G$ es un grupo, entonces $J(\R[G])=0$. 
% \end{corollary}

% \begin{proof}
% 	Sea $\iota\colon \R[G]\to\C[G]$ la inclusión canónica. Como 
% 	\[
% 	\C[G]=\R[G]+i\R[G],
% 	\]
% 	el lema~\ref{lem:Rickart} y el teorema de Rickart implican que
% 	$\iota(J(\R[G]))\subseteq J(\C[G])=0$. Luego $J(\R[G])=0$ pues $\iota$ es
% 	inyectiva. 
% \end{proof}



\chapter{}
\label{09}

\topic{Local rings}

In this section, we will consider arbitrary rings with one. 

\begin{definition}
    \index{Ring!local}
    A ring is said to be \textbf{local} if it contains only one maximal left ideal. 
\end{definition}

Division rings are local rings. 

% \begin{exercise}
%     Let $R$ be a commutative ring with one and $P$ be a prime ideal. Let  
%     $S=R\setminus P$. Then the localization $S^{-1}R$ is a local ring with maximal ideal $S^{-1}P$. 
% \end{exercise}

\begin{theorem}
\label{thm:local}
    Let $R$ be a ring and $I=R\setminus\mathcal{U}(R)$. The following
    statements are equivalent:
    \begin{enumerate}
        \item $R$ is local.
        \item $R/J(R)$ is a division ring.
        \item $I=J(R)$.
        \item $I$ is an ideal of $R$.
    \end{enumerate}
\end{theorem}

\begin{proof}
    We first prove $1)\implies2)$. Let $M$ be the maximal left ideal of $R$. Then $J(R)=M$. 
    Let $x\not\in M$. Then $R=Rx+M$, so $1=rx+m$ for some $r\in R$ and $m\in M$. Thus  
    $r+M$ is a left inverse of $x+M$. In particular, 
    $r\not\in M$. Since $R=Rr+M$, there exists $y\in R$ such that $1=yr$. Therefore
    $y+M$ is a left inverse of $r+M$. Thus 
    \begin{align*}
    y+M&=(y+M)(1+M)=(y+M)(r+M)(x+M)\\
    &=(yr+M)(x+M)=(1+M)(x+M)=x+M
    \end{align*}
    and hence $x+M$ is invertible. 

    Now we prove $2)\implies3)$. Clearly $J(R)\subseteq I$. 
    Conversely, let $x\in I$. If $x\not\in J(R)$, then
    $x+J(R)\ne J(R)$. Since $R/J(R)$ is a division ring, 
    $x+J(R)\in\mathcal{U}(R/J(R))$. In particular, $1-xy\in J(R)$ and hence 
    $xy=1-(1-xy)\in\mathcal{U}(R)$. Thus $1=(xy)z=x(yz)$ for some $z\in R$ and therefore $x\not\in I$, a contradiction. 
    
    It is trivial that $3)\implies4)$. 

    Finally, we prove $4)\implies 1)$. 
    Let $M$ be a maximal left ideal of $R$. Then $M\subseteq I$. Since $M$ 
    is maximal and $I$ is in particular a left ideal of $R$, 
    it follows that $M=I$. 
\end{proof}

\begin{definition}
    \index{Idempotent}
    An element $x$ of a ring is said to be \textbf{idempotent} 
    if $x^2=x$.   
\end{definition}

\index{Trivial idempotent}
Examples of idempotents are 0 and 1. 
An idempotent $x$ is said to be \textbf{non-trivial} if $x\not\in\{0,1\}$. 

\begin{exercise}
\label{xca:idempotents_modpm}
    Let $p$ be a prime number and $m>0$. 
    Prove that the only idempotents of $\Z/p^m$ are 0 and 1. 
\end{exercise}


\begin{exercise}
    \label{xca:idempotents_modn}
    How many idempotent does $\Z/n$ have?
\end{exercise}

\begin{exercise}
\label{xca:lifting_idempotents}
    Let $R$ be a ring with one and $I$ be an ideal of $R$. 
    We say that an idempotent $x\in R/I$ can be lifted if $x=e+I$ for
    some idempotent $e$ of $R$. 
    Prove that if every element of $I$ is nilpotent, then every 
    idempotent of $R/I$ can be lifted. 
\end{exercise}

The previous exercise shows that if $R$ is left artinian, 
every idempotent of $R/J(R)$ can be lifted to $R$. 

\begin{lemma}
\label{lem:J(R)_nil}
    Let $R$ be a left artinian ring. Then $J(R)$ is nil. 
\end{lemma}

\begin{proof}
    Let $x\in J(R)$. The sequence $Rx\supseteq Rx^2\supseteq\cdots$ stabilizes, so
    $Rx^n=Rx^{n+1}$ for some $n$. In particular, there exists $r\in R$ 
    such that $x^n=rx^{n+1}$. This implies that $(1-rx)x^n=0$. Since $x\in J(R)$, 
    the element $1-rx$ is invertible. Hence $x^n=0$.  
\end{proof}

\begin{theorem}
\label{thm:local_idempotent}
    Let $R$ be a left artinian ring. Then $R$ is local if and only if 
    $R$ has no non-trivial idempotents. 
\end{theorem}

\begin{proof}
    Let us first prove $\implies$. For this implication, we do not need to use that 
    $R$ is left artinian. Let $x\in R$ be an idempotent. Then $x(1-x)=0$. If $x\in\mathcal{U}(R)$, then 
    $x=1$. If $1-x\in\mathcal{U}(R)$, then $x=0$. If $x\not\in \mathcal{U}(R)$ and $1-x\not\in\mathcal{U}(R)$, 
    then, since $R\setminus\mathcal{U}(R)$ is an ideal of $R$, 
    it follows that 
    $1=x+1-x\not\in\mathcal{U}(R)$, a contradiction. 

    Now we prove $\impliedby$. By the previous lemma, $J(R)$ is nil. 
    By the previous exercise, every idempotent of $R/J(R)$ can be lifted. Thus $R/J(R)$ has 
    no non-trivial idempotents. On the other hand, by Artin--Wedderburn, 
    \[
    R/J(R)\simeq\prod_{i=1}^kM_{n_i}(D_i)
    \]
    for some $n_1,\dots,n_k\geq1$ and division rings $D_1,\dots,D_k$. Then 
    $k=n_1=1$, as $R/J(R)$ has no non-trivial idempotents. Since $R/J(R)$ is a division ring, 
    $R$ is local by the previous theorem. 
\end{proof}

\begin{theorem}
    The center of a local ring is local. 
\end{theorem}

\begin{proof}
    Let $R$ be a local ring. By Theorem \ref{thm:local}, $J(R)=R\setminus\mathcal{U}(R)$.  
    We need to prove that $Z(R)\setminus\mathcal{U}(Z(R))=J(Z(R))$. 
    We first note that
    \begin{equation}
    \label{eq:U(Z(R))}
        \mathcal{U}(Z(R))=Z(R)\cap\mathcal{U}(R).
    \end{equation}
    
    We claim that $Z(R)\cap J(R)\subseteq J(Z(R))$.  
    Let $x\in Z(R)\cap J(R)$. Let $z\in Z(R)$. Since $x\in J(R)$, $1-zx\in\mathcal{U}(R)$.
    Moreover, $1-zx\in Z(R)$. Thus 
    \[
    1-zx\in Z(R)\cap\mathcal{U}(R)=\mathcal{U}(Z(R)).
    \]
    Hence $x\in J(Z(R))$. 

    To prove the theorem it is enough to show that 
    $Z(R)\setminus\mathcal{U}(Z(R))=J(Z(R))$. Let us prove the non-trivial inclusion. 
    Let $x\in Z(R)\setminus\mathcal{U}(Z(R))$.  Then 
    \eqref{eq:U(Z(R))} implies that 
    $x\not\in\mathcal{U}(R)$. 
    By Theorem \ref{thm:local}, 
    $x\in J(R)$. Then $x\in J(R)\cap Z(R)\subseteq J(Z(R))$. 
\end{proof}

\begin{exercise}
\label{eq:local_center}
    Let $R$ be a local ring. Prove that 
    $Z(R)=J(R)\subseteq J(Z(R))$.  
\end{exercise}

\begin{exercise}
\label{xca:local_right}
    Prove that a ring is local if and only if it contains only one maximal right ideal.
\end{exercise}

\begin{exercise}
\label{xca:non_local1}
    Find a non-local ring with a unique maximal ideal. 
\end{exercise}

\begin{exercise}
\label{xca:non_local2}
    Let $R$ be a ring with at least three elements. 
    If $|\mathcal{U}(R)|=1$, then $R$ is not local. 
\end{exercise}

\index{Ring!Von Neumann regular}
A ring $R$ is said to be \textbf{Von Neumann regular} if  
for every non-zero $r\in R$, $r=rxr$ for some $x\in R$. 

\begin{exercise}
\label{xca:VonNeumann_local}
    Prove that a ring $R$ is local if and only if $R$ is a division ring. 
\end{exercise}

\begin{exercise}
\label{xca:nilp_or_unit}
    Let $R$ be a ring such that every element of $R$ is either 
    nilpotent or a unit. Prove that $R$ is local. 
\end{exercise}

\index{Ring!semilocal}
A ring $R$ is said to be \textbf{semilocal} if $R/J(R)$ is left artinian. 

\begin{exercise}
\label{xca:semilocal}
    Prove the following statements:
    \begin{enumerate}
        \item Every local ring is semilocal.
        \item $R$ is semilocal if and only if $R/J(R)$ is semisimple.
        \item If $R$ has finitely many maximal ideals, then $R$ is semilocal. 
        \item If $R_1,\dots,R_k$ are rings, then $\oplus_{i=1}^k R_i$ is semilocal
            if and only if each $R_i$ is semilocal. 
    \end{enumerate}
\end{exercise}

\begin{example}
\label{xca:semilocal_commutative}
    Let $R$ be a ring such that $R/J(R)$ is commutative. Prove
    that $R$ is semilocal if and only if $R$ has finitely many maximal ideals. 
\end{example}

\topic{*When is group algebra local?}

\begin{proposition}
    \label{pro:augmentation}
    Let $R$ be a commutative ring with one. 
    Let $f\colon G\to H$ be a group homomorphism with kernel $K$. Then
    \[
    \varphi\colon R[G]\to R[H],
    \quad 
    \sum\lambda_ig_i\mapsto \sum\lambda_if(g),
    \]
    is a ring homomorphism with kernel the ideal 
    of $R[G]$ generated by $\{k-1:k\in K\}$. 
\end{proposition}

\begin{proof}
    A direct calculation shows that the map $\varphi$ is a well-defined ring homomorphism. Let 
    $S=\{k-1:k\in K\}$. Then $(S)\subseteq \ker\varphi$. 
    
    Let us show that 
    $\ker\varphi\subseteq (S)$. Let $\alpha=\sum r_ig_i\in\ker\varphi$. Then 
    \[
    \varphi(\alpha)=\sum r_if(g_i)=0.
    \]
    Let 
    $\{Kg_{i_1},\dots,Kg_{i_k}\}$ be the subset of pairwise distinct cosets 
    of $Kg_1,\dots,Kg_n$. Write  
    \[
    \alpha=\sum\sum s_{ij}k_{ij}g_{i_j}
    \]
    for some $s_{ij}\in R$ and $k_{ij}\in K$. Then 
    \begin{equation}
    \label{eq:0=s_ij}
    0=\varphi(\alpha)=\sum\sum s_{ij}\varphi(k_{ij}g_{i_j})
    =\sum\sum s_{ij}f(g_{i_j}),
    \end{equation}
    as $K=\ker f$. Note that 
    \[
    f(g_{i_j})=f(g_{i_k})\implies 
    g_{i_j}g_{i_k}^{-1}\in K\implies 
    g_{i_j}K=g_{i_k}K.
    \]
    Thus $f(g_{i_j})\ne f(g_{i_k})$ for $j\ne k$. Since $R[H]$ is a free $R$-module 
    with basis $\{h:h\in H\}$, Equality
    \eqref{eq:0=s_ij}
    implies that $\sum_i s_{ij}=0$ for all $j$. Thus
    \[
    \alpha=\sum\sum s_{ij}k_{ij}g_{i_j}=\sum\sum s_{ij}(k_{ij}-1)g_{i_j}\in (S).\qedhere
    \]
\end{proof}

\begin{corollary}
\label{cor:R[G/N]}
    Let $R$ be a commutative ring with one. If 
    $G$ is a group and $N$ is a normal subgroup of $G$, then
    \[
    R[G/N]\simeq R[G]/I,
    \]
    where $I$ is the ideal of $R[G]$ generated by $\{n-1:n\in N\}$. 
\end{corollary}

\begin{proof}
    Apply the previous proposition to the canonical map $\pi\colon G\to G/N$ to get
    a ring homomorphism $\varphi\colon R[G]\to R[G/N]$. The kernel of $\varphi$ is the ideal $I$ 
    generated by the set $\{g-1:g\in\ker\pi=N\}$. Since 
    $\pi$ is surjective, $\varphi$ is surjective. Then 
    the claim follows from the first isomorphism theorem. 
\end{proof}

Let $K$ be a field and $G$ be a group. We write $A(K[G])$ to denote
the ideal of $K[G]$ generated by the set $\{g-1:g\in G\}$. This ideal is known as the
\textbf{augmentation ideal} of $K[G]$. 

\begin{corollary}
\label{cor:local_K[N]_and_K[G/N]}
    Let $K$ be a field. 
    Let $G$ be a group and $N$ be a central subgroup of $G$. If $K[N]$ and $K[G/N]$ are local, 
    then $K[G]$ is local. 
\end{corollary}

\begin{proof}
    By Corollary \ref{cor:R[G/N]}, $K[G/N]\simeq K[G]/I$, where 
    $I$ is the ideal of $K[G]$ generated by $\{n-1:n\in N\}$. Since $N\subseteq Z(G)$, 
    $I$ is central in $K[G]$. Note that 
    \[
    I=A(K[N])K[G].
    \]
    Let $\alpha\in A(K[G])$. 
    Since $K[G/N]$ is local, $A(K[G/N])$ is nil by Theorem \ref{thm:local}. Since $K[G]/I\simeq K[G/N]$, this implies that 
    there exists $m$ such that $\alpha^m\in I$. Since $K[N]$ is local,  
    $A(K[N])$ is nil by Theorem \ref{thm:local}. Moreover, $K[N]$ is central in $K[G]$, because $N\subseteq Z(G)$. This implies that $I=A(K[N])K[G]$ is also nil. In particular, 
    $\alpha$ is nil. Hence 
    $K[G]$ is nil and therefore $K[G]$ is local by Theorem \ref{thm:local}. 
\end{proof}


\begin{exercise}
\label{xca:augmentation}
    Let $R$ be a commutative ring with one and $G$ be a group. 
    Prove that 
    the map $R[G]\to R$, $\sum_{g\in G}r_gg\mapsto\sum_{g\in G}r_g$, is a surjective
    ring homomorphism with kernel $A(R[G])$.  
\end{exercise}

\begin{lemma}
    Let $K$ be a field and $G$ be a finite group. 
    The following statements are equivalent: 
    \begin{enumerate}
        \item $K[G]$ is local. 
        \item $A(K[G])\subseteq J(K[G])$. 
        \item $A(K[G])$ is nil.
        \item $A(K[G])=J(K[G])$. 
    \end{enumerate}
\end{lemma}

\begin{proof}
    Let us prove that $1)\implies 2)$. Since $K[G]$ is local, 
    $R\setminus\mathcal{U}(K[G])=J(K[G])$ by Theorem \ref{thm:local}. Since $K[G]\setminus\mathcal{U}(K[G])$ contains 
    every proper ideal of $K[G]$, $A(K[G])\subseteq J(K[G])$. 

    We now prove that $2)\implies 3)$. Since $G$ is finite, $K[G]$ is artinian. By Lemma \ref{lem:J(R)_nil}, 
    $J(K[G])$ is nil. Hence $A(K[G])$ is nil. 

    We now prove that $3)\implies 4)$. Since $J(K[G])$ contains every nil ideal (see Proposition~\ref{pro:nilJ}), 
    $A(K[G])\subseteq J(K[G])$. On the other hand, $K[G]/A(K[G])\simeq K$. Since $K$ is a field, the correspondence theorem
    implies that $A(K[G])=J(K[G])$. 

    Finally, we prove that $4)\implies 1)$. Since $A(K[G])=J(K[G])$, Exercise \ref{xca:augmentation} implies that 
    $K[G]/J(K[G])\simeq K$. Since $K$ is a field, 
    it is, in particular, a division ring. Thus $K[G]$ is local by Theorem \ref{thm:local}.     
\end{proof}


\begin{exercise}
\label{xca:C_p:local}
    Let $p$ be a prime number, $K$ be a field of characteristic $p$ and 
    $G$ be a cyclic group of order $p$. Prove that $K[G]$ is local. 
\end{exercise}


\begin{exercise}
\label{xca:K[G]_domain_easy}
    Let $K$ be a field and $G$ be a finite group. Then 
    $K[G]$ is a domain if and only if $|G|=1$. 
\end{exercise}


\begin{theorem}
    Let $K$ be a field and $G$ be a non-trivial finite group. 
    Then $K[G]$ is local if and only if $K$ is of characteristic $p>0$ and $G$ is a $p$-group. 
\end{theorem}

\begin{proof}
    Let us first prove $\implies$. Assume first that $K$ is a field of characteristic zero. 
    By Maschke's theorem, $J(K[G])=\{0\}$. By Theorem \ref{thm:local}, 
    $K[G]$ is a division ring. In particular, $K[G]$ is a domain, 
    a contradiction (see Exercise \ref{xca:K[G]_domain_easy}).

    Assume now that $K$ is of characteristic $p>0$. 
    Let $q$ be a prime divisor of $|G|$ and $g\in G$ an element of order $q$. 
    Since 
    \[
    (1-g)(1+\cdots+g^{q-1})=1-g^q=0,
    \]
    $1-g\not\in\mathcal{U}(K[G])$ and $1+\cdots+g^{q-1}\not\in\mathcal{U}(K[G])$. It follows
    that $1-g^m\not\in\mathcal{U}(K[G])$ for all $m\geq0$. By Theorem \ref{thm:local}, 
    $K[G]\setminus J(K[G])$ is an ideal. Thus  
    \[
    q1_G=1+\cdots+g^{q-1}+\sum_{m=1}^{q-1}(1-g^m)\not\in\mathcal{U}(K[G])
    \]
    If $q\ne 0$ in $K$, then $q1_G\in\mathcal{U}(K[G])$. Hence $q=0$ in $K$ and
    therefore $p$ divides $q$. We conclude that $G$ is a $p$-group. 

    We now prove $\impliedby$. Let $G$ be a $p$-group and $K$ be a field of characteristic $p>0$. We proceed
    by induction on $|G|$. 
    If $|G|=p$, $K[G]$ is a local ring (see Exercise \ref{xca:C_p:local}).
    % \[
    % K[G]\simeq K[X]/(X^p-1)\simeq K[X]/((X-1)^p), 
    % \]
    % as $X^p-1=(X-1)^p$. But $K[X]/((X-1)^p)$ is a commutative local ring  
    If $|G|>p$, let $Z=Z(G)$. Since $G$ is a $p$-group, $|Z|\geq p$. Let $N$ be a subgroup of $Z$ of order $p$. 
    Then $|N|<|G|$ and $|G/N|<|G|$. By the inductive hypothesis, both 
    $K[N]$ and $K[G/N]$ are local. By Corollary \ref{cor:local_K[N]_and_K[G/N]}, $K[G]$ is local too. 
\end{proof}

\topic{*Hurewitz' theorem}

\begin{theorem}[Hurewicz]
    \label{thm:Hurewicz}
    \index{Hurewicz' theorem}
    Let $G$ be a group and $I$ be the augmentation ideal of $\Z[G]$. 
    Then $G/[G,G]\simeq I/I^2$ as (abelian) groups. 
\end{theorem}

\begin{proof}
    Let $\varphi\colon G\to I/I^2$, $g\mapsto g-1_G+I^2$. Since $g-1_G\in I$ for all $g\in G$, $\varphi$ is well-defined. The map $\varphi$ is a group homomorphism. Since 
    $(g-1_G)(h-1_G)\in I^2$, 
    \begin{align*}
    \varphi(gh) &= gh-1_G+I^2\\
    &=gh-1_G-(g-1_G)(h-1_G)+I^2+I^2\\
    &=g-1_G+h-1_G+I^2\\
    &=\varphi(g)+\varphi(h)
    \end{align*}
    holds for all $g,h\in G$. 

    Since $[G,G]\subseteq\ker\varphi$, there exists a group homomorphism
    \[
    \overline{\varphi}\colon G/[G,G]\to I/I^2,\quad 
    g[G,G]\mapsto g-1_G+I^2.
    \]
    We claim that $\overline{\varphi}$ is an isomorphism. 
    Let us construct the inverse of $\overline{\varphi}$. Let 
    \[
    \psi\colon I\to G/[G,G],\quad 
    \sum_{g\in G}m_g(g-1_G)\mapsto \left(\prod_{g\in G}g^{m_g}\right)[G,G].
    \]
    Since $G/[G,G]$ is abelian, the map $\psi$ is well-defined, that is
    the order of the factors in $\prod_{g\in G}g^{m_g}$ does not matter. Note that 
    $I^2\subseteq\ker\psi$, as 
    $\{(g-1_G)(h-1_G):g,h\in G\}$ generates the additive group $I^2$ 
    and 
    \begin{align*}
        \psi((g-1_G)(h-1_G))&=\psi( (gh-1_G)-(g-1_G)-(h-1_G))\\
        &=(ghg^{-1}h^{-1})[G,G]\\
        &=[G,G].
    \end{align*}
    Therefore there exists a group homomorphism
    \[
    \overline{\psi}\colon I/I^2\to G/[G,G],\quad 
    \sum_{g\in G}m_g(g-1_G)+I^2\mapsto \left(\prod_{g\in G}g^{m_g}\right)[G,G].
    \]
    A direct calculation shows that $\overline{\psi}$ is the inverse 
    of $\overline{\varphi}$. 
\end{proof}

\topic{Andrunakevic--Rjabuhin's theorem}
%\begin{exercise}
%\label{xca:reduced}
%     Let $R$ be a ring and $I$ be an ideal of $R$.
%     Prove that $I$ is prime if and only if $xRy\subseteq I$ implies
%     either $x\in I$ or $y\in I$. 
% \end{exercise}

% \begin{sol}{xca:reduced}
%     Let $A$ and $B$ be ideals such that $AB\subseteq I$. If 
%     $A\not\subseteq I$ and $B\not\subseteq J$, let 
%     $x\in A\setminus P$ and $y\in B\setminus P$. Then
%     $xRy\subseteq AB\subseteq I$, a contradiction. 
%     Conversely, if $xRy\subseteq I$ and $x\not I$ and $y\not\in I$, 
%     then $A=(x)\not\subseteq I$ and $B=(y)\not\subseteq P$.
% \end{sol}

\begin{definition}
\index{Ring!reduced}
     A ring $R$ is \textbf{reduced} if 
     has no non-zero nilpotent elements. 
\end{definition}

Every commutative domain is reduced. 

\begin{example}
    The ring $\Z\times\Z$ with the usual operations 
    is reduced but not a domain. 
\end{example}

\begin{example}
    The ring $\Z/6$ is reduced. However, $\Z/4$ is not reduced. 
\end{example}

\begin{exercise}
\label{xca:reduced}
    Prove that a ring $R$ is \textbf{reduced} if and only 
    if for all $r\in R$ such that $r^2=0$ one has $r=0$.
\end{exercise}

%\begin{exercise}
%\label{xca:reduced_Zn}
%    Let $n\geq2$. Then $\Z/n$ is reduced 
%    but not a domain if and only if $n$ is square-free 
%    but not prime.
%\end{exercise}

\begin{exercise}
    \label{xca:reduced_RX}
    Let $R$ be a commutative ring that is reduced but not a domain.
    Prove that $R[X]$ is reduced but not a domain. 
\end{exercise}

The previous exercise and induction 
shows that if $R$ is reduced but not a domain, 
then so is $R[X_1,\dots,X_n]$. 

\begin{example}
    Let $R=\Z/3\times\Z/3$ with
    operations $(a,b)+(c,d)=(a+c,b+d)$ and 
    $(a,b)(c,d)=(ac,ad+bc)$. Then $R$ is a commutative
    ring with identity $(1,0)$. Since 
    $(0,1)$ is a non-zero nilpotent element, $R$ is not reduced. 
\end{example}

\begin{definition}
\index{Ideal!reduced}
    Let $R$ be a ring and $I$ be an ideal of $R$. 
    Then $I$ is \textbf{reduced} if $R/I$ is a reduced ring. 
\end{definition}

Let $R$ be a ring and 
$I$ be a reduced ideal of $R$. If $ab\in I$, then 
$ba\in I$. In fact, since $ab\in I$, 
$(ba)^2=b(ab)a\in I$.
Since $R/I$ is reduced, $ba\in I$. 

 \begin{theorem}[Andrunakevic--Rjabuhin]
 \label{thm:AndrunakevicRjabuhin}
 \index{Andrunakevic--Rjabuhin's theorem}
 	Let $R$ be a non-zero ring. If $R$ is reduced, there exists
 	an ideal $I$ of $R$ such that 
 	then $R/I$ has no non-zero zero-divisors. 
 \end{theorem}

 Let $R$ be a ring and $I$ be an ideal of $R$. If $S$ 
 is a subset of $R$, the \emph{left annihilator} of $S$
 modulo $I$ is the set $\{r\in R:rS\subseteq I\}$.  

 \begin{lemma}
    Let $R$ be a ring and $I$ be a reduced ideal. 
    If $S\subseteq R$ is a subset, then 
    the left annihilator of $S$ modulo $I$ 
    is a reduced ideal. 
 \end{lemma}

\begin{proof}
    We need to show that $A=\{r\in R:rS\subseteq I\}$ 
    is a reduced ideal. 
    A straightforward calculation shows that $A$ is 
    a left ideal. We claim that $A$ is a right ideal. Let $r\in R$
    and $a\in A$. Then 
    $as\in I$ for all $s\in S$. Since $I$ is reduced, $sa\in I$ for all $s\in S$. Since 
    $I$ is an ideal of $R$, $sar\in I$
    for all $s\in S$. Using again 
    that $I$ is reduced, 
    $ars\in I$ for all $s\in S$. Thus 
    $ar\in A$. 
    
    We now claim that $A$ is reduced. If $a^2\in A$, then 
    $aas=a^2s\in I$ for all $s\in S$. 
    Since $I$ is reduced, $asa\in I$ for
    all $s\in S$. Thus $(as)^2=(asa)s\in I$ for all $s\in S$. 
    Since $I$ is reduced, $as\in I$ for all $s\in S$. Hence $a\in A$. 
\end{proof}

Similarly, if $S$ is a subset of a ring $R$, then 
the \emph{right annihilator} 
$\{r\in R:Sr\subseteq I\}$ 
of $S$ modulo $I$ 
is a reduced ideal. 

\begin{proof}[Proof of Theorem \ref{thm:AndrunakevicRjabuhin}]
    Let $x\in R\setminus\{0\}$. Let $X$ 
    be the set of reduced ideals $I$ such that 
    $x\not\in I$. Since $R$ is reduced, $\{0\}$ 
    is a reduced ideal and hence $X\ne\emptyset$. 
    A standard application of Zorn's lemma shows that
    there exists a maximal element $M\in X$. 
    
    We claim that $R/M$ has no non-zero divisors. If not, 
    there exist $a,b\in R$ such that $ab\in M$, $a\not\in M$ 
    and $b\not\in M$. Let $A$ be the left annihilator of $\{b\}$ 
    modulo $M$ and $B$ be the right annihilator of $\{a\}$ 
    modulo $M$. By the previous lemma, $A$ and $B$ 
    are reduced ideals of $R$. Since  
    $a\in A$, $M\subsetneq A$. Similarly, since 
    $b\in B$, $M\subsetneq B$. Moreover, $AB\subseteq M$. 
    Since $x\in A\cap B$, $x^2\in AB\subseteq M$. Since 
    $M$ is reduced, $x\in M$, a contradiction. 
\end{proof}

\begin{exercise}
    Prove that a reduced ring is a subdirect product
    of rings without no non-zero divisors. 
\end{exercise}

\begin{exercise}
    Is the ring $\C[\Z/2]$ reduced? 
\end{exercise}

% Let $G=\langle g:g^2=1\rangle$. If $(a+bg)^2=0$, then
% $(a^2+b^2)+(2ab)g=0$. Thus $a=b=0$. 

\begin{problem}
\label{prob:reduced}
    Let $G$ be a torsion-free group. Is
    $K[G]$ is reduced?
\end{problem}

Problem \ref{prob:reduced} is related to other important
open problems about group algebras 
(e.g. zero-divisors, units, 
indempotents and semisimplicity of group
rings).

\begin{exercise}
\label{xca:reduced_central}
    Prove that idempotents of reduced rings are central. 
\end{exercise}

The previous exercise is used to solve the following problem.

\begin{exercise}
\label{xca:x^3=x}
    Let $R$ be a ring such that $x^3=x$ for all $x\in R$. Prove that
    $R$ is commutative. 
\end{exercise}

Exercise \ref{xca:x^3=x} is hard. 
Even harder is the following exercise:

\begin{exercise}
\label{xca:x^4=x}
    Let $R$ be a ring such that $x^4=x$ for all $x\in R$. Prove
    that $R$ is commutative. 
\end{exercise}

%Other exercises about reduced rings. 

%\begin{exercise}
%\label{xca:reduced}
%    Prove that a ring is reduced if 
%    and only it has no non-zero nilpotent elements. 
%\end{exercise}

\begin{exercise}
\label{xca:reduced=>semiprime}
    Reduced rings are semiprime.
\end{exercise}
 
\begin{theorem}
\label{thm:reduced}
    Let $K$ be a field and $G$ be a group. If $K[G]$
    is reduced, then every finite subgroup of $G$ is normal. 
\end{theorem}

\begin{proof}
    Let $H=\{h_1,\dots,h_n\}$ be a finite normal subgroup of $G$. 
    We claim that $n=|H|$ is invertible in $K$. If $\ch K=0$, this 
    is clear. If $\ch K=p>0$ and $n$ is not invertible in $K$, 
    then $p$ divides $n=|H|$. By Cauchy's theorem, 
    there exists an element $h\in H$ of order $n$, that is 
    $|h|=n$. Since $(1-h)^p=1-h^p=0$ and $K[G]$ is reduced,
    $h=1$, a contradiction. 
    
    Let $\alpha=\frac{1}{n}\sum_{i=1}^nh_i\in K[G]$. Then
    \[
    \alpha^2=\frac{1}{n^2}\sum_{i=1}^n\sum_{j=1}^nh_ih_j
    =\frac{1}{n^2}\sum_{i=1}^nn\alpha=\alpha.
    \]
    Thus $\alpha$ is idempotent. As idempotent 
    element of reduced rings are central (Exercise \ref{xca:reduced_central}), 
    $g\alpha g^{-1}=\alpha$ for all $g\in G$. If $g\in G$, 
    then 
    \[
    \sum_{i=1}^n gh_ig^{-1}=\sum_{i=1}^n h_i.
    \]
    It follows that $H$ is normal in $G$, 
    as for each $i\in\{1,\dots,n\}$ 
    there exists $j\in\{1,\dots,n\}$ such that 
    $gh_ig^{-1}=h_j\in H$. 
\end{proof}

\begin{example}
    If $K$ is a field, then $K[\Sym_3]$ is not reduced. 
    In fact, 
    if 
    \[
    \alpha=(12)+(123)-(132)-(13),
    \]
    then 
    $\alpha^2=0$. 
\end{example}

\begin{exercise}
    Prove that the converse of Theorem \ref{thm:reduced} 
    does not hold. 
\end{exercise}

\topic{Rickart's theorem}

We now consider Jacobson's semisimplicity problem. 

\begin{openproblem}
\label{Jacobson's semisimplicity problem}
Let $G$ be a group and $K$ be a field. When $J(K[G])=\{0\}$?
\end{openproblem}

As an application of Amitsur's theorem \ref{thm:Amitsur}, 
we prove that 
complex group algebras have null Jacobson radical.
This is known as 
Rickart's theorem. The original proof found by Rickart 
uses complex analysis. Here, however, 
we present an algebraic proof. 

\begin{theorem}[Rickart]
\index{Rickart's theorem}
\label{thm:Rickart}
    Let $G$ be a group. Then $J(\C[G])=\{0\}$.
\end{theorem}

To prove the theorem, we need a lemma.

\begin{lemma}
Let $G$ be a group. Then $J(\C[G])$ is nil.        
\end{lemma}

\begin{proof}
    We need to show that every element of $J(\C[G])$ is nilpotent. 
    If $G$ is countable, then the result follows from Amitsur's theorem \ref{thm:Amitsur}. So assume that 
    $G$ is not countable. Let $\alpha\in J(\C[G])$, say
    \[
    \alpha=\sum_{i=1}^n\lambda_ig_i,
    \]
    where $\lambda_1,\dots,\lambda_n\in\C$ and $g_1,\dots,g_n\in G$. Let $H=\langle g_1,\dots,g_n\rangle$.
    Then $\alpha\in \C[H]$ and $H$ is countable. We claim that $\alpha\in J(\C[H])$. Decompose
    $G$ as a disjoint union 
    \[
    G=\bigcup_\lambda x_\lambda H
    \]
    of cosets of $H$ in $G$. Then $\C[G]=\bigoplus_\lambda x_\lambda\C[H]$ and
    hence $\C[G]=\C[H]\oplus K$ for some right module $K$ over $\C[H]$ (this follows
    from the fact that one of the cosets is that of $H$). Since $\alpha\in J(\C[G])$, for each 
    $\beta\in\C[H]$ there exists $\gamma\in\C[G]$ such that 
    $\gamma(1-\beta\alpha)=1$. Write $\gamma=\gamma_1+\kappa$ for $\gamma_1\in\C[H]$ and $\kappa\in K$. Then
    \[
    1=\gamma(1-\beta\alpha)=\gamma_1(1-\beta\alpha)+\kappa(1-\beta\alpha)
    \]
    and hence $\kappa(1-\beta\alpha)\in K\cap \C[H]=\{0\}$, as $\beta\in\mathbb{C}[H]$. 
    Since $1=\gamma_1(1-\beta\alpha)$, it follows that
    $\alpha\in J(\C[H])$ and the lemma follows from Amitsur's theorem \ref{thm:Amitsur}.  
\end{proof}

We now prove the theorem. 

\begin{proof}[Proof of Theorem \ref{thm:Rickart}]
    For $\alpha=\sum_{i=1}^n\lambda_ig_i\in\C[G]$ let 
    \[
    \alpha^*=\sum_{i=1}^n\overline{\lambda_i}g_i^{-1}.
    \]
    Then $\alpha\alpha^*=0$ if and only if $\alpha=0$ and, moreover, 
    $(\alpha\beta)^*=\beta^*\alpha^*$ for all $\beta\in\C[G]$. 
    Assume that $J(\C[G])\ne\{0\}$ and let $\alpha\in J(\C[G])\setminus\{0\}$. Then
    $\beta=\alpha\alpha^*\in J(\C[G])$, as $J(\C[G])$ is an ideal of $\C[G]$. Moreover, the previous 
    lemma implies that $\beta$ is nilpotent. Note that $\beta\ne 0$, as $\alpha\ne0$. Now  
    \[
    (\beta^m)^*=(\beta^*)^m=\beta^m
    \]
    for all $m\geq1$. If there exists $k\geq2$ such that $\beta^k=0$ and $\beta^{k-1}\ne 0$, then
    \[
    \beta^{k-1}\left(\beta^{k-1}\right)^*=\beta^{2k-2}=0
    \]
    and hence $\beta^{k-1}=0$, a contradiction. Thus $\beta=0$ and therefore $\alpha=0$. 
\end{proof}

\begin{exercise}
	If $G$ is a group, then $J(\R[G])=0$. 
\end{exercise}

% To obtain a consequence of Rickart's theorem we need two lemmas. 

% \begin{lemma}[Nakayama]
% 	\label{lem:Nakayama}
% 	\index{Nakayama's lemma}
% 	Let $R$ be a unitary ring and $M$ be a finitely generated module. If 
% 	$J(R)\cdot M=M$, then $M=\{0\}$.
% \end{lemma}

% \begin{proof}
%     Since $M$ is finitely generated, we may assume that 
% 	$M=(x_1,\dots,x_n)$. Since $x_n\in M=J(R)\cdot M$, 
% 	there exist $r_1,\dots,r_n\in J(R)$ such that $x_n=r_1\cdot x_1+\cdots+r_n\cdot x_n$, that is 
% 	$(1-r_n)\cdot x_n=\sum_{j=1}^{n-1}r_j\cdot x_j$. 
% 	Since $1-r_n$ is invertible, there exists $s\in R$ such that $s(1-r_n)=1$. Thus 
% 	$x_n=\sum_{j=1}^{n-1}(sr_j)\cdot x_j$ 
% 	and hence $M=(x_1,\dots,x_{n-1})$. Repeating this procedure several times 
% 	one obtains $M=\{0\}$.
% \end{proof}

% \begin{lemma}
% 	\label{lem:Rickart}
% 	Let $\iota\colon R\to S$ be a homomorphism of unitary rings. If	
% 	\[
% 	S=\iota(R)x_1+\cdots+\iota(R)x_n,
% 	\]
% 	where each $x_j$ is such that $x_jy=yx_j$ for all $y\in\iota(R)$, then 
% 	$\iota(J(R))\subseteq J(S)$.
% \end{lemma}

% \begin{proof}
% 	We claim that $J=\iota(J(R))$ acts trivially on each simple $S$-module $M$.
% 	If is $M$ is a simple module over $S$, then, in particular, $M=S\cdot m$ for some $m\ne0$. 
% 	Now $M$ is a module over $R$ with $r\cdot m=\iota(r)\cdot m$. Since 
% 	\[
% 		M=S\cdot m=(\iota(R)x_1+\cdots+\iota(R)x_n)\cdot m=\iota(R)\cdot (x_1\cdot m)+\cdots+\iota(R)\cdot (x_n\cdot m),
% 	\]
% 	it follows that 
% 	$M$ is finitely generated as a module over $\iota(R)$. Moreover, 
% 	\[
% 	J(R)\cdot
% 	M=J\cdot M=\iota(J)\cdot M
% 	\]
% 	is an $S$-submodule of $M$, as 
% 	\[
% 		x_j\cdot (J\cdot M)=(x_j J)\cdot M=(J x_j)\cdot M=J\cdot (x_j\cdot M)\subseteq J\cdot M.
% 	\]
% 	Since $M\ne\{0\}$, Nakayama's lemma implies that $J(R)\cdot M\subsetneq M$. The simplicity of 
% 	the $S$-module $M$ implies that $J(R)\cdot M=\{0\}$.
% \end{proof}

% We now obtain the following consequence of Rickart's theorem. 

% \begin{theorem}
% 	If $G$ is a group, then $J(\R[G])=0$. 
% \end{theorem}

% \begin{proof}
% 	Let $\iota\colon \R[G]\to\C[G]$ be the canonical inclusion. Since 
% 	\[
% 	\C[G]=\R[G]+i\R[G],
% 	\]
% 	Lemma~\ref{lem:Rickart} and Rickart's theorem imply that 
% 	$\iota(J(\R[G]))\subseteq J(\C[G])=0$. Thus $J(\R[G])=0$, as $\iota$ is injective. 
% \end{proof}


\begin{definition}
	\index{Ring!semiprime}
	A ring $R$ \textbf{semiprime} if 
	$aRa=\{0\}$ implies $a=0$.
\end{definition}

\begin{proposition}
	Let $R$ be a ring. The following statements are equivalent: 
	\begin{enumerate}
		\item $R$ is semiprime.
		\item If $I$ is a left ideal such that $I^2=\{0\}$, then $I=\{0\}$.
		\item If $I$ is an ideal such that $I^2=\{0\}$, then $I=\{0\}$.
		\item $R$ does not contain non-zero nilpotent ideals.
	\end{enumerate}
\end{proposition}

\begin{proof}
	We first prove that $1)\implies2)$. If $I^2=\{0\}$ y $x\in I$, then
	$xRx\subseteq I^2=\{0\}$ and thus $x=0$. The implications $2)\implies3)$
	and $4)\implies3)$ are both trivial. Let us prove that $3)\implies4)$.  If
	$I$ is a non-zero nilpotent ideal, let $n\in\Z_{>0}$ be minimal such that
	$I^n=\{0\}$.  Since $(I^{n-1})^2=\{0\}$, it follows that $I^{n-1}=\{0\}$, a
	contradiction.  Finally, we prove that $3)\implies1)$. Let $a\in R$ be such
	that $aRa=\{0\}$. Then $I=RaR$ is an ideal of $R$ such that $I^2=\{0\}$. Thus 
	$RaR=\{0\}$. This means that $Ra$ and $aR$ are ideals such that
	$(Ra)R=R(aR)=\{0\}$ (for example, $R(aR)\subseteq RaR=\{0\}\subseteq aR$). 
	Moreover, since $(Ra)(Ra)=\{0\}$ and $(aR)(aR)=\{0\}$, it follows that
	$aR=Ra=\{0\}$. 
	This implies that $\Z a$ is an ideal of $R$, as $R(\Z a)\subseteq \Z(Ra)=\{0\}$ and 
	$(\Z a)R\subseteq aR=\{0\}$. Now $(\Z a)(\Z a)\subseteq (\Z a)R=\{0\}$ and hence
	$a=0$, as $\Z a=\{0\}$. 
\end{proof}

Two consequences:

\begin{exercise}
	A commutative ring is semiprime if and only if it does not contain non-zero
	nilpotent elements. 
\end{exercise}

% tomar $\alpha$ tal que $\alpha^2=0$ y sea $A=K[G]\alpha$. Como $A^2=0$, $A=0$ y entonces $\alpha=0$.

\begin{exercise}
\label{xca:D_semiprime_semiprimitive}
	Let $D$ be a division ring. 
	\begin{enumerate}
		\item $D[X]$ is semiprime and semiprimitive.
		\item $D[\![X]\!]$ is semiprime and it is not semiprimitive.
	\end{enumerate}
\end{exercise}

% We will prove 
% in Lecture \ref{09} (Corollary \ref{cor:C[G]_semiprime}) 
% that the if $G$ is a group, 
% then the ring $\C[G]$ is semiprime. 

\begin{corollary}
\label{cor:C[G]_semiprime}
	The ring $\C[G]$ is semiprime.
\end{corollary}

\begin{proof}
	Since $J(\C[G])=\{0\}$ by Rickart's theorem and the Jacobson radical
	contains every nil ideal by Proposition~\ref{pro:nilJ}, it follows that
	$\C[G]$ does not contain non-trivial nil ideals. Thus $\C[G]$ does not
	contain non-trivial nilpotent ideals and hence $\C[G]$ is semiprime.
\end{proof}

\begin{exercise}
	Prove that $Z(\C[G])$ is semiprime.
\end{exercise}

We now characterize when complex group algebras 
are left artinian. For that purpose,
we need a lemma. This is similar to one of the implications proved in Proposition \ref{pro:semisimple}. However,
in the arbitrary setting we are considering, we need to use Zorn's lemma. 

\begin{lemma}
    Let $M$ be a semisimple module and $N$ be a submodule. 
    Then $N$ is a direct summand.
\end{lemma}

\begin{proof}[Sketch of the proof]
    Let $M=\oplus_{i\in I}M_i$ be a direct sum of simple modules  
    and let $i\in I$. 
    Since $N\cap M_i$ is a submodule of $M_i$ and $M_i$ is simple, it follows
    that $N\cap M_i=\{0\}$ or $N\cap M_i=M_i$. If
    $N\cap M_i=M_i$ for all $i\in I$, then $N=M$ and the lemma is proved. So we may assume
    that there exists $i\in I$ such that $N\cap M_i=\{0\}$. Let $X$ be the set
    of subsets $J$ of $I$ such that $N\cap (\oplus_{j\in J}M_j)=\{0\}$. Our assumptions
    imply that $X$ is non-empty. Zorn's lemma implies the existence of 
    a maximal element $K$. Let $N_1=\oplus_{k\in K}M_k$. We claim that
    $N\oplus N_1=M$. If not, there exists $i\in I$ such that
    $M_i\not\subseteq N\oplus N_1$. The simplicity of $M_i$ implies that
    $M_i\cap (N\oplus N_1)=\{0\}$, which contradicts the maximality of $K$. 
\end{proof}

A direct application of the lemma proves that
complex group algebras of infinite groups are never semisimple. 

\begin{proposition}
    \label{pro:KGsemisimple}
    If $G$ is an infinite group, then $\C[G]$ is not semisimple. 
\end{proposition}

\begin{proof}
	Assume that $R=\C[G]$ is semisimple.  Let $I$ 
	be the augmentation ideal of $R$, that is
	\[
	I=\left\{\alpha=\sum_{g\in G}\lambda_gg\in R:\sum_{g\in G}\lambda_g=0\right\}.
	\]
	By the previous lemma, 
	there exists a non-zero ideal $J$ such that 
	$R=I\oplus J$. Since $R$ is unitary, there exist $e\in I$ and $f\in J$ such that
	$1=e+f$. If
	$x\in I$, then $x=xe+xf$ and hence $xf=x-xe\in I\cap J=\{0\}$. Since 
	$x=xe$ for all $x\in I$, it follows that $e=e^2$. Similarly, one proves
	that $f^2=f$. Moreover, $ef=0$, as $ef\in I\cap J=\{0\}$.  Since $I$ 
	is the augmentation ideal of $R$ and $If=(Re)f=R(ef)=\{0\}$ (note that $I=Re$ because $x=xe$ for all $x\in I$), we conclude that
	$(g-1)f=0$
	for all $g\in G$, as $g-1\in I$. If $f=\sum_{h\in
	G}\lambda_hh$ (finite sum), then  
	\[
	f=gf=\sum_{h\in G}\lambda_h(gh)=\sum_{h\in
	G}\lambda_{g^{-1}h}h.
	\]
	Thus $\lambda_h=\lambda_{g^{-1}h}$ for all $g,h\in G$. Since $G$ 
	is infinite, some $\lambda_g=0$ and hence $f=0$. Thus $e=1$ and $I=\C[G]$, a contradiction. 
% 	If $f=0$, then $e=1$ and $I=\C[G]$, a contradiction.  
% 	, a contradiction because 
% 	$f\ne 0$ implies that the sum that defines $f$ should be an infinite sum.
\end{proof}

% The ideal $I(G)$ used in the proof of the previous proposition 
% is known as the \textbf{augmentation ideal} 
% of $\C[G]$.

\begin{theorem}
	Let $G$ be a group. Then $\C[G]$ 
	is left artinian if and only if 
	$G$ is finite. 
\end{theorem}

\begin{proof}
    If $G$ is finite, then $\C[G]$ is left artinian because $\dim\C[G]=|G|<\infty$. So assume that 
    $G$ is infinite. By Rickart's theorem,   
	$J(\C[G])=0$. Moreover, $\C[G]$
	is not semisimple by the previous proposition. Thus
	$\C[G]$ is not left artinian by Theorem~\ref{thm:SSartin=J}.
\end{proof}



\chapter{}

\topic{Prime rings}

In commutative algebra domains play a fundamental role. In non-commutative
algebra certain things could be quite different. 
For example, the ring $M_n(\C)$ is not a domain.
We need a non-commutative generalization of domains.

\begin{definition}
	\index{Ring!prime} 
	Let $R$ be a ring (not necessarily with one). Then $R$ is
	\textbf{prime} if for $x,y\in R$ such that $xRy=\{0\}$ it follows that $x=0$ or 
	$y=0$.
\end{definition}

\begin{example}
    \index{Domain}
	A ring $R$ is a \textbf{domain} if $xy=0$ implies
	$x=0$ or $y=0$. Each domain is trivially a prime ring.
\end{example}

\begin{example}
    A commutative ring is prime if and only if it is a domain, as $ab=0$ 
    if and only if $aRb=\{0\}$.
\end{example}

\begin{example}
    A non-zero ideal of a prime ring is a prime ring.
\end{example}

A characterization of prime rings:

\begin{proposition}
    Let $R$ be a ring. The following statements are equivalent:
	\begin{enumerate}
		\item $R$ is prime.
		\item If $I$ and $J$ are left ideals such that $IJ=\{0\}$, then 
			$I=\{0\}$ or $J=\{0\}$.
		\item If $I$ and $J$ are ideals such that $IJ=\{0\}$, then $I=\{0\}$ or
			$J=\{0\}$.
	\end{enumerate}
\end{proposition}

\begin{proof}
	We first prove that $1)\implies2)$. Let $I$ and $J$ be left ideals such that
	$IJ=\{0\}$. Then $IRJ=I(RJ)\subseteq IJ=\{0\}$. If $J\ne
	\{0\}$, $u\in I$ and $v\in J\setminus\{0\}$, then $uRv\in IRJ=\{0\}$. Hence 
	$u=0$.

	The implication $2)\implies3)$ is trivial. 

    Let us prove that $3)\implies1)$. Let $x,y\in R$ be such that $xRy=\{0\}$.
	Let $I=RxR$ and $J=RyR$. Since $IJ=(RxR)(RyR)=R(xRy)R=\{0\}$, 
	we may assume that $I=\{0\}$. In particular, $Rx$ and $xR$ are ideals, as 
	$R(xR)=(Rx)R=\{0\}$. Then $\Z x$ is an ideal of $R$ such that $(\Z x)R=\{0\}$. 
	Thus $x=0$. 
\end{proof}

Simple rings are trivially prime. The converse is not true:

\begin{example} 
    $\Z$ is a domain, so it is a prime ring. Clearly, it is not simple.
\end{example}

\begin{example}
	If $R_1$ and $R_2$ are rings, $R=R_1\times R_2$ is not prime, as 
	$I=R_1\times\{0\}$ and $J=\{0\}\times R_2$ are non-zero ideals such that $IJ=\{0\}$.
\end{example}

\begin{lemma}
	\label{lem:primoizqmin=>prim}
	Let $R$ be a prime ring and $L$ be a minimal left ideal of $R$.
	Then $R$ is primitive. 
\end{lemma}

\begin{proof}
	Since $L$ is a minimal left ideal, it is simple as a module over $R$. 
	We claim that $L$ is faithful. Let $y\in L\setminus\{0\}$ and
	$x\in\Ann_R(L)$. Since $xRy\in xRL\subseteq xL=\{0\}$, it follows that 
	$x=0$.
\end{proof}

\begin{lemma}
	\label{lem:denso_artiniano}
	Let $D$ be a division ring and $R$ be a dense ring in a module $V$ over $D$. 
	If $R$ is left artininian, then $\dim_DV<\infty$.
\end{lemma}

\begin{proof}
	Assume that $\dim_DV=\infty$ and let $\{u_1,u_2,\dots,\}$ be linearly independent. 
	Since $R\subseteq\End_D(V)$, it follows that 
	$V$ is a module over $R$ with $f\cdot v=f(v)$, where $f\in R$ y $v\in V$. 
	For $n\in\Z_{>0}$ let 
	\[
		I_n=\Ann_R(\{u_1,\dots,u_n\}).
	\]
	Each $I_j$ is a left ideal of $R$ and $I_1\supseteq
	I_2\supseteq\cdots\supseteq I_n\supseteq\cdots$. Let 
	$n\in\Z_{>0}$ and $v\in V\setminus\{0\}$. Since $R$ is dense
	in $V$, there exists $f\in R$ such that $f(u_j)=0$ for all $j\in\{1,\dots,n\}$ and 
	$f(u_{n+1})=v\ne0$. Thus $I_1\supsetneq I_2\supsetneq\cdots\supsetneq
	I_n\supsetneq\cdots$, a contradiction.
\end{proof}

\begin{theorem}[Wedderburn]
	\index{Wedderburn's theorem}
	Let $R$ be a left artinian ring. The following statements are equivalent:
	\begin{enumerate}
		\item $R$ is simple.
		\item $R$ is prime.
		\item $R$ is primitive.
		\item $R\simeq M_n(D)$ for some $n$ and some division ring $D$.
	\end{enumerate}
\end{theorem}

\begin{proof}
	The implication $1)\implies2)$ is trivial. 
	
	To show that $2)\implies3)$ first note that 
	$R$ contains a minimal left ideal, as $R$ is left artinian. 
	By Lemma~\ref{lem:primoizqmin=>prim}, $R$ is primitive. 

	Now we prove that $3)\implies4)$. If $R$ is primitive, 
	Jacobson's density theorem implies that there exists a division
	ring $D$ such that  
	$R$ is isomorphic to a ring $S$ that is dense in a vector space $V$ over $D$.
	Since $R$ is left artinian, Lemma~\ref{lem:denso_artiniano} implies that  
	$R=\End_D(V)\simeq M_n(D)$, as $\dim_DV<\infty$. 

	Finally, $4)\implies1)$ is trivial, as $M_n(D)$ is simple. 
%	$D$-espacio vectorial $V$ de dimensión finita. Si $u_1,\dots,u_m\in V$ son
%	linealmente independientes sobre $D$, existen $f_1,\dots,f_m\in S$ tales
%	que $f_i(u_i)\ne0$ y $f_i(u_j)=0$ si $i\ne j$. Como los $f_j$ son
%	linealmente independientes sobre $k$, $\dim_DV\leq \dim A$. Luego $A=\End_DV\simeq M_n(D^{\op})$ 
%	por la proposición\dots
\end{proof}

We now prove Artin--Wedderburn's theorem. We will assume that our ring
is a unitary left artinian ring. One could prove
Artin--Wedderburn's theorem for arbitrary rings --see for example \cite{MR600654}--  
but when dealing with unitary rings the proof 
is simpler. We will prove
that left artinian semiprimitive unitary rings
are isomorphic to a direct product
of finitely many matrix rings. The idea of the proof goes as follows. 
We know that if 
$R$ is semiprimitive, then $R$ is a subdirect product
of primitive rings, that is  
there exists an injective map
\[
R\to \prod_{i\in I}R/I_i
\]
where each $I_i$ is a primitive ideal. Since $R$ is left artinian, 
the set $I$ will be a finite set. Moreover, 
by Wedderburn's theorem, 
$R/I_i\simeq M_{n_i}(D_i)$ for some division ring $D_i$. Finally,
a sort of non-commutative version of the Chinese remainder theorem
is used to prove that the map is fact surjective. 

\begin{definition}
    An ideal $I$ of $R$ is prime if $xRy\subseteq I$ implies
    $x\in I$ or $y\in I$.
\end{definition}

Note that a ring $R$ is prime if and only if $\{0\}$ is a prime ideal. 
Moreover, 
an ideal $I$ of $R$ is prime if and only if 
the ring $R/I$ is prime. 

\begin{lemma}
    If $R$ is left artinian and $I$ is a primitive ideal, then
    $I$ is prime. 
\end{lemma}

% no necesito hacerlo para artiniano

\begin{proof}
    Since $I$ is primitive, then $R/I$ is primitive. By Wedderburn theorem, 
    $R/I$ is prime and hence $I$ is prime. 
\end{proof}

\begin{theorem}[Artin--Wedderburn]
\label{thm:ArtinWedderburn_version2}
\index{Artin--Wedderburn's theorem}
    Let $R$ be a semiprimitive left artinian unitary ring. Then
    $R\simeq\prod_{i=1}^kM_{n_i}(D_i)$ for finitely many
    division rings $D_1,\dots,D_k$. 
\end{theorem}

We shall need the following lemmas.

\begin{lemma}
\label{lem:primitive=>maximal}
    Let $R$ be a left artinian ring and $I$ be a primitive ideal. 
    Then $I$ is maximal. 
\end{lemma}

\begin{proof}
    If $I$ is a primitive ideal of $R$, then $R/I$ is a primitive ring
    by Lemma \ref{lemma:primitivo}. By Wedderburn's theorem, $R/I$ is
    simple. Thus $I$ is maximal by Proposition \ref{proposition:R/I}. 
\end{proof}

\begin{lemma}
    Let $I_1,\dots,I_k$ be finitely many distinct maximal ideals of $R$. 
    Then $I_2\cdots I_k\not\subseteq I_1$.   
\end{lemma}

\begin{proof}
    Suppose the result is not true and let $k$ be minimal
    such that $I_2\cdots I_k\subseteq I_1$. Since the result is clearly
    true for two distinct maximal ideals, $k\geq3$. Let $I=I_2\cdots I_{k-1}$. 
    Since $I\not\subseteq I_1$, there exists $x\in I\setminus I_1$. Moreover,  
    there exists $y\in I_k\setminus I_1$, as 
    $I_k\ne I_1$. 
    Then 
    $(xR)y\subseteq II_k\subseteq I_1$. Since $I_1$ is prime, 
    it follows that either $x\in I_1$ or $y\in I_1$, a contradiction.   
\end{proof}

\begin{lemma}
    Let $R$ be a left artinian ring. Then $R$ has only 
    finitely many primitive ideals.
\end{lemma}

\begin{proof}
    If $I_1,I_2\dots$ are infinitely many primitive ideals. 
    Since $R$ is left artinian, the sequence 
    $I_1\supseteq I_1I_2\supseteq\cdots$ stabilizes, so there
    exists $n$ such that 
    \[
    I_1I_2\cdots I_n=I_1I_2\cdots I_nI_{n+1}\subseteq I_{n+1},
    \]
    a contradiction to the previous lemma, 
    as each $I_j$ is a maximal ideal. 
\end{proof}

Now we are ready to prove the theorem. 

\begin{proof}[Proof of Theorem \ref{thm:ArtinWedderburn_version2}]
    Let $I_1,\dots,I_k$ be the (distinct) primitive ideals of $R$. 
    We know that each $I_i$ is a maximal ideal. Thus $I_i+I_j=R$ for
    $i\ne j$. Since $R$ is semiprimitive, 
    $I_1\cap\cdots\cap I_k=J(R)=\{0\}$. Let 
    \[
    \varphi\colon R\to \prod_{i=1}^k R/I_i,\quad
    x\mapsto (x+I_1,\dots,x+I_k).
    \]
    Then $\varphi$ is a ring homomorphism with kernel $I_1\cap\cdots\cap I_k=\{0\}$, so
    $\varphi$ is injective. We need to prove that $\varphi$ is surjective. 
    
    We first claim that 
    $I_1+( I_2\cdots I_k) = R$. In fact, 
    since $I_1,\dots,I_k$ are maximal ideals, $I_2\cdots I_k\not\subseteq I_1$. This implies
    that $I_1+(I_2\cdots I_k)$ is an ideal of $R$ that contains $I_1$. Since $I_1$ is maximal, 
    $I_1+(I_2\cdots I_k)=R$. 
    
    Since $I_1+( I_2\cdots I_k) = R$, 
    there exists $x_1\in \prod_{j=2}^kI_j$ such that $1\in x_1+I_1$. Note that
    $x_1=(1+I_1)\cap (I_2\cdots I_k)\subseteq I_j$ for all $j\in\{2,\dots,k\}$. 
    Thus 
    \[
    \varphi(x_1)=(x+I_1,I_2,\dots,I_k)=(1+I_1,I_2,\dots, I_k).
    \]
    Similarly,
    there exists $x_2\in 1+I_2,\dots, x_k\in 1+I_k$ such that 
    \begin{align*}
    \varphi(x_2)&=(I_1,1+I_2,\dots,I_k),\\
    &\vdots\\
    \varphi(x_k)&=(I_1,I_2,\dots,1+I_k).
    \end{align*}
    From this it follows that $\varphi$ is surjective. Each $R/I_i$ 
    is primitive and hence isomorphic to $M_{n_i}(D_i)$ for some 
    $n_i$ and some division ring $D_i$. Therefore
    \[
    R\simeq R/I_1\times\cdots\times R/I_k\simeq \prod_{i=1}^kM_{n_i}(D_i).\qedhere 
    \]
\end{proof}

\topic{Wedderburn's little theorem}

\begin{definition}
	The $n$-th cyclotomic polynomial 
	is defined as the polynomial
	\begin{equation}
		\label{eq:ciclotomico}
		\Phi_n(X)=\prod(X-\zeta),
	\end{equation}
	where the product is taken over all 
	$n$-th primitive roots of one. 
\end{definition}

Some examples:
	\begin{align*}
		&\Phi_2=X-1,\\
		&\Phi_3=X^2+X+1,\\
		&\Phi_4=X^2+1,\\
		&\Phi_5=X^4+X^3+X^2+X+1,\\
		&\Phi_6=X^2-X+1,\\
		&\Phi_7=X^6+X^5+\cdots+X+1.
	\end{align*}

\begin{lemma}
	If $n\in\Z_{>0}$, then
	\[
		X^n-1=\prod_{d\mid n}\Phi_d(X).
	\]
\end{lemma}

\begin{proof}
	Write 
	\[
		X^n-1=\prod_{j=1}^n (X-e^{2\pi ij/n})
		=\prod_{d\mid n}\prod_{\substack{1\leq j\leq n\\\gcd(j,n)=d}}(X-e^{2\pi ij/n})
		=\prod_{d\mid n}\Phi_d(X).\qedhere 
	\]
\end{proof}

\begin{lemma}
	If $n\in\Z_{>0}$, then $\Phi_n(X)\in\Z[X]$.
\end{lemma}

\begin{proof}
	We proceed by induction on $n$. The case where $n=1$ is trivial, as 
	$\Phi_1(X)=X-1$. Assume that $\Phi_d(X)\in\Z[X]$ for all $d<n$.
	Then 
	\[
		\prod_{d\mid n,d\ne n}\Phi_d(X)\in\Z[X]
	\]
	is a monic polynomial. Thus $\Phi_n(X)/\prod_{d\mid
	n,d<n}\Phi_d(X)\in\Z[X]$.
\end{proof}

\begin{theorem}[Wedderburn]
	\index{Wedderburn's little theorem}
	Every finite division ring is a field. 
\end{theorem}

\begin{proof}
    Let $D$ be a finite division rin  
	and $K=Z(D)$. Then $K$ is a finite field, say $|K|=q$. Note that
	$K$ is a $D$-vector space. Let 
	$n=\dim_KD$.  We claim that $n=1$. If $n>1$, the 
	class equation for the group 
	$D^\times=D\setminus\{0\}$ implies that 
	\begin{equation}
		\label{eq:clases}
		q^n-1=q-1+\sum_{j=1}^m \frac{q^n-1}{q^{d_j}-1},
	\end{equation}
	where $1<\frac{q^n-1}{q^{d_j}-1}\in\Z$ for all $j\in\{1,\dots,m\}$. 
	Since $d^{d_j}-1$ divides $q^n-1$, each $d_j$ divides $n$. In particular,
	~\eqref{eq:ciclotomico} implies that 
	\begin{equation}
		\label{eq:trick_ciclotomico}
		X^n-1=\Phi_n(X)(X^{d_j}-1)h(X)
	\end{equation}
	for some $h(X)\in\Z[X]$. 
	By evaluating~\eqref{eq:trick_ciclotomico} in $X=q$  
	we obtain that $\Phi_n(q)$ divides $q^n-1$ and that $\Phi_n(q)$
	divides $\frac{q^n-1}{q^{d_j}-1}$. By~\eqref{eq:clases}, 
	$\Phi_n(q)$ divides $q-1$. Thus  
	\[
		q-1\geq |\Phi_n(q)|=\prod |q-\zeta|>q-1,
	\]
	as each $|q-\zeta|>q-1$, 
	a contradiction. 
\end{proof}

% todo:  (draw a picture of $q$ and $\zeta$ in the complex plane)
% explicar mejor!

% Veamos como corolario una aplicación al último teorema de Fermat en anillos
% finitos. Demostraremos el siguiente resultado:

% \begin{theorem}
% 	Sea $R$ un anillo unitario finito. Entonces para todo $n\geq1$ existen $x,y,z\in
% 	R\setminus\{0\}$ tales que $x^n+y^n=z^n$ si y sólo si $R$ no es un anillo
% 	de división.
% \end{theorem}

% \begin{proof}
% 	Supongamos primero que $R$ es de división. Por el teorema de Wedderburn,
% 	$R$ es entonces un cuerpo finito, digamos $|R|=q$. Como entonces
% 	$x^{q-1}=1$ para todo $x\in R\setminus\{0\}$, se concluye que la ecuación
% 	$x^{q-1}+y^{q-1}=z^{q-1}$ no tiene solución.

% 	Supongamos ahora que $R$ no es de división. Como entonces, en particular,
% 	$R$ no es un cuerpo, $|R|>2$ y luego $x+y=z$ tiene solución en
% 	$R\setminus\{0\}$ (tomar por ejemplo $x=1$, $y=z-1$ y $z\not\in\{0,1\}$).
% 	Como $R$ es finito, $R$ es artiniano a izquierda y entonces el radical de
% 	Jacobson $J(R)$ es nilpotente. Si $J(R)\ne 0$, existe entonces $a\in
% 	R\setminus\{0\}$ tal que $a^2=0$ y luego $a^n=0$ para todo $n\geq2$. En
% 	este caso, la ecuación $x^n+y^n=z^n$ tiene solución en $R\setminus\{0\}$ si
% 	$n\geq 2$ (tomar por ejemplo $x=a$, $y=z=1$). Si $J(R)=0$, entonces, $R$ es
% 	semisimple y luego, por el teorema de Wedderburn,
% 	\[
% 		R\simeq \prod_{i=1}^k M_{n_i}(D_i)
% 	\]
% 	donde los $D_i$ son cuerpos finitos (por ser anillos de división finitos).
% 	Como $R$ no es un cuerpo, hay dos posibilidades: o bien $n_i>1$ para algún
% 	$i\in\{1,\dots,k\}$, o bien $k\geq 2$ y $n_i=1$ para todo
% 	$i\in\{1,\dots,k\}$. En el primer caso, como $M_{n_i}(D_i)$ tiene elementos
% 	no nulos cuyo cuadrado es cero, $R$ también los tiene, y luego, tal como se
% 	hizo antes, vemos que $x^n+y^n=z^n$ tiene solución. En el segundo caso,
% 	$x=(1,0,0,\dots,0)$, $y=(0,1,0,\dots,0)$ y $z=(1,1,0,\dots,0)$ es una
% 	solución de $x^n+y^n=z^n$.
% \end{proof}

\chapter{}

\topic{Frobenius's theorem}

\begin{theorem}[Frobenius]
	\label{thm:Frobenius}
	\index{Frobenius'!theorem}
	Every finite-dimensional real division algebra is isomorphic to $\R$, $\C$
	or $\H$.
\end{theorem}

We present an elementary proof. We shall need some lemmas. 

\begin{lemma}
	\label{lem:trick_frobenius1}
	Let $D$ be a real division algebra such that $\dim D=n$. If $x\in D$, then
	there exists $\lambda\in\R$ such that $x^2+\lambda x\in\R$.
\end{lemma}

\begin{proof}
	Since $\dim D=n$, the set $\{1,x,x^2,\dots,x^n\}$ is linearly dependent. So
	there exists a non-zero polynomial $f(X)\in\R[X]$ of degree $\leq n$ such
	that $f(x)=0$. Without loss of generality, we may assume that the leading
	coefficient of $f(X)$ is one. Then we can write $f(X)$ as a product of
	polynomials of degree $\leq2$, say 
	\[
		f(X)=(X-\alpha_1)\cdots (X-\alpha_r)(X^2+\lambda_1 X+\mu_1)\cdots (X^2+\lambda_s X+\mu_s).
	\]
	Since $D$ is a division algebra and $f(x)=0$, some factor of $f(X)$ is
	zero. If $x-\lambda_j\ne 0$ for all $j$, then $x$ is a root of some
	$X^2+\lambda_k X+\mu_k$. In any case, there exists $\lambda\in\R$ such that
	$x^2+\lambda x\in\R$. 
\end{proof}

\begin{lemma}
	\label{lem:trick_frobenius2}
	Let $D$ be a real division algebra of dimension $n$. Then
	\[
		V=\{x\in D:x^2\in\R_{\leq0}\}
		%,x^2\leq 0\}
	\]
	is a subspace of $D$ such that $D=\R\oplus V$.
\end{lemma}

\begin{proof}
	Let $x\in D\setminus V$ be such that $x^2\in\R$. Since $x^2>0$, it follows
	that $x^2=\alpha^2$ for some $\alpha\in\R$. Thus $x=\pm\alpha\in\R$, as $D$
	is a division algebra and $(x-\alpha)(x+\alpha)=x^2-\alpha^2=0$.

	We claim that $V$ is a subspace of $D$.  Note that $0\in V$ and 
	that if $x\in V$, then $\lambda x\in V$ for all $\lambda\in\R$.  Let 
	$x,y\in V$. If $\{x,y\}$ is linearly dependent, then $x+y\in V$.
	If not, we claim that 
	$\{1,x,y\}$ is linearly independent. If there exist 
	$\alpha,\beta,\gamma\in\R$ such that $\alpha x+\beta y+\gamma=0$, then 
	\[
	\alpha^2x^2=\beta^2y^2+2\beta\gamma y+\gamma^2=(-\beta y-\gamma)^2.
	\]
	This implies that $2\beta\gamma y\in\R$ and thus $\beta\gamma=0$. Hence 
	$\alpha=\beta=\gamma=0$. The previous lemma implies that there exist 
	$\lambda,\mu\in\R$ such that 
	\[
		(x+y)^2+\lambda(x+y)\in\R,\quad
		(x-y)^2+\mu(x-y)\in\R.
	\]
	Since 
	\[
		(x+y)^2+(x-y)^2=2x^2+2y^2\in\R,
	\]
	it follows that $(\lambda+\mu)x+(\lambda-\mu)y\in\R$. Since  $\{1,x,y\}$ is linearly 
	independent, 
	$\lambda=\mu=0$. Thus $(x+y)^2\in\R$. If 
	$x+y\not\in V$, then, the first paragraph of the proof implies that 
	$x+y\in\R$, a contradiction. 

	Clearly, $\R\cap V=0$. If $x\in D\setminus\R$, then the previous lemma 
	implies that $x^2+\lambda x\in\R$ for some 
	$\lambda\in\R$. We claim that $x+\lambda/2\in V$. If not, since 
	\[
	(x+\lambda/2)^2=x^2+\lambda x+(\lambda/2)^2\in\R,
	\]
	it follows that $x+\lambda/2\in\R$ and thus $x\in\R$. Hence 
	$x=-\lambda/2+(x+\lambda/2)\in\R\oplus V$.
\end{proof}

\begin{lemma}
	\label{lem:trick_frobenius3}
	Let $D$ be a real algebra of (real) dimension $n$. If $n>2$, then
	there exist $i,j,k\in D$ such that $\{1,i,j,k\}$ is linearly independent and 
	\begin{align}
	\label{eq:H}
	&i^2=j^2=k^2=-1, && ij=-ji=k, && ki=-ik=j, && jk=-kj=i.
	\end{align}
\end{lemma}

\begin{proof}
	Let $V=\{x\in D:x^2\in\R,x^2\leq 0\}$ be the subspace of Lemma~\ref{lem:trick_frobenius2}. 
	For $x,y\in V$ let $x\circ
	y=xy+yx=(x+y)^2-x^2-y^2\in\R$. If $x\ne0$, then $x\circ
	x=2x^2\ne0$. Since $\dim V=n-1$, there exist $y,z\in V$ such that $\{y,z\}$ is 
	linearly independent. Let 
	\[
		x=z-\frac{z\circ y}{y\circ y}y.
	\]
	Since $\{y,z\}$ is linearly independent, $x\ne0$. Moreover, since 
	\[
		x\circ y
		=\left(z-\frac{z\circ y}{y\circ y}\right)\circ y
		=zy-\frac{z\circ y}{y\circ y}y^2+yz-\frac{z\circ y}{y\circ y}y^2
		=z\circ y-\frac{z\circ y}{y\circ y}y\circ y=0,
	\]
	it follows that $xy=-yx$. 
	Let  
	\[
		i=\frac{1}{\sqrt{-x^2}}x,
		\quad
		j=\frac{1}{\sqrt{-y^2}}y,
		\quad
		k=ij. 
	\]
	A direct calculation shows that the formulas of ~\eqref{eq:H} hold. For example, 
	\[
		ji=\frac{1}{\sqrt{-y^2}}\frac{1}{\sqrt{-x^2}}yx=\frac{1}{\sqrt{-x^2}}\frac{1}{\sqrt{-y^2}}(-xy)=-k.\qedhere
	\]
\end{proof}

Now we are finally 
ready to prove the theorem: 

\begin{proof}[Proof of \ref{thm:Frobenius}]
	Let $D$ be a real division algebra and let $n=\dim D$. If $n=1$, then 
	$D\simeq\R$. If $n=2$, the subspace $V$ of Lemma~\ref{lem:trick_frobenius2} 
	is non-zero and thus there exists $i\in D$ such that 
	$i^2=-1$. Hence $D\simeq\C$. Lemma~\ref{lem:trick_frobenius3}
	implies that $n\ne3$. If $n=4$, then $D\simeq\H$. Suppose that 
	$n>4$.  By Lemma~\ref{lem:trick_frobenius3} there exist
	$i,j,k\in D$ such that $\{1,i,j,k\}$ is linearly independent 
	and that the formulas of~\eqref{eq:H} hold. Let 
	\[
		V=\{x\in D:x^2\in\R_{\leq0}\}.
	\]
	By Lemma~\ref{lem:trick_frobenius2}, $\dim V=n-1$. Thus there exists 
	$x\in V\setminus\langle i,j,k\rangle$. Let 
	\[
		e=x+\frac{i\circ x}{2}i+\frac{j\circ x}{2}j+\frac{k\circ x}{2}k\in V\setminus\{0\}.
	\]
	A direct calculation shows that $i\circ e=j\circ e=k\circ e=0$. Then 
	\[
		ek=e(ij)=(ei)j=-(ie)j=-i(ej)=i(je)=(ij)e=ke,
	\]
	a contradiction. 
\end{proof}

\topic{Jacobson's commutativity theorem}

\begin{exercise}
    \index{Ring!boolean}
    A ring $R$ is \textbf{boolean} if $x^2=x$ for all $x\in R$. 
    Prove that boolean rings are commutative. 
\end{exercise}

To prove this fact, note that $1=(-1)^2=-1$. This means that $R$ has
characteristic two. Let $x,y\in R$. Since $x+y=(x+y)^2=x^2+xy+yx+y^2$. 
it follows that $0=xy+yx$ and hence $xy=yx$.

%\begin{definition}
%	\index{Ring!reduced}
%	A ring $R$ is \textbf{reduced} if $x^2=0$ implies $x=0$. 
%\end{definition}
%
%For example, boolean rings and domains are reduced. Moreover, the ring 
%$\Z^n$ with point-wise multiplication is reduced (and has zero divisors).

\begin{proposition}
\label{pro:Jacobson}
    Let $R$ be a finite ring such that 
    for each $x\in R$ there exists $n(x)\geq2$ such that 
    $x^{n(x)}=x$. Then $R$ is commutative. 
\end{proposition}

\begin{proof}
    Since $R$ is finite, $R$ is artinian and hence $J(R)$ is nil. 
    Since $R$ is reduced, $J(R)=\{0\}$. 
    By the Artin--Wedderburn theorem, $R\simeq\prod_{i=1}^k M_{n_i}(D_i)$ for some division rings $D_1,\dots,D_k$. 
    Since $R$ is finite, each $D_i$ is finite. By Wedderburn's theorem,
    every $D_i$ is a field. Again, since $R$ is reduced, 
    $n_i=1$ for all $i$. Therefore $R$ is commutative, as it is 
    direct product of finitely many fields.
\end{proof}

In this lecture, we will prove extend the 
result of 
Proposition \ref{pro:Jacobson} to arbitrary 
(i.e. non-finite) rings. 

\begin{theorem}[Jacobson]
\label{thm:commutativity}
\index{Jacobson's commutativity theorem}
	Let $R$ be a ring such that 
	for each $x\in R$ there exists $n(x)\geq2$ such that 
	$x^{n(x)}=x$. Then $R$ is commutative. 
\end{theorem}

We shall need the following lemma.

% The structure 
% of reduced rings is described by the following result:

% \begin{theorem}[Andrnakievich--Ryabukhin]
% \index{Andrnakievich--Ryabukhin's theorem}
% 	Let $R$ be a non-zero ring. Then $R$ is reduced if and only if
% 	$R$ is a subdirect product of domains. 
% \end{theorem}
%https://ysharifi.wordpress.com/2010/06/04/about-reduced-rings-2/
% \begin{proof}
% 	Supongamos que $R$ es reducido.\framebox{?}

% 	Supongamos ahora que $R$ es producto subdirecto de la familia $\{R_i:i\in I\}$ de dominios. Sea 
% 	$f\colon R\to \prod_{i\in I}R_i$, $f(x)=(x_i)_{i\in I}$, el morfismo inyectivo. 
% 	Si $x\in R$ es tal que $x^2=0$ entonces 
% 	\[
% 		(0)_{i\in I}=f(0)=f(x^2)=f(x)^2=(x_i^2)_{i\in I}
% 	\]
% 	y luego, como cada $R_i$ es un dominio, se concluye que $x_i=0$ para todo
% 	$i\in I$.
% \end{proof}

% Let x,y \in R. Suppose that R is a domain. Then R is reduced and if xRy=(0), then xy=0 and thus, since R is a domain, either x=0 or y=0. So R is a prime ring. Now suppose that R is both prime and reduced and xy=0. Then, by Remark 1, yRx=(0) and so, since R is prime, either y=0 or x=0. To prove that every reduced ring is semiprime, suppose that xRx=(0) for some x \in R. Then x^2 \in xRx =(0) and so x=0. \ \Box    

% \begin{lemma}
% 	\label{lem:reducido}
% 	Todo idempotente de un anillo reducido es central.
% \end{lemma}

% \begin{proof}
% 	Sea $e\in R$ tal que $e^2=e$ y sea $x\in R$. Como
% 	\[
% 	(ex-exe)^2=exex-exexe-exex+exexe=0
% 	\]
% 	y el anillo $R$ es reducido, $ex=exe$. 
% 	Similarmente la igualdad 
% 	$(xe-exe)^2=0$ implica que $xe=exe$. Luego $ex=exe=xe$ para todo $x\in R$.
% \end{proof}


% \begin{proposition}
% 	Sea $R$ un anillo tal que $x^3=x$ para todo $x\in R$. Entonces $R$ es
% 	conmutativo.
% \end{proposition}

% \begin{proof}
% 	El anillo $R$ es reducido pues si $x^2=0$ entonces $x=x^3=0$.  Como en $R$
% 	todo cuadrado es idempotente (pues $x^2=x^4=(x^2)^2$ para todo $x\in R$),
% 	el lema~\ref{lem:reducido} implica que todo cuadrado es central. 
% 	Si $x\in R$, entonces 
% 	$2x=(x^2+x)-2x^2$ 
% 	es central. Como además 
% 	\[
% 		1+x=(1+x)^3=1+3x+3x^2+x,
% 	\]
% 	se tiene que $3x=-3x^2$ es un elemento central. Luego $x=3x-2x$ es central.
% \end{proof}



% \begin{proposition}
% 	Sea $R$ un anillo tal que $x^4=x$ para todo $x\in R$. Entonces $R$ es conmutativo.
% \end{proposition}

% \begin{proof}
% 	El anillo $R$ es reducido pues si $x^2=0$ entonces $x=x^4=0$.  Como
% 	$x=x^4=(-x)^4=-x$ para todo $x\in R$, el anillo $R$ tiene característica
% 	dos. Todo elemento de la forma $z^2+z$ es idempotente pues
% 	$(z^2+z)^2=z^4+2z^3+z^2=z^2+z$. Luego, por el 
% 	lema~\ref{lem:reducido}, todo elemento de la forma $z^2+z$ es central. Si $x,y\in R$, entonces 
% 	\[
% 		x^2y+yx^2=(x^2+y)^2-(x^2+y)-(x^2+x)-(y^2+y)
% 	\]
% 	es central. Como entonces $x^2(x^2y+yx^2)=(x^2y+yx^2)x^2$, se concluye que $xy=yx$.
% \end{proof}

% Parece difícil poder extender los resultados anteriores a otros exponentes. Antes 
% de enunciar y demostrar un teorema de Jacobson que generaliza los resultados mencionados, necesitamos
% dos lemas:

\begin{lemma}
	\label{lem:k_finito}
	Let $K$ be a finite field of characteristic $p>0$. There exists 
	$n\in\Z_{>0}$ such that $|K|=p^n$ and $x^{p^n}=x$ for all $x\in K$. Moreover, 
	if $K\setminus\{0\}=\{x_1,\dots,x_{p^n-1}\}$, then 
	$X^{p^n}-X=(X-x_1)\cdots(X-x_{p^n-1})X$. 
\end{lemma}

\begin{proof}
	The field $K$ is a $(\Z/p)$-vector space. If $\dim_{\Z/p} K=n$, then 
	$|K|=p^n$. In particular, $K\setminus\{0\}$ is an abelian
	group of order $p^{n}-1$ and hence, by Lagrange's theorem, 
	$x^{p^n-1}=1$ for all $x\in K\setminus\{0\}$. Thus $x^{p^n}=x$ for all $x\in K$
	and hence every $x\in K$ is a root of the polynomial 
	$X^{p^n}-X$ of degree $p^n$. 
\end{proof}

Let $R$ be a ring. For each $r\in R$ the map $\ad{r}\colon
R\to R$, $x\mapsto rx-xr$, is a derivation. This means that
$\ad(xy)=(\ad x)y+x(\ad y)$ for all $x,y\in R$. 
By induction one proves that 
\begin{equation}
	\label{eq:Leibniz}
	(\ad{r})^n(x)=\sum_{k=0}^n(-1)^k\binom{n}{k}r^{n-k}xr^k
\end{equation}
for all $x\in R$ and $n\in\Z_{>0}$. If $p$ 
is a prime number, 
$p$ divides $\binom{p}{k}$ for all $k\in\{1,\dots,p-1\}$. This fact
is needed to solve the following exercise:

\begin{exercise}
    Let $p$ be a prime number and $R$ be a ring of characteristic $p$. 
    Prove that $(\ad{r})^{p^n}=\ad{r^{p^n}}$. 
\end{exercise}

% \begin{lemma}
% 	\label{lem:Jacobson}
% 	Sean $p$ un primo, 
% 	$R$ un anillo de característica $p$ y $r\in R$. 
% 	Entonces $(\ad{r})^{p^n}=\ad{r^{p^n}}$. 
% \end{lemma}

% \begin{proof}
% 	Procederemos por inducción en $n$. Supongamos que $n=1$. Si $k\in\{1,\dots,p-1\}$, entonces 
% 	$p$ no divide a $k!(p-k)!$ y luego $p$ divide a $\binom{p}{k}$. La igualdad~\eqref{eq:Leibniz} se
% 	transforma entonces en\dots
% \end{proof}

Now we are ready to prove Jacobson's commutativity theorem. 

\begin{proof}[Proof of Theorem \ref{thm:commutativity}]
    We divide the proof in several steps and claims. We may assume that 
    $R$ is non-zero. 
    
    \begin{claim}
        $J(R)=\{0\}$. 
    \end{claim}
    
    Let $x\in J(R)$ and $n=n(x)$. Since $-x^{n-1}\in J(R)$, 
    there exists $y\in R$ such that $-x^{n-1}\circ y=-x^{n-1}+y-x^{n-1}y=0$. Thus
    \[
    -x^{n-1}+y=x^{n-1}y\implies
    -x+xy=x(-x^{n-1}+y)=x^ny=xy.
    \]
    This implies that $x=0$.
    
    \begin{claim}
        Without loss of generality we may assume that $R$ is primitive. 
    \end{claim}
	
	Let $\{P_i:i\in I\}$ be the collection of primitive ideals of $R$. 
	The map 
	$R\to \prod_{i\in I}R/P_i$, $r\mapsto (r+P_i)_{i\in I}$, 
	is an injective homomorphism, since
	its kernel is 
	\[
	\bigcap_{i\in I} P_i=J(R)=\{0\}.
	\]
	Note that
	$R$ is commutative if and only if each $R/P_i$ is commutative. Moreover, 
	each $R/P_i$ 
	satisfies the assumption, that is 
	$(x+P_i)^{n(x)}=x^{n(x)}+P_i=x+P_i$, and 
	and is a primitive ring. 
	
    \begin{claim}
        $R$ is a division ring. 
    \end{claim}

    By Jacobson's density theorem,
    there exists a division ring $D$ and 
    a $D$-vector space $V$ such that 
    $R$ is dense in $V$. We claim that $\dim_DV=1$. If $\dim_DV\geq 2$, 
    let $\{v_1,v_2\}\subseteq V$ be a linearly independent set. Then
    there exists $f\in R$ such that $f(v_1)=v_2$ and $f(v_2)=0$. This implies
    that $f^{k}(v_1)=0$ for all $k\geq 2$ and $f(v_1)\ne 0$. This contradicts
    the fact that $f^{n}=f$ for $n=n(f)$. Thus $R\simeq D^{\op}$, 
    a division ring. 
    
    \begin{claim}
        $R$ has positive characteristic. 
    \end{claim}

    Since $R$ is a division ring, $2=1+1\in R$. There exists $n\geq2$ 
    such that $2^n=2$. In particular, 
    $2(2^{n-1}-1)=0$. This implies the claim. 
    
    \begin{claim}
        Every non-zero subring of $R$ is a division ring. 
    \end{claim}
    
    Let $S\subseteq R$ is a non-zero subring of $R$. If $x\in S\setminus\{0\}$, 
    then 
    $x^{n(x)}=x$. In particular, $x^{-1}=x^{n(x)-2}\in S$.
    
    \begin{claim}
        $R$ is commutative.
    \end{claim}
    
    Let us assume that $R$ is not commutative. Let $x\in R\setminus Z(R)$. 
    Since $R$ has positive characteristic, there exists $m>0$ such that 
    $mx=0$. Moreover, since $R$ is a division ring and 
    $x^{n(x)}=x$, it follows that $x^{n(x)-1}=1$. These facts imply that
    the subring $K$ of $R$ generated by
    $x$ is finite. By Wedderburn's theorem, $K$ is 
    a finite field. 
    Thus 
    $|K|=p^k$ for some prime number $p$ and some $k>0$ and 
    \[
    x^{p^k}=x.
    \]
    Note that $R$ is a $K$-vector space
    and $\delta=\ad x\colon R\to R$, $y\mapsto xy-yx$, is a $K$-linear map. Moreover, 
    by the lemma, 
    \[
    \delta^{p^k}=(\ad x)^{p^k}=\ad \left(x^{p^k}\right)=\ad x=\delta
    \]
    and 
    \begin{equation}
        \label{eq:delta}
        \delta(\delta-x_1\id)\cdots (\delta-x_{p^{k-1}\id})=0
    \end{equation}
    if $K=\{0,x_1,\dots,x_{p^k-1}\}$. Since $x$ is not central, 
    $\delta$ is non-zero. So there exists $y\in R$ such that $\delta(y)\ne 0$. 
    Evaluating \eqref{eq:delta} in $y$ and using that $R$ is a division ring 
    we obtain that 
    \[
    x_iy=\delta(y)=xy-yx
    \]
    for some $i$. Let $R_0$ be the subring of $R$ generated by $x$ and $y$. 
    Since $xy-yx=\delta(y)\ne 0$, the ring $R_0$ is a 
    non-commutative division ring. Note that 
    $yx=(x-x_i)y\in Ky$, as $x\in K$ and $x_i\in K$. By induction one proves
    that $yx^j\subseteq Ky$ for all $j\geq 1$ and hence
    $y^iK\subseteq Ky^i$ for
    all $i\geq1$. This implies that
    \[
    K+Ky+\cdots+Ky^{n(y)-2}\subseteq R
    \]
    is a subring. It follows that $K+Ky+\cdots+Ky^{n(y)-2}=R_0$, 
    as it is a subring of $R$ included in $R_0$ 
    that contains $x$ and $y$. Since 
    $R_0$ is a finite division ring, it is a field 
    by Wedderburn's theorem, a contradiction since
    it is non-commutative.
\end{proof}

There are elementary proofs of Jacobson's commutativity theorem.
See for example \cite{MR347890}. 

\section{Skolem--Noether theorem}

\begin{definition}
    Let $K$ be a field. 
	An algebra $A$ (over $K$) is \textbf{central} if $Z(A)=K$. 
\end{definition}

If $K$ is a field, then $M_n(K)$ is a central algebra.

\begin{proposition}
	Let $A$ be a unitary algebra and $n\geq1$. Then
	$A$ is central if and only if $M_n(A)$ is central.
\end{proposition}

\begin{proof}
	If $M_n(A)$ is central and $z\in Z(A)$, then
	$zI\in Z(M_n(A))=KI$. Thus
	$z\in K$. Conversely, if $X\in Z(M_n(A))$, then, since 
	$XE_{kl}=E_{kl}X$ for all $k\ne l$, $X=aI$ for some $a\in A$. 
	Moreover, 
	$XaE_{11}=aE_{11}X$. Hence $a\in Z(A)=K1$. 
\end{proof}

\begin{example}
	$\H$ is a real central algebra.
\end{example}

\begin{example}
	$\C$ is a complex central algebra but it is not a real central 
	algebra. 
\end{example}

Frobenius' theorem~\ref{thm:Frobenius} translates into 
the following statement: Every finite-dimensional  
real central division algebra is isomorphic to $\R$ or $\H$. 

\begin{proposition}
	Every simple unitary ring is an algebra over its center. 
\end{proposition}

\begin{proof}
	Let $R$ be a simple unitary ring. It is enough to show that
	$Z(R)$ is a field. If $z\in
	Z(R)\setminus\{0\}$ then $zR$ is a non-zero ideal of $R$. Since $R$
	is simple, $zR=R$. Thus $z$ is invertible. 
\end{proof}

For an algebra $A$, let $L\colon A\to\End_k(A)$,
$a\mapsto L_a$, and $R\colon A\to\End_k(A)$, $a\mapsto R_a$, be given by 
$L_a(x)=ax$ and $R_a(x)=xa$. Then both $L$ and $R$ are linear maps such that 
\begin{align*}
	L_{ab}=L_aL_b, && R_{ab}=R_bR_a, &&	L_aR_b=R_bL_a
\end{align*}
for all $a,b\in A$.

\begin{definition}
	\index{Algebra!of multipliers}
	Let $A$ be an algebra. The \textbf{algebra of multipliers} of $A$ 
	is 
	\[
		M(A)=\left\{\sum_{j=1}^n L_{a_i}R_{b_i}:n\in\Z_{\geq0},\,a_1,\dots,a_n,b_1,\dots,b_n\in A\right\}.
	\]
\end{definition}

It is an exercise to show that $M(A)$ is a subalgebra of $\End_K(A)$. Moreover,
if $A$ is unitary, then $M(A)$ is generated by the $L_a$ and the $R_b$ for 
$a,b\in A$.

\begin{remark}
	\label{rem:SkolemNoether}
	For $f\in M(A)$,
	there are $a_1,\dots,a_n,b_1,\dots,b_n\in A$ such that 
	\[
		f=\sum_{i=1}^n L_{a_i}R_{b_i}
	\]
	and the set $\{b_1,\dots,b_n\}$ is linearly independent. En efecto,
	si tomamos $n$ minimal entonces los $b_j$ son linealmente independientes:
	si $b_n=\sum_{j=1}^{n-1}\lambda_jb_j$ entonces
	$f=\sum_{i=1}^{n-1}L_{a_i+\lambda_ia_n}R_{b_i}$, que contradice la
	minimalidad de $n$.
\end{remark}

\begin{lemma}
	\label{lem:SkolemNoether1}
	Sea $A$ un álgebra central simple. Si $\sum_{i=1}^n L_{a_i}R_{b_i}=0$ y el
	conjunto $\{b_1,\dots,b_n\}$ (resp. $\{a_1,\dots,a_n\}$) es linealmente
	independiente, entonces $a_i=0$ (resp. $b_i=0$) para todo
	$i\in\{1,\dots,n\}$.
\end{lemma}

\begin{proof}
	Primero observemos que el resultado es válido para $n=1$. Queremos
	demostrar que si $a_1xb_1=0$ para todo $x\in A$ y $b_1\ne0$ entonces
	$a_1=0$. Supongamos que $a_1\ne 0$. Entonces el ideal de $A$ generado por
	$a_1$ es no nulo y luego es igual a $A$. Esto implica que existen
	$u_1,\dots,u_m,v_1,\dots,v_m\in A$ tales que $1=\sum_{j=1}^m u_ja_1v_j$.
	Podemos escribir entonces 
	\[
		0=\sum_{j=1}^m L_{u_j}(L_{a_1}R_{b_1})L_{v_j}=\sum_{j=1}^m L_{u_ja_1v_j}R_{b_1}=R_{b_1}
	\]
	y luego $b_1=0$. 

	Supongamos que el lema no es cierto y sea $n>1$ el menor entero positivo
	donde el lema es falso. Supongamos que $a_n\ne 0$. Como $A$ es simple, el
	ideal generado por $a_n$ es $A$ y luego existen
	$u_1,\dots,u_m,v_1,\dots,v_m\in A$ tales que $1=\sum_{j=1}^m u_ja_1v_j$.
	Entonces
	\[
		0=\sum_{j=1}^m L_{u_j}\left(\sum_{i=1}^n L_{a_i}R_{b_i}\right)L_{v_j}=\sum_{i=1}^n\sum_{j=1}^m L_{u_ja_iv_j}R_{b_i}=\sum_{i=1}^n L_{c_i}R_{b_i},
	\]
	donde $c_i=\sum_{j=1}^m u_ja_iv_j$ y obviamente $c_n=1$. Como 
	\[
		0=L_x\left(\sum_{i=1}^n L_{c_i}R_{b_i}\right)-\left(\sum_{i=1}^n L_{c_i}R_{b_i}\right)L_x=\sum_{i=1}^{n-1}L_{xc_i-c_ix}R_{b_i}
	\]
	para todo $x\in A$, la minimalidad de $n$ implica que $xc_i-c_ix=0$ para
	todo $x\in A$. Luego, como $A$ es central, $c_i\in k$ para todo
	$i\in\{1,\dots,n-1\}$. Al evaluar $0=\sum_{i=1}^n L_{c_i}R_{b_i}$ en $1_A$
	se obtiene que $0=c_1b_1+\cdots+c_nb_n$, una contradicción a la
	independencia lineal de $\{b_1,\dots,b_n\}$. 
\end{proof}

\begin{lemma}
	\label{lem:SkolemNoether2}
	Si $A$ es un álgebra central simple de dimensión finita, entonces $M(A)=\End_k(A)$. 
\end{lemma}

\begin{proof}
	Sea $\{a_1,\dots,a_n\}$ una base de $A$. El conjunto
	$\{L_{a_i}R_{a_j}:1\leq i,j\leq n\}$ es linealmente independiente: si
	$\sum_{i,j=1}^n\lambda_{ij}L_{a_i}R_{a_j}=0$ entonces
	$\sum_{i=1}^nL_{a_i}R_{c_i}=0$, donde
	$c_i=\sum_{j=1}^n\lambda_{ij}R_{a_j}$. Como los $a_i$ son linealmente
	independientes, el lema~\ref{lem:SkolemNoether1} implica que $c_i=0$ para
	todo $i\in\{1,\dots,n\}$, una contradicción a la independencia lineal de los $a_j$.  
	Luego $\dim_kM(A)\geq n^2=\dim\End_k(A)$.
\end{proof}

\begin{definition}
	\index{Automorfismo!interior}
	Sea $R$ un anillo unitario. Un automorfismo $f\in\Aut(R)$ se dice
	\textbf{interior} si existe un elemento inversible $r\in R$ tal que
	$f(x)=rxr^{-1}$ para todo $x\in R$.
\end{definition}

\begin{example}
	El automorfismo $\C\to\C$ dado por $z\mapsto\overline{z}$ no es interior.
\end{example}

\begin{example}
	Sea $\lambda\in k\setminus\{0\}$ y sea $R=k[X]$. El automorfismo $k[X]\to
	k[X]$, $f(X)\mapsto f(X+\lambda)$, no es interior.
\end{example}

\begin{example}
	Sea $R$ un anillo. El automorfismo $R\times R\to R\times R$, $(x,y)\mapsto
	(y,x)$, no es interior.
\end{example}

\begin{theorem}[Skolem--Noether]
	\index{Teorema!de Skolem--Noether}
	\label{thm:SkolemNoether}
	Si $A$ es un álgebra central simple de dimensión finita, todo automorfismo
	de $A$ es interior.
\end{theorem}

\begin{proof}	
	Sea $f\in\Aut(A)$. Gracias al lema~\ref{lem:SkolemNoether2}, 
	$f=\sum_{i=1}^n	L_{a_i}R_{b_i}$. 
	Sin perder generalidad podemos suponer que $a_1\ne 0$ y
	que $\{b_1,\dots,b_n\}$ es linealmente independiente
	(observación~\ref{rem:SkolemNoether}). Como $f$ es morfismo,
	$L_{f(x)}f=fL_x$ para todo $x\in A$. Entonces
	\[
		0=\sum_{i=1}^n L_{f(x)a_i-a_ix}R_{b_i}
	\]
	y luego, por el lema~\ref{lem:SkolemNoether1}, $f(x)a_1-a_1x=0$ para todo
	$x\in A$. Para terminar la demostración basta ver que $a_1$ es inversible:
	Como $a_1\ne 0$ y $A$ es simple, el ideal de $A$ generado por $a_1$ es $A$; esto nos permite escribir 
	$1=\sum_{i=1}^m u_ja_1v_j$ y luego $a_1$ es inversible pues 
	\[
		\left(\sum_{j=1}^m u_jf(v_j)\right)a_1=a_1\left(\sum_{j=1}^m f^{-1}(u_j)v_j\right)=1.
	\]
\end{proof}


%\begin{corollary}
%	\label{cor:SkolemNoether1}
%	Sea $A$ es un álgebra de dimensión finita unitaria. Si $a\in A$, entonces
%	$a$ es inversible o es un divisor de cero. 
%\end{corollary}
%
%\begin{proof}
%	Como $A$ es de dimensión finita, $A$ es algebraica. Existe entonces un
%	polinomio $f=\sum_{j=1}^n \lambda_jX^j\in k[X]$ (que podemos suponer de grado
%	mínimo) tal que $f(a)=0$. Al escribir
%	\[
%		0=f(a)=a(\lambda_na^{n-1}+\cdots+\lambda_2a+\lambda_1)+a_0
%	\]
%	vemos que existe un polinomio $g=\lambda_nX^{n-1}+\cdots+\lambda_1\in k[X]$
%	tal que $g(a)\ne 0$ (por la minimalidad de $n$) y $ag(a)=-\lambda_0$. Si
%	$a$ no es un divisor de cero, entonces $\lambda_0\ne 0$ y luego
%	$a^{-1}=-\lambda_0^{-1}g(a)$. 
%\end{proof}
%
%\begin{corollary}
%	Sea $A$ un álgebra de dimensión finita unitaria y sean $a,b\in A$. Si
%	$ab=1$, entonces $ba=1$.
%\end{corollary}
%
%\begin{proof}
%	Es consecuencia inmediata del corolario~\ref{cor:SkolemNoether1}.
%\end{proof}
%
%\begin{corollary}
%	Sea $D$ un álgebra de división de dimensión finita y sea $A$ una subálgebra
%	de $D$. Entonces $A$ es un álgebra de división.
%\end{corollary}
%
%\begin{proof}
%	Sea $a\in A\setminus\{0\}$. Como existe $d\in D$ tal que $ad=1$. Como $a$ es algebraico, 
%	existe $f\in k[X]$ de grado mínimo tal que $f(a)=0$. Luego $a$ es inversible con 
%	$a^{-1}=-\lambda_0^{-1}g(a)$ para algún $g\in k[X]$ tal que $g(a)\ne 0$. En
%	particular, $a^{-1}\in A$ y además $A$ es unitaria. 
%\end{proof}




\backmatter

\chapter{Some hints}

\section*{Lecture 1}
\section*{Lecture 2}
\section*{Lecture 3}
\section*{Lecture 4}
\section*{Lecture 5}

\begin{sol}{xca:K[G]notsimple}
Consider the proper non-zero ideal 
\[
	I(G)=\left\{\sum_{g\in G}\lambda_gg\in K[G]:\sum_{g\in G}\lambda_g=0\right\}.
\]
%es un ideal propio y no nulo de $K[G]$ (pues $\dim I(G)=\dim K[G]-1$). Este
%conjunto se conoce como el \textbf{ideal de aumentación} de $K[G]$.
\end{sol}
\section*{Lecture 6}
\section*{Lecture 7}
\section*{Lecture 8}
\section*{Lecture 9}
\section*{Lecture 10}
\section*{Lecture 9}
\section*{Lecture 10}
\section*{Lecture 11}
\section*{Lecture 12}
\section*{Lecture 13}

\addcontentsline{lec}{chapter}{Some hints}

\chapter*{Some solutions}

%\section*{Lecture 1}
%\section*{Lecture 2}
%\section*{Lecture 3}

% ?????

% 1.10
\begin{sol}{xca:G_zero_divisors}
    Let $g\in G$ be an element of order $n>1$.  
    Then $G$ has zero divisors, as 
    \[
    0=g^n-1=(g-1)(g^{n-1}+\cdots+g+1).
    \]
\end{sol}

% 1.11
\begin{sol}{xca:units_UP}
    Let 
    \[
    \varphi\colon K[G]\to A,\quad  
    \varphi\left(\sum_{g\in G}\lambda_gg\right)=\sum_{g\in G}\lambda_gf(g).
    \]
    Routine calculations show that $\varphi$ is a well-defined
    algebra homomorphism. By definition, $\varphi(g)=f(g)$ for all $g\in G$, that is $\varphi|_G=f$. 
\end{sol}

\begin{sol}{xca:K_cyclic}
    Let $\varphi\colon K[X]\to K[G]$, $\sum\lambda_iX^i\mapsto \sum\lambda_ig^i$. Then 
    $\varphi$ is a subjective 
    algebra homomorphism with kernel $(X^n-1)$. To prove that 
    $\ker\varphi=(X^n-1)$ we proceed as follows. The inclusion 
    $(X^n-1)\subseteq\ker\varphi$ is trivial. Conversely, 
    if $p(X)\in\ker\varphi$, divide $p(X)$ by 
    $X^n-1$ to obtain polynomials $q(X)$ and $r(X)$ such that
    \[
    p(X)=(X^n-1)q(X)+r(X),
    \]
    where $r(X)=0$ or $\deg r(X)<n$. Write $r(X)=\sum_{i=0}^{n-1}\lambda_iX^i$. 
    Then
    \[
    0=\varphi(p(X))=\varphi(X^n-1)\varphi(q(X))+\varphi(r(X))
    =\sum_{i=1}^{n-1}\lambda_ig^i.
    \]
    Since $\{1,g,\dots,g^{n-1}\}$ is linearly independent, 
    $\lambda_0=\cdots=\lambda_{n-1}=0$ and hence $r(X)=0$. 
    Thus $p(X)\in (X^n-1)$. The isomorphism theorem 
    now implies that 
    \[
    K[X]/(X^n-1)\simeq K[G].
    \]
\end{sol}

\begin{sol}{xca:invertible_subgroups}
Note that $K[H]$ is included in $K[G]$. This proves the easy implications. For the non-trivial implication, we use the map 
\[ 
\pi_H\colon K[G]\to K[H],\quad \pi_H\left(\sum_{g\in G}\lambda_gg\right)=\sum_{g\in H}\lambda_gg.
\] 
Assume that $\alpha\in K[G]$ is invertible. Then there exists $\beta\in K[G]$ such that \[ 
\alpha\beta=\beta\alpha=1.
\]
Then $\alpha\pi_H(\beta)=\pi_H(\alpha\beta)=\pi_H(1)=1$ and similarly $\pi_H(\beta)\alpha=\pi_H(\beta\alpha)=1$. 

Now if $\alpha\beta=0$ for some non-zero $\beta\in K[G]$, then take $g\in G$ with $1\in\supp \beta g$. Since $\alpha(\beta g)=0$, it follows that 
\[ 
\alpha\pi_H(\beta g)=\pi_H(\alpha\beta g)=\pi_H(0)=0. 
\]
But $\pi_H(\beta g)\ne 0$, as $1\in\supp\beta g$. 
\end{sol}

\begin{sol}{xca:isos_dihedral}
Let $G$ be the group $\langle r,s:r^3=s^2=1,srs=r^{-1}\rangle$ 
and $\omega$ be a primitive cubic root of one. 
\begin{enumerate}
    \item Consider the map 
\[
\varphi\colon \C[G]\to \C\times \C\times M_2(\C),\quad 
r\mapsto (1,1,R),\quad 
s\mapsto (1,-1,S),
\]
where 
\[
R=\begin{pmatrix}
    \omega & 0\\
    0 & \omega^2
\end{pmatrix},\quad 
S=\begin{pmatrix}
    0 & 1 \\
    1 & 0
    \end{pmatrix}. 
\] 
Since $R^3=S^2=\begin{pmatrix}1&0\\0&1\end{pmatrix}$ and $SRS=R^{-1}$, the map 
is a well-defined bijective algebra homomorphism. 
\item Consider the map 
\begin{align*} 
\Q[G]\to\Q\times\Q\times M_2(\Q),&&
r\mapsto \left(1,1,\begin{pmatrix}0&1\\-1&-1\end{pmatrix}\right),&&
s\mapsto \left(1,-1,\begin{pmatrix}1&1\\0&-1\end{pmatrix}\right). 
\end{align*}
This map is a bijective algebra homomorphism. 
\end{enumerate}
\end{sol}


\begin{sol}{xca:simple=>prim} 
	Since $R$ is unitary, there exists a maximal left ideal $I$ and $R$ is regular.
	By Proposition~\ref{proposition:R/I}, $R/I$ is a simple $R$-module. 
	Since $\Ann_R(R/I)$ is an ideal of $R$ and $R$ is simple, either $\Ann_R(R/I)\in\{0\}$ or 
	$\Ann_R(R/I)=R$. Moreover, since 
	$1\not\in\Ann(R/I)$, it follows that 
	$\Ann_R(R/I)=\{0\}$. 
\end{sol}

\begin{sol}{xca:prim+conm=cuerpo}
	If $R$ is a field, then $R$ is primitive because it is a unitary simple ring, see  
	Exercise~\ref{xca:simple=>prim}. If $R$ is a primitive commutative ring, Proposition~\ref{proposition:R/I} implies that there exists a maximal regular ideal $I$
	such that  
	$R/I$ is a faithful simple $R$-module. 
	Since $I\subseteq \Ann_R(R/I)=\{0\}$ and $I$ is regular, there exists $e\in R$ such that 
	$r=re=er$. Therefore $R$ is a unitary commutative ring. Since $I=\{0\}$ is a maximal ideal, 
	$R$ is a field. 
\end{sol}

\begin{sol}{xca:maximal=>primitive}
	Let $R$ be a ring with identity and $M$ be a maximal ideal of $R$. Then 
	$R/M$ is a simple unitary ring by 
	Proposition~\ref{proposition:R/I}. Then $R/M$ is primitive by
	Exercise~\ref{xca:simple=>prim}. By Lemma~\ref{lemma:primitivo}, 
	$M$ is primitive. 
\end{sol}




% \begin{sol}{xca:M_n(R)primitive}
%     Assume that $M_n(R)$ is primitive. Let $W$ be a faithful 
%     simple $M_n(R)$-module. Let 
%     $V=\{w\in W:L_1v=0\}$, where $L_1$ is a matrix
%     with the first column equal to zero. Then $V$ is a subgroup
%     of $W$.
    
%     We claim that $V\ne\{0\}$. If 
%     $x$ is a matrix where all the rows are zero except the first one, 
%     then $L_1x=0$. Hence $0\ne x\in V$....
% \end{sol}

\begin{sol}{xca:M_n(R)primitive}
    Let $W$ be a faithful simple $M_n(R)$-module. Let 
    $L_1$ be the subset of $M_n(R)$ of matrices 
    with the first column equal to zero and 
    $V=\{w\in W:L_1v=0\}$. Then $V$ is a subgroup of $W$. 

    We claim that $V\ne\{0\}$. Let $x$ be 
    a non-zero matrix where only the first row is non-zero. Then 
    $L_1x=0$ and hence $x\in V$. 

    We claim that $xW\ne\{0\}$. 
    Let $R\times V\to W$, $(r,v)\mapsto E_{11}(r)v$, where $E_{11}(r)$ is the matrix 
    with $r$ in position $(1,1)$ and zero elsewhere. Since 
    $L_1E_{11}(r)=0$, $rv\in V$. Routine calculations show
    that $V$ is an $R$-module. 
    
    We claim that $V$ is simple. 
    If $v\in V\setminus\{0\}$, then 
    $M_n(R)V=W$. In particular, if $u\in V$, 
    there exists $a\in M_n(R)$ 
    such that $av=u$. Write $a=E_{11}(r)+l_1+c$ for
    some $r\in R$, $l_1\in L_1$ and a matrix 
    $c=(c_{ij})$ with $c_{i1}=0$ for all $i\geq2$. Then
    \[
    u=av=E_{11}(r)v+l_1v+cv.
    \]
    Since $l_1v=0$, $cv=u-E_{11}(r)v\in V$. Let $b\in M_n(R)$ 
    be such that $b=l_1'+d$ for $l_1'\in L_1$ and a matrix $d$
    with only the first column different from zero. Since $cv\in V$, 
    $l_1'cv=0$ and $dvc=0$. It follows that
    $bcv=(m+d)cv=0$. Hence $M_n(R)cv=0$ and 
    therefore $cv=0$, so $u=E_{11}(r)v=rv$. This implies that 
    $Rv=V$ and $RV=V$. Thus $\{0\}$ and $V$ are the only 
    submodules of $V$. 

    Now we prove that $V$ is faithful. If $rV=\{0\}$, then
    $E_{11}(r)V=\{0\}$. Let $v\in V\setminus\{0\}$. Then
    $M_n(R)v=W$. If $w\in W$, then there exists 
    a matrix $f$ with only the first column different from zero
    such that $fv=w$. Then
    \[
    E_{11}(r)w=E_{11}(r)fv=E_{11}(r)E_{11}(f_{11})v=0.
    \]
    It follows that $E_{11}rW=\{0\}$ and
    hence $E_{11}(r)=0$, which means $r=0$.
\end{sol}
%

% 12.6
\begin{sol}{xca:Z_semiprimitive}
    We show that $\Z$ is a non-trivial subdirect product of the family $\{\Z/p:p\text{ prime}\}$. 
    Let $f\colon\Z\to \prod_{p}\Z/p$, $f(m)=(m\bmod p)_{p}$. Since $\cap_p\Z/p=\{0\}$, routine calculations show that 
    $f$ is an injective homomorphism. For each prime number $q$, let $\pi_q\colon\prod_{p}\Z/p\to\Z/q$ be the canonical map. A straightforward computation 
    shows that 
    $\pi_pf$ is a surjective ring homomorphism. 
\end{sol}

% 12.13
\begin{sol}{xca:D_semiprime_semiprimitive}
\begin{enumerate}
    \item Let $R=D[X]$. We first prove that $R$ is semiprime. If $fRf=0$, in particular, 
    $f^2=0$. This implies that $f=0$. 
    We now prove that $R$ is semiprimitive. Let $f\in J(R)$. Then $f$ is not invertible. By Exercise \ref{xca:Jcon1}, $1-f$ is invertible. Thus $\deg(1-f)=0$ and therefore
    $f=0$. 

    \item Let $R=D[\![X]\!]$. Since $R$ is a domain, 
    it is semiprime. Let us prove that, however, $R$ is not
    semiprimitive. Let $I$ be ideal of $R$ 
    generated by $X$. 
    Since $f=\sum_{n\geq0}a_nX^n\in R$ is invertible if and only 
    if $a_0\ne 0$, one obtains that $f\in I$ for every non-invertible $f$. 
    This implies
    that every proper left ideal is contained in $I$. Hence 
    $I$ is the unique maximal left ideal of $R$. Therefore 
    $J(R)=I$. 
    \end{enumerate}
\end{sol}


\begin{sol}{xca:idempotents_modpm}
    Let $x\ne 0$ be an idempotent of $\Z/p^m$. Then $x^2\equiv x\bmod p^m$. 
    Write $x=p^ry$ for some $r$ and $y$ such that $\gcd(p,y)=1$. Then 
    \[
    p^{2r}y^2=x^2\equiv x=p^ry\bmod p^m,
    \]
    which means that $p^{m-r}$ divides $y(p^ry-1)$. Since 
    $\gcd(p,y)=1$, it follows that 
    $p^ry\equiv 1\bmod p^{m-r}$. This implies that $r=0$ and 
    therefore $x=y=1\bmod p^m$.
\end{sol}

\begin{sol}{xca:idempotents_modn}
    Let $n=p_1^{\alpha_1}\cdots p_k^{\alpha_k}$ be the decompositions 
    of $n$ into primes. For each $i$, let $R_i=\Z/p_i^{\alpha_i}$. By the 
    Chinese remainder theorem, $R\simeq R_1\times\cdots\times R_k$. Exercise~\ref{xca:idempotents_modpm} states that each $R_i$ has exactly  
    two idempotents. Thus $R$ has $2^k$ idempotents. 
\end{sol}

\begin{sol}{xca:lifting_idempotents}
    Let $x+I\in R/I$ be an idempotent. Then $x^2-x\in I$. Since $I$ is nil, 
    there exists $n$ such that $(x^2-x)^n=0$. Thus 
    \begin{align*}    
    0=(x^2-x)^n&=\sum_{i=0}^n (-1)^i\binom{n}{i}x^{n-i}
    =\sum_{i=0}^n(-1)^i\binom{n}{i}x^{n+i}\\
    &=x^n+\sum_{i=1}^n(-1)^i\binom{n}{i}x^{n+i}
    =x^n-x^{n+1}\sum_{i=1}^n\binom{n}{i}(-1)^{i-1}x^{i-1}.
    \end{align*}
    Let $y=\sum_{i=1}^n (-1)^{i-i}\binom{n}{i}r^{n-i}$. Then 
    $xy=yx$ and $x^n=x^{n+1}y$. Let $e=(xy)^n$. 

    We claim that $e^2=e$. Since $x$ and $y$ commute, 
    \[
    e^2=(xy)^{2n}=x^{2n}y^{2n}=x^{n+1}yx^{n-1}y^{2n-1}=x^{2n-1}y^{2n-1}=\cdots=x^ny^n=e.
    \]
    
    Since $x+I=x^2+I$, 
    by induction one proves that $x+I=x^k+I$ for all $k\geq1$. Since 
    \[
    x+I=x^n+I=x^{n+1}y+I=(x^{n+1}+I)(y+I)=xy+I, 
    \]
    one obtains that 
    \[
    x+I=x^n+I=(x+I)^n=(xy+I)^n=(xy)^n+I=e+I.
    \]
\end{sol}

%\section*{Lecture 5}
%\section*{Lecture 6}
%\section*{Lecture 7}
%\section*{Lecture 8}
%\section*{Lecture 9}
%\section*{Lecture 10}

\begin{sol}{xca:invertible_algebraic}
	Since $a$ is algebraic, 
	\[
		a^n(1+\lambda_1a+\cdots+\lambda_ma^m)=0
	\]
	for some minimal $n\geq0$ and scalars $\lambda_1,\dots,\lambda_m$. If  
	$n>0$, then 
	\[
	b=(1+\lambda_1a+\cdots+\lambda_ma^m)a^{n-1}\ne 0
	\]
	is such that $ab=ba=0$. If $n=0$, then  
	\[
		c=-\lambda_1-\lambda_2a-\cdots-\lambda_ma^{m-1}\ne 0
	\]
	is such that $ac=ca=1$. 
\end{sol}

\begin{sol}{xca:C_p:local}
    Note that $K[G]\simeq K[X]/(X^p-1)$ by 
    Exercise \ref{xca:K_cyclic}. Since $K$ has characteristic $p>0$, 
    $(X^p-1)=(X-1)^p$. Thus 
    \[
    K[G]\simeq K[X]/(X^p-1)=K[X]/( (X-1)^p)
    \]
    is a commutative local ring with maximal ideal $(X-1)/(X-1)^p$. 
\end{sol}

\begin{sol}{xca:K[G]_domain_easy}
    If $G$ is trivial, then $K[G]\simeq K$ is a domain. Conversely, 
    assume that $G$ is not trivial. Let $g\in G$ be an element of
    order $n>1$. Then $g^n=1$ and 
    $1,g,\dots,g^{n-1}$ are all distinct. Then $K[G]$ is not a domain, as 
  \[
  (1-g)(1+g+\cdots+g^{n-1})=1-g^n=0
  \]   
  but $1-g\ne 0$ and $1+g+\cdots+g^{n-1}\ne 0$. 
\end{sol}


\begin{sol}{xca:reduced}

\end{sol}

\begin{sol}{xca:reduced_RX}

\end{sol}

\begin{sol}{xca:reduced_central}
\end{sol}

\begin{sol}{xca:x^3=x}
\end{sol}

%\section*{Lecture 9}
%\section*{Lecture 10}
%\section*{Lecture 11}
%\section*{Lecture 12}
%\section*{Lecture 13}

\addcontentsline{lec}{chapter}{Some solutions}


\bibliographystyle{abbrv}
\bibliography{refs}
\addcontentsline{lec}{chapter}{References}


\printindex     
\phantom{Trick}
\addcontentsline{lec}{chapter}{\indexname}

\end{document}





