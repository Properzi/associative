\thispagestyle{plain}
\section*{Introduction}

The notes correspond to the master  
course \textbf{Associative Algebra} of the 
Vrije Universiteit Brussel, 
Faculty of Sciences, 
Department of Mathematics and Data Sciences. The course
is divided into twelve two-hour lectures. 

% \bigskip 
% \textbf{Exercises}
\subsection*{Mandatory and bonus exercises}

The notes include many exercises, some with full detailed solutions. Mandatory exercises have a \colorbox{green!5!white}{green background}, while optional ones (bonus exercises) have a \colorbox{yellow!15!white}{yellow background}.

\subsection*{Prerequisites}

The reader should have a solid understanding of an undergraduate-level abstract algebra course. For reference, you can review my notes for the VUB courses: \href{https://github.com/vendramin/group}{Group Theory} and \href{https://github.com/vendramin/rings}{Ring and Module Theory}.

The content presented here draws heavily from \cite{MR3308118}, \cite{MR1449137}, 
and \cite{MR798076}. Additionally, I have 
followed the outstanding \href{https://ysharifi.wordpress.com}{blog} on 
abstract algebra by Yaghoub Sharif.


\subsection*{Final projects} 

The document contains various topics for final projects, including some that could be expanded into bachelor's or master's theses. Here are some additional topics.

% \subsection*{Rickart's theorem}

% In Lecture \ref{09} we presented an algebraic proof of Rickart's theorem. 
% The original proof uses analysis; see Appendix \ref{section:Rickart} or \cite[(6.4) of Chapter II]{MR1838439}. 

%\subsubsection*{Connel's theorem}
%
%In Lecture \ref{11} we presented the statement of Connel's theorem, which
%characterizes prime group rings over fields of characteristic zero 
%(see Theorem \ref{thm:Connel}); the proof of this  
%result appears for example in \cite[Theorem 2.10 of Chapter 4]{MR798076}. 
%As a corollary, one obtains 
%that, if $K$ is a field of characteristic zero,
%then the group ring $K[G]$ is left artinian if and only if the group
%$G$ is finite; see 
%\cite[Theorem 1.1 of Chapter 10]{MR798076} for a proof. 

\subsubsection*{Kolchin's theorem}

Let $U_n(\C)$ be the subgroup of $\GL_n(\C)$ 
of matrices $(u_{ij})$ such that 
\[
u_{ij}=\begin{cases}
1&\text{if $i=j$},\\
0&\text{if $i>j$}.\end{cases}
\]

A matrix $a\in\GL_n(\C)$ is said to be \emph{unipotent} 
if its characteristic polynomial is of the form $(X-1)^n$. 
A subgroup $G$ of $\GL_n(\C)$ is said to be \emph{unipotent} if
each $g\in G$ is unipotent. 

An important theorem of Kolchin states that 
every unipotent subgroup of $\GL_n(\C)$ is conjugate
of some subgroup of $U_n(\C)$. The theorem and its proof 
appear, for example, 
in the 
VUB course \href{https://github.com/vendramin/representation}{Representation theory of algebras}.

\subsubsection*{The double centralizer theorem}

Let $R$ be a ring. 
The \emph{centralizer} of a subring $S$ of $R$ 
is 
$C_R(S)=\{r\in R: rs=sr\text{ for all $s\in S$}\}$. 
Clearly, $C_R(C_R(S))\supseteq S$, but equality does not always hold. 
The double centralizer theorem give conditions under which one can conclude that the equality occurs; see \cite[Chapter 4]{MR3308118}. 

\subsubsection*{The Amitsur--Levitzki theorem}

The theorem states that 
if $A$ is a commutative algebra, then 
the matrix algebra 
$M_n(K)$ satisfies the identity 
\[
s_{2n}(a_{1},\dots ,a_{2n})=0,
\]
where 
\[
s_{n}(X_1,\dots,X_n)=\sum_{\sigma\in\Sym_n}\sgn(\sigma)X_{\sigma(1)}\cdots X_{\sigma(n)}.
\]
See \cite[Theorem 6.39]{MR3308118} for the beautiful 
proof found by Rosset. 

\subsubsection*{Non-commutative Hilbert's basis theorem}

There exists a non-commutative version of the celebrated
Hilbert's basis theorem. It is based on the theory of Ore's extensions (also known as \emph{skew polynomial rings}). The theorem
appears in \cite[I.8.3]{MR1321145}; see \cite[I.7]{MR1321145} 
for the basic theory of Ore's extensions. 

\subsubsection*{The Golod--Shafarevich theorem}

This is an important theorem of non-commutative algebra
with several interesting applications, for example, in group theory. 
A quick proof (and some applications) can be found in the book \cite{MR1449137} of Herstein. 

% \subsection*{The Brauer group}

% The Brauer group is a helpful tool to classify division algebras over fields. It can also be defined in terms of Galois cohomology. 
% See \cite{MR1233388} for the definition and some properties. 

\subsubsection*{The Weyl algebra}

The Weyl algebra is the quotient of the free algebra on two generators
$X$ and $Y$ by the ideal generated by the element
$YX-XY-1$. The Weyl algebra is a simple ring that is 
not a matrix ring over a division ring. It is also a non-commutative domain and an Ore extension. See \cite{MR1838439} for more information. 
In 1968, Dixmier conjectured that any 
endomorphism of a Weyl algebra is an automorphism; the conjecture
is still open. 

\subsection*{Acknowledgments}

Thanks go to Ilaria Colazzo, 
Luca Descheemaeker, Lukas Simons.   

\bigskip 
This version 
was compiled on \today~at~\currenttime.
Please send comments and corrections to me at \url{Leandro.Vendramin@vub.be}. 


