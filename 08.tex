\chapter{}



\topic{Semisimple modules}

% todo: va antes de la densidad de Jacboson

In the first lectures, we studied semisimple modules over finite-dimensional 
algebras. Let us now review the theory of semisimple modules over rings. 
A (finitely generated) module $M$ (over a ring $R$) is \textbf{semisimple} 
if it is isomorphic to a (finite) direct sum of simple modules. 

\begin{definition}
    Let $R$ be a ring. A left ideal $L$ is said to be \textbf{minimal}
    if $L\ne\{0\}$ and there is no left ideal $L_1$
    such that $\{0\}\subsetneq L_1\subsetneq L$.
\end{definition}

The ring $\Z$ contains no minimal left ideals. If $I$ is a non-zero 
left ideal of $\Z$, then
$I=(n)$ for some $n>0$ and $I=(n)\supsetneq (2n)$. 

\begin{proposition}
    Let $R$ be a left artinian ring. 
    Then every non-zero left ideal contains a minimal left ideal. 
\end{proposition}

\begin{proof}
    Let $X$ be the family of non-zero left ideals contained in $I$. Then $X$ is non-empty, as 
    $I\in X$. Then $X$ contains a minimal element by Proposition \ref{pro:artinian_minimal}. 
\end{proof}

% \begin{proposition}
% 	Let $R$ be a unitary ring and $M$ be a unitary semisimple module. 
% 	The following statements are equivalent:
% 	\begin{enumerate}
% 		\item $M$ is noetherian.
% 		\item $M$ is artinian.
% 		\item $M$ is a direct sum of finitetely many simple modules. 
% 	\end{enumerate}
% \end{proposition}

% \begin{proof}
% 	We first prove that $3)\Longleftrightarrow1)$ and that 
% 	$3)\Longleftrightarrow2)$. Como cada submódulo simple es artiniano y
% 	noetheriano, $M$ resulta artiniano y noetheriano. Recíprocamente, si $M$ es
% 	artiniano, $I$ debe ser finito pues de lo contrario podríamos elegir
% 	elementos $i_1,i_2,i_3,\dots$ de $I$ tales que la sucesión
% 	\[
% 		\bigoplus_{i\in I}M_i\supsetneq \bigoplus_{i\in I\setminus\{i_1\}}M_i\supsetneq\bigoplus_{i\in I\setminus\{i_1,i_2\}}M_i\supsetneq\cdots
% 	\]
% 	nunca se estabiliza. Análogamente, si $M$ es noetheriano, podríamos elegir
% 	elementos $i_1,i_2,i_3,\dots\in I$ tales que la sucesión
% 	\[
% 		M_{i_1}\subsetneq M_{i_1}\oplus M_{i_2}\subsetneq\cdots
% 	\]
% 	nunca se estabiliza.
% \end{proof}

\begin{definition}
    \index{Ring!semisimple}
    A ring $R$ with identity is \textbf{semisimple} if it is a direct sum of (finitely many) 
    minimal left ideals. 
\end{definition}

Why finitely many minimal left ideals?
Suppose that $R=\oplus_{i\in I}L_i$, where $\{L_i:i\in I\}$ is a collection of
minimal left ideals of $R$. 
Since $R$ is unitary, $1=\sum_{i\in I}e_i$ (finite sum) for 
some $e_i\in L_i$. This means that the set $J=\{i\in I:e_i\ne 0\}$ is finite.
Note that $R=\oplus_{j\in J}L_j$, as 
if $x\in R$, then 
\[
x=x1=\sum_{j\in J}xe_j\in \bigoplus_{j\in J}L_j.
\]

Note
that $\prescript{}{R}{R}$ is finitely generated by $\{1\}$. Minimal left ideals of $R$ 
are exactly the simple submodules of $\prescript{}{R}{R}$. 
This means that 
the ring $R$ is semisimple if and only if the module
$\prescript{}{R}{R}$ is semisimple.  

\begin{proposition}
    Let $R$ be a semisimple ring. Then $R$ is noetherian and artinian.
\end{proposition}

\begin{proof}
    Write $R$ as a direct sum $R=L_1\oplus\cdots\oplus L_n$ of minimal left ideals. Since 
    each $L_j$ is a simple submodule of $\prescript{}{R}{R}$, it follows that 
    \[
    L_1\oplus\cdots\oplus L_n\supsetneq L_2\oplus\cdots\oplus L_n\supsetneq\cdots\supsetneq L_n\supsetneq\{0\}
    \]
    is a composition series for $\prescript{}{R}{R}$ with composition factors
    $L_1,\dots,L_n$. Since the module $\prescript{}{R}{R}$ admits a composition
    series, it is artinian and noetherian by Theorem \ref{thm:serie_de_composicion}. It follows
    from the definitions that $R$ is left artinian and left noetherian. 
\end{proof}

\begin{exercise}
\label{xca:semisimple}
    If $R$ is a semisimple ring, every $R$-module is semisimple. 
\end{exercise}

% \begin{proof}
%     Since $R$ is semisimple, the module $\prescript{}{R}{R}$ is semisimple. 
%     Let $M$ be a module over $R$. Then $R$ is isomorphic to 
%     a quotient of a free module $F\simeq \oplus_{i\in I}{R}$. Since $F$
%     is semisimple, $M$ is semisimple. 
% \end{proof}

% Now it is possible to prove Artin--Wedderburn's theorem for rings. 
% If $R$ is a semisimple ring, then
% \[
% R\simeq \prod_{i=1}^k M_{n_i}(D_i)
% \]
% for some $n_1,\dots,n_k\geq1$ and some
% division rings $D_1,\dots,D_k$. 
% The proof is somewhat
% the same we did for finite-dimensional algebras.

\begin{exercise}
\label{xca:M_n(D)_semisimple}
    Prove that if $D$ is a division ring, then 
    $M_n(D)$ is semisimple. 
\end{exercise}

To see a concrete example, note that  
$M_2(\R)$ is semisimple, as  
\[
M_2(\R)=\left\{\begin{pmatrix}
a&b\\
c&d
\end{pmatrix}\right\}
=
\left\{\begin{pmatrix}
a&0\\
b&0
\end{pmatrix}\right\}
\oplus
\left\{\begin{pmatrix}
0&c\\
0&d
\end{pmatrix}\right\}\simeq D\oplus D
\]
and $D$ is a minimal left ideal of $M_2(\R)$. 

\begin{theorem}
	\label{thm:SSartin=J}
	Let $R$ be a unitary ring. Then $R$ is semisimple if and only if 
	$R$ is left artinian and $J(R)=\{0\}$.
\end{theorem}

\begin{proof}
    If $R$ is semisimple, then $R$ is left artinian by the previous proposition. 
    Moreover, 
    there are finitely many 
    minimal left ideals $L_1,\dots,L_k$ of $R$ 
    such that $R\simeq L_1\oplus\cdots\oplus L_k$. We claim that
    for each $i\in\{1,\dots,k\}$, the ideal
    $M_i=\sum_{j\ne i}L_j$ of $R$ is maximal. For example, let us prove
    that $M_1$ is maximal. If not, there exists an ideal $I$ of $R$ 
    such that $M_1\subsetneq I$. Let $x\in I\setminus M_1$ and write 
    \[
    x=x_1+x_2+\cdots+x_k
    \]
    for $x_j\in L_j$. Since $x_2+\cdots+x_k\in M_1\subseteq I$, 
    it follows that $x_1\in I\cap I_1$, a contradiction.  
    
    Conversely, if $R$ is left artinian and $J(R)=\{0\}$, then
    $R\simeq M_{n_1}(D_1)\times\cdots\times M_{n_k}(D_k)$ 
    for division rings $D_1,\dots,D_k$, this is Artin--Wedderburn theorem. 
    Since each $M_{n_j}(D_j)$ 
    is semisimple, it follows that $R$ is semisimple. 
\end{proof}

% We shall need a lemma.

% \begin{lemma}
% 	\label{lem:Jartiniano}
% 	Let $R$ be a unitary left artinian ring. There exists finitely many maximal ideals 
% 	$I_1,\dots,I_n$ of $R$ such that 
% 	$J(R)=I_1\cap\cdots\cap I_n$.
% \end{lemma}

% \begin{proof}
%     The set $X$ of left ideals of the form
% 	$I_1\cap\cdots\cap I_n$ for finitely many maximal ideals $I_1,\dots,I_n$ of $R$
% 	is non-empty, as $R$ contains maximal ideals since it is a unitary ring. 
%     Since $R$ is left artinian,
% 	Proposition~\ref{pro:artinian_minimal} implies that $X$ 
% 	contains a minimal element, say 
% 	$J=\bigcap_{i=1}^k I_i$. We claim that $J=J(R)$. 
% 	Since $R$ is
% 	unitary, $J(R)$ is the intersection of all maximal ideals of $R$ and hence 
% 	$J(R)\subseteq J$. Let us now prove that $J\subseteq J(R)$. 
% 	If not, let $x\in
% 	J\setminus J(R)$. Then there exists a maximal ideal $M$ such that $x\not\in
% 	M$. This implies that $J\cap M\subsetneq J$, a contradiction to the minimality of 
%     $J$. 
% \end{proof}

% We now prove the theorem. 

% \begin{proof}[Proof of Theorem \ref{thm:SSartin=J}]
% 	Assume first that $R$ is semisimple. By Artin--Wedderburn's theorem, 
% 	\[
% 		R\simeq\prod_{i=1}^kM_{n_i}(D_i)
% 	\]
% 	for some $n_1,\dots,n_k\geq1$ and some division rings $D_1,\dots,D_k$. 
% 	In particular, $R$ is left artinian and $J(R)=\prod_{i=1}^kJ(M_{n_i}(D_i))=\{0\}$
% 	because each $M_{n_i}(D_i)$ is simple. 

%     Conversely, the previous lemma implies that $\{0\}=J(R)=I_1\cap\cdots\cap I_k$ for some
%     maximal ideals $I_1,\dots,I_k$. Since each $R/I_i$ is simple, it follows that 
%     $\prod_{i=1}^k R/I_i$ is semisimple. The map 
% 	\[
% 	R\to \prod_{i=1}^k R/I_i,\quad
% 	x\mapsto (x+I_1,\dots,x+I_k),
% 	\]
% 	is an ring homomorphism with 
% 	kernel $I_1\cap\cdots\cap I_k=\{0\}$. Thus it is injective and hence 
% 	it follows that $R$ 
% 	is also semisimple. 
% \end{proof}

% % Como consecuencia tenemos el siguiente resultado:

% % \begin{proposition}
% % 	Sea $G$ un grupo. Entonces $\C[G]$ es artiniana a izquierda si y sólo si
% % 	$G$ es finito. 
% % \end{proposition}

% % \begin{proof}
% % 	Si $G$ es finito sabemos que $\C[G]$ es artiniano a izquierda por ser de
% % 	dimensión finita.  Recíprocamente, si $G$ es infinito, sabemos que
% % 	$J(\C[G])=0$ (por el teorema de Rickart) y que $\C[G]$ no es semisimple
% % 	(por la proposición~\ref{pro:nunca_SS}). Luego $\C[G]$ no es artiniana a izquierda por el
% % 	teorema~\ref{thm:SSartin=J}.
% % \end{proof}

% % Concluimos la sección con el siguiente teorema:

% We now present an important result that uses 
% semisimplicity. 

\topic{Hopkins--Levitski's theorem}

\begin{theorem}[Hopkins--Levitszki]
	\label{thm:Hopkins-Levitski}
	\index{Hopkins--Levitski theorem}
	Let $R$ be a unitary left artinian ring. Then $R$ is left noetherian.
\end{theorem}

\begin{proof}
	Let $J=J(R)$. Since $R$ is left artinian, $J$ is a nilpotent ideal 
	by Theorem~\ref{thm:Jnilpotente}. Let $n$ be such that $J^n=\{0\}$. Now consider the sequence 
	\[
		R\supsetneq J\supsetneq J^2\supsetneq\cdots\supsetneq J^{n-1}\supsetneq J^n=\{0\}.
	\]
	Each $J^{i}/J^{i+1}$ is a module over $R$ annihilated by $J$, 
	that is $J\cdot (J^i/J^{i+1})=\{0\}$, as 
	\[
	x\cdot (y+J^{i+1})=xy+J^{i+1}\subseteq JJ^i+J^{i+1}=J^{i+1} 
	\]
	if $x\in J$ and $y\in J^i$. 
	Thus each  
	$J^i/J^{i+1}$ is a module over $R/J$. Since $R/J$ is left artinian and 
	$J(R/J)=\{0\}$ by Theorem \ref{thm:J(R/J)=0}, it follows that $R/J$ is semisimple. 
	In particular, since every $R/J$-module is semisimple,  
	each $J^{i}/J^{i+1}$ 
	is semisimple and hence it is left noetherian. 
	
	Now suppose that $R$ is not left noetherian. Let $m$ be the largest non-negative integer
	such that $J^m$ is not left noetherian. Note that $0\leq m<n$. The sequence
	\[
	 \begin{tikzcd}
        0 & J^{m+1} & J^m & J^{m}/J^{m+1} & 0
        \arrow[from=1-1, to=1-2]
        \arrow[from=1-2, to=1-3]
        \arrow[from=1-3, to=1-4]
        \arrow[from=1-4, to=1-5]
    \end{tikzcd}
	\]
	is exact. Since $J^{m+1}$ is left noetherian by the definition of $m$ 
	and $J^m/J^{m+1}$ is left noetherian, it follows that 
	$J^m$ is noetherian, a contradiction. 
\end{proof}

% Más adelante uso la semisimplicidad del álgebra de grupo, como 
% que todo ideal tenga complemento. Así necesito definir un anillo semisimple, con la representación regular.
% Hopkins-Levitksy tiene que hacerse después de Artin-Wederburn, sin usar nada. Puedo consultar
% las notas de Bell. 
% También puede mirarse el Hungerford. ¿Dónde se usa el 1 en Hopkins-Levitski?
% Lo del álgebra de grupo tiene que moverse después de Jacobson
% quizá alcance con escribir bien la sucesión exacta


% % agregar como corolario el teorema de fermat! 
% \topic{Hopkins--Levitski's theorem}

% Note that if $R$ is semiprimitive and left artinian, then 
% any artinian $R$-module is noetherian, as 
% $R\simeq M_{n_1}(D_1)\times\cdots\times M_{n_k}(D_k)$ 
% for some division rings $D_1,\dots,D_k$. Each $M_{n_i}(D_i)$ 
% is a finite-dimensional vector space over $D_i$. Since 
% each  $M_{n_i}(D_i)$ is noetherian, it follows
% that $R$ is noetherian, as $R$ is the 
% direct product of finitely many noetherian rings.

