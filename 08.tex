\section{Lecture: 21/11/2024}

\subsection{Prime rings}

In commutative algebra, domains play a fundamental role. In non-commutative
algebra, certain things could be quite different. 
For example, the ring $M_n(\C)$ is not a domain.
We need a non-commutative generalization of domains.

\begin{definition}
	\index{Ring!prime} 
	Let $R$ be a ring (not necessarily with one). Then $R$ is
	\emph{prime} if for $x,y\in R$ such that $xRy=\{0\}$ it follows that $x=0$ or 
	$y=0$.
\end{definition}

\index{Domain}
A ring $R$ is a \emph{domain} if $xy=0$ implies
$x=0$ or $y=0$. Each domain is trivially a prime ring.

\begin{example}
    A commutative ring is prime if and only if it is a domain, as $ab=0$ 
    if and only if $aRb=\{0\}$.
\end{example}

\begin{example}
    A non-zero ideal of a prime ring is a prime ring.
\end{example}

\begin{exercise}
\label{xca:domain<=>prime+reduced}
    A ring is a domain if and only if
    it is both prime and reduced. 
\end{exercise}

A characterization of prime rings:

\begin{proposition}
    Let $R$ be a ring. The following statements are equivalent:
	\begin{enumerate}
		\item $R$ is prime.
		\item If $I$ and $J$ are left ideals such that $IJ=\{0\}$, then 
			$I=\{0\}$ or $J=\{0\}$.
		\item If $I$ and $J$ are ideals such that $IJ=\{0\}$, then $I=\{0\}$ or
			$J=\{0\}$.
	\end{enumerate}
\end{proposition}

\begin{proof}
	We first prove that $1)\implies2)$. Let $I$ and $J$ be left ideals such that
	$IJ=\{0\}$. Then $IRJ=I(RJ)\subseteq IJ=\{0\}$. If $J\ne
	\{0\}$, $u\in I$ and $v\in J\setminus\{0\}$, then $uRv\in IRJ=\{0\}$. Hence 
	$u=0$.

	The implication $2)\implies3)$ is trivial. 

    Let us prove that $3)\implies1)$. Let $x,y\in R$ be such that $xRy=\{0\}$.
	Let $I=RxR$ and $J=RyR$. Since $IJ=(RxR)(RyR)\subseteq R(xRy)R=\{0\}$, 
	we may assume that $I=\{0\}$. In particular, $Rx$ and $xR$ are ideals, as 
	$R(xR)=(Rx)R=\{0\}$. Then $\Z x$ is an ideal of $R$ such that $(\Z x)R=\{0\}$. 
	Thus $x=0$. 
\end{proof}

Simple rings are trivially prime. The converse is not true. For example,
$\Z$ is a domain, so it is a prime ring but is not simple.

\begin{example}
	If $R_1$ and $R_2$ are rings, $R=R_1\times R_2$ is not prime, as 
	$I=R_1\times\{0\}$ and $J=\{0\}\times R_2$ are non-zero ideals such that $IJ=\{0\}$.
\end{example}

A theorem of Connel states (see Theorem \ref{thm:Connel}) that 
if $K$ is a field of characteristic zero and $G$ 
is a group, then $K[G]$ is prime if and only if 
$G$ does not contain non-trivial finite normal subgroups. 

% \begin{theorem}[Connel]
% \index{Connel's theorem}
% \label{thm:Connel}
%     Let $K$ be a field of characteristic zero and
%     $G$ be a group. Then $K[G]$ is prime
%     if and only if $G$ does not contain 
%     non-trivial finite normal subgroups. 
% \end{theorem}

% \begin{proof}
%     See for example \cite[Theorem 2.10 of Chapter 4]{MR798076}.
% \end{proof}

\begin{lemma}
	\label{lem:primoizqmin=>prim}
	Let $R$ be a prime ring and $L$ be a minimal left ideal of $R$.
	Then $R$ is primitive. 
\end{lemma}

\begin{proof}
	Since $L$ is a minimal left ideal, it is simple as a module over $R$. 
	We claim that $L$ is faithful. Let $y\in L\setminus\{0\}$ and
	$x\in\Ann_R(L)$. Since $xRy\in xRL\subseteq xL=\{0\}$, it follows that 
	$x=0$.
\end{proof}

\begin{lemma}
	\label{lem:denso_artiniano}
	Let $D$ be a division ring and $R$ be a dense ring in a module $V$ over $D$. 
	If $R$ is left artininian, then $\dim_DV<\infty$.
\end{lemma}

\begin{proof}
	Assume that $\dim_DV=\infty$ and let $\{u_1,u_2,\dots,\}$ be a linearly independent set. 
	Since $R\subseteq\End_D(V)$, it follows that 
	$V$ is a module over $R$ with $f\cdot v=f(v)$, where $f\in R$ y $v\in V$. 
	For $n\in\Z_{>0}$ let 
	\[
		I_n=\Ann_R(\{u_1,\dots,u_n\}).
	\]
	Each $I_j$ is a left ideal of $R$ and $I_1\supseteq
	I_2\supseteq\cdots\supseteq I_n\supseteq\cdots$. Let 
	$n\in\Z_{>0}$ and $v\in V\setminus\{0\}$. Since $R$ is dense
	in $V$, there exists $f\in R$ such that $f(u_j)=0$ for all $j\in\{1,\dots,n\}$ and 
	$f(u_{n+1})=v\ne0$. Thus $I_1\supsetneq I_2\supsetneq\cdots\supsetneq
	I_n\supsetneq\cdots$, a contradiction.
\end{proof}

\begin{theorem}[Wedderburn]
	\index{Wedderburn's theorem}
	Let $R$ be a left artinian ring. The following statements are equivalent:
	\begin{enumerate}
		\item $R$ is simple.
		\item $R$ is prime.
		\item $R$ is primitive.
		\item $R\simeq M_n(D)$ for some $n$ and some division ring $D$.
	\end{enumerate}
\end{theorem}

\begin{proof}
	The implication $1)\implies2)$ is trivial. 
	
	To show that $2)\implies3)$ first note that 
	$R$ contains a minimal left ideal, as $R$ is left artinian. 
	By Lemma~\ref{lem:primoizqmin=>prim}, $R$ is primitive. 

	Now we prove that $3)\implies4)$. If $R$ is primitive, 
	Jacobson's density theorem implies that there exists a division
	ring $D$ such that  
	$R$ is isomorphic to a ring $S$ that is dense in a vector space $V$ over $D$.
	Since $R$ is left artinian, Lemma~\ref{lem:denso_artiniano} implies that  
	$R=\End_D(V)\simeq M_n(D)$, as $\dim_DV<\infty$. 

	Finally, $4)\implies1)$ is trivial, as $M_n(D)$ is simple. 
\end{proof}

We now prove Artin--Wedderburn theorem. We will assume that our ring
is a unitary left artinian ring. One could prove
Artin--Wedderburn's theorem for arbitrary rings --see for example \cite{MR600654}--  
but when dealing with unitary rings, the proof 
is simpler. We will prove
that left artinian semiprimitive unitary rings
are isomorphic to a direct product
of finitely many matrix rings. The idea of the proof goes as follows. 
We know that if 
$R$ is semiprimitive, then $R$ is a subdirect product
of primitive rings; that is  
there exists an injective map
\[
R\to \prod_{i\in I}R/I_i
\]
where each $I_i$ is a primitive ideal. Since $R$ is left artinian, 
the set $I$ will be finite. Moreover, 
by Wedderburn's theorem, 
$R/I_i\simeq M_{n_i}(D_i)$ for some division ring $D_i$. Finally,
a non-commutative version of the Chinese remainder theorem
implies that the map is fact surjective. 

\begin{definition}
\index{Ideal!prime}
    An ideal $I$ of $R$ is \emph{prime} if $xRy\subseteq I$ implies
    $x\in I$ or $y\in I$.
\end{definition}

Note that a ring $R$ is prime if and only if $\{0\}$ is a prime ideal. 
Moreover, 
an ideal $I$ of $R$ is prime if and only if 
the ring $R/I$ is prime. 

\begin{lemma}
    If $R$ is left artinian and $I$ is a primitive ideal, then
    $I$ is prime. 
\end{lemma}

% no necesito hacerlo para artiniano

\begin{proof}
    Since $I$ is primitive, then $R/I$ is primitive. By Wedderburn theorem, 
    $R/I$ is prime and hence $I$ is prime. 
\end{proof}

\begin{theorem}[Artin--Wedderburn]
\label{thm:ArtinWedderburn_version2}
\index{Artin--Wedderburn's theorem}
    Let $R$ be a semiprimitive left artinian unitary ring. Then
    $R\simeq\prod_{i=1}^kM_{n_i}(D_i)$ for finitely many
    division rings $D_1,\dots,D_k$. 
\end{theorem}

We shall need the following lemmas.

\begin{lemma}
\label{lem:primitive=>maximal}
    Let $R$ be a left artinian ring and $I$ be a primitive ideal. 
    Then $I$ is maximal. 
\end{lemma}

\begin{proof}
    If $I$ is a primitive ideal of $R$, then $R/I$ is a primitive ring
    by Lemma \ref{lemma:primitivo}. By Wedderburn's theorem, $R/I$ is
    simple. Thus $I$ is maximal by Proposition \ref{proposition:R/I}. 
\end{proof}

\begin{lemma}
    Let $R$ be a left artinian unitary ring.
    Let $I_1,\dots,I_k$ be finitely many distinct maximal ideals of $R$. 
    Then $I_2\cdots I_k\not\subseteq I_1$.   
\end{lemma}

\begin{proof}
    Suppose the result is not true and let $k$ be minimal
    such that $I_2\cdots I_k\subseteq I_1$. Since the result is clearly
    true for two distinct maximal ideals, $k\geq3$. Let $I=I_2\cdots I_{k-1}$. 
    Since $I\not\subseteq I_1$, there exists $x\in I\setminus I_1$. Moreover,  
    there exists $y\in I_k\setminus I_1$, as 
    $I_k\ne I_1$. 
    Then 
    $(xR)y\subseteq II_k\subseteq I_1$. Note that $I_1$ is prime: if $xy\in xRy\subseteq I_1$ and $x\not\in I_1$, 
    then $(x,M)=R$. Thus $1=rx+m$ for some $r\in R$ and $m\in M$. By multiplying by $y$ on the left, 
    $y=r(xy)+my\in M$. Now that we know that $I_1$ is prime, 
    it follows that either $x\in I_1$ or $y\in I_1$, a contradiction.   
\end{proof}

\begin{lemma}
    Let $R$ be a left artinian unitary ring. Then $R$ has only 
    finitely many primitive ideals.
\end{lemma}

\begin{proof}
    If $I_1,I_2\dots$ are infinitely many primitive ideals. 
    Since $R$ is left artinian, the sequence 
    $I_1\supseteq I_1I_2\supseteq\cdots$ stabilizes, so there
    exists $n$ such that 
    \[
    I_1I_2\cdots I_n=I_1I_2\cdots I_nI_{n+1}\subseteq I_{n+1}.
    \]
    This contradicts the previous lemma, 
    as each $I_j$ is a maximal ideal. 
\end{proof}

Now we are ready to prove the theorem. 

\begin{proof}[Proof of Theorem \ref{thm:ArtinWedderburn_version2}]
    Let $I_1,\dots,I_k$ be the (distinct) primitive ideals of $R$. 
    We know that each $I_i$ is a maximal ideal. Thus $I_i+I_j=R$ for
    $i\ne j$. Since $R$ is semiprimitive, 
    $I_1\cap\cdots\cap I_k=J(R)=\{0\}$. Let 
    \[
    \varphi\colon R\to \prod_{i=1}^k R/I_i,\quad
    x\mapsto (x+I_1,\dots,x+I_k).
    \]
    Then $\varphi$ is a ring homomorphism with kernel $I_1\cap\cdots\cap I_k=\{0\}$, so
    $\varphi$ is injective. We need to prove that $\varphi$ is surjective. 
    
    We first claim that 
    $I_1+( I_2\cdots I_k) = R$. In fact, 
    since $I_1,\dots,I_k$ are maximal ideals, $I_2\cdots I_k\not\subseteq I_1$. This implies
    that $I_1+(I_2\cdots I_k)$ is an ideal of $R$ that contains $I_1$. Since $I_1$ is maximal, 
    \[
    I_1+(I_2\cdots I_k)=R.
    \]
    
    Since $I_1+( I_2\cdots I_k) = R$, 
    there exists $x_1\in \prod_{j=2}^kI_j$ such that $1\in x_1+I_1$. Note that
    \[
    x_1=(1+I_1)\cap (I_2\cdots I_k)\subseteq I_j
    \]
    for all $j\in\{2,\dots,k\}$. 
    Thus 
    \[
    \varphi(x_1)=(x+I_1,x+I_2,\dots,x+I_k)=(1+I_1,I_2,\dots, I_k).
    \]
    Similarly,
    there exists $x_2\in 1+I_2,\dots, x_k\in 1+I_k$ such that 
    \begin{align*}
    \varphi(x_2)&=(I_1,1+I_2,\dots,I_k),\\
    &\vdots\\
    \varphi(x_k)&=(I_1,I_2,\dots,1+I_k).
    \end{align*}
    From this, it follows that $\varphi$ is surjective. Each $R/I_i$ 
    is primitive and hence isomorphic to $M_{n_i}(D_i)$ for some 
    $n_i$ and some division ring $D_i$. Therefore
    \[
    R\simeq R/I_1\times\cdots\times R/I_k\simeq \prod_{i=1}^kM_{n_i}(D_i).\qedhere 
    \]
\end{proof}



\subsection{Semisimple modules}

% todo: va antes de la densidad de Jacboson

In the first lectures, we studied semisimple modules over finite-dimensional 
algebras. Let us now review the theory of semisimple modules over rings. 
A (finitely generated) module $M$ (over a ring $R$) is \emph{semisimple} 
if it is isomorphic to a (finite) direct sum of simple modules. 

\begin{definition}
    Let $R$ be a ring. A left ideal $L$ is said to be \emph{minimal}
    if $L\ne\{0\}$ and there is no left ideal $L_1$
    such that $\{0\}\subsetneq L_1\subsetneq L$.
\end{definition}

The ring $\Z$ contains no minimal left ideals. If $I$ is a non-zero 
left ideal of $\Z$, then
$I=(n)$ for some $n>0$ and $I=(n)\supsetneq (2n)$. 

\begin{proposition}
    Let $R$ be a left artinian ring. 
    Then every non-zero left ideal contains a minimal left ideal. 
\end{proposition}

\begin{proof}
    Let $I$ be a non-zero left ideal and 
    $X$ be the family of non-zero left ideals contained in $I$. Then $X$ is non-empty, as 
    $I\in X$. Then $X$ contains a minimal element by Proposition \ref{pro:artinian_minimal}. 
\end{proof}

% \begin{proposition}
% 	Let $R$ be a unitary ring and $M$ be a unitary semisimple module. 
% 	The following statements are equivalent:
% 	\begin{enumerate}
% 		\item $M$ is noetherian.
% 		\item $M$ is artinian.
% 		\item $M$ is a direct sum of finitetely many simple modules. 
% 	\end{enumerate}
% \end{proposition}

% \begin{proof}
% 	We first prove that $3)\Longleftrightarrow1)$ and that 
% 	$3)\Longleftrightarrow2)$. Como cada submódulo simple es artiniano y
% 	noetheriano, $M$ resulta artiniano y noetheriano. Recíprocamente, si $M$ es
% 	artiniano, $I$ debe ser finito pues de lo contrario podríamos elegir
% 	elementos $i_1,i_2,i_3,\dots$ de $I$ tales que la sucesión
% 	\[
% 		\bigoplus_{i\in I}M_i\supsetneq \bigoplus_{i\in I\setminus\{i_1\}}M_i\supsetneq\bigoplus_{i\in I\setminus\{i_1,i_2\}}M_i\supsetneq\cdots
% 	\]
% 	nunca se estabiliza. Análogamente, si $M$ es noetheriano, podríamos elegir
% 	elementos $i_1,i_2,i_3,\dots\in I$ tales que la sucesión
% 	\[
% 		M_{i_1}\subsetneq M_{i_1}\oplus M_{i_2}\subsetneq\cdots
% 	\]
% 	nunca se estabiliza.
% \end{proof}

\begin{definition}
    \index{Ring!semisimple}
    A ring $R$ with identity is \emph{semisimple} if it is a direct sum of (finitely many) 
    minimal left ideals. 
\end{definition}

Why finitely many minimal left ideals?
Suppose that $R=\oplus_{i\in I}L_i$, where $\{L_i:i\in I\}$ is a collection of
minimal left ideals of $R$. 
Since $R$ is unitary, $1=\sum_{i\in I}e_i$ (finite sum) for 
some $e_i\in L_i$. This means that the set $J=\{i\in I:e_i\ne 0\}$ is finite.
Note that $R=\oplus_{j\in J}L_j$, as 
if $x\in R$, then 
\[
x=x1=\sum_{j\in J}xe_j\in \bigoplus_{j\in J}L_j.
\]

Note
that $\prescript{}{R}{R}$ is finitely generated by $\{1\}$. Minimal left ideals of $R$ 
are exactly the simple submodules of $\prescript{}{R}{R}$. 
This means that 
the ring $R$ is semisimple if and only if the module
$\prescript{}{R}{R}$ is semisimple.  

\begin{proposition}
    Let $R$ be a semisimple ring. Then $\prescript{}{R}{R}$ is noetherian and artinian.
\end{proposition}

\begin{proof}
    Write $R$ as a direct sum $R=L_1\oplus\cdots\oplus L_n$ of minimal left ideals. Since 
    each $L_j$ is a simple submodule of $\prescript{}{R}{R}$, it follows that 
    \[
    L_1\oplus\cdots\oplus L_n\supsetneq L_2\oplus\cdots\oplus L_n\supsetneq\cdots\supsetneq L_n\supsetneq\{0\}
    \]
    is a composition series for $\prescript{}{R}{R}$ with composition factors
    $L_1,\dots,L_n$. Since the module $\prescript{}{R}{R}$ admits a composition
    series, it is artinian and noetherian by Theorem \ref{thm:serie_de_composicion}. 
\end{proof}

The previous proposition shows 
that every semisimple ring is left artinian and left noetherian. 

\begin{exercise}
\label{xca:semisimple}
    If $R$ is a semisimple ring, every $R$-module is semisimple. 
\end{exercise}

% \begin{proof}
%     Since $R$ is semisimple, the module $\prescript{}{R}{R}$ is semisimple. 
%     Let $M$ be a module over $R$. Then $R$ is isomorphic to 
%     a quotient of a free module $F\simeq \oplus_{i\in I}{R}$. Since $F$
%     is semisimple, $M$ is semisimple. 
% \end{proof}

% Now it is possible to prove Artin--Wedderburn's theorem for rings. 
% If $R$ is a semisimple ring, then
% \[
% R\simeq \prod_{i=1}^k M_{n_i}(D_i)
% \]
% for some $n_1,\dots,n_k\geq1$ and some
% division rings $D_1,\dots,D_k$. 
% The proof is somewhat
% the same we did for finite-dimensional algebras.

\begin{exercise}
\label{xca:M_n(D)_semisimple}
    Prove that if $D$ is a division ring, then 
    $M_n(D)$ is semisimple. 
\end{exercise}

To see a concrete example, note that  
$M_2(\R)$ is semisimple, as  
\[
M_2(\R)=\left\{\begin{pmatrix}
a&b\\
c&d
\end{pmatrix}\right\}
=
\left\{\begin{pmatrix}
a&0\\
b&0
\end{pmatrix}\right\}
\oplus
\left\{\begin{pmatrix}
0&c\\
0&d
\end{pmatrix}\right\}\simeq D\oplus D
\]
and $D$ is a minimal left ideal of $M_2(\R)$. 

\begin{theorem}
	\label{thm:SSartin=J}
	Let $R$ be a unitary ring. Then $R$ is semisimple if and only if 
	$R$ is left artinian and $J(R)=\{0\}$.
\end{theorem}

\begin{proof}
    If $R$ is semisimple, then $R$ is left artinian by the previous proposition. 
    Moreover, 
    there are finitely many 
    minimal left ideals $L_1,\dots,L_k$ of $R$ 
    such that $R\simeq L_1\oplus\cdots\oplus L_k$. We claim that
    for each $i\in\{1,\dots,k\}$, the left ideal
    $M_i=\sum_{j\ne i}L_j$ of $R$ is maximal. For example, let us prove
    that $M_1$ is maximal. If not, there exists a left ideal $I$ of $R$ 
    such that $M_1\subsetneq I$. Let $x\in I\setminus M_1$ and write 
    \[
    x=x_1+x_2+\cdots+x_k
    \]
    for $x_j\in L_j$. Since $x_2+\cdots+x_k\in M_1\subseteq I$, 
    it follows that $x_1\in I\cap L_1$, a contradiction. Now the claim follows, as
    $J(R)\subseteq M_1\cap\cdots\cap M_k=\{0\}$. 
    
    Conversely, if $R$ is left artinian and $J(R)=\{0\}$, then
    $R\simeq M_{n_1}(D_1)\times\cdots\times M_{n_k}(D_k)$ 
    for division rings $D_1,\dots,D_k$, this is Artin--Wedderburn theorem. 
    Since each $M_{n_j}(D_j)$ 
    is semisimple, it follows that $R$ is semisimple. 
\end{proof}

% We shall need a lemma.

% \begin{lemma}
% 	\label{lem:Jartiniano}
% 	Let $R$ be a unitary left artinian ring. There exists finitely many maximal ideals 
% 	$I_1,\dots,I_n$ of $R$ such that 
% 	$J(R)=I_1\cap\cdots\cap I_n$.
% \end{lemma}

% \begin{proof}
%     The set $X$ of left ideals of the form
% 	$I_1\cap\cdots\cap I_n$ for finitely many maximal ideals $I_1,\dots,I_n$ of $R$
% 	is non-empty, as $R$ contains maximal ideals since it is a unitary ring. 
%     Since $R$ is left artinian,
% 	Proposition~\ref{pro:artinian_minimal} implies that $X$ 
% 	contains a minimal element, say 
% 	$J=\bigcap_{i=1}^k I_i$. We claim that $J=J(R)$. 
% 	Since $R$ is
% 	unitary, $J(R)$ is the intersection of all maximal ideals of $R$ and hence 
% 	$J(R)\subseteq J$. Let us now prove that $J\subseteq J(R)$. 
% 	If not, let $x\in
% 	J\setminus J(R)$. Then there exists a maximal ideal $M$ such that $x\not\in
% 	M$. This implies that $J\cap M\subsetneq J$, a contradiction to the minimality of 
%     $J$. 
% \end{proof}

% We now prove the theorem. 

% \begin{proof}[Proof of Theorem \ref{thm:SSartin=J}]
% 	Assume first that $R$ is semisimple. By Artin--Wedderburn's theorem, 
% 	\[
% 		R\simeq\prod_{i=1}^kM_{n_i}(D_i)
% 	\]
% 	for some $n_1,\dots,n_k\geq1$ and some division rings $D_1,\dots,D_k$. 
% 	In particular, $R$ is left artinian and $J(R)=\prod_{i=1}^kJ(M_{n_i}(D_i))=\{0\}$
% 	because each $M_{n_i}(D_i)$ is simple. 

%     Conversely, the previous lemma implies that $\{0\}=J(R)=I_1\cap\cdots\cap I_k$ for some
%     maximal ideals $I_1,\dots,I_k$. Since each $R/I_i$ is simple, it follows that 
%     $\prod_{i=1}^k R/I_i$ is semisimple. The map 
% 	\[
% 	R\to \prod_{i=1}^k R/I_i,\quad
% 	x\mapsto (x+I_1,\dots,x+I_k),
% 	\]
% 	is an ring homomorphism with 
% 	kernel $I_1\cap\cdots\cap I_k=\{0\}$. Thus it is injective and hence 
% 	it follows that $R$ 
% 	is also semisimple. 
% \end{proof}

% % Como consecuencia tenemos el siguiente resultado:

% % \begin{proposition}
% % 	Sea $G$ un grupo. Entonces $\C[G]$ es artiniana a izquierda si y sólo si
% % 	$G$ es finito. 
% % \end{proposition}

% % \begin{proof}
% % 	Si $G$ es finito sabemos que $\C[G]$ es artiniano a izquierda por ser de
% % 	dimensión finita.  Recíprocamente, si $G$ es infinito, sabemos que
% % 	$J(\C[G])=0$ (por el teorema de Rickart) y que $\C[G]$ no es semisimple
% % 	(por la proposición~\ref{pro:nunca_SS}). Luego $\C[G]$ no es artiniana a izquierda por el
% % 	teorema~\ref{thm:SSartin=J}.
% % \end{proof}

% % Concluimos la sección con el siguiente teorema:

% We now present an important result that uses 
% semisimplicity. 

\subsection{The Hopkins--Levitski theorem}

\begin{theorem}[Hopkins--Levitszki]
	\label{thm:Hopkins-Levitski}
	\index{Hopkins--Levitski theorem}
	Let $R$ be a unitary left artinian ring. Then $R$ is left noetherian.
\end{theorem}

\begin{proof}
	Let $J=J(R)$. Since $R$ is left artinian, $J$ is a nilpotent ideal 
	by Theorem~\ref{thm:Jnilpotente}. Let $n$ be such that $J^n=\{0\}$. Now consider the sequence 
	\[
		R\supsetneq J\supseteq J^2\supseteq\cdots\supseteq J^{n-1}\supseteq J^n=\{0\}.
	\]
	Each $J^{i}/J^{i+1}$ is a module over $R$ annihilated by $J$, 
	that is $J\cdot (J^i/J^{i+1})=\{0\}$, as 
	\[
	x\cdot (y+J^{i+1})=xy+J^{i+1}\subseteq JJ^i+J^{i+1}=J^{i+1} 
	\]
	if $x\in J$ and $y\in J^i$. 
	Thus each  
	$J^i/J^{i+1}$ is a module over $R/J$. Since $R/J$ is left artinian and 
	$J(R/J)=\{0\}$ by Theorem \ref{thm:J(R/J)=0}, it follows that $R/J$ is semisimple. 
	In particular, since every $(R/J)$-module is semisimple,  
	each $J^{i}/J^{i+1}$ 
	is semisimple and hence it is left noetherian. 
	
	Now suppose that $R$ is not left noetherian. Let $m$ be the largest non-negative integer
	such that $J^m$ is not left noetherian. Note that $0\leq m<n$. The sequence
	\[
	 \begin{tikzcd}
        0 & J^{m+1} & J^m & J^{m}/J^{m+1} & 0
        \arrow[from=1-1, to=1-2]
        \arrow[from=1-2, to=1-3]
        \arrow[from=1-3, to=1-4]
        \arrow[from=1-4, to=1-5]
    \end{tikzcd}
	\]
	is exact. Since $J^{m+1}$ is left noetherian by the definition of $m$ 
	and $J^m/J^{m+1}$ is left noetherian, it follows that 
	$J^m$ is noetherian, a contradiction. 
\end{proof}

% Más adelante uso la semisimplicidad del álgebra de grupo, como 
% que todo ideal tenga complemento. Así necesito definir un anillo semisimple, con la representación regular.
% Hopkins-Levitksy tiene que hacerse después de Artin-Wederburn, sin usar nada. Puedo consultar
% las notas de Bell. 
% También puede mirarse el Hungerford. ¿Dónde se usa el 1 en Hopkins-Levitski?
% Lo del álgebra de grupo tiene que moverse después de Jacobson
% quizá alcance con escribir bien la sucesión exacta


% % agregar como corolario el teorema de fermat! 
% \subsection{Hopkins--Levitski's theorem}

% Note that if $R$ is semiprimitive and left artinian, then 
% any artinian $R$-module is noetherian, as 
% $R\simeq M_{n_1}(D_1)\times\cdots\times M_{n_k}(D_k)$ 
% for some division rings $D_1,\dots,D_k$. Each $M_{n_i}(D_i)$ 
% is a finite-dimensional vector space over $D_i$. Since 
% each  $M_{n_i}(D_i)$ is noetherian, it follows
% that $R$ is noetherian, as $R$ is the 
% direct product of finitely many noetherian rings.

% \begin{theorem}[Connel]
% \label{thm:Connel_artinian}
%     Let $K$ be a field of characteristic zero and 
%     $G$ be a group. Then $K[G]$ is left artinian if and only
%     if $G$ is finite. 
% \end{theorem}

% \begin{proof}
%     It follows from Theorem \ref{thm:Connel} and
%     Hopkins--Levitzky theorem; see 
%     \cite[Theorem 1.1 of Chapter 10]{MR798076}. 
% \end{proof}

