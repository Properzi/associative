\section{}



\subsection{Semiprimitive rings}

\begin{definition}
	A ring $R$ is \textbf{semiprimitive} (or Jacobson semisimple) if  $J(R)=\{0\}$.
\end{definition}

In Lecture \ref{03} we defined primitive rings as
those rings that have a faithful simple module.  We claim that primitive rings
are semiprimitive. If $R$ is primitive, then $\{0\}$ is a primitive ideal. Since
$J(R)$ is the intersection of primitive ideals, it follows that $J(R)=\{0\}$.

\begin{example}
	If $R=\prod_{i\in I}R_i$ is a direct product of semiprimitive rings, then
	$R$ is semiprimitive. In fact, 
	\[
		J(R)=J\left(\prod_{i\in I}R_i\right)=\prod_{i\in I}J(R_i)=\{0\}.
	\]
\end{example}

\begin{example}
$\Z$ is semiprimitive, as $J(\Z)=\cap_{p}p\Z=\{0\}$.
\end{example}

\begin{example}
	\label{exa:C[a,b]}
	Let $R=C[a,b]$ be the ring of continuous maps $f\colon [a,b]\to\R$. 
	In this case $J(R)$ is the intersection of all maximal ideals of $R$. Note that 
	each maximal ideal of $R$ is of the form 
	\[
		U_c=\{f\in C[a,b]:f(c)=0\}
	\]
	for some $c\in[a,b]$. 
	%Each $U_c$ is a maximal ideal, as 
	%$C[a,b]/U_c\simeq\R$ is a field.  
	Thus $J(R)=\cap_{a\leq c\leq
	b}U_c=\{0\}$.
\end{example}

We proved in Theorem~\ref{thm:J(R/J)=0} (Lecture \ref{04}) 
that $R/J(R)$ is semiprimive. 

%El teorema de densidad de Jacobson nos permite entonces obtener el siguiente resultado:
%
%\begin{theorem}
%	Sea $R$ un anillo no radical. Entonces $R/J(R)$ es isomorfo a un producto
%	subdirecto de anillos densos en espacios vectoriales sobre anillos de
%	división.	
%\end{theorem}
%
%\begin{proof}
%	Si $R$ no es radical, $J(R)\ne R$. Luego $R/J(R))$ es semiprimitivo por el
%	teorema~\ref{thm:semiprimitivo}. El teorema~\ref{thm:subdirecto} y el
%	teorema de densidad de Jacobson completan la demostración del teorema.
%\end{proof}


%\begin{definition}
%	Let $\{R_i:i\in I\}$ be a collection of rings. A subring $R$ of
%	$\prod_{i\in I}R_i$ is said to be a \textbf{subdirect product} of %the
%	collection if each $\pi_j\colon R\to R_j$, $(r_i)_{i\in I}\mapsto %r_j$, is surjective. 
%\end{definition}

\begin{definition}
    \index{Subdirect product}
    Let $\{R_i:i\in I\}$ be an arbitrary family of rings. For each 
    $j\in I$, let 
    \[
    \pi_j\colon \prod_{i\in I}R_i\to R_j
    \]
    be the
    canonical map. We say that $R$ is a \textbf{subdirect product}
    of $\{R_i:i\in I\}$ if the following conditions hold:
    \begin{enumerate}
        \item There exists an injective ring homomorphism $f\colon R\to \prod_{i\in I}R_i$. 
        \item For each $j$, the composition 
        $\pi_jf\colon R\to R_j$ is surjective. 
    \end{enumerate}
\end{definition}

Direct products and direct sums of rings are all examples of subdirect products of rings. 

\begin{exercise}
\label{xca:Z_semiprimitive}
    Write (if possible) $\Z$ as a non-trivial subdirect product. 
\end{exercise}

\begin{example}
    Let $R$ be a ring,  
    $\{I_j:j\}$ be a collection of ideals of $R$ and
    \[
    f\colon R\to \prod_i R/I_i,
    \quad
    r\mapsto (r+I_i)_i.
    \]
    For each
    $i$, let $R_i=R/I_i$. Then $R$ is a subdirect product
    of the $R_i$ if and only if $f$ is injective. 
\end{example}


%El siguiente teorema justifica que indistintamente llamemos anillos
%semiprimitivos a los anillos semisimples Jacobson:

\begin{theorem}
	\label{thm:subdirecto}
	Let $R$ be a non-zero ring. Then $R$ is semiprimitive if and only if
	$R$ is isomorphic to a subdirect product of primitive rings. 
\end{theorem}

\begin{proof}
	Suppose first that $R$ is semiprimitive and let $\{P_i:i\in I\}$ be the collection of 
	primitive ideals of $R$. This collection is non-empty, as 
        $R$ is non-zero and semiprimitive. Each $R/P_j$ is primitive and 
	\[
 \{0\}=J(R)=\cap_{i\in I}P_i.
 \]
 For $j$ let $\lambda_j\colon R\to
	R/P_j$ and $\pi_j\colon \prod_{i\in I}R/P_i\to R/P_j$ be canonical maps. 
	The ring homomorphism 
	\[
		\phi\colon R\to\prod_{i\in I}R/P_i,\quad
		r\mapsto \{\lambda_i(r):i\in I\},
	\]
	is injective and satisfies $\pi_j\phi(R)=R/P_j$ for all 
	$j$.

	Assume now that $R$ is isomorphic to a subdirect product of primitive rings 
	$R_j$ and let 
    \[
    \varphi\colon R\to\prod_{i\in I}R_i
    \]
    be an injective homomorphism 
	such that $\pi_j(\varphi(R))=R_j$ for all $j$. For $j$ 
	let $P_j=\ker\pi_j\varphi$. Since $R/P_j\simeq R_j$, each $P_j$ is a primitive ideal. 
	If $x\in\cap_{i\in I}P_i$, then $\varphi(x)=0$ and thus $x=0$.
	Hence $J(R)\subseteq\cap_{i\in I} P_i=0$. 
\end{proof}

%\begin{example}
%	$\Z$ is isomorphic to a subdirect %product of the fields $\Z/p$, where 
%	$p$ runs over all prime numbers. 
%\end{example}

\begin{example}
	The ring $C[a,b]$ 
	of Example \ref{exa:C[a,b]}
	is isomorphic to a subdirect product of the fields 
	$C[a,b]/U_c\simeq\R$.
\end{example}


%\section{Anillos semiprimitivos}
%
%\begin{lemma}
%	\label{lem:Iunitario}
%	Sea $R$ un anillo y sea $I$ un ideal de $R$ unitario. Sea $e\in I$ la
%	unidad de $I$. Entonces $e$ es un idempotente central de $R$, $I=eR$ y
%	existe un ideal $J$ de $R$ tal que $R=I\oplus J$. Además $R\simeq I\times
%	J$.
%\end{lemma}
%
%\begin{proof}
%	Como $e\in I$, $eR\subseteq I$. Luego $I=eR$ pues $I=eI\subseteq eR$. Como
%	$ex\in I$ y $xe\in I$ para todo $x\in R$, $ex=(ex)e$ y $xe=e(xe)$. Luego
%	$ex=xe$ y entonces $e$ es central e idempotente. Sea $J=\{x-ex:x\in R\}$.
%	Es fácil demostrar que $J$ es un ideal tal que $R=I\oplus J$. Además
%	$R\simeq I\times J$, via $x\mapsto (ex,x-ex)$,
%\end{proof}
%
%A continuación daremos una demostración muy sencilla del teorema de Wedderburn
%en el caso de álgebras de dimensión finita.
%
%\begin{theorem}[Artin--Wedderburn]
%	Sea $R$ un anillo artiniano a izquierda y no nulo. Entonces $R$ es
%	semiprimo si y sólo si existen $n_1,\dots,n_r\in\N$ y existen anillos de
%	división $D_1,\dots,D_r$ tales que $R\simeq M_{n_1}(D_1)\times\cdots\times
%	M_{n_r}(D_r)$.
%\end{theorem}
%
%\begin{proof}
%	Procederemos por inducción en $\dim A$. Si $\dim A=1$\dots\framebox{} 
%
%	Supongamos entonces que $\dim A>1$. Si $A$ es un álgebra prima, el
%	resultado se sigue inmediatamente del teorema de Wedderburn. Supongamos
%	entonces que existe $a\in A\setminus\{0\}$ tal que $I=\{x\in A:aAx=0\}$ es
%	no nulo. Como $I$ es un ideal de $A$, $I$ es un álgebra semiprima.
%	\framebox{?} Como $a\not\in I$, $\dim I<\dim A$, y entonces, por hipótesis
%	inductiva, existen $n_1,\dots,n_s\in\N$ y álgebras de división
%	$D_1,\dots,D_s$ tales que 
%	\[
%		I\simeq M_{n_1}(D_1)\times\cdots\times M_{n_s}(D_s).
%	\]
%	En particular, $I$ es unitario. Por el lema~\ref{lem:Iunitario}, existe un
%	ideal $J$ de $A$ tal que $A\simeq I\times J$. Como $\dim J<\dim A$, la hipótesis inductiva
%	implica que existen $n_{s+1},\dots,n_r\in\N$ y álgebras de división $D_{s+1},\dots,D_r$ tales que
%	\[
%		J\simeq M_{n_{s+1}}(D_{s+1})\times\cdots\times M_{n_r}(D_r).
%	\]
%	Luego $A\simeq I\times J\simeq \prod_{j=1}^s M_{n_j}(D_j)$.
%\end{proof}
%
%\begin{corollary}
%	Sea $A$ un álgebra no nula de dimensión finita. Si $A$ es semiprima,
%	entonces $A$ es unitaria.
%\end{corollary}
%
%%Gracias al teorema de Wedderburn se puede ir un poco más lejos:
%%\begin{corollary}
%%	Sea $A$ un álgebra unitaria. Son equivalentes:
%%	\begin{enumerate}
%%		\item $A$ es semiprima.
%%		\item Todo $A$-módulo unitario es semisimple.
%%		\item $A$ es semisimple como $A$-módulo.
%%		\item Todo ideal a izquierda de $A$ es de la forma $Ae$ para algún
%%			idempotente $e\in A$. 
%%	\end{enumerate}
%%\end{corollary}
%%
%%\begin{proof}
%%	La implicación $(1)\implies(2)$ es el teorema de Wedderburn. 
%%	
%%\end{proof}
%
%\begin{example}
%	Por el teorema de Maschke sabemos que si $G$ es un grupo finito, 
%	$\C[G]$ es un álgebra semiprimitiva y luego semisimple.
%\end{example}
%




%\section{Viejo!}
%
%\begin{theorem}[Artin--Wedderburn]
%	\index{Teorema!de Artin--Wedderburn}
%	\label{thm:ArtinWedderburn}
%	Si $R$ es un anillo, las siguientes afirmaciones son equivalentes:
%	\begin{enumerate}
%		\item $R$ es un anillo no nulo semiprimitivo y artiniano a izquierda.
%		\item Existen anillos de división $D_1,\dots,D_r$ y tales que
%			\[
%				R\simeq\prod_{i=1}^r R_i,
%			\]
%			donde $R_i=\End_{D_i}(V_i)$
%		\item Existen anillos de división $D_1,\dots,D_r$ y enteros positivos
%			$n_1,\dots,n_r$ tales que 
%			\[
%			R\simeq M_{n_1}(D_1)\times\cdots\times M_{n_r}(D_r).
%		\]
%	\end{enumerate}
%\end{theorem}
%
%\begin{proof}
%	Demostremos que $(1)\implies(2)$. Como $R\ne0$ y $J(R)=0$, $R$ admite
%	ideales primitivos. Supongamos que existe un número finito de ideales
%	primitivos distintos, digamos $P_1,\dots,P_t$. Cada $R/P_j$ es un anillo
%	primitivo y es artiniano a izquierda \framebox{?}. Entonces, por el teorema
%	de Wedderburn, para cada $j\in\{1,\dots,t\}$ existen un anillo de división
%	$D_j$ y un entero positivo $n_j$ tales que $R/P_j\simeq M_{n_j}(D_j)$. En
%	particular, cada $R/P_j$ es simple y entonces $P_j$ es un ideal maximal de
%	$R$. Como $R/P_j$ es simple, $(R/P_j)^2\ne 0$ y luego $R^2\not\subseteq
%	P_j$. Por maximalidad, $R^2+P_j=R$ y además $P_i+P_j=R$ para todo $i\ne j$.
%	Por el teorema chino del resto,
%	\[
%		R=R/0=R/J(R)=R/\cap_{j=1}^t P_j\simeq R/P_1\times\cdots\times R/P_t.
%	\]
%	Sea $\iota_k\colon R/P_k\to \prod_{j=1}^t R/P_j$ la inclusión canónica.
%	Cada $\iota_k(R/P_k)$ es un ideal simple (es decir, que como anillo es
%	simple) de $\prod_{j=1}^t R/P_j\simeq R$. Luego las imágenes, digamos
%	$I_k$, de los $\iota_k(R/P_k)$ dan ideales simples de $R$ y
%	$R=I_1\times\cdots\times I_t$.
%
%	Demostremos ahora que $(3)\implies(1)$. Para cada $j$ sea
%	$R_j=M_{n_j}(D_j)$. Como cada $R_j$ es primitivo por el teorema de
%	Wedderburn, $J(R_j)=\{0\}$ para todo $j$. Luego
%	$J(R)=\prod_{i=1}^rJ(R_j)=\{0\}$ y entonces $R$ es semiprimitivo. Además
%	$R$ es artiniano a izquierda.\framebox{?}
%\end{proof}
%
%\begin{corollary}
%	Sea $R$ un anillo semiprimitivo.
%	\begin{enumerate}
%		\item Si $R$ es artiniano a izquierda, entonces $R$ es unitario.
%		\item $R$ es artiniano a izquierda si y sólo si es artiniano a derecha.
%		\item Si $R$ es artiniano a izquierda es noetheriano.
%	\end{enumerate}
%\end{corollary}
%
%\begin{proof}
%	La primera afirmación es consecuencia inmediata del teorema de
%	Artin--Wedderburn~\ref{thm:ArtinWedderburn}.
%\end{proof}
%
%\begin{corollary}
%	Sea $R$ un anillo semiprimitivo artiniano a izquierda y sea $I$ un ideal de
%	$R$. Entonces $I=Re$ para algún idempotente $e\in R$ tal que $e\in Z(R)$.
%\end{corollary}
%
%\begin{proof}
%		
%\end{proof}
%
%\begin{proposition}
%	Sea $R$ un anillo semisimple artiniano a izquierda. 
%	\begin{enumerate}
%		\item $R=I_1\times\cdots\times I_n$ donde los $I_j$ son ideales simples.
%		\item Si $J\subseteq R$ es un ideal simple, entonces existe $k\in\{1,\dots,n\}$ tal que $J=I_k$.
%		\item Si $R=J_1\times\cdots\times J_m$ donde los $J_k$ son ideales simples, entonces $n=m$ y existe
%			$\sigma\in\Sym_n$ tal que $I_k=J_{\sigma(k)}$ para todo $k\in\{1,\dots,n\}$.
%	\end{enumerate}
%\end{proposition}
%
%\begin{proof}
%\end{proof}
%
%\begin{theorem}
%	Sea $R$ un anillo unitario no nulo. Las siguientes afirmaciones son
%	equivalentes:
%	\begin{enumerate}
%		\item $R$ es semiprimitivo y artiniano a izquierda.
%		\item Todo $R$-módulo unitario es proyectivo.
%		\item Todo $R$-módulo unitario es inyectivo.
%		\item Toda sucesión exacta de $R$-módulos unitarios se parte.
%		\item Todo $R$-módulo unitario no nulo es semisimple.
%		\item $\prescript{}{R}R$ es unitario y semisimple.
%		\item Todo ideal a izquierda de $R$ es de la forma $Re$ para algún $e\in R$ indempotente.
%		\item $\prescript{}{R}R$ es suma directa de ideales a izquierda
%			minimales $L_1,\dots,L_m$ donde cada $L_j$ es de la forma $Re_j$, y
%			los $e_j$ son idempotentes ortogonales tales que
%			$e_1+\cdots+e_m=1$. 
%	\end{enumerate}
%\end{theorem}
%
%\begin{proof}
%	Veamos que $(4)\implies(5)$. Sea $M$ un módulo unitario y sea $N$ un
%	submódulo no nulo de $M$. Como la sucesión $0\to N\to M\to M/N\to 0$ es
%	exacta, se parte. Luego $M=N\oplus X$ para algún submódulo $X$ de $N$ tal
%	que $X\simeq M/N$. Como $M$ es unitario, $Rm\ne 0$ para todo $m\in
%	M\setminus\{0\}$. Luego $M$ es semisimple por el teorema~\framebox{?}.
%
%	Veamos ahora que $(5)\implies(4)$. Sea 
%	\[
%	\begin{tikzcd}
%		0 \arrow[r]
%		& N \arrow[r]
%		& M \arrow[r]
%		& X \arrow[r]
%		& 0
%	\end{tikzcd}
%	\]
%	una sucesión exacta corta de $R$-módulos. Como $f\colon N\to f(N)$ es un
%	isomorfismo y entonces $f(N)$ es un submódulo del semisimple $M$, $f(N)$ es
%	sumando directo de $M$. Sea $\pi\colon M\to f(N)$ el morfismo canónico.
%	Entonces $\pi f=f$ y $f^{-1}\pi\colon M\to A$ es un morfismo tal que
%	$(f^{-1}\pi)f=\id_N$.\framebox{?}
%
%	Demostremos que $(5)\implies(7)$. Sea $L$ un ideal a izquierda de $R$. Como
%	los ideales a izquierda de $R$ son los submódulos de $\prescript{}{R}R$,
%	existe un ideal a izquierda $N$ de $R$ tal que $R=L\oplus N$. Existen
%	entonces $e_1\in L$ y $e_2\in N$ tales que $1=e_1+e_2$. Si $x\in L$,
%	entonces $x=xe_1+xe_2$ y luego $xe_2=x-xe_1\in L\cap N=\{0\}$. Demostramos
%	entonces que $x=xe_1$ para todo $x\in L$. En particular, $e_1e_1=e_1$ y
%	$L=Re_1$. 
%
%	Demostremos que $(7)\implies(6)$. Sea $L$ un submódulo de
%	$\prescript{}{R}R$. Como entonces $L$ es un ideal a izquierda de $R$,
%	$L=Re$ para algún idempotente $e\in R$. Como el conjunto $R(1-e)$ es un
%	ideal a izquierda de $R$ tal que $R=Re\oplus R(1-e)$, se concluye que
%	$\prescript{}{R}R$ es semisimple.\framebox{?}
%
%	Veamos que $(6)\implies(1)$. Supongamos que $\prescript{}{R}R=\sum_{i\in
%	I}N_i$, donde los $N_j$ son submódulos simples de $\prescript{}{R}R$.
%	Reordenando los $N_j$ si fuera necesario, podemos suponer que existe
%	$k\in\N$ tal que $1=e_1+\cdots+e_k$ y $e_j\in N_j$ para todo
%	$j\in\{1,\dots,k\}$. Si $r\in R$, entonces $r=re_1+\cdots+re_k\in
%	\sum_{i=1}^k N_i$. Luego $R=\sum_{i=1}^k N_i$.
%	Veamos que $J(R)=0$. Si $r\in J(R)$ entonces, como $rN_i=0$ para todo
%	$i\in\{1,\dots,k\}$, se concluye que $r=r1=re_1+\cdots+re_k=0$. Probamos
%	entonces que $R$ es semiprimitivo. Falta ver que $R$ es artiniano a
%	izquierda. Para eso basta obvervar que, como
%	\[
%		\frac{N_1\oplus\cdots\oplus N_i}{N_1\oplus\cdots\oplus N_{i-1}}\simeq N_i
%	\]
%	para cada $i\in\{1,\dots,k\}$, la serie
%	\[
%	R=N_1\oplus\cdots\oplus N_k\supsetneq N_1\oplus\cdots\oplus N_{k-1}\supsetneq\cdots\supsetneq N_1\oplus N_2\supsetneq N_1\supsetneq 0
%	\]
%	es una serie de composición.\framebox{?}
%
%	Veamos que $(1)\implies(8)$. Sin perder generalidad podemos suponer que
%	\[
%	R=\prod_{i=1}^k M_{n_j}(D_j),
%	\]
%	donde los $D_j$ son anillos de división.\framebox{?}
%
%	Veamos que $(8)\implies(5)$. Sea $M$ un módulo unitario no nulo. Si $m\in
%	M$ entonces $L_im$ es un submódulo de $M$. Los $L_jm$ generan a $M$ pues 
%	\[
%	m=1m=e_1m+\cdots+e_km\in\sum L_im.
%	\]
%	Veamos que cada $L_jm$ es simple. Fijado $i$, la función $f\colon L_i\to
%	L_im$, $x\mapsto xm$, es un morfismo sobreyectivo. Como $L_i$ es un ideal a
%	izquierda minimal, $L_i$ es un submódulo simple. Luego $m\ne0$ implica que
%	$f$ es un isomorfismo. Probamos entonces que el conjunto $\{L_jm:1\leq
%	j\leq k,m\in M,L_jm\ne 0\}$ es una familia de submódulos simples cuya suma
%	es $M$.
%\end{proof}


\subsection{Jacobson's density theorem}

At this point, it is convenient to recall that
modules over division rings are pretty much as vector spaces over fields. 
Modules over division rings are usually called vector spaces over division rings. 

\begin{definition}
	Let $D$ be a division ring, and $V$ be a vector space over $D$. A subring 
	$R\subseteq\End_D(V)$ is a \textbf{dense ring of linear operators} 
	of $V$ (or simple, \textbf{dense} in $V$) if for every  
	$n\in\Z_{>0}$, every linearly independent set $\{u_1,\dots,u_n\}\subseteq V$ 
	and every (not necessarily linearly independent) subset $\{v_1,\dots,v_n\}\subseteq V$ 
	there exists $f\in R$ such that $f(u_j)=v_j$ for all 
	$j\in\{1,\dots,n\}$.
\end{definition}

%\begin{lemma}
%	Sea $R$ un subanillo de $\End_D(V)$. Entonces $R$ es denso en $V$ si y sólo
%	si para todo $g\in\End_D(V)$ y todo subespacio $U$ de $V$ de dimensión
%	finita existe $f\in R$ tal que $f|_U=g|_U$.
%\end{lemma}
%
%\begin{proof}
%	Supngamos que $R$ es denso en $V$. Sean $g\in\End_D(V)$ y $U\subseteq V$ un
%	subespacio de dimensión finita. Sea $\{u_1,\dots,u_n\}$ una base de $U$.
%	Como $R$ es denso, existe $f\in R$ tal que $f(u_j)=g(u_j)$ para todo
%	$j\in\{1,\dots,n\}$ y luego $f|_U=g|_U$.
%
%	Recíprocamente, sea $n\in\N$ y sean $\{u_1,\dots,u_n\}\subseteq V$ un
%	conjunto linealmente independiente y $\{v_1,\dots,v_n\}\subseteq V$. Sean
%	$g\in\End_D(V)$ tal que $g(u_j)=v_j$ para todo $j\in\{1,\dots,n\}$ y $U$ el
%	subespacio de $V$ generado por $u_1,\dots,u_n$. Por hipótesis existe $f\in
%	R$ tal que $f|_U=g|_U$ y luego $f(u_j)=g(u_j)=v_j$ para todo
%	$j\in\{1,\dots,n\}$.
%\end{proof}

\begin{proposition}
	\label{pro:unique_dense}
	Let $D$ be a division ring and 
    $V$ be a finite-dimensional $D$-vector space.  
	Then $\End_D(V)$ is the only dense ring of $V$.
\end{proposition}

\begin{proof}
	Let $R$ be dense in $V$ and let $\{v_1,\dots,v_n\}$ be a basis of $V$. By definition,
	\[
    R\subseteq\End_D(V).
    \]
    If $g\in\End_D(V)$ then, since $R$ is dense in $V$, there
	exists $f\in R$ such that $f(v_j)=g(v_j)$ for all 
	$j\in\{1,\dots,n\}$. Hence $g=f\in R$.
\end{proof}


%Ahora demostraremos el teorema de densidad de Jacobson. 
%Necesitaremos el siguiente
%lema:
%\begin{lemma}
%	\label{lem:densidad}
%	Sea $M$ un $R$-módulo simple y $D=\End_R(M)$.  Si $N$ es un subespacio de
%	$M$ tal que $\dim_DN<\infty$ y $m\in M\setminus N$, entonces existe $r\in
%	R$ tal que $rm\ne 0$ y $rN=0$.
%\end{lemma}
%
%\begin{proof}
%	Supongamos que la afirmación no es cierta y sea $N$ un contraejemplo de la
%	mínima dimensión posible. Entonces $\dim_DN\geq1$ (pues el resultado es
%	verdadero en el caso $N=0$). Sea $N_0$ un subespacio de $N$ tal que $\dim
%	N_0=\dim N-1$ y sea
%	\[
%		L=\{r\in R:rN_0=0\}.
%	\]
%	Como por la minimalidad de $N$ nuestra afirmación es cierta para $N_0$,
%	para cualquier $x\in N\setminus N_0$ se tiene que $Lx=N$ (pues existe $r\in
%	L$ tal que $rx=\ne 0$). Como $L$ es ideal a izquierda de $R$ y
%	$Lx\subseteq N$ es un submódulo, $Lx=N$ pues $N$ es simple.
%
%	Sea $w\in V\setminus U$ tal que nuestra afirmación no es cierta y sea $u\in
%	U\setminus U_0$.  La función 
%	\[
%		\delta\colon V\to V,\quad
%		v\mapsto lw,
%	\]
%	donde $v=lu\in Lu=V$ (que depende de $u$ y $w$) 
%	está bien definida: si $l_1,l_2\in L$ son tales que $v=l_1u=l_2u$ entonces $(l_1-l_2)u=0$ y luego
%	\[
%		0=\delta(0)=\delta((l_1-l_2)u)=(l_1-l_2)w=l_1w-l_2w. 
%	\]
%	Además $\delta$ es morfismo de $R$-módulos pues si $l\in L$ es tal que $v=lu$ entonces
%	\[
%		\delta(rv)=\delta(r(lu))=\delta( (rl)u)=(rl)w=r(lw)=r\delta(v)
%	\]
%	para todo $r\in R$.
%
%
%\end{proof}


\begin{theorem}[Jacobson]
	\label{thm:density}
	\index{Jacobson's density theorem}
	A ring is primitive if and only if it is isomorphic to a dense ring on 
	a vector space over a division ring.
\end{theorem}

We shall need the following lemma. 

\begin{lemma}
	\label{lem:ideal_denso}
	Let $D$ be a division ring and $V$ be a $D$-vector space. 
	If $R$ is dense in $V$ and $I$ is a non-zero ideal of $R$, then 
	$I$ is dense on $V$. 
\end{lemma}

\begin{proof}
    Fix $n\in\Z_{>0}$. Let 
    $\{u_1,\dots,u_n\}\subseteq V$ be a linearly independent set 
	and let $\{v_1,\dots,v_n\}\subseteq V$. We want to find $\gamma\in I$ such that 
	$\gamma(u_i)=v_i$ for all $i$. Since $I\ne\{0\}$, there exists 
	$h\in I\setminus\{0\}$. This means that 
	$h(u)=v\ne0$ for some $u\ne 0$. Since $R$ is 
	dense on $V$, there exist $g_1,\dots,g_n\in R$ such that 
	\[
	g_i(u_j)=\begin{cases} 
	u & \text{if $i=j$},\\
	0 & \text{otherwise}.
	\end{cases}
	\]
	Further, since $\{v\}$ is a linearly independent subset of $V$, 
	there exist $f_1,\dots,f_n\in R$ such that 
	$f_i(v)=v_i$ for all $i$. Thus $\gamma=\sum_{i=1}^n f_ihg_i\in I$ is such that 
	$\gamma(u_j)=v_j$ for all $j\in\{1,\dots,n\}$.
\end{proof}

Now we are ready to prove Jacobson's density theorem. 

\begin{proof}[Proof of Theorem \ref{thm:density}]
    Let $R$ be a ring. 
	If $R$ is isomorphic to a dense ring in $V$, where
	$V$ is a $D$-vector space for some division ring $D$, then $R$
	is primitive, as $V$ is a simple and faithful $R$-module. Why faithful? If 
	$f\in\Ann_R(V)$, then $f=0$ since $f(v)=0$ for all $v\in V$. Why simple? 
	If $W\subseteq V$ is a non-zero submodule, let $v\in V$ and $w\in
	W\setminus\{0\}$. There exists $f\in R$ such that $v=f(w)\in W$. 

	Now assume that $R$ is primitive. Let $V$ be a simple faithful module.
	Schur's lemma implies that $D=\End_R(V)$ is a division ring. Thus $V$ is
	a $D$-vector space with 
	\[
	D\times V\to V,\quad
	(\delta,v)\mapsto \delta v=\delta(v).
	\]
	For $r\in R$ let  
	\[
		\gamma_r\colon V\to V,\quad
		v\mapsto rv.
	\]
	A straightforward calculation shows that $\gamma_r\in\End_D(V)$ and that
    $\gamma\colon R\to\End_D(V)$,
	$r\mapsto\gamma_r$, is a ring homomorphism. Since $V$ is faithful,
	$R\simeq\gamma(R)=\{\gamma_r:r\in R\}$. In fact, if $\gamma_r=\gamma_s$, then 
	$rv=\gamma_r(v)=\gamma_s(v)=sv$ for all $v\in V$ and hence $r=s$, as
	$(r-s)v=0$ for all $v\in V$.

	\begin{claim}
		If $U$ is a finite-dimensional subspace of $V$, 
		for each $w\in V\setminus U$ there exists $r\in R$ such that 
		$\gamma_r(U)=\{0\}$ and $\gamma_r(w)\ne0$.
	\end{claim}

	Suppose the claim is not true. Let $U$ be a counterexample of minimal  
	dimension. Then
	$\dim_DU\geq1$, as the claim holds for the zero subspace. (For this, we only
        need to show that for each non-zero $v\in V$, there exists $r\in R$ such that $rv\ne 0$. 
        Since $V$ is simple, $V=(v)$. If $rv=0$ for all $r\in R$, then 
        $V=\{0\}$, a contradiction to the simplicity of $V$.) Let now 
	$U_0$ be a subspace of $U$ such that 
	$\dim U_0=\dim U-1$ and let 
	\[
		L=\{l\in R:\gamma_l(U_0)=\{0\}\}.
	\]
	The minimality of the dimension of $U$ shows that the claim is true for $U_0$, so
	any $v\in V\setminus U_0$ is such that $Lv=V$. Since there exists $l\in
	L$ such that $lv=\gamma_l(v)\ne 0$ and $L$ is a left ideal of $R$, it follows
	that $Lv\subseteq V$ is a submodule and the claim follows from the simplicity of
	$V$.
	
	Let $w\in V\setminus U$ be such that the claim is not true. Let $u\in
	U\setminus U_0$. The map  
	\[
		\delta\colon V\to V,\quad
		v\mapsto lw,
	\]
	where $v=lu\in Lu=V$ (that depends both on $u$ and $w$) 
	is well-defined: if $l_1,l_2\in L$ are such that 
	$v=l_1u=l_2u$, then $(l_1-l_2)u=0$ and thus 
	\[
		0=(l_1-l_2)w=l_1w-l_2w,
	\]
        as for all $r\in R$, if $rU=\{0\}$, then $rw=0$. 
	Further, $\delta$ is a homomorphism of modules over $R$, as if 
	$l\in L$ is such that $v=lu$, then 
	\[
		\delta(rv)=\delta(r(lu))=\delta( (rl)u)=(rl)w=r(lw)=r\delta(v)
	\]
	for all $r\in R$.

	For every $l\in L$,  
	\[
		l(\delta(u)-w)=l\delta(u)-lw=\delta(lu)-lw=0.
	\]
	Thus $L(\delta(u)-w)=\{0\}$. This implies that $\delta(u)-w\not\in V\setminus U_0$, 
	that is $\delta(u)-w\in U_0$. Therefore  
	\[
		w=\delta(u)-(\delta(u)-w)\in Du+U_0=U,
	\]
	a contradiction.
	
	Now the theorem follows from the claim. Let 
	$u_1,\dots,u_n\in V$ be linearly independent vectors and let 
	$v_1,\dots,v_n\in V$ arbitrary vectors. Fix $i\in\{1,\dots,n\}$. 
	The previous claim with 
	\[
		U=\langle u_1,\dots,u_{i-1},u_{i+1},\dots,u_n\rangle
	\]
	and $w=u_i$ implies that there exists $r_i\in R$ such that $\gamma_{r_i}(u_j)=0$ if 
	$j\ne i$ and $\gamma_{r_i}(u_i)\ne 0$. Since there exists $s_i\in R$ such that 
	$\gamma_{s_i}\gamma_{r_i}(u_i)=v_i$, it follows that 
	$r=\sum_{j=1}^n s_jr_j\in R$ is such that $\gamma_r(u_i)=v_i$ for all 
	$i\in\{1,\dots,n\}$.
\end{proof}

%\begin{exercise}
%	Sea $R$ un anillo denso en $V$. Demuestre que $R$ es artiniano a izquierda
%	si y sólo si $V$ es de dimensión finita.
%\end{exercise}
% si V es de dimensión finita, es fácil por el lema~\ref{lem:unico_denso}
% hungerford pag 419

\begin{corollary}
	If $R$ is a primitive ring, then either there exists a division ring $D$
	such that $R\simeq\End_D(V)$ for some finite-dimensional vector space $V$ over $D$ or 
	for all $m\in\Z_{>0}$ there exists a subring $R_m$ of 
	$R$ and a surjective ring homomorphism $R_m\to\End_D(V_m)$ for some vector space  
	$V_m$ over $D$ such that $\dim_DV_m=m$.
\end{corollary}

\begin{proof}
	The ring $R$ admits a simple faithful module $V$. Furthermore, by Jacobson's density 
	theorem we may assume that there exists a division ring $D$ 
	such that $R$ is dense in a vectoe space $V$ over $D$. 
	Let $\gamma\colon R\to\End_D(V)$, $r\mapsto\gamma_r$, where 
	$\gamma_r(v)=rv$. Since $V$ is faithful, $\gamma$ is injective. Thus 
	$R\simeq\gamma(R)$. 

	If $\dim_DV<\infty$, the result follows from Proposition \ref{pro:unique_dense}. 
	Assume that $\dim_DV=\infty$ and let $\{u_1,u_2,\dots\}$ be a linearly independent set. 
	For each $m\in\Z_{>0}$ let $V_m$ be the subspace generated by $\{u_1,\dots,u_m\}$
	and $R_m=\{r\in R:rV_m\subseteq V_m\}$. Then $R_m$ is a subring of $R$. 
	Since $R$ is dense in $V$, the map 
	\[
		R_m\to \End_D(V_m),\quad
		r\mapsto\gamma_r|_{V_m}
	\]
	is a surjective ring homomorphism. 
\end{proof}

