\section{Project: Passman's theorem}
\label{section:Passman}

% We now describe some other well-known problems
% in the theory of group rings.

\index{Trivial units in group algebras}
Let $K$ be a field and $G$ be a group. A unit $u\in K[G]$ is said
to be \emph{trivial} if $u=\lambda g$ for some $\lambda\in K\setminus\{0\}$ an
d
$g\in G$.

\begin{exercise}
\label{xca:non_trivial:C2andC5}
        Prove that $\C[C_2]$ and $\C[C_5]$ have non-trivial units.
\end{exercise}

The following question is usually attributed to Kaplansky.

\begin{question}[Units in groups algebras]
        \label{question:units}
        Let $K$ be a field and $G$ be a torsion-free group. Is it true that all
        units of $K[G]$ are
        trivial?
\end{question}

Question \ref{question:units} was negatively answered by Gardam.

\begin{theorem}[Gardam]
\label{thm:Gardam_char2}
\index{Gardam's theorem}
    Let $\F_2$ be the field of two elements. Consider the elements
    $x=a^2$, $y=b^2$ and $z=(ab)^2$ of $P$ and let
    \begin{align*}
        &p=(1+x)(1+y)(1+z^{-1}),
        &&q = x^{-1}y^{-1}+x+y^{-1}z+z,\\
        &r = 1+x+y^{-1}z+xyz,
        &&s=1+(x+x^{-1}+y+y^{-1})z^{-1}.
    \end{align*}
    Then $u=p+qa+rb+sab$ is a non-trivial unit in $\F_2[P]$.
\end{theorem}

\begin{proof}
    See \cite{MR4334981}.
\end{proof}

\begin{definition}
\index{Ring!reduced}
A ring $R$ is \emph{reduced} if for all $r\in R$ such that 
$r^2=0$ one has $r=0$.
\end{definition}

Integral domains and boolean rings are reduced. The ring $\Z/8$ of integers
modulo eight 
and $M_2(\R)$ are not reduced. 

\begin{example}
    The ring over the abelian group $\Z^n$ with multiplication  \[
    (a_1,\dots,a_n)(b_1,\dots,b_n)=(a_1b_1,\dots,a_nb_n)\]
    is reduced. 
\end{example}

The structure of 
reduced rings is described by the 
Andrunakevic--Rjabuhin theorem. It states
that a ring is reduced if and only if
it is a subdirect products of domains. See
\cite[3.20.5]{MR2015465} for a proof. 

\begin{question}[Reduced group algebras]
	\label{question:reduced}
	Let $K$ be a field and $G$ be a torsion-free group. Is it true that 
	$K[G]$ is reduced? 
\end{question}

Recall that if $R$ is a unitary ring, one proves that 
the Jacobson radical $J(R)$ is 
the set of elements $x$ such that
$1+\sum_{i=1}^n r_ixs_i$ is invertible 
for all $n$ and all $r_i,s_i\in R$.

\begin{question}[Semisimple group algebras]
	\label{question:J}
	Let $K$ be a field and $G$ be a torsion-free group. It is true that 
	$J(K[G])=\{0\}$ if $G$ is non-trivial?
\end{question}

\index{Idempotent}
Recall that an element $e$ of a ring is said to be \emph{idempotent} 
if $e^2=e$. Examples of idempotents are $0$ and $1$ and 
these are known as the \emph{trivial idempotents}. 

\begin{question}[Idempotents in group algebras]
	\label{question:idempotente}
	Let $G$ be a torsion-free group and $\alpha\in K[G]$ be an idempotent. 
	Is it true that $\alpha\in\{0,1\}$?
\end{question}

\begin{exercise}
	Prove that if $K[G]$ has no zero-divisors and $\alpha\in K[G]$ is an
	idempotent, then $\alpha\in\{0,1\}$.
\end{exercise}

\begin{exercise}
    Let $K$ be a field of characteristic two. 
	Prove that $K[C_4]$ contains non-trivial zero divisors and every
	idempotent of $K[C_4]$ is trivial. What happens if the characteristic of $K$ is not two?
\end{exercise}

For completeness, let us 
recall the following important question. 
%restate Conjecture~\ref{conjecture:zero} as follows:

\begin{question}[Zero divisors in group algebras]
    \label{question:zero}
 	Let $K$ be a field and 
  $G$ be a torsion-free group. Is it true that 
 	$K[G]$ is a domain?
\end{question}

Our goal is the prove
the following implications:
\[
\ref{question:J}\Longleftarrow\ref{question:units}
\Longrightarrow\ref{question:reduced}
\Longleftrightarrow\ref{question:zero}
\]

We first prove that an affirmative solution to Question~\ref{question:units} 
yields a solution to Question~\ref{question:reduced}. 

\begin{theorem}
\label{thm:units=>reduced}
    Let $K$ be a field of characteristic $\ne2$ 
	and $G$ be a non-trivial group. Assume that $K[G]$ has only trivial units.
	Then $K[G]$ is reduced. 
\end{theorem}

\begin{proof}
	Let $\alpha\in K[G]$ be such that $\alpha^2=0$. We claim that 
	$\alpha=0$. Since $\alpha^2=0$, 
	\[
		(1-\alpha)(1+\alpha)=1-\alpha^2=1, 
	\]
	it follows that $1-\alpha$ is a unit of $K[G]$. Since units of $K[G]$ are 
	trivial, there exist $\lambda\in K\setminus\{0\}$ and $g\in G$ such that 
	$1-\alpha=\lambda g$. We claim that $g=1$. If not, 
	\[
		0=\alpha^2=(1-\lambda g)^2=1-2\lambda g+\lambda^2g^2,
	\]
	a contradiction. Therefore $g=1$ and hence $\alpha=1-\lambda\in K$. Since
	$K$ is a field, one concludes that $\alpha=0$.
\end{proof}

\begin{exercise}
    What happens in Theorem \ref{thm:units=>reduced} if $K$ is a field of characteristic two?
\end{exercise}

We now prove that an affirmative solution to Question~\ref{question:units} 
also yields a solution to Question~\ref{question:J}. 

\begin{theorem}
	Let $K$ be a field and $G$ be a non-trivial group. Assume that $K[G]$ has only trivial units. 
	If $|K|>2$ or $|G|>2$, then $J(K[G])=\{0\}$.
\end{theorem}

\begin{proof}
	Let $\alpha\in J(K[G])$. There exist $\lambda\in K\setminus\{0\}$ and $g\in
	G$ such that $1-\alpha=\lambda g$. We claim that $g=1$. Assume $g\ne 1$. 
	If $|K|\geq3$,
	then there exist $\mu\in K\setminus\{0,1\}$ such that
	\[
		1-\alpha\mu=1-\mu+\lambda\mu g 
	\]
	is a non-trivial unit, a contradiction.
	If $|G|\geq3$, there exists $h\in G\setminus\{1,g^{-1}\}$ such that
	\[
        1-\alpha h=1-h+\lambda gh
    \]
    is a non-trivial unit, a contradiction.  Thus
	$g=1$ and hence $\alpha=1-\lambda\in K$. Therefore $1+\alpha h$ is a
	trivial unit for all $h\ne 1$ and hence 	$\alpha=0$.
\end{proof}

\begin{exercise}
	Prove that if $G=\langle g\rangle\simeq\Z/2$, then 
	$J(\F_2[G])=\{0,g-1\}\ne\{0\}$. 
\end{exercise}


We now want to prove that an affirmative answer to 
Question~\ref{question:reduced} yields an affirmative answer to Question~\ref{question:zero}. We first need some preliminaries. 

For a group $G$ we consider
\[
        \Delta(G)=\{g\in G:(G:C_G(g))<\infty\}.
\]

\begin{exercise}
    Prove that $\Delta(\Delta(G))=\Delta(G)$.
\end{exercise}

\begin{proposition}
    If $G$ is a group, then $\Delta(G)$
    is a characteristic subgroup of $G$.
\end{proposition}

\begin{proof}
        We first prove that $\Delta(G)$ is a subgroup of $G$. If $x,y\in\Delta(G)$
        and $g\in G$, then 
        \[
        g(xy^{-1})g^{-1}=(gxg^{-1})(gyg^{-1})^{-1}.
        \]
        Moreover,
        $1\in\Delta(G)$. Let us show now that $\Delta(G)$ is characteristic in $G$. If
        $f\in\Aut(G)$ and $x\in G$, then, since
        \[
        f(gxg^{-1})=f(g)f(x)f(g)^{-1},
        \]
        it follows that $f(x)\in\Delta(G)$.
\end{proof}


\begin{exercise}
        Prove that if $G=\langle r,s:s^2=1,srs=r^{-1}\rangle$ is the
        infinite dihedral group, then $\Delta(G)=\langle r\rangle$.
\end{exercise}

The following exercise uses the transfer map and
is need 
in Proposition \ref{pro:FCabeliano}. 

\begin{exercise}
        \label{xca:center}
        If $G$ is a group such that $Z(G)$ has finite index $n$, then
        $(gh)^n=g^nh^n$ for all $g,h\in G$.
\end{exercise}

\begin{proposition}
	\label{pro:FCabeliano}
	If $G$ is a torsion-free group such 
	that $\Delta(G)=G$, then $G$ is abelian.
\end{proposition}

\begin{proof}
	Let $x,y\in G=\Delta(G)$ and $S=\langle x,y\rangle$. The group $Z(S)=C_S(x)\cap C_S(y)$ has 
	finite index, say $n$, in $S$. By Exercise~\ref{xca:center}, 
	the map $S\to Z(S)$, $s\mapsto s^n$, is a group homomorphism. Thus  
	\[
		[x,y]^n=(xyx^{-1}y^{-1})^n=x^ny^nx^{-n}y^{-n}=1
	\]
	as $x^n\in Z(S)$. Since $G$ is torsion-free, $[x,y]=1$.
\end{proof}

\begin{lemma}[Neumann]
	\index{Neumann's!lemma}
	\label{lem:Neumann}
	Let $H_1,\dots,H_m$ be subgroups of $G$. 
	Assume there are finitely many elements
	$a_{ij}\in G$, $1\leq i\leq m$, $1\leq j\leq n$, such that 
	\[
		G=\bigcup_{i=1}^m\bigcup_{j=1}^n H_ia_{ij}.
	\]
	Then some $H_i$ has finite index in $G$.
\end{lemma}

\begin{proof}
	We proceed by induction on $m$. The case $m=1$ is trivial. 
	Let us assume that $m\geq2$. If $(G:H_1)=\infty$, there exists $b\in G$
	such that 
	\[
		H_1b\cap\left(
	\bigcup_{j=1}^nH_1a_{1j}\right)=\emptyset.
	\]
	Since $H_1b\subseteq\bigcup_{i=2}^m\bigcup_{j=1}^n H_ia_{ij}$, 
	it follows that 
	\[
		H_1a_{1k}\subseteq \bigcup_{i=2}^m\bigcup_{j=1}^n H_ia_{ij}b^{-1}a_{1k}.
	\]
	Hence $G$ can be covered by finitely many cosets of $H_2,\dots,H_m$. By the inductive hypothesis, 
	some of these $H_j$ has finite index in $G$.
\end{proof}

We now consider a projection operator of group algebras. If $G$ 
is a group and $H$ is a subgroup of $G$, let 
\[
	\pi_H\colon K[G]\to K[H],\quad
	\pi_H\left(\sum_{g\in G}\lambda_gg\right)=\sum_{g\in H}\lambda_gg.
\]

If $R$ and $S$ are rings, a $(R,S)$-bimodule is an abelian group
$M$ that is both a left $R$-module and a right $S$-module 
and the compatibility condition 
\[
(rm)s = r(ms)
\]
holds for all $r\in R$, $s\in S$ and $m\in M$.

\begin{exercise}
	Let $G$ be a group and $H$ be a subgroup of $G$. Prove that
	if $\alpha\in
	K[G]$, then $\pi_H$ is a $(K[H],K[H])$-bimodule homomorphism
	with usual left and right multiplications,
	\[
		\pi_H(\beta\alpha\gamma)=\beta\pi_H(\alpha)\gamma
	\]
	for all $\beta,\gamma\in K[H]$.
\end{exercise}

%\begin{proof}
%	Supongamos que $\alpha=\sum_{g\in G}\lambda_gg=\alpha_1+\alpha_2$, donde
%	$\alpha_1=\sum_{g\not\in H}\lambda_gg$ y $\alpha_2=\sum_{g\in
%	H}\lambda_gg=\pi_H(\alpha)$. Entonces
%	$\beta\alpha\gamma=\beta\alpha_1\gamma+\beta\alpha_2\gamma$, donde
%	$\beta\alpha_1\gamma\not\in K[H]$ y $\beta\alpha_2\gamma\in K[H]$.
%\end{proof}

\begin{lemma}
	\label{lem:escritura}
	Let $X$ be a left transversal of $H$ in $G$. Every $\alpha\in K[G]$ can be written
	uniquely as 
	\[
	\alpha=\sum_{x\in X}x\alpha_x,
	\]
	where $\alpha_x=\pi_H(x^{-1}\alpha)\in K[H]$.
\end{lemma}

\begin{proof}
	Let $\alpha\in K[G]$. Since $\supp\alpha$ is finite, $\supp\alpha$ is contained 
    in finitely many cosets of $H$, say $x_1H,\dots,x_nH$, where each 
	$x_j$ belongs to $X$. Write $\alpha=\alpha_1+\cdots+\alpha_n$,
	where $\alpha_i=\sum_{g\in x_iH}\lambda_gg$. If $g\in x_iH$, then 
	$x_i^{-1}g\in H$ and hence 
	\[
		\alpha=\sum_{i=1}^n x_i(x_i^{-1}\alpha_i)=\sum_{x\in X}x\alpha_x
	\]
	with $\alpha_x\in K[H]$ for all $x\in X$. For the uniqueness, note that 
	for each  $x\in X$ the previous exercise implies that  
	\[
		\pi_H(x^{-1}\alpha)
		=\pi_H\left(\sum_{y\in X}x^{-1}y\alpha_y\right)
		=\sum_{y\in X}\pi_H(x^{-1}y)\alpha_y=\alpha_x, 
	\]
	as  
	\[
		\pi_H(x^{-1}y)=\begin{cases}
		1 & \text{if $x=y$},\\
		0 & \text{if $x\ne y$}.
	\end{cases}\qedhere 
	\]
\end{proof}

\begin{lemma}
	\label{lem:ideal_pi}
	Let $G$ be a group and $H$ be a subgroup of $G$. If $I$ is a non-zero 
	left ideal
	of $K[G]$, then  $\pi_H(I)\ne\{0\}$.
\end{lemma}

\begin{proof}
	Let $X$ be a left transversal of $H$ in $G$ and $\alpha\in I\setminus\{0\}$. By Lemma
	\ref{lem:escritura} we can write $\alpha=\sum_{x\in
	X}x\alpha_x$ with $\alpha_x=\pi_H(x^{-1}\alpha)\in K[H]$ for all $x\in X$.
	Since $\alpha\ne0$, there exists $y\in X$ such that $0\ne
	\alpha_y=\pi_H(y^{-1}\alpha)\in\pi_H(I)$ ($y^{-1}\alpha\in I$ since $I$ is 
    a left ideal).
\end{proof}

Lemma \ref{lem:ideal_pi} 
can also be proved directly, without using Lemma~\ref{lem:escritura}. The proof goes as follows: if $\alpha\ne 0$, then by taking $g$ in the
support of $\alpha$, we have that $\pi_H (g^{-1}\alpha)\ne0$. 

\begin{exercise}
	Let $G$ be a group, $H$ be a subgroup of $G$ and $\alpha\in K[H]$. The following statements hold:
	\begin{enumerate}
		\item $\alpha$ is invertible in $K[H]$ if and only if $\alpha$ is
			invertible in $K[G]$.
		\item $\alpha$ is a zero divisor of $K[H]$ if and only if $\alpha$ is  
			a zero divisor of $K[G]$.
	\end{enumerate}
\end{exercise}

% \begin{sol}
% 	If $\alpha$ is invertible in $K[G]$, there exists $\beta\in K[G]$ such that 
% 	$\alpha\beta=\beta\alpha=1$. Apply $\pi_H$ and use that $\pi_H$ 
% 	is a $(K[H],K[H])$-bimodule homomorphism to obtain  
% 	\[
% 		\alpha\pi_H(\beta)=\pi_H(\alpha\beta)=\pi_H(1)=1=\pi_H(1)=\pi_H(\beta\alpha)=\pi_H(\beta)\alpha.
% 	\]
	
% 	Assume now that $\alpha\beta=0$ for some $\beta\in K[G]\setminus\{0\}$. Let $g\in G$
% 	be such that $1\in\supp(\beta g)$. Since $\alpha(\beta g)=0$, 
% 	\[
% 		0=\pi_H(0)=\pi_H(\alpha(\beta g))=\alpha\pi_H(\beta g),
% 	\]
% 	where $\pi_H(\beta g)\in K[H]\setminus\{0\}$, as $1\in\supp(\beta g)$. 
% \end{sol}

\begin{lemma}[Passman]
	\index{Passman's!lemma}
	\label{lem:Passman}
	Let $G$ be a group and 
	$\gamma_1,\gamma_2\in K[G]$ be such that $\gamma_1K[G]\gamma_2=\{0\}$.
	Then $\pi_{\Delta(G)}(\gamma_1)\pi_{\Delta(G)}(\gamma_2)=\{0\}$.
\end{lemma}

\begin{proof}
	It is enough to show that $\pi_{\Delta(G)}(\gamma_1)\gamma_2=\{0\}$, 
	as in this case
	\[
		\{0\}=\pi_{\Delta(G)}(\pi_{\Delta(G)}(\gamma_1)\gamma_2)=\pi_{\Delta}(\gamma_1)\pi_{\Delta(G)}(\gamma_2).
	\]
	Write $\gamma_1=\alpha_1+\beta_1$, where 
	\begin{align*}
		&\alpha_1=a_1u_1+\cdots+a_ru_r, && u_1,\dots,u_r\in\Delta(G),\\
		&\beta_1=b_1v_1+\cdots+b_sv_s, && v_1,\dots,v_s\not\in\Delta(G),\\
		&\gamma_2=c_1w_1+\cdots+c_tw_t,&& w_1,\dots,w_t\in G.
	\end{align*}
	The subgroup $C=\bigcap_{i=1}^rC_G(u_i)$ has finite index in $G$.
	Assume that 
	\[
		0\ne \pi_{\Delta}(\gamma_1)\gamma_2=\alpha_1\gamma_2. 
	\]
	Let $g\in\supp(\alpha_1\gamma_2)$. 
	If $v_i$ is a conjugate in $G$ of some 
	$gw_j^{-1}$, let $g_{ij}\in G$ be such that
	$g_{ij}^{-1}v_ig_{ij}=gw_j^{-1}$. If $v_i$ and $gw_j^{-1}$ 
	are not conjugate, 
	we take $g_{ij}=1$. 

	For every $x\in C$ it follows that
	$\alpha_1\gamma_2=(x^{-1}\alpha_1x)\gamma_2$. Since  
	\[
		x^{-1}\gamma_1x\gamma_2\in x^{-1}\gamma_1K[G]\gamma_2=0,
	\]
	it follows that
	\begin{align*}
		(a_1u_1+\cdots+a_ru_r)\gamma_2&=
		\alpha_1\gamma_2=x^{-1}\alpha_1x\gamma_2=-x^{-1}\beta_1x\gamma_2\\
		&=-x^{-1}(b_1v_1+\cdots+b_sv_r)x(c_1w_1+\cdots+c_tw_t).
	\end{align*}
	Now $g\in\supp(\alpha_1\gamma_2)$ implies that there exist $i,j$ such that
	$g=x^{-1}v_ixw_j$.
	Thus $v_i$ and $gw_j^{-1}$ are conjugate and hence
	$x^{-1}v_ix=gw_j^{-1}=g_{ij}^{-1}v_ig_{ij}$, that is
	$x\in C_G(v_i)g_{ij}$. This proves that 
	\[
		C\subseteq\bigcup_{i,j}C_G(v_i)g_{ij}. 
	\]
	Since $C$ has finite index in $G$, it follows that 
	$G$ can be covered by finitely many cosets of 
	the $C_G(v_i)$. Every $v_i\not\in\Delta(G)$, so 
	each $C_G(v_i)$ has infinite index in $G$, a contradiction 
	to Neumann's lemma.
\end{proof}


\begin{exercise}
    Let $K$ be a field and $G$ be a torsion-free abelian group. Prove that 
    $K[G]$ has no non-zero divisors. 
\end{exercise}


\begin{theorem}[Passman]
\index{Passman's!theorem}
\label{thm:Passman}
	Let $K$ be a field and $G$ be a torsion-free group. If 
	$K[G]$ is reduced, then $K[G]$ is a domain.
\end{theorem}

\begin{proof}
	Assume that $K[G]$ is not a domain. Let $\gamma_1,\gamma_2\in K[G]\setminus\{0\}$
	be such that $\gamma_2\gamma_1=0$. If $\alpha\in K[G]$, then
	\[
		(\gamma_1\alpha\gamma_2)^2=\gamma_1\alpha\gamma_2\gamma_1\alpha\gamma_2=0
	\]
	and thus $\gamma_1\alpha\gamma_2=0$, as $K[G]$ is reduced. In particular, 
	$\gamma_1K[G]\gamma_2=\{0\}$. Let $I$ be the left ideal of $K[G]$ generated 
	by $\gamma_2$. Since $I\ne\{0\}$, it follows
	from Lemma~\ref{lem:ideal_pi} that 
	$\pi_{\Delta(G)}(I)\ne\{0\}$. Hence 
	$\pi_{\Delta(G)}(\beta\gamma_2)\ne\{ 0\}$ for some $\beta\in K[G]$. 
	Similarly, 
	$\pi_{\Delta(G)}(\gamma_1\alpha)\ne\{ 0\}$ for some $\alpha\in K[G]$. Since 
	\[
		\gamma_1\alpha K[G]\beta\gamma_2\subseteq \gamma_1 K[G]\gamma_2=\{0\},
	\]
    it follows that $\pi_{\Delta(G)}(\gamma_1\alpha)\pi_{\Delta(G)}(\beta\gamma_2)=\{0\}$
    by Passman's lemma. Hence $K[\Delta(G)]$ has zero divisors, a contradictions
    since $\Delta(G)$ is an abelian group.
\end{proof}



